\documentclass[allowframebreaks]{beamer}
\usepackage{amssymb,amsmath,amsfonts,amsthm,amstext}
\usepackage{enumitem}
\usepackage{times}
\usepackage{lmodern}

\newcommand{\cp}[1]{\mathbb{#1}}

\mode<presentation>
{
\usetheme{Warsaw}
}

\newtheorem{teorema}{Teorema}
\newtheorem{definicao}{Definição}

\begin{document}
\frame{
  \begin{teorema}
  Considere o sistema
  \[
    AX = B
  \]
  onde $A$ \'e uma matriz $m \times n$ e $B$ \'e uma matriz $m \times 1$, ambas com entradas no corpo $\cp{K}$ e $X$ \'e uma matriz $n \times 1$. Ent\~ao:
  \begin{enumerate}[label={\roman*})]
    \item O sistema tem solu\c{c}\~ao se, e somente se, o posto da matriz ampliada \'e igual ao posto da matriz dos coeficientes.

    \item Se a matriz ampliada e a matriz dos coeficientes t\^em o mesmo posto $p$ e $p = n$, ent\~ao a solu\c{c}\~ao \'e \'unica.

    \item Se a matriz ampliada e a matriz dos coeficientes t\^em o mesmo posto $p$ e $p < n$, ent\~ao podemos escolher $n - p$ vari\'aveis, e as outras $p$ vari\'aveis ser\~ao dadas em fun\c{c}\~ao destas $n - p$ vari\'aveis escolhidas.
  \end{enumerate}
  O n\'umero $n - p$ \'e chamado de \textbf{grau de liberdade} e as $n - p$ vari\'aveis s\~ao chamadas de \textbf{vari\'aveis livres}.
\end{teorema}
}

\frame{
  \begin{definicao}
    Um conjunto n\~ao vazio $V$ \'e um \textbf{espa\c{c}o vetorial}\index{Espa\c{c}o Vetorial} sobre um corpo $\cp{K}$ se em seus elementos, chamados \textbf{vetores}, estiverem definidas duas opera\c{c}\~oes satisfazendo:\pause
    \begin{itemize}
      \item A cada par $u$, $w \in V$ corresponde um vetor $u + w \in V$, chamado \textbf{soma} de $u$ e $w$, de modo que:\pause
      \item[1)] $u + w = w + u$, para todos $u$, $w \in V$;\pause
      \item[2)] $(u + w) + x = u + (w + x)$, para todos $u$, $w$ e $x \in V$;\pause
      \item[3)] Existe em $V$ um vetor, denominado \textbf{vetor nulo} e denotado por $0_V$, tal que\pause
      \[
        0_V + u = u
      \]
      para todo $u \in V$.\pause
      \item[4)] Para cada vetor $u \in V$, existe um vetor em $V$, denotado por $-u$ tal que\pause
      \[
        u + (-u) = 0_V.
      \]
    \end{itemize}
  \end{definicao}
}
\frame{
  \begin{definicao}
    \begin{itemize}
      \item A cada par $\alpha \in \cp{K}$ e $u \in V$, corresponde um vetor $\alpha \cdot u \in V$, denominado \textbf{produto por escalar} de $\alpha$ por $u$ de modo que:\pause
      \item[5)] $(\alpha\beta)\cdot u = \alpha(\beta\cdot u)$ para todos $\alpha$, $\beta \in \cp{K}$ e todo $u \in V$;\pause
      \item[6)] $1_\cp{K}\cdot u = u$ para todo $u \in V$, onde $1_\cp{K}$ \'e o elemento neutro da multiplica\c{c}\~ao em $\cp{K}$.\pause
      \item[7)] $\alpha\cdot(u + w) = \alpha\cdot u + \alpha\cdot w$, para todo $\alpha \in \cp{K}$ e todos $u$, $w \in V$;\pause
      \item[8)] $(\alpha + \beta)\cdot u = \alpha\cdot u + \beta\cdot u$, para todos $\alpha$, $\beta \in \cp{K}$ e todo $u \in V$.\pause
    \end{itemize}
  \end{definicao}
}
\frame{
  \begin{definicao}
  Um \textbf{bloco de Jordan} $r \times r$ em $\lambda$ \'e a matrix $J_r(\lambda)$ em $\cp{M}_n(\cp{K})$ que tem $\lambda$ na diagonal principal e 1 na diagonal abaixo da principal, isto \'e,
  \[
    J_r(\lambda) = \begin{bmatrix}
      \lambda & 0 & 0 & \cdots & 0 & 0\\
      1 & \lambda & 0 & \cdots & 0 & 0\\
      0 & 1 & \lambda & \cdots & 0 & 0\\
      \vdots\\
      0 & 0 & 0 & \cdots & 1 & \lambda
    \end{bmatrix}_{r \times r}.
  \]
\end{definicao}
}
\frame{
  \begin{definicao}
    Seja $V$ um $\cp{K}$-espa\c{c}o vetorial. Para subespa\c{c}os $W_1$, \dots, $W_t$ de $V$ dizemos que $V$ \'e uma \textbf{soma direta} de $W_1$, \dots, $W_t$ se
    \begin{enumerate}[label={\roman*})]
      \item $W_1 + \cdots + W_t = \{u_1 + \cdots + u_t \mid u_i \in W_i,\ i = 1,\dots, t\} = V$\pause
      \item $W_i \cap (W_1 + \cdots + W_{i - 1} + W_{i + 1} + \cdots + W_t) = \{0_V\}$, $i = 1$, \dots, $t$.\pause
    \end{enumerate}
    Neste caso escrevemos
    \[
      V = W_1 \oplus \cdots \oplus W_t.\pause
    \]
    Se $V$ \'e um $\cp{K}$-espa\c{c}o vetorial de dimens\~ao finita tal que $V = W_1 \oplus \cdots \oplus W_t$, ent\~ao
    \[
      \dim_\cp{K}V = \sum_{i = 1}^t\dim_\cp{K}W_i.
    \]
  \end{definicao}
}
\frame{
  \begin{teorema}\label{operadornilpotente}
    Seja $T : V \to V$ um operador linear nilpotente com {\'\i}ndice de nilpot\^encia $r \ge 1$, onde $V$ \'e um $\cp{K}$-espa\c{c}o vetorial de dimens\~ao finita. Ent\~ao existem n\'umeros positivos $p$, $m_1$, \dots, $m_p$ e vetores $u_1$, \dots, $u_p$ tais que
    \begin{enumerate}[label={\roman*})]
      \item $r = m_1 \ge m_2 \ge \cdots \ge m_p$.
      \item O conjunto $\mathcal{B} = \{u_1, T(u_1), \dots, T^{m_1 - 1}(u_1); u_2, T(u_2), \dots, T^{m_2 - 1}(u_2); \dots; u_p, T(u_p), \dots, \linebreak T^{m_p - 1}(u_p)\}$ \'e uma base de $V$.
      \item $T^{m_i}(u_i) = 0_V$ para cada $i = 1$, \dots, $p$.
      \item Se $S$ for um operador linear em um $\cp{K}$-espa\c{c}o vetorial $W$ de dimens\~ao finita, ent\~ao os inteiros $p$, $m_1$, \dots, $m_p$ associados a $S$ e a $T$ s\~ao iguais se, e somente se, existir um isomorfismo $\Phi : V \to W$ com $\Phi T \Phi^{-1} = S$.
    \end{enumerate}
  \end{teorema}
}
\frame{
  \begin{teorema}\label{formadejordan}
    Seja $T : V \to V$ um operador linear, onde $V$ \'e um $\cp{K}$-espa\c{c}o vetorial de dimens\~ao finita. Suponha que
    \[
      p_T(x) = (x - \lambda_1)^{m_1}\dots(x - \lambda_1)^{m_r}
    \]
    onde $m_i \ge 1$ e $\lambda_i \ne \lambda_j$ se $i \ne j$. Ent\~ao $V = W_1 \oplus \cdots \oplus W_r$ onde para cada $i = 1$, \dots, $r$ temos:
    \begin{enumerate}[label={\roman*})]
      \item $\dim_\cp{K} W_i = m_i$
      \item O subespa\c{c}o $W_i$ \'e $T-invariante$
      \item A restri\c{c}\~ao do operador $T - \lambda_i Id$ \`a $W_i$ \'e nilpotente.
    \end{enumerate}
  \end{teorema}
}
\end{document}

