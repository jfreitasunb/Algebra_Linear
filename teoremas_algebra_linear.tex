\documentclass{beamer}
\usepackage{amssymb,amsmath,amsfonts,amsthm,amstext}
\usepackage{enumitem}
\usepackage{times}
\usepackage{lmodern}

\newcommand{\cp}[1]{\mathbb{#1}}

\mode<presentation>
{
\usetheme{Warsaw}
}

\newtheorem{teorema}{Teorema}

\begin{document}
\frame{
  \begin{teorema}
  Considere o sistema
  \[
    AX = B
  \]
  onde $A$ \'e uma matriz $m \times n$ e $B$ \'e uma matriz $m \times 1$, ambas com entradas no corpo $\cp{K}$ e $X$ \'e uma matriz $n \times 1$. Ent\~ao:
  \begin{enumerate}[label={\roman*})]
    \item O sistema tem solu\c{c}\~ao se, e somente se, o posto da matriz ampliada \'e igual ao posto da matriz dos coeficientes.

    \item Se a matriz ampliada e a matriz dos coeficientes t\^em o mesmo posto $p$ e $p = n$, ent\~ao a solu\c{c}\~ao \'e \'unica.

    \item Se a matriz ampliada e a matriz dos coeficientes t\^em o mesmo posto $p$ e $p < n$, ent\~ao podemos escolher $n - p$ vari\'aveis, e as outras $p$ vari\'aveis ser\~ao dadas em fun\c{c}\~ao destas $n - p$ vari\'aveis escolhidas.
  \end{enumerate}
  O n\'umero $n - p$ \'e chamado de \textbf{grau de liberdade} e as $n - p$ vari\'aveis s\~ao chamadas de \textbf{vari\'aveis livres}.
\end{teorema}
}
\end{document}

