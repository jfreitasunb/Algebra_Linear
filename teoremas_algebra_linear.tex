\documentclass[allowframebreaks]{beamer}
\usepackage{amssymb,amsmath,amsfonts,amsthm,amstext}
\usepackage{enumitem}
\usepackage{times}
\usepackage{lmodern}

\newcommand{\cp}[1]{\mathbb{#1}}

\mode<presentation>
{
\usetheme{Warsaw}
}

\newtheorem{teorema}{Teorema}
\newtheorem{definicao}{Definição}

\begin{document}
\frame{
  \begin{teorema}
  Considere o sistema
  \[
    AX = B
  \]
  onde $A$ \'e uma matriz $m \times n$ e $B$ \'e uma matriz $m \times 1$, ambas com entradas no corpo $\cp{K}$ e $X$ \'e uma matriz $n \times 1$. Ent\~ao:
  \begin{enumerate}[label={\roman*})]
    \item O sistema tem solu\c{c}\~ao se, e somente se, o posto da matriz ampliada \'e igual ao posto da matriz dos coeficientes.

    \item Se a matriz ampliada e a matriz dos coeficientes t\^em o mesmo posto $p$ e $p = n$, ent\~ao a solu\c{c}\~ao \'e \'unica.

    \item Se a matriz ampliada e a matriz dos coeficientes t\^em o mesmo posto $p$ e $p < n$, ent\~ao podemos escolher $n - p$ vari\'aveis, e as outras $p$ vari\'aveis ser\~ao dadas em fun\c{c}\~ao destas $n - p$ vari\'aveis escolhidas.
  \end{enumerate}
  O n\'umero $n - p$ \'e chamado de \textbf{grau de liberdade} e as $n - p$ vari\'aveis s\~ao chamadas de \textbf{vari\'aveis livres}.
\end{teorema}
}

\frame{
  \begin{definicao}
    Um conjunto n\~ao vazio $V$ \'e um \textbf{espa\c{c}o vetorial}\index{Espa\c{c}o Vetorial} sobre um corpo $\cp{K}$ se em seus elementos, chamados \textbf{vetores}, estiverem definidas duas opera\c{c}\~oes satisfazendo:\pause
    \begin{itemize}
      \item A cada par $u$, $w \in V$ corresponde um vetor $u + w \in V$, chamado \textbf{soma} de $u$ e $w$, de modo que:\pause
      \item[1)] $u + w = w + u$, para todos $u$, $w \in V$;\pause
      \item[2)] $(u + w) + x = u + (w + x)$, para todos $u$, $w$ e $x \in V$;\pause
      \item[3)] Existe em $V$ um vetor, denominado \textbf{vetor nulo} e denotado por $0_V$, tal que\pause
      \[
        0_V + u = u
      \]
      para todo $u \in V$.\pause
      \item[4)] Para cada vetor $u \in V$, existe um vetor em $V$, denotado por $-u$ tal que\pause
      \[
        u + (-u) = 0_V.
      \]
    \end{itemize}
  \end{definicao}
}
\frame{
  \begin{definicao}
    \begin{itemize}
      \item A cada par $\alpha \in \cp{K}$ e $u \in V$, corresponde um vetor $\alpha \cdot u \in V$, denominado \textbf{produto por escalar} de $\alpha$ por $u$ de modo que:\pause
      \item[5)] $(\alpha\beta)\cdot u = \alpha(\beta\cdot u)$ para todos $\alpha$, $\beta \in \cp{K}$ e todo $u \in V$;\pause
      \item[6)] $1_\cp{K}\cdot u = u$ para todo $u \in V$, onde $1_\cp{K}$ \'e o elemento neutro da multiplica\c{c}\~ao em $\cp{K}$.\pause
      \item[7)] $\alpha\cdot(u + w) = \alpha\cdot u + \alpha\cdot w$, para todo $\alpha \in \cp{K}$ e todos $u$, $w \in V$;\pause
      \item[8)] $(\alpha + \beta)\cdot u = \alpha\cdot u + \beta\cdot u$, para todos $\alpha$, $\beta \in \cp{K}$ e todo $u \in V$.\pause
    \end{itemize}
  \end{definicao}
}
\end{document}

