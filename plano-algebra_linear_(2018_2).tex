%!TEX program = xelatex 
%!TEX encoding = ISO-8859-1
\documentclass[12pt]{article}
%\usepackage[leqno]{amsmath}
%\usepackage{makeidx,graphics}
\usepackage{graphicx}
%\usepackage{color}
% \usepackage[latin1]{inputenc}
\usepackage[portuges]{babel}
\usepackage{graphicx}
\graphicspath{{/ArquivosUbuntu/Dropbox/imagens-latex/}{D:/Dropbox/imagens-latex/}}
\usepackage{enumerate}
\usepackage{url}

%\DeclareGraphicsRule{jpg}{*[}{}{`jpeg2eps #1.jpg}
%\input{seteps}
%\input{setbmp-dvips}

\setlength{\topmargin}{-1.0in}
\setlength{\oddsidemargin}{0in}
\setlength{\textheight}{10.1in}
\setlength{\textwidth}{6.5in}
\setlength{\baselineskip}{12mm}

\begin{document}
\pagestyle{empty}

\begin{figure}[h]
    \begin{minipage}[c]{1.7cm}
    \includegraphics[width=1.7cm]{unb.pdf}
    \end{minipage}%
    \hspace{0pt}
    \begin{minipage}[c]{4in}
    {Universidade de Bras{{\'\i}}lia} \\
    {Departamento de Matem\'atica}
    \end{minipage}
\end{figure}
\vspace{-0.9cm}
\hrule

\begin{center}
{\large\bf Plano de Ensino -- 2$^{o}$/2018} \\
{\large\bf \'Algebra Linear -- Turma A}\\
Prof. Jos\'e Ant\^onio O. Freitas
\end{center}
\hrule
\vspace{0.25cm}
\noindent {\bf{PROGRAMA:}}
\begin{enumerate}[1)]
\item \textit{Preliminares:} Corpos. Sistemas Lineares. Opera\c{c}\~oes Elementares. Solu\c{c}\~oes de um Sistema Linear. Determinante. Desenvolvimento de Laplace. Matriz Adjunta. Matriz inversa.

\item \textit{Espa\c{c}os Vetoriais:} Espa\c{c}os Vetoriais sobre um corpo. Bases. Espa\c{c}o Vetoriais de Dimens\~ao Finita. Subespa\c{c}os. Completamento de Bases. Somas Diretas.


\item \textit{Transforma\c{c}\~oes Lineares:} Conceitos b\'asicos. N\'ucleo e Imagem. Isomorfismos. Matrizes de Transforma\c{c}\~ao. Mudan\c{c}a de Base. O Espa\c{c}o $L(U,V)$.

\item \textit{Formas Can\^onicas:} Autovalor. Autovetor. Operadores Diagonaliz\'aveis. Subespa\c{c}os T-invariantes. Polin\^omios minimais e o Teorema de Cayley Hamilton. Espa\c{c}os Vetoriais T-c{\'\i}clicos. Operadores Nilpotentes. Formas de Jordan.

\item \textit{T\'opicos Adicionais:} Espa\c{c}os com Produto Interno.

\end{enumerate}

\noindent {\bf{BIBLIOGRAFIA:}}
\begin{itemize}
 \item \textit{Um curso de \'Algebra Linear}, Fl\'avio U. Coelho e Mary L. Louren\c{c}o,  Editora EdUSP, 2a edi\c{c}\~ao, 2007. 

\item \textit{\'Algebra Linear}, K. Hoffman e R. Kunze. Livros T\'ecnicos e Cient{\'\i}ficos Editora, 1976.

\item \textit{\'Algebra Linear}, J. Boldrini, S. Costa, V. Figueiredo, H. Wetzler, Editora Harbra, 3a edi\c{c}\~ao, 1980.

\end{itemize}

\noindent {\bf{SISTEMA DE AVALIA\c{C}\~AO:}} Ser\~ao realizadas tr\^es prova cada um valendo 10 pontos, as quais s\~ao atribu{\'\i}das as notas $P_1$, $P_2$ e $P_3$, nas datas detalhadas a seguir.

\begin{center}
    \begin{tabular}{|c|c|c|}
        \hline\hline
        \hspace{1cm}{\bf Prova}\hspace{1cm} & \hspace{3cm}{\bf Data}\hspace{3cm} & \hspace{1.7cm}{\bf Hor\'ario}\hspace{1.7cm} \\
        \hline\hline
        $P_1$ & 13/04/17 (quinta-feira) \phantom{x} & 16:00 - 17:50 \\
        \hline
        $P_2$ & 25/05/17 (quinta-feira) \phantom{x} & 16:00 - 17:50 \\
        \hline
        $P_3$ & 29/06/17 (quinta-feira) \phantom{x} & 16:00 - 17:50 \\
        \hline\hline
    \end{tabular}
\end{center}

{\bf \noindent Nota Final:} A partir das notas das provas mencionadas neste texto, a nota final ($NF$) de cada estudante \'e dada
por
\vspace{-0.15cm}
\[
NF = \frac{ P_1 + P_2 + P_3}{3} 
\]
e ser\'a considerado aprovado o estudante que obtiver $NF \geq 5,00$.

{\bf \noindent Men\c{c}\~ao Final:} ser\'a obtida da $NF$ de
acordo com as normas da UnB.
\begin{center}
    \begin{tabular}{c|c}
        \hline\hline
        \hspace{1cm}{Nota}\hspace{1cm} & \hspace{0.25cm}{Men\c{c}\~ao}\hspace{0.25cm}\\
        \hline\hline
        9,00 a 10,0 & SS \\
        \hline
        7,00 a 8,99 & MS \\
        \hline
        5,00 a 6,99 & MM \\
        \hline
        3,00 a 4,99 & MI \\
        \hline
        0,00 a 2,99  & II \\
        \hline\hline
    \end{tabular}
\end{center}
Receber\'a a men{\c c}\~ao {\bf SR} quem estiver reprovado por faltar mais de 25\%
das aulas.

\vspace{0.5cm}
\noindent {\bf{P\'{A}GINA DA TURMA:}} A p\'agina da disciplina est\'a dispon{\'\i}vel no endere\c{c}o
\begin{center}
    \url{moodle.mat.unb.br}.
\end{center}


\begin{itemize}
\item Toda a comunica\c{c}\~ao oficial do curso, inclusive a divulga\c{c}\~ao de
notas e gabaritos, se dar\'a atrav\'es do {\em F\'orum de Not{\'\i}cias} do
MOODLE.\vspace{-0.20cm}
\item No {\em F\'orum de Debates} do MOODLE poder\~ao ser
postadas d\'uvidas que ser\~ao respondidas on-line pelos seus
colegas ou pelo monitor dessa turma.
\end{itemize}

\noindent {\bf{OBSERVA\c{C}\~OES IMPORTANTES:}}

\begin{itemize}
\item[1)] As provas ser\~ao individuais e sem qualquer tipo de
aux{\'\i}lio (calculadora, livros etc.), sendo vedado o empr\'estimo de
qualquer material entre os alunos durante as avalia\c{c}\~oes. As
tentativas de fraude ser\~ao reprimidas com m\'aximo rigor.
\vspace{-0.25cm}

\item[2)] \'E vedado o uso de telefones celulares e quaisquer dispositivos eletr\^onicos pessoais durante a realiza\c{c}\~ao das atividades do curso em sala de aula. \vspace{-0.25cm}

\item[3)] Ser\'a exigido documento de identifica\c{c}\~ao dos estudantes nos
dias de provas e testes. \vspace{-0.25cm}

\item[4)] A aus\^encia acarretar\'a nota zero em qualquer uma das
avalia\c{c}\~oes. \vspace{-0.25cm}

\item[5)] A crit\'erio do professor, as datas das provas poder\~ao
ser alteradas. \vspace{-0.25cm}

\item[6)] A lista de presen\c{c}a ser\'a passada apenas uma vez
durante cada aula e est\'a sujeita a confirma\c{c}\~ao oral. O
estudante deve assin\'a-la com sua rubrica. {\'E} proibido assinar
com suas iniciais e \'e proibido assinar por outra pessoa.
\vspace{-0.25cm}

\item[7)] Haver\'a avalia{\c c}\~ao quanto {\`a} clareza, apresenta{\c
c}\~ao e formaliza{\c c}\~ao na  resolu{\c c}\~ao das quest\~oes de
cada prova. A nota do aluno poder\'a ser alterada em raz\~ao da
inobserv{\^a}ncia desses par{\^a}metros.

\item[8)] A comunica\c{c}\~ao entre o professor/monitores e estudantes ser\'a, preferencialmente,
 estabelecida pelo f\'orum do MOODLE.
\end{itemize}

\vfill
\hrule
\end{document}
