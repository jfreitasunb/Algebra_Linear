%!TEX program = xelatex
% !TEX encoding = ISO-8859-1

\documentclass[12pt]{exam}

\usepackage{caption}
\usepackage{amssymb}
\usepackage{amsmath,amsfonts,amsthm,amstext}
\usepackage[brazil]{babel}
% \usepackage[latin1]{inputenc}
\usepackage{graphicx}
\graphicspath{{/ArquivosUbuntu/Dropbox/imagens-latex/}{D:/Dropbox/imagens-latex/}}
\usepackage{enumitem}
\usepackage{multicol}
 \usepackage{answers}
\usepackage{tikz,ifthen}
\usetikzlibrary{lindenmayersystems}
\usetikzlibrary[shadings]

\Newassociation{solucao}{Solution}{ans}
\newtheorem{exercicio}{}

\setlength{\topmargin}{-1.0in}
\setlength{\oddsidemargin}{0in}
\setlength{\textheight}{10.1in}
\setlength{\textwidth}{6.5in}
\setlength{\baselineskip}{12mm}

\newcounter{exercicios}
\setcounter{exercicios}{0}
\newcommand{\questao}{
\addtocounter{exercicios}{1}
\noindent{\bf Exerc{\'\i}cio \Roman{exercicios}: }}

\newcommand{\resp}[1]{
\noindent{\bf Exerc{\'\i}cio #1: }}

\newcommand{\im}{{\rm Im\,}}
\newcommand{\sub}{\subseteq}
\newcommand{\n}{\mathbb{N}}
\newcommand{\integer}{\mathbb{Z}}
\newcommand{\rac}{\mathbb{Q}}
\newcommand{\real}{\mathbb{R}}
\newcommand{\complex}{\mathbb{C}}
\newcommand{\cp}[1]{\mathbb{#1}}
\newcommand{\ch}{\mbox{\textrm{car\,}}\nobreak}
\newcommand{\dlim}[2]{\displaystyle\lim_{#1\rightarrow #2}}
\newcommand{\minf}{+\infty}
\newcommand{\ninf}{-\infty}
\renewcommand{\sin}{{\rm sen\,}}
\renewcommand{\sinh}{{\rm senh\,}}
\renewcommand{\tan}{{\rm tg\,}}
\renewcommand{\csc}{{\rm cossec\,}}
\renewcommand{\cot}{{\rm cotg\,}}
\newcommand{\din}[4]{\displaystyle\int_{#1}^{#2}{#3}{d#4}}

\newcommand{\se}[1]{\displaystyle\sum_{n = 1}^\infty{#1}}
\newcommand{\slim}{\displaystyle\lim_{n \rightarrow \infty}}
\newcommand{\seq}[1]{\{{#1_n\}}}
\newcommand{\seg}[1]{\displaystyle\sum_{n = 1}^\infty{#1_n}}
\newcommand{\sei}[2]{\displaystyle\sum_{#1}^\infty{#2}}

\newcommand{\vesp}[1]{\vspace{ #1  cm}}

\newcommand{\compcent}[1]{\vcenter{\hbox{$#1\circ$}}}
\newcommand{\comp}{\mathbin{\mathchoice
{\compcent\scriptstyle}{\compcent\scriptstyle}
{\compcent\scriptscriptstyle}{\compcent\scriptscriptstyle}}}

\begin{document}
\pagestyle{empty}

\Opensolutionfile{ans}[ans1]

\begin{figure}[h]
        \begin{minipage}[c]{1.7cm}
        \includegraphics[width=1.7cm]{unb.pdf}
        \end{minipage}%
        \hspace{0pt}
        \begin{minipage}[c]{4in}
          {Universidade de Bras{\'\i}lia} \\
          {Departamento de Matem{\'a}tica}
\end{minipage}
\end{figure}

\vesp{-0.35} \hrule

\begin{center}
{\Large\bf \'Algebra Linear - Turma A -- 2$^{o}$/2018} \\ \vspace{9pt} {\large\bf
  $7^{\underline{a}}$ Lista de Exerc{\'\i}cios -- Diagonaliza\c{c}\~ao}\\ \vspace{9pt} Prof. Jos{\'e} Ant{\^o}nio O. Freitas
\end{center}
\hrule

\vesp{.6}

\begin{exercicio}
  Em cada um dos casos abaixo, decida se o operador linear $T : \cp{K}^n \to \cp{K}^n$ dado por sua matriz $[T]_\mathcal{B}$ \'e diagonaliz\'avel. Em caso positivo, calcule uma base de autovetores e a sua forma diagonal.
  \begin{enumerate}[label=({\alph*})]
    \item $[T]_\mathcal{B} = \begin{bmatrix} 1 & 0\\ 0 & 0\end{bmatrix}$, $\cp{K} = \complex$, $n = 2$
    \item $[T]_\mathcal{B} = \begin{bmatrix} 1 & -2\\ 1 & -1\end{bmatrix}$, $\cp{K} = \complex$, $n = 2$
    \item $[T]_\mathcal{B} = \begin{bmatrix} 5 & -1\\ 1 & \phantom{-} 3\end{bmatrix}$, $\cp{K} = \real$, $n = 2$
    \item $[T]_\mathcal{B} = \begin{bmatrix} \phantom{-} 1 & 0 & 2\\ -1 & 0 & 1\\ \phantom{-} 1 & 1 & 2\end{bmatrix}$, $\cp{K} = \real$, $n = 3$
    \item $[T]_\mathcal{B} = \begin{bmatrix} -1 & -2 & 0\\ \phantom{-} 0 & -1 & 1\\ \phantom{-} 1 & \phantom{-} 0 & 0\end{bmatrix}$, $\cp{K} = \real$, $n = 3$
    \item $[T]_\mathcal{B} = \begin{bmatrix} -1 & -2 & 0\\ \phantom{-} 0 & -1 & 1\\ \phantom{-} 1 & \phantom{-} 0 & 0\end{bmatrix}$, $\cp{K} = \complex$, $n = 3$
    \item $[T]_\mathcal{B} = \begin{bmatrix} 1 & \phantom{-} 2\\ 0 & -1\end{bmatrix}$, $\cp{K} = \real$, $n = 2$
    \item $[T]_\mathcal{B} = \begin{bmatrix} \phantom{-} 1 & 2 & \phantom{-} 2\\ \phantom{-} 1 & 2 & -1\\ -1 & 1 & \phantom{-} 4\end{bmatrix}$, $\cp{K} = \real$, $n = 3$
    \item $[T]_\mathcal{B} = \begin{bmatrix} \phantom{-} 1 & 3 & 3\\ \phantom{-} 0 & 4 & 0\\ -3 & 3 & 1\end{bmatrix}$, $\cp{K} = \real$, $n = 3$
    \item $[T]_\mathcal{B} = \begin{bmatrix} \phantom{-} 14 & -28 & -44\\ -7 & -14 & -23\\ \phantom{-} 9 & \phantom{-} 18 & \phantom{-} 29\end{bmatrix}$, $\cp{K} = \real$, $n = 3$
  \end{enumerate}
  \begin{solucao}
      \begin{enumerate}[label=({\alph*})]
          \item $\mathcal{A} = \{(1,0);(0,1)\}$; $[T]_\mathcal{A} = \begin{bmatrix} 1 & 0\\ 0 & 0\end{bmatrix}$
          \item $\mathcal{A} = \{(2,1 - i);(2,1 + i)\}$; $[T]_\mathcal{A} = \begin{bmatrix} i & \phantom{-} 0\\ 0 & -i\end{bmatrix}$
          \item T n\~ao \'e diagonaliz\'avel.
          \item $\mathcal{A} = \{(-1,1,0);(1,2,-1);(1,0,1)\}$; $[T]_\mathcal{A} = \begin{bmatrix} 1 & \phantom{-} 0 & 0\\ 0 & -1 & 0\\0 & \phantom{-} 0 & 3\end{bmatrix}$
          \item O operador $T$ n\~ao \'e diagonaliz\'avel.
          \item $\mathcal{A} = \{(2,1,-1);(-1 + i, 1, 1 + i);(-1 - i, 1, 1 - i)\}$; $[T]_\mathcal{A} = \begin{bmatrix} -2 & 0 & \phantom{-} 0\\ \phantom{-} 0 & i & \phantom{-} 0\\ \phantom{-} 0 & 0 & -i\end{bmatrix}$
          \item $\mathcal{A} = \{(1,0);(-1,1)\}$; $[T]_\mathcal{A} = \begin{bmatrix} 1 & 0\\ 0 & -1\end{bmatrix}$
          \item $\mathcal{A} = \{(1,1,0);(1,0,1);(2,1,1)\}$; $[T]_\mathcal{A} = \begin{bmatrix} 3 & 0 & 0\\ 0 & 3 & 0\\ 0 & 0 & 1\end{bmatrix}$
          \item $\mathcal{A} = \{(1,1,0);(-1,0,1);(1,0,1)\}$; $[T]_\mathcal{A} = \begin{bmatrix} 4 & 0 & \phantom{-} 0\\ 0 & 4 & \phantom{-} 0\\ 0 & 0 & -2\end{bmatrix}$
          \item $\mathcal{A} = \{(-8,-1,3);(-2,1,0);(-1,-1,1)\}$; $[T]_\mathcal{A} = \begin{bmatrix} -1 & 0 & 0\\ \phantom{-} 0 & 0 & 0\\ \phantom{-} 0 & 0 & 2\end{bmatrix}$
        \end{enumerate}
  \end{solucao}
\end{exercicio}

\begin{exercicio}
  Seja $T : V \to V$ um operador linear. Mostre que se todo vetor de $V$ for autovetor de $T$, ent\~ao existe $\lambda \in \cp{K}$ tal que $T(u) = \lambda u$ para todo $u \in V$.
\end{exercicio}

\begin{exercicio}
  Seja $T : V \to V$ um operador linear. Mostre que se $\dim_\cp{K}\im T = m$, ent\~ao $tem$ T no m\'aximo $m + 1$ autovalores.
\end{exercicio}

\begin{exercicio}
  Seja $T : \cp{K}^2 \to \cp{K}^2$ um operador linear tal que $T \circ T = 0$. Mostre que
  \begin{enumerate}[label=({\alph*})]
    \item $\im T \sub \ker T$
    \item Se $T \ne 0$, ent\~ao existe uma base $\mathcal{B}$ de $\cp{K}^2$ tal que
    \[
      [T]_\mathcal{B} = \begin{bmatrix} 0 & 0\\ 1 & 0\end{bmatrix}.
    \]
  \end{enumerate}
\end{exercicio}

\begin{exercicio}
  Determine, se existir, uma matriz $P$ com coeficientes em $\real$ e invert{\'\i}vel tal que $P^{-1}AP$ seja diagonal para cada uma das seguintes matrizes:
    \begin{enumerate}[label=({\alph*})]
      \item $A = \begin{bmatrix} 0 & 1\\ 1 & 0\end{bmatrix}$
      \item $A = \begin{bmatrix} 2 & 3\\ 1 & 4\end{bmatrix}$
      \item $A = \begin{bmatrix} \phantom{-} 1 & 2 & -2\\ -2 & 5 & -2\\ -6 & 6 & -3\end{bmatrix}$
      \item $A = \begin{bmatrix} 1 & 0 & 0\\ 1 & 2 & 1\\ 1 & 0 & 2\end{bmatrix}$
      \item $A = \begin{bmatrix} 1 & a \\ a & 1\end{bmatrix}$, $a \in \real$.
    \end{enumerate}
    \begin{solucao}
      \begin{enumerate}[label=({\alph*})]
        \item $P = \begin{bmatrix}
          \phantom{-} 1 & 1\\ -1 & 1
        \end{bmatrix}$ A resposta n\~ao \'e \'unica.
        \item $P = \begin{bmatrix}
          -3 & 1\\ \phantom{-} 1 & 1
        \end{bmatrix}$ A resposta n\~ao \'e \'unica.
        \item $P = \begin{bmatrix}
          1 & -1 & 1\\ 1 & \phantom{-} 0 & 1\\ 3 & \phantom{-} 1 & 0
        \end{bmatrix}$ A resposta n\~ao \'e \'unica.
        \item $A$ n\~ao \'e diagonaliz\'avel.
        \item $P = \begin{bmatrix} 1/2 & -1/2\\ 1/2 & \phantom{-} 1/2\end{bmatrix}$. A resposta n\~ao \'e \'unica.
      \end{enumerate}
    \end{solucao}

\end{exercicio}

\begin{exercicio}
  Seja $T : \real^2 \to \real^2$ uma transforma\c{c}\~ao linear que tem como autovetores $(3,1)$ e $(-2,1)$ associados aos autovalores $-2$ e $3$, respectivamente. Calcule $T(x,y)$.
  \begin{solucao}
    $T(x,y) = (-6y, y -x)$
  \end{solucao}
\end{exercicio}

\begin{exercicio}
  Sejam $T : V \to V$ e $S : V \to V$ transforma\c{c}\~oes lineares. Suponha que $u \in V$ \'e um autovetor de $T$ e de $S$ associado aos autovalores $\lambda_1$ e $\lambda_2$ de $T$ e $S$, respectivamente. Ache um autovetor e um autovalor de:
  \begin{enumerate}[label=({\alph*})]
    \item $\alpha S + \beta T$, onde $\alpha$, $\beta \in \real$.
    \item $S \circ T$.
  \end{enumerate}
\end{exercicio}

\begin{exercicio}
  \begin{enumerate}[label=({\alph*})]
    \item Mostre que se $B$, $M \in \cp{M}_n(\cp{K})$, com $M$ invert{\'\i}vel, ent\~ao
    \[
        (M^{-1}BM)^n = (M^{-1}B^nM)
    \]
    para todo $n \in \cp{N}$.
    \item Calcule $A^n$, $n \in \cp{N}$, onde
    \[
      A = \begin{bmatrix}
        2 & 4\\
        3 & 13
      \end{bmatrix}.
    \]
  \end{enumerate}
  \begin{solucao}
    \begin{enumerate}
    \item[b)] $A^n = \dfrac{1}{13}\begin{bmatrix}
      14^n + 12 & 4(14^n - 1)\\3(14^n - 1) & 12.14^n + 1
    \end{bmatrix}$, $n \in \cp{N}$
  \end{enumerate}
  \end{solucao}
\end{exercicio}

\begin{exercicio}
  Seja $T : \cp{M}_2(\real) \to \cp{M}_2(\real)$ um operador linear cuja matriz em rela\c{c}\~ao \`a base
  \[
    \mathcal{B} = \left\{\begin{bmatrix}1 & 0\\ 1 & 0\end{bmatrix}; \begin{bmatrix}1 & 0\\ 0 & 0\end{bmatrix}; \begin{bmatrix}0 & 1\\ 0 & 1\end{bmatrix}; \begin{bmatrix}0 & 0\\ 0 & 1\end{bmatrix}\right\}
  \]
  \'e dada por
  \[
    [T]_\mathcal{B} = \begin{bmatrix}
      -1 & -4 & -2 & -2\\
      -4 & -1 & -2 & -2\\
      \phantom{-} 2 & \phantom{-} 2 & \phantom{-} 1 & \phantom{-} 4\\
      \phantom{-} 2 & \phantom{-} 2 & \phantom{-} 4 & \phantom{-} 1
    \end{bmatrix}.
  \]
  Determine uma matriz invert{\'\i}vel $M \in \cp{M}_4(\real)$ tal que $M^{-1}[T]_\mathcal{B}M$ seja uma matriz diagonal.
\begin{solucao}
  $M = \begin{bmatrix}
    \phantom{-} 1 & \phantom{-} 1 & \phantom{-} 1 & \phantom{-} 1\\
    \phantom{-} 1 & \phantom{-} 0 & \phantom{-} 1 & \phantom{-} 1\\
    \phantom{-} 1 & -1 & \phantom{-} 0 & -2\\
    -2 & -1 & -1 & -2
  \end{bmatrix}$. A resposta n\~ao \'e \'unica.
\end{solucao}
\end{exercicio}

\begin{exercicio}
  Determine todos os valores de $a$, $b$ e $c \in \complex$ para os quais a matriz abaixo seja diagonaliz\'avel:
  \[
    A = \begin{bmatrix}
      a & b & 1\\
      0 & c & 0\\
      0 & 0 & 1
    \end{bmatrix}.
  \]
  \begin{solucao}
    $a = c$ e $a \ne 1$.
  \end{solucao}
\end{exercicio}

\begin{exercicio}
  Quando uma matriz $2 \times 2$
  \[
    \begin{pmatrix}
      a & b\\c & d
    \end{pmatrix}
  \]
  é diagonalizável?
\end{exercicio}

\begin{exercicio}
  Sejam $T : V \to V$ e $G : V \to V$ transformações lineares sobre um $\cp{K}$-espaço vetorial de dimensão finita $V$. Seja $\mathcal{B}$ uma base de $V$ tal que
  \[
    [T]_\mathcal{B}[G]_\mathcal{B} = [G]_\mathcal{B}[T]_\mathcal{B}.
  \]
  Suponha que $T$ possui um autovetor $w$ associado ao autovalor $\lambda \in \cp{K}$.
  \begin{enumerate}[label=({\alph*})]
    \item Mostre que $G(w)$ também é um autovalor de $T$ associado ao autovalor $\lambda$.
    \item Suponha que $T$ é diagonalizável com distintos autovalores. Qual é a dimensão de cada autoespaço de $T$.
    \item Mostre que $w$ também é um autovetor de $G$.
  \end{enumerate}
\end{exercicio}

\newpage
\Closesolutionfile{ans}
\hrule
\begin{center}
{\large\bf RESPOSTAS}
\end{center}
\hrule
\input{ans1}

\end{document}