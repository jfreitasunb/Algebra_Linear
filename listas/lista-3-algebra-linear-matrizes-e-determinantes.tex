% !TEX encoding = ISO-8859-1

\documentclass[12pt]{exam}

\usepackage{caption}
\usepackage{amssymb}
\usepackage{amsmath,amsfonts,amsthm,amstext}
\usepackage[brazil]{babel}
% \usepackage[latin1]{inputenc}
%\usepackage[pdftex]{graphicx}
\usepackage{graphicx}
\graphicspath{{/home/jfreitas/Dropbox/imagens-latex/}{/Volumes/Vader/Dropbox/imagens-latex/}{D:/Dropbox/imagens-latex/}}
\usepackage{enumitem}
\usepackage{multicol}
 \usepackage{answers}
\usepackage{tikz,ifthen}
\usetikzlibrary{lindenmayersystems}
\usetikzlibrary[shadings]

\Newassociation{solucao}{Solution}{ans}
\newtheorem{exercicio}{}

\setlength{\topmargin}{-1.0in}
\setlength{\oddsidemargin}{0in}
\setlength{\textheight}{10.1in}
\setlength{\textwidth}{6.5in}
\setlength{\baselineskip}{12mm}

\newcounter{exercicios}
\setcounter{exercicios}{0}
\newcommand{\questao}{
\addtocounter{exercicios}{1}
\noindent{\bf Exerc{\'\i}cio \Roman{exercicios}: }}

\newcommand{\resp}[1]{
\noindent{\bf Exerc{\'\i}cio #1: }}

\newcommand{\sub}{\subseteq}
\newcommand{\n}{\mathbb{N}}
\newcommand{\integer}{\mathbb{Z}}
\newcommand{\rac}{\mathbb{Q}}
\newcommand{\real}{\mathbb{R}}
\newcommand{\complex}{\mathbb{C}}
\newcommand{\cp}[1]{\mathbb{#1}}
\newcommand{\ch}{\mbox{\textrm{car\,}}\nobreak}
\newcommand{\dlim}[2]{\displaystyle\lim_{#1\rightarrow #2}}
\newcommand{\minf}{+\infty}
\newcommand{\ninf}{-\infty}
\renewcommand{\sin}{{\rm sen\,}}
\renewcommand{\sinh}{{\rm senh\,}}
\renewcommand{\tan}{{\rm tg\,}}
\renewcommand{\csc}{{\rm cossec\,}}
\renewcommand{\cot}{{\rm cotg\,}}
\newcommand{\din}[4]{\displaystyle\int_{#1}^{#2}{#3}{d#4}}

\newcommand{\se}[1]{\displaystyle\sum_{n = 1}^\infty{#1}}
\newcommand{\slim}{\displaystyle\lim_{n \rightarrow \infty}}
\newcommand{\seq}[1]{\{{#1_n\}}}
\newcommand{\seg}[1]{\displaystyle\sum_{n = 1}^\infty{#1_n}}
\newcommand{\sei}[2]{\displaystyle\sum_{#1}^\infty{#2}}

\newcommand{\vesp}[1]{\vspace{ #1  cm}}

\newcommand{\compcent}[1]{\vcenter{\hbox{$#1\circ$}}}
\newcommand{\comp}{\mathbin{\mathchoice
{\compcent\scriptstyle}{\compcent\scriptstyle}
{\compcent\scriptscriptstyle}{\compcent\scriptscriptstyle}}}

\begin{document}
\pagestyle{empty}

\Opensolutionfile{ans}[ans1]

\begin{figure}[h]
        \begin{minipage}[c]{1.7cm}
        \includegraphics[width=1.7cm]{../../../imagens/unb.pdf}
        \end{minipage}%
        \hspace{0pt}
        \begin{minipage}[c]{4in}
          {Universidade de Bras{\'\i}lia} \\
          {Departamento de Matem{\'a}tica}
\end{minipage}
\end{figure}

\vesp{-0.35} \hrule

\begin{center}
{\Large\bf \'Algebra Linear - Turma A -- 2$^{o}$/2013} \\ \vspace{9pt} {\large\bf
  $3^{\underline{a}}$ Lista de Exerc{\'\i}cios -- Matrizes e Determinantes}\\ \vspace{9pt} Prof. Jos{\'e} Ant{\^o}nio O. Freitas
\end{center}
\hrule

\vesp{.6}

Nos exerc{\'\i}cios \ref{matrizinicio} \`a \ref{matrizfim}, mostre que $B$ \'e a inversa de $A$.
\begin{exercicio}\label{matrizinicio}
\[
  A =\begin{bmatrix}
    1 & -1\\
    2 & 3
  \end{bmatrix}, B =\begin{bmatrix}
    3/5 & 1/5\\
    -2/5 & 1/5
  \end{bmatrix}
\]
\end{exercicio}

\begin{exercicio}
\[
  A =\begin{bmatrix}
    -2 & 2 & 3\\
    1 & -1 & 0\\
    0 & 1 & 4
  \end{bmatrix}, B =\begin{bmatrix}
    -4/3 & -5/3 & 1\\
    -4/3 & -8/3 & 1\\
    1/3 & 2/3 & 0
  \end{bmatrix}
\]
\end{exercicio}

\begin{exercicio}\label{matrizfim}
\[
  A =\begin{bmatrix}
    2 & -17 & 11\\
    -1 & 11 & -7\\
    0 & 3 & -2
  \end{bmatrix}, B =\begin{bmatrix}
    1 & 1 & 2\\
    2 & 4 & -3\\
    3 & 6 & -5
  \end{bmatrix}
\]
\end{exercicio}

Nos exerc{\'\i}cios \ref{matrizinversainicio} \`a \ref{matrizinversafim}, encontre a inversa da matriz dada, se existir.

\begin{exercicio}\label{matrizinversainicio}
$A =\begin{bmatrix}
    1 & 2\\
    3 & 7
  \end{bmatrix}$
\begin{solucao}
  $A^{-1} =\begin{bmatrix}
    7 & -2\\
    -3 & 1
  \end{bmatrix}$
\end{solucao}
\end{exercicio}

\begin{exercicio}
$A =\begin{bmatrix}
    -7 & 33\\
    4 & 19
  \end{bmatrix}$
\begin{solucao}
  $A^{-1} =\begin{bmatrix}
    -19 & -33\\
    -4 & -7
  \end{bmatrix}$
\end{solucao}
\end{exercicio}

\begin{exercicio}
$A = \begin{pmatrix}
    \overline{4} & \overline{0} & \overline{1}\\
    \overline{1} & \overline{1} & \overline{0}\\
    \overline{3} & \overline{0} & \overline{1}
  \end{pmatrix}$ em $\integer_5$.
\begin{solucao}
  $A^{-1} = \begin{pmatrix}
    \overline{1} & \overline{0} & \overline{4}\\
    \overline{4} & \overline{1} & \overline{1}\\
    \overline{2} & \overline{0} & \overline{4}
  \end{pmatrix}$
\end{solucao}
\end{exercicio}

\begin{exercicio}
$A = \begin{bmatrix}
    \overline{1} & \overline{4} & \overline{3} & \overline{3}\\
    \overline{1} & \overline{6} & \overline{0} & \overline{0}\\
    \overline{1} & \overline{6} & \overline{4} & \overline{4}\\
    \overline{1} & \overline{6} & \overline{5} & \overline{5}
  \end{bmatrix}$ em $\integer_7$.
\begin{solucao}
  N\~ao existe inversa.
\end{solucao}
\end{exercicio}

\begin{exercicio}
$A =\begin{bmatrix}
    2 & 4\\
    4 & 8
  \end{bmatrix}$
\begin{solucao}
  N\~ao possui inversa.
\end{solucao}
\end{exercicio}

\begin{exercicio}
$A =\begin{bmatrix}
    1 & 1 & 1\\
    3 & 5 & 4\\
    3 & 6 & 5
  \end{bmatrix}$
\begin{solucao}
  $A^{-1} =\begin{bmatrix}
    1 & 1 & -1\\
    -3 & 2 & -1\\
    3 & -3 & 2
  \end{bmatrix}$
\end{solucao}
\end{exercicio}

\begin{exercicio}
$A =\begin{bmatrix}
    1 & 2 & -1\\
    3 & 7 & -10\\
    7 & 16 & -21
  \end{bmatrix}$
\begin{solucao}
  N\~ao tem inversa.
\end{solucao}
\end{exercicio}

\begin{exercicio}
$A =\begin{bmatrix}
    -8 & 0 & 0 & 0\\
    0 & 1 & 0 & 0\\
    0 & 0 & 0 & 0\\
    0 & 0 & 0 & -5
  \end{bmatrix}$
\begin{solucao}
 N\~ao existe inversa.
\end{solucao}
\end{exercicio}

\begin{exercicio}\label{matrizinversafim}
$A =\begin{bmatrix}
    1 & 1 & 2\\
    3 & 1 & 0\\
    -2 & 0 & 3
  \end{bmatrix}$
\begin{solucao}
  $A^{-1} =\begin{bmatrix}
    -3/2 & 3/2 & 1\\
    9/2 & -7/2 & -3\\
    -1 & 1 & 1
  \end{bmatrix}$
\end{solucao}
\end{exercicio}

Nos exerc{\'\i}cios \ref{determinanteinicio} \`a \ref{determinantefim}, encontre o determinante da matriz dada.

\begin{exercicio}\label{determinanteinicio}
$A = \begin{bmatrix}
    2 & 1\\
    3 & 4
    \end{bmatrix}$
\begin{solucao}
  $\det A = 5$
\end{solucao}
\end{exercicio}

\begin{exercicio}
$A = \begin{bmatrix}
    \overline{1} & \overline{1}\\
    \overline{2} & \overline{1}
    \end{bmatrix}$ em $\integer_3$.
\begin{solucao}
  $\det A = \overline{2}$
\end{solucao}
\end{exercicio}

\begin{exercicio}
$A = \begin{bmatrix}
    \overline{0} & \overline{1} & \overline{0}\\
    \overline{1} & \overline{1} & \overline{0}\\
    \overline{1} & \overline{1} & \overline{1}\\
    \end{bmatrix}$ em $\integer_2$.
\begin{solucao}
  $\det A = \overline{1}$
\end{solucao}
\end{exercicio}

\begin{exercicio}
$A = \begin{bmatrix}
    \overline{1} & \overline{6} & \overline{0} & \overline{0}\\
    \overline{1} & \overline{7} & \overline{0} & \overline{0}\\
    \overline{2} & \overline{3} & \overline{1} & \overline{2}\\
    \overline{3} & \overline{8} & \overline{0} & \overline{0}\\
    \end{bmatrix}$ em $\integer_{11}$.
\begin{solucao}
  $\det A = \overline{0}$
\end{solucao}
\end{exercicio}


\begin{exercicio}
$A = \begin{bmatrix}
    1 & 4 & -2\\
    3 & 2 & 0\\
    -1 & 4 & 3
    \end{bmatrix}$
\begin{solucao}
  $\det A = -58$
\end{solucao}
\end{exercicio}

\begin{exercicio}
$A = \begin{bmatrix}
    3 & 6 & -5 & 4\\
    -2 & 0 & 6 & 0\\
    1 & 1 & 2 & 2\\
    0 & 3 & -1 & -1
    \end{bmatrix}$
\begin{solucao}
  $\det A = -108$
\end{solucao}
\end{exercicio}

\begin{exercicio}
$A = \begin{bmatrix}
    5 & 8 & -4 & 2\\
    0 & 0 & 6 & 0\\
    0 & 0 & 2 & 2\\
    0 & 0 & 0 & -1
    \end{bmatrix}$
\begin{solucao}
  $\det A = 0$
\end{solucao}
\end{exercicio}

\begin{exercicio}
$A = \begin{bmatrix}
    -1 & 4 & 2 & 1 & -3\\
    0 & 3 & -4 & 5 & 2\\
    0 & 0 & 2 & 7 & 0\\
    0 & 0 & 0 & 5 & -1\\
    0 & 0 & 0 & 0 & -1
    \end{bmatrix}$
\begin{solucao}
  $\det A = -30$
\end{solucao}
\end{exercicio}

\begin{exercicio}\label{determinantefim}
$A = \begin{bmatrix}
    5 & 2 & 0 & 0 & -2\\
    0 & 1 & 4 & 3 & 2\\
    0 & 0 & 2 & 6 & 3\\
    0 & 0 & 3 & 4 & 1\\
    0 & 0 & 0 & 0 & 2
  \end{bmatrix}$
\begin{solucao}
  $\det A = -100$
\end{solucao}
\end{exercicio}

\newpage
\Closesolutionfile{ans}
\hrule
\begin{center}
{\large\bf RESPOSTAS}
\end{center}
\hrule
\input{ans1}

\end{document}