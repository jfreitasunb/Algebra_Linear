%!TEX program = xelatex
% !TEX encoding = ISO-8859-1
\def\ano{2019}
\def\semestre{1}

\documentclass[12pt]{exam}

\usepackage{caption}
\usepackage{amssymb}
\usepackage{amsmath,amsfonts,amsthm,amstext}
\usepackage[brazil]{babel}
% \usepackage[latin1]{inputenc}
\usepackage{graphicx}
\graphicspath{{/ArquivosUbuntu/Dropbox/imagens-latex/}{D:/Dropbox/imagens-latex/}}
\usepackage{enumitem}
\usepackage{multicol}
\usepackage{answers}
\usepackage{tikz,ifthen}
\usetikzlibrary{lindenmayersystems}
\usetikzlibrary[shadings]
\Newassociation{solucao}{Solution}{ans}
\newtheorem{exercicio}{}

\setlength{\topmargin}{-1.0in}
\setlength{\oddsidemargin}{0in}
\setlength{\textheight}{10.1in}
\setlength{\textwidth}{6.5in}
\setlength{\baselineskip}{12mm}

\extraheadheight{0.7in}
\firstpageheadrule
\runningheadrule
\lhead{
        \begin{minipage}[c]{1.7cm}
        \includegraphics[width=1.7cm]{unb.pdf}
        \end{minipage}%
        \hspace{0pt}
        \begin{minipage}[c]{4in}
          {Universidade de Bras{\'\i}lia} --
          {Departamento de Matem{\'a}tica}
\end{minipage}
\vspace*{-0.8cm}
}
% \chead{Universidade de Bras{\'\i}lia - Departamento de Matem{\'a}tica}
% \rhead{}
% \vspace*{-2cm}

\extrafootheight{.5in}
\footrule
\lfoot{\'Algebra Linear \semestre$^o$/\ano}
\cfoot{}
\rfoot{P\'agina \thepage\ de \numpages}

\newcounter{exercicios}
\setcounter{exercicios}{0}
\newcommand{\questao}{
\addtocounter{exercicios}{1}
\noindent{\bf Exerc{\'\i}cio \Roman{exercicios}: }}

\newcommand{\resp}[1]{
\noindent{\bf Exerc{\'\i}cio #1: }}
\newcommand{\im}{{\rm Im\,}}
\newcommand{\sub}{\subseteq}
\newcommand{\n}{\mathbb{N}}
\newcommand{\integer}{\mathbb{Z}}
\newcommand{\rac}{\mathbb{Q}}
\newcommand{\real}{\mathbb{R}}
\newcommand{\complex}{\mathbb{C}}
\newcommand{\cp}[1]{\mathbb{#1}}
\newcommand{\ch}{\mbox{\textrm{car\,}}\nobreak}
\newcommand{\dlim}[2]{\displaystyle\lim_{#1\rightarrow #2}}
\newcommand{\minf}{+\infty}
\newcommand{\ninf}{-\infty}
\renewcommand{\sin}{{\rm sen\,}}
\renewcommand{\sinh}{{\rm senh\,}}
\renewcommand{\tan}{{\rm tg\,}}
\renewcommand{\csc}{{\rm cossec\,}}
\renewcommand{\cot}{{\rm cotg\,}}
\newcommand{\din}[4]{\displaystyle\int_{#1}^{#2}{#3}{d#4}}

\newcommand{\se}[1]{\displaystyle\sum_{n = 1}^\infty{#1}}
\newcommand{\slim}{\displaystyle\lim_{n \rightarrow \infty}}
\newcommand{\seq}[1]{\{{#1_n\}}}
\newcommand{\seg}[1]{\displaystyle\sum_{n = 1}^\infty{#1_n}}
\newcommand{\sei}[2]{\displaystyle\sum_{#1}^\infty{#2}}

\newcommand{\vesp}[1]{\vspace{ #1  cm}}

\newcommand{\compcent}[1]{\vcenter{\hbox{$#1\circ$}}}
\newcommand{\comp}{\mathbin{\mathchoice
{\compcent\scriptstyle}{\compcent\scriptstyle}
{\compcent\scriptscriptstyle}{\compcent\scriptscriptstyle}}}

\begin{document}

\Opensolutionfile{ans}[ans1]
\begin{center}
{\Large\bf \'Algebra Linear - Turma A -- \semestre$^{o}$/\ano} \\ \vspace{9pt} {\large\bf
  $7^{\underline{a}}$ Lista de Exerc{\'\i}cios -- Diagonaliza\c{c}\~ao e Forma de Jordan}\\ \vspace{9pt} Prof. Jos{\'e} Ant{\^o}nio O. Freitas
\end{center}
\hrule

\vesp{.6}

\begin{exercicio}
  Em cada um dos casos abaixo, decida se o operador linear $T : \cp{K}^n \to \cp{K}^n$ dado por sua matriz $[T]_\mathcal{B}$ \'e diagonaliz\'avel. Em caso positivo, calcule uma base de autovetores e a sua forma diagonal.
  \begin{enumerate}[label=({\alph*})]
    \item $[T]_\mathcal{B} = \begin{bmatrix} 1 & 0\\ 0 & 0\end{bmatrix}$, $\cp{K} = \complex$, $n = 2$
    \item $[T]_\mathcal{B} = \begin{bmatrix} 1 & -2\\ 1 & -1\end{bmatrix}$, $\cp{K} = \complex$, $n = 2$
    \item $[T]_\mathcal{B} = \begin{bmatrix} 5 & -1\\ 1 & \phantom{-} 3\end{bmatrix}$, $\cp{K} = \real$, $n = 2$
    \item $[T]_\mathcal{B} = \begin{bmatrix} \phantom{-} 1 & 0 & 2\\ -1 & 0 & 1\\ \phantom{-} 1 & 1 & 2\end{bmatrix}$, $\cp{K} = \real$, $n = 3$
    \item $[T]_\mathcal{B} = \begin{bmatrix} -1 & -2 & 0\\ \phantom{-} 0 & -1 & 1\\ \phantom{-} 1 & \phantom{-} 0 & 0\end{bmatrix}$, $\cp{K} = \real$, $n = 3$
    \item $[T]_\mathcal{B} = \begin{bmatrix} -1 & -2 & 0\\ \phantom{-} 0 & -1 & 1\\ \phantom{-} 1 & \phantom{-} 0 & 0\end{bmatrix}$, $\cp{K} = \complex$, $n = 3$
    \item $[T]_\mathcal{B} = \begin{bmatrix} 1 & \phantom{-} 2\\ 0 & -1\end{bmatrix}$, $\cp{K} = \real$, $n = 2$
    \item $[T]_\mathcal{B} = \begin{bmatrix} \phantom{-} 1 & 2 & \phantom{-} 2\\ \phantom{-} 1 & 2 & -1\\ -1 & 1 & \phantom{-} 4\end{bmatrix}$, $\cp{K} = \real$, $n = 3$
    \item $[T]_\mathcal{B} = \begin{bmatrix} \phantom{-} 1 & 3 & 3\\ \phantom{-} 0 & 4 & 0\\ -3 & 3 & 1\end{bmatrix}$, $\cp{K} = \real$, $n = 3$
    \item $[T]_\mathcal{B} = \begin{bmatrix} \phantom{-} 14 & -28 & -44\\ -7 & -14 & -23\\ \phantom{-} 9 & \phantom{-} 18 & \phantom{-} 29\end{bmatrix}$, $\cp{K} = \real$, $n = 3$
  \end{enumerate}
  \begin{solucao}
      \begin{enumerate}[label=({\alph*})]
          \item $\mathcal{A} = \{(1,0);(0,1)\}$; $[T]_\mathcal{A} = \begin{bmatrix} 1 & 0\\ 0 & 0\end{bmatrix}$
          \item $\mathcal{A} = \{(2,1 - i);(2,1 + i)\}$; $[T]_\mathcal{A} = \begin{bmatrix} i & \phantom{-} 0\\ 0 & -i\end{bmatrix}$
          \item T n\~ao \'e diagonaliz\'avel.
          \item $\mathcal{A} = \{(-1,1,0);(1,2,-1);(1,0,1)\}$; $[T]_\mathcal{A} = \begin{bmatrix} 1 & \phantom{-} 0 & 0\\ 0 & -1 & 0\\0 & \phantom{-} 0 & 3\end{bmatrix}$
          \item O operador $T$ n\~ao \'e diagonaliz\'avel.
          \item $\mathcal{A} = \{(2,1,-1);(-1 + i, 1, 1 + i);(-1 - i, 1, 1 - i)\}$; $[T]_\mathcal{A} = \begin{bmatrix} -2 & 0 & \phantom{-} 0\\ \phantom{-} 0 & i & \phantom{-} 0\\ \phantom{-} 0 & 0 & -i\end{bmatrix}$
          \item $\mathcal{A} = \{(1,0);(-1,1)\}$; $[T]_\mathcal{A} = \begin{bmatrix} 1 & 0\\ 0 & -1\end{bmatrix}$
          \item $\mathcal{A} = \{(1,1,0);(1,0,1);(2,1,1)\}$; $[T]_\mathcal{A} = \begin{bmatrix} 3 & 0 & 0\\ 0 & 3 & 0\\ 0 & 0 & 1\end{bmatrix}$
          \item $\mathcal{A} = \{(1,1,0);(-1,0,1);(1,0,1)\}$; $[T]_\mathcal{A} = \begin{bmatrix} 4 & 0 & \phantom{-} 0\\ 0 & 4 & \phantom{-} 0\\ 0 & 0 & -2\end{bmatrix}$
          \item $\mathcal{A} = \{(-8,-1,3);(-2,1,0);(-1,-1,1)\}$; $[T]_\mathcal{A} = \begin{bmatrix} -1 & 0 & 0\\ \phantom{-} 0 & 0 & 0\\ \phantom{-} 0 & 0 & 2\end{bmatrix}$
        \end{enumerate}
  \end{solucao}
\end{exercicio}

\begin{exercicio}
  Seja $T : V \to V$ um operador linear. Mostre que se todo vetor de $V$ for autovetor de $T$, ent\~ao existe $\lambda \in \cp{K}$ tal que $T(u) = \lambda u$ para todo $u \in V$.
\end{exercicio}

\begin{exercicio}
  Seja $T : V \to V$ um operador linear. Mostre que se $\dim_\cp{K}\im T = m$, ent\~ao $tem$ T no m\'aximo $m + 1$ autovalores.
\end{exercicio}

\begin{exercicio}
  Seja $T : \cp{K}^2 \to \cp{K}^2$ um operador linear tal que $T \circ T = 0$. Mostre que
  \begin{enumerate}[label=({\alph*})]
    \item $\im T \sub \ker T$
    \item Se $T \ne 0$, ent\~ao existe uma base $\mathcal{B}$ de $\cp{K}^2$ tal que
    \[
      [T]_\mathcal{B} = \begin{bmatrix} 0 & 0\\ 1 & 0\end{bmatrix}.
    \]
  \end{enumerate}
\end{exercicio}

\begin{exercicio}
  Determine, se existir, uma matriz $P$ com coeficientes em $\real$ e invert{\'\i}vel tal que $P^{-1}AP$ seja diagonal para cada uma das seguintes matrizes:
    \begin{enumerate}[label=({\alph*})]
      \item $A = \begin{bmatrix} 0 & 1\\ 1 & 0\end{bmatrix}$
      \item $A = \begin{bmatrix} 2 & 3\\ 1 & 4\end{bmatrix}$
      \item $A = \begin{bmatrix} \phantom{-} 1 & 2 & -2\\ -2 & 5 & -2\\ -6 & 6 & -3\end{bmatrix}$
      \item $A = \begin{bmatrix} 1 & 0 & 0\\ 1 & 2 & 1\\ 1 & 0 & 2\end{bmatrix}$
      \item $A = \begin{bmatrix} 1 & a \\ a & 1\end{bmatrix}$, $a \in \real$.
    \end{enumerate}
    \begin{solucao}
      \begin{enumerate}[label=({\alph*})]
        \item $P = \begin{bmatrix}
          \phantom{-} 1 & 1\\ -1 & 1
        \end{bmatrix}$ A resposta n\~ao \'e \'unica.
        \item $P = \begin{bmatrix}
          -3 & 1\\ \phantom{-} 1 & 1
        \end{bmatrix}$ A resposta n\~ao \'e \'unica.
        \item $P = \begin{bmatrix}
          1 & -1 & 1\\ 1 & \phantom{-} 0 & 1\\ 3 & \phantom{-} 1 & 0
        \end{bmatrix}$ A resposta n\~ao \'e \'unica.
        \item $A$ n\~ao \'e diagonaliz\'avel.
        \item $P = \begin{bmatrix} 1/2 & -1/2\\ 1/2 & \phantom{-} 1/2\end{bmatrix}$. A resposta n\~ao \'e \'unica.
      \end{enumerate}
    \end{solucao}

\end{exercicio}

\begin{exercicio}
  Seja $T : \real^2 \to \real^2$ uma transforma\c{c}\~ao linear que tem como autovetores $(3,1)$ e $(-2,1)$ associados aos autovalores $-2$ e $3$, respectivamente. Calcule $T(x,y)$.
  \begin{solucao}
    $T(x,y) = (-6y, y -x)$
  \end{solucao}
\end{exercicio}

\begin{exercicio}
  Sejam $T : V \to V$ e $S : V \to V$ transforma\c{c}\~oes lineares. Suponha que $u \in V$ \'e um autovetor de $T$ e de $S$ associado aos autovalores $\lambda_1$ e $\lambda_2$ de $T$ e $S$, respectivamente. Ache um autovetor e um autovalor de:
  \begin{enumerate}[label=({\alph*})]
    \item $\alpha S + \beta T$, onde $\alpha$, $\beta \in \real$.
    \item $S \circ T$.
  \end{enumerate}
\end{exercicio}

\begin{exercicio}
  \begin{enumerate}[label=({\alph*})]
    \item Mostre que se $B$, $M \in \cp{M}_n(\cp{K})$, com $M$ invert{\'\i}vel, ent\~ao
    \[
        (M^{-1}BM)^n = (M^{-1}B^nM)
    \]
    para todo $n \in \cp{N}$.
    \item Calcule $A^n$, $n \in \cp{N}$, onde
    \[
      A = \begin{bmatrix}
        2 & 4\\
        3 & 13
      \end{bmatrix}.
    \]
  \end{enumerate}
  \begin{solucao}
    \begin{enumerate}
    \item[b)] $A^n = \dfrac{1}{13}\begin{bmatrix}
      14^n + 12 & 4(14^n - 1)\\3(14^n - 1) & 12.14^n + 1
    \end{bmatrix}$, $n \in \cp{N}$
  \end{enumerate}
  \end{solucao}
\end{exercicio}

\begin{exercicio}
  Seja $T : \cp{M}_2(\real) \to \cp{M}_2(\real)$ um operador linear cuja matriz em rela\c{c}\~ao \`a base
  \[
    \mathcal{B} = \left\{\begin{bmatrix}1 & 0\\ 1 & 0\end{bmatrix}; \begin{bmatrix}1 & 0\\ 0 & 0\end{bmatrix}; \begin{bmatrix}0 & 1\\ 0 & 1\end{bmatrix}; \begin{bmatrix}0 & 0\\ 0 & 1\end{bmatrix}\right\}
  \]
  \'e dada por
  \[
    [T]_\mathcal{B} = \begin{bmatrix}
      -1 & -4 & -2 & -2\\
      -4 & -1 & -2 & -2\\
      \phantom{-} 2 & \phantom{-} 2 & \phantom{-} 1 & \phantom{-} 4\\
      \phantom{-} 2 & \phantom{-} 2 & \phantom{-} 4 & \phantom{-} 1
    \end{bmatrix}.
  \]
  Determine uma matriz invert{\'\i}vel $M \in \cp{M}_4(\real)$ tal que $M^{-1}[T]_\mathcal{B}M$ seja uma matriz diagonal.
\begin{solucao}
  $M = \begin{bmatrix}
    \phantom{-} 1 & \phantom{-} 1 & \phantom{-} 1 & \phantom{-} 1\\
    \phantom{-} 1 & \phantom{-} 0 & \phantom{-} 1 & \phantom{-} 1\\
    \phantom{-} 1 & -1 & \phantom{-} 0 & -2\\
    -2 & -1 & -1 & -2
  \end{bmatrix}$. A resposta n\~ao \'e \'unica.
\end{solucao}
\end{exercicio}

\begin{exercicio}
  Determine todos os valores de $a$, $b$ e $c \in \complex$ para os quais a matriz abaixo seja diagonaliz\'avel:
  \[
    A = \begin{bmatrix}
      a & b & 1\\
      0 & c & 0\\
      0 & 0 & 1
    \end{bmatrix}.
  \]
  \begin{solucao}
    $a = c$ e $a \ne 1$.
  \end{solucao}
\end{exercicio}

\begin{exercicio}
  Quando uma matriz $2 \times 2$
  \[
    \begin{pmatrix}
      a & b\\c & d
    \end{pmatrix}
  \]
  \'e diagonaliz\'avel?
\end{exercicio}

\begin{exercicio}
  Sejam $T : V \to V$ e $G : V \to V$ transforma\c{c}\~oes lineares sobre um $\cp{K}$-espa\c{c}o vetorial de dimens\~ao finita $V$. Seja $\mathcal{B}$ uma base de $V$ tal que
  \[
    [T]_\mathcal{B}[G]_\mathcal{B} = [G]_\mathcal{B}[T]_\mathcal{B}.
  \]
  Suponha que $T$ possui um autovetor $w$ associado ao autovalor $\lambda \in \cp{K}$.
  \begin{enumerate}[label=({\alph*})]
    \item Mostre que $G(w)$ tamb\'em \'e um autovetor de $T$ associado ao autovalor $\lambda$.
    \item Suponha que $T$ \'e diagonaliz\'avel com distintos autovalores. Qual \'e a dimens\~ao de cada autoespa\c{c}o de $T$.
    \item Mostre que $w$ tamb\'em \'e um autovetor de $G$.
  \end{enumerate}
\end{exercicio}

\begin{exercicio}
  Sejam $U$ e $W$ subespa\c{c}os de $\real^3$ definidos por
  \begin{align*}
    U &= \{(a,b,c) \mid a = b = c\}\\
    W &= \{(0,b,c) \mid b, c \in \real\}.
  \end{align*}
  Mostre que $\real^3 = U \oplus W$.
\end{exercicio}

\begin{exercicio}
  Sejam $U$, $V$ e $W$ os seguintes subespa\c{c}os de $\real^3$:
  \begin{align*}
    U &= \{(a,b,c) \mid a + b + c = 0\}\\
    V &= \{(a,b,c) \mid a = c\}\\
    W &= \{(0,0,c) \mid c \in \real\}.
  \end{align*}
  Mostre que:
  \begin{enumerate}[label=({\alph*})]
    \item $\real^3 = U + V$
    \item $\real^3 = U + W$
    \item $\real^3 = V + W$.
  \end{enumerate}
  Em qual caso a soma \'e direta?
\end{exercicio}

\begin{exercicio}
  Seja $V$ o espa\c{c}o vetorial de todas as fun\c{c}\~oes de $\real$ em $\real$. Seja $U$ o subespa\c{c}o das fun\c{c}\~oes pares e $W$ o subespa\c{c}o das fun\c{c}\~oes {\'\i}mpares. Mostre que $V = U \oplus W$. [Lembre-se que f \'e par se, e somente se, $f(-x) = f(x)$ e que $g$ \'e {\'\i}mpar se, e somente se, $g(-x) = -g(x)$.]
\end{exercicio}

\begin{exercicio}
  Seja $W = \{(z,z) \mid z \in \complex\} \sub \complex^2$. Mostre que $W$ \'e um subespa\c{c}o de $\complex^2$ e encontre subespa\c{c}os $U$ e $V$ de $\complex^2$ tais que $W \oplus V = W \oplus U = \complex^2$ e $U \cap V = \{(0,0\}$.
\end{exercicio}

Seja $V$ um $\cp{K}$-espa\c{c}o vetorial. Para subespa\c{c}os $W_1$, \dots, $W_t$ de $V$ dizemos que $V$ \'e uma \textbf{soma direta} de $W_1$, \dots, $W_t$ se
\begin{itemize}
  \item $W_1 + \cdots + W_t = \{u_1 + \cdots + u_t \mid u_i \in W_i,\ i = 1,\dots, t\}$
  \item $W_i \cap (W_1 + \cdots + W_{i - 1} + W_{i + 1} + \cdots + W_t) = \{0_V\}$, $i = 1$, \dots, $t$.
\end{itemize}
Neste caso escrevemos
\[
  V = W_1 \oplus \cdots \oplus W_t.
\]
Se $V$ \'e um $\cp{K}$-espa\c{c}o vetorial de dimens\~ao finita tal que $V = W_1 \oplus \cdots \oplus W_t$, ent\~ao
\[
  \dim_\cp{K}V = \sum_{i = 1}^t\dim_\cp{K}W_i.
\]

\begin{exercicio}
  Seja $V = W_1 \oplus \cdots \oplus W_t$ e sejam $\mathcal{B}_i \sub W_i$, para cada $i = 1$, \dots, $t$. Considere $\mathcal{B} = \mathcal{B}_1 \cup \dots \cup \mathcal{B}_t$.
  \begin{enumerate}[label=({\alph*})]
    \item Mostre que se $\mathcal{B}_i$ for L.I. para cada $i = 1$, \dots, $t$, ent\~ao $\mathcal{B}$ \'e L.I..
    \item Mostre que se $\mathcal{B}_i$ uma base de $W_i$ para cada $i = 1$, \dots, $t$, ent\~ao $\mathcal{B}$ \'e uma base de $V$.
  \end{enumerate}
\end{exercicio}

\begin{exercicio}
  Mostre que todo espa\c{c}o vetorial finitamente gerado sobre $\cp{K}$ \'e uma soma direta de subespa\c{c}os vetorias de dimens\~ao 1.
\end{exercicio}

\begin{exercicio}
Seja $T \in \mathcal{L}(V,V)$ um operador linear, onde $V$ é um $\cp{K}$-espaço vetorial de dimensão finita. Mostre que se $T = T_1 \oplus T_2$, ent\~ao $p_T(x) = p_{T_1}(x)p_{T_2}(x)$.
\end{exercicio}

\begin{exercicio}
  Sejam $T : V \to V$ um operador linear, $W \sub V$ um subespa\c{c}o de $V$ e $\lambda \in \cp{K}$. Mostre que $W$ \'e $(\lambda Id - T)$-invariante se, e somente se, $W$ for $T$-invariante.
\end{exercicio}

\begin{exercicio}
  Seja $T \in \mathcal{L}(V,V)$ um operador linear com polin\^omio caracter{\'\i}stico $p_T(x) = x^n$. Mostre que existe $m \ge 1$ tal que $T^m = 0$.
\end{exercicio}


\begin{exercicio}
  Mostre que se $T : V \to V$ \'e um operador linear nilpotente, ent\~ao $\ker T \ne \{0_V\}$.
\end{exercicio}

\begin{exercicio}
  Mostre que os seguintes operadores s\~ao nilpotentes e encontre seu \'indice de nilpot\^encia:
  \begin{enumerate}[label=({\alph*})]
    \item Seja $D : \mathcal{P}_3(\real) \to \mathcal{P}_3(\real)$ o operador deriva\c{c}\~ao.
    \item Seja $D : \mathcal{P}_n(\real) \to \mathcal{P}_n(\real)$ o operador deriva\c{c}\~ao.
    \item Seja $T : \real^2 \to \real^2$ o operador linear tal que
    \[
      [T] = \begin{bmatrix}
        0 & 0\\
        1 & 0
      \end{bmatrix}.
    \]
  \end{enumerate}
\end{exercicio}

\begin{exercicio}
  Encontre todas as possibilidades para o polin\^omio minimal de um operador $T : \real^5 \to \real^5$ com polin\^omio caracter{\'\i}stico:
  \begin{enumerate}[label=({\alph*})]
    \item $p_T(x) = (x - 3)^3(x - 2)^2$
    \item $p_T(x) = (x - 1)(x - 2)(x - 3)(x - 4)(x - 5)$
    \item $p_T(x) = (x - 1)^m$, $m \ge 1$
  \end{enumerate}
  \'E poss{\'\i}vel concluir que algum deles \'e necessariamente diagonaliz\'avel?
  \begin{solucao}
    \begin{enumerate}[label=({\alph*})]
      \item $m_T(x) = (x - 3)(x - 2)$ ou $m_T(x) = (x - 3)(x - 2)^2$ ou $m_T(x) = (x - 3)^2(x - 2)$ ou $m_T(x) = (x - 3)^2(x - 2)^2$ ou $m_T(x) = (x - 3)^3(x - 2)$ ou $m_T(x) = (x - 3)^3(x - 2)^2$
      \item $m_T(x) = p_T(x)$ e neste caso \'e diagonaliz\'avel.
      \item Existem $m$ possibilidades que s\~ao: $m_T(x) = (x - 1)$, $m_T(x) = (x - 1)^2$, \dots, $m_T(x) = (x - 1)^m$
      \end{enumerate}
  \end{solucao}
\end{exercicio}

\begin{exercicio}
  Encontre os polin\^omios caracter{\'\i}stico e minimal das seguintes matrizes:
  \begin{enumerate}[label=({\roman*})]
    \item $A = \begin{bmatrix}
        \phantom{-} 1 & \phantom{-} 1 & \phantom{-} 0 & 0\\
        -1 & -1 & \phantom{-} 0 & 0\\
        -2 & -2 & \phantom{-} 2 & 1\\
        \phantom{-} 1 & \phantom{-} 1 & -1 & 0
      \end{bmatrix}$
      \item $B = \begin{bmatrix}
        2 & 5 & 0 & 0 & 0\\
        0 & 2 & 0 & 0 & 0\\
        0 & 0 & 4 & 2 & 0\\
        0 & 0 & 3 & 5 & 0\\
        0 & 0 & 0 & 0 & 7
      \end{bmatrix}$
      \item $C = \begin{bmatrix}
        3 & 1 & 0 & 0 & 0\\
        0 & 3 & 0 & 0 & 0\\
        0 & 0 & 3 & 1 & 0\\
        0 & 0 & 0 & 3 & 1\\
        0 & 0 & 0 & 0 & 3
      \end{bmatrix}$
      \item $D = \begin{bmatrix}
        \lambda & 0 & 0 & 0 & 0\\
        0 & \lambda & 0 & 0 & 0\\
        0 & 0 & \lambda & 0 & 0\\
        0 & 0 & 0 & \lambda & 0\\
        0 & 0 & 0 & 0 & \lambda
      \end{bmatrix}$
      \item $E = \begin{bmatrix}
        1 & 1 & 0\\
        0 & 2 & 0\\
        0 & 0 & 1
      \end{bmatrix}$
      \item $F = \begin{bmatrix}
        2 & 0 & 0\\
        0 & 2 & 2\\
        0 & 0 & 1
      \end{bmatrix}$
  \end{enumerate}
  \begin{solucao}
    \begin{enumerate}[label=({\roman*})]
      \item $p_A(x) = m_A(x) = x^2(x - 1)^2$.
      \item $p_B(x) = (x - 2)^3(x - 7)^2$ $m_A(x) = (x - 2)^2(x - 7)$.
      \item $p_C(x) = (x - 3)^5$ $m_A(x) = (x - 3)^3$.
      \item $p_D(x) = (x - \lambda)^5$ $m_A(x) = (x - \lambda)$.
      \item $p_E(x) = (x - 1)^2(x - 2)$ $m_A(x) = (x - 1)(x - 2)$.
      \item $p_F(x) = (x - 1)(x - 2)^2$ $m_A(x) = (x - 1)(x - 2)$.
    \end{enumerate}
  \end{solucao}
\end{exercicio}

\begin{exercicio}
  Sejam $V$ um $\cp{K}$-espa\c{c}o vetorial de dimens\~ao finita e $T : V \to V$ um operador linear. Mostre que se para algum $l > 0$, temos que $\ker T^l = \ker T^{l + 1}$, ent\~ao $\ker T^l = \ker T^{l + i}$, para todo $i \ge 0$.
\end{exercicio}

% \begin{exercicio}
%   Seja $T : \cp{K}^\cp{N} \to \cp{K}^\cp{N}$ dada por $T((x_n)) = (0,x_1,x_2,\dots,x_n,\dots)$. Mostre que $T$ n\~ao se escreve como uma soma direta de um operador nilpotente com um operador invert{\'\i}vel.
% \end{exercicio}

\begin{exercicio}
  Seja $T : \real^5 \to \real^5$ o operador linear dado por
  \[
      T(x_1,x_2,x_3,x_4,x_5) = (3x_1 -2x_5, 0 , 2x_3 - x_4 + x_5, x_5 - x_1, 2x_1 - x_5).
  \]
  Determine a decomposi\c{c}\~ao $T = T_1 \oplus T_2$ onde $T_1$ \'e nilpotente e $T_2$ \'e invert{\'\i}vel.
\end{exercicio}

\begin{exercicio}
  Determine o n\'umero de matrizes n\~ao semelhantes $A$ em $\cp{M}_5(\real)$ que satisfazem a equa\c{c}\~ao $(A + I_5)^3 = 0$.
  \begin{solucao}
    Existem 2 tipos de matrizes diferentes.
  \end{solucao}
\end{exercicio}

\begin{exercicio}
  Seja $A \in \cp{M}_4(\real)$ tal que $A^4 - 8A^2 + 16I = 0$. Quais s\~ao as poss{\'\i}veis formas de Jordan n\~ao semelhantes para $A$?
  \begin{solucao}
    \[
      [A] = \begin{bmatrix}
        \phantom{-}2 \\
        & \phantom{-}2 \\
        & & -2\\
        & & & -2
      \end{bmatrix},\qquad [A] = \begin{bmatrix}
        \phantom{-}2 & \phantom{-}0\\
        \phantom{-}1& \phantom{-}2 \\
        & & -2\\
        & & & -2
      \end{bmatrix}\]
      \[
      [A] = \begin{bmatrix}
        \phantom{-}2\\
        & \phantom{-}2 \\
        & & -2 & \phantom{-}0\\
        & & \phantom{-}1& -2
      \end{bmatrix},\qquad
      [A] = \begin{bmatrix}
        \phantom{-}2 & \phantom{-}0\\
        \phantom{-}1& \phantom{-}2 \\
        & & -2 & \phantom{-}0\\
        & & \phantom{-}1& -2
      \end{bmatrix}
    \]
  \end{solucao}
\end{exercicio}

\begin{exercicio}
  Verifique se as matrizes seguintes s\~ao semelhantes
  \[
    A = \begin{bmatrix}
      -1 & 0 & 0 & -2\\
      \phantom{-} 0 & 1 & 0 & \phantom{-} 4\\
      -1 & 0 & 1 & \phantom{-} 1\\
      \phantom{-} 0 & 0 & 0 & \phantom{-} 1
    \end{bmatrix}, \quad B = \begin{bmatrix}
      \phantom{-} 1 & 0 & 0 & \phantom{-} 0\\
      -1 & 1 & 0 & \phantom{-} 0\\
      \phantom{-} 0 & 1 & 1 & \phantom{-} 0\\
      \phantom{-} 0 & 0 & 0 & -1
    \end{bmatrix}.
  \]
  \begin{solucao}
    N\~ao s\~ao semelhantes.
  \end{solucao}
\end{exercicio}

\begin{exercicio}
  Ache a forma de Jordan das seguintes matrizes
  \[
      A = \begin{bmatrix}
      0 & -9 & 0 & 0\\
      1 & \phantom{-} 6 & 0 & 0\\
      0 & \phantom{-} 0 & 3 & 0\\
      0 & \phantom{-} 0 & 0 & 3
    \end{bmatrix}, \quad B = \begin{bmatrix}
      \phantom{-} 5 & -9 & -4\\
      \phantom{-} 6 & -11 & -5\\
      -7 & \phantom{-} 13 & \phantom{-} 6
    \end{bmatrix}.
  \]
  \begin{solucao}
    \[
      J(A) = \begin{bmatrix}
        3 & 0\\
        1 & 3\\
        & & 3\\
        & & & 3
      \end{bmatrix}, \qquad J(B) = \begin{bmatrix}
        0 & 0 & 0\\
        1 & 0 & 0\\
        0 & 1 & 0
      \end{bmatrix}
    \]
  \end{solucao}
\end{exercicio}

\begin{exercicio}
  Seja $A$ uma matriz real $9 \times 9$ cujo polin\^omio caracter{\'\i}stico \'e $(x - 3)^5(x - 2)^4$ e cujo polin\^omio minimal \'e $(x - 3)^3(x - 2)^2$. D\^e as poss{\'\i}veis formas de Jordan de $A$.
  \begin{solucao}
    \[
      [J_1] = \begin{bmatrix}
        3 & 0 & 0\\
        1 & 3 & 0\\
        0 & 1 & 3\\
        & & & 3\\
        & & & & 3\\
        & & & & & 2 & 0\\
        & & & & & 1 & 2\\
        & & & & & & & 2\\
        & & & & & & & & 2
      \end{bmatrix}\qquad
      [J_2] = \begin{bmatrix}
        3 & 0 & 0\\
        1 & 3 & 0\\
        0 & 1 & 3\\
        & & & 3 & 0\\
        & & & 1 & 3\\
        & & & & & 2 & 0\\
        & & & & & 1 & 2\\
        & & & & & & & 2\\
        & & & & & & & & 2
      \end{bmatrix}
    \]
    \[
      [J_3] = \begin{bmatrix}
        3 & 0 & 0\\
        1 & 3 & 0\\
        0 & 1 & 3\\
        & & & 3\\
        & & & & 3\\
        & & & & & 2 & 0\\
        & & & & & 1 & 2\\
        & & & & & & & 2 & 0\\
        & & & & & & & 1 & 2
      \end{bmatrix}\qquad
      [J_4] = \begin{bmatrix}
        3 & 0 & 0\\
        1 & 3 & 0\\
        0 & 1 & 3\\
        & & & 3 & 0\\
        & & & 1 & 3\\
        & & & & & 2 & 0\\
        & & & & & 1 & 2\\
        & & & & & & & 2 & 0\\
        & & & & & & & 1 & 2
      \end{bmatrix}
    \]
  \end{solucao}
\end{exercicio}

\begin{exercicio}
  Seja $T$ um operador linear sobre um espa\c{c}o de dimens\~ao finita. Mostre que se $m_T(x)$ for um produto de polin\^omios de grau 1 e sem ra{\'\i}zes repetidas, ent\~ao $T$ \'e diagonaliz\'avel.
\end{exercicio}

\begin{exercicio}
  Encontre todas as poss{\'\i}veis formas de Jordan para o operador linear $T$ cujo polin\^omios caracter{\'\i}stico e minimal s\~ao como seguem:
  \begin{enumerate}[label=({\alph*})]
    \item $p_T(x) = (x - 2)^4(x - 3)^3$, $m_T(x) = (x - 2)^2(x - 3)^2$
    \item $p_T(x) = (x - 7)^5$, $m_T(x) = (x - 7)^2$
    \item $p_T(x) = (x - 2)^7$, $m_T(x) = (x - 2)^3$
    \item $p_T(x) = (x - 3)^4(x - 5)^4$, $m_T(x) = (x - 3)^2(x - 5)^2$
  \end{enumerate}
  \begin{solucao}
    \begin{enumerate}[label=({\alph*})]
      \item $[T] = \begin{bmatrix}
        2 & 0 \\
        1 & 2 \\
        & & 2 & 0\\
        & & 1 & 2\\
        & & & & 3 & 0\\
        & & & & 1 & 3\\
        & & & & & & 3
      \end{bmatrix},\quad [T] = \begin{bmatrix}
        2 & 0 \\
        1 & 2 \\
        & & 2\\
        & & & 2\\
        & & & & 3 & 0\\
        & & & & 1 & 3\\
        & & & & & & 3
      \end{bmatrix}$
      \item $[T] = \begin{bmatrix}
        7 & 0 \\
        1 & 7 \\
        & & 7 & 0\\
        & & 1 & 7\\
        & & & & 7
      \end{bmatrix},\quad [T] = \begin{bmatrix}
        7 & 0 \\
        1 & 7 \\
        & & 7\\
        & & & 7\\
        & & & & 7
      \end{bmatrix}$
      \item $[T] = \begin{bmatrix}
        2 & 0 & 0\\
        1 & 2 & 0\\
        0 & 1 & 2\\
        & & & 2 & 0 & 0\\
        & & & 1 & 2 & 0\\
        & & & 0 & 1 & 2\\
        & & & & & & 2
      \end{bmatrix},\quad [T] = \begin{bmatrix}
        2 & 0 & 0\\
        1 & 2 & 0\\
        0 & 1 & 2\\
        & & & 2 & 0\\
        & & & 1 & 2\\
        & & & & & 2 & 0\\
        & & & & & 1 & 2
      \end{bmatrix}$\\
      $[T] = \begin{bmatrix}
        2 & 0 & 0\\
        1 & 2 & 0\\
        0 & 1 & 2\\
        & & & 2 & 0\\
        & & & 1 & 2\\
        & & & & & 2\\
        & & & & & & 2
      \end{bmatrix},\quad [T] = \begin{bmatrix}
        2 & 0 & 0\\
        1 & 2 & 0\\
        0 & 1 & 2\\
        & & & 2 \\
        & & & & 2\\
        & & & & & 2 \\
        & & & & & & 2
      \end{bmatrix}$
      \item $[T] = \begin{bmatrix}
        3 & 0\\
        1 & 3\\
        & & 3 & 0\\
        & & 1 & 3\\
        & & & & 5 & 0\\
        & & & & 1 & 5\\
        & & & & & & 5 & 0\\
        & & & & & & 1 & 5
      \end{bmatrix},\quad [T] = \begin{bmatrix}
        3 & 0\\
        1 & 3\\
        & & 3 & 0\\
        & & 1 & 3\\
        & & & & 5 & 0\\
        & & & & 1 & 5\\
        & & & & & & 5\\
        & & & & & & & 5
      \end{bmatrix}$\\
      $[T] = \begin{bmatrix}
        3 & 0\\
        1 & 3\\
        & & 3\\
        & & & 3\\
        & & & & 5 & 0\\
        & & & & 1 & 5\\
        & & & & & & 5 & 0\\
        & & & & & & 1 & 5
      \end{bmatrix},\quad [T] = \begin{bmatrix}
        3 & 0\\
        1 & 3\\
        & & 3\\
        & & & 3\\
        & & & & 5 & 0\\
        & & & & 1 & 5\\
        & & & & & & 5\\
        & & & & & & & 5
      \end{bmatrix}$
    \end{enumerate}
  \end{solucao}
\end{exercicio}

\begin{exercicio}
  Se $A$ \'e uma matriz $5 \times 5$ complexa com polin\^omio caracter{\'\i}stico $p_A(x) = (x - 2)^3(x + 7)^2$ e polin\^omio minimal $m_A(x) = (x - 2)^2(x + 7)$, qual \'e a forma de Jordan de $A$?
  \begin{solucao}
    \[
      \begin{bmatrix}
        2 & 0\\
        1 & 2\\
        & & 2\\
        & & & 7\\
        & & & & 7
      \end{bmatrix}
    \]
  \end{solucao}
\end{exercicio}

\begin{exercicio}
  Quantas formas de Jordan s\~ao poss{\'\i}veis para a matriz complexa $6 \times 6$ cujo polin\^omio caracter{\'\i}stico \'e $p_A(x) = (x + 2)^4(x - 1)^2$?
  \begin{solucao}
    Existem 10 poss{\'\i}veis formas de Jordan para $A$.
  \end{solucao}
\end{exercicio}

\begin{exercicio}
  O operador deriva\c{c}\~ao sobre o espa\c{c}os dos polin\^omios reais de grau menor ou igual a 3 \'e representado em rela\c{c}\~ao \`a base can\^onica pela matriz
  \[
      [D]_\mathcal{B} = \begin{bmatrix}
        0 & 1 & 0 & 0\\
        0 & 0 & 2 & 0\\
        0 & 0 & 0 & 3\\
        0 & 0 & 0 & 0
      \end{bmatrix}.
  \]
  Qual a forma de Jordan deste operador?
  \begin{solucao}
    \[
      \begin{bmatrix}
        0 & 0 & 0 & 0\\
        1 & 0 & 0 & 0\\
        0 & 1 & 0 & 0\\
        0 & 0 & 1 & 0
      \end{bmatrix}
    \]
  \end{solucao}
\end{exercicio}

\begin{exercicio}
  Seja $A$ a matriz complexa
  \[
      A = \begin{bmatrix}
        \phantom{-} 2 & 0 & 0 & 0 & 0 & \phantom{-} 0\\
        \phantom{-} 1 & 2 & 0 & 0 & 0 & \phantom{-} 0\\
        -1 & 0 & 2 & 0 & 0 & \phantom{-} 0\\
        \phantom{-} 0 & 1 & 0 & 2 & 0 & \phantom{-} 0\\
        \phantom{-} 1 & 1 & 1 & 1 & 2 & \phantom{-} 0\\
        \phantom{-} 0 & 0 & 0 & 0 & 1 & -1
      \end{bmatrix}.
  \]
  Determinar a forma de Jordan de $A$.
  \begin{solucao}
    \[
      \begin{bmatrix}
        2 & 0 & 0 & 0 \\
        1 & 2 & 0 & 0\\
        0 & 1 & 2 & 0\\
        0 & 0 & 1 & 2\\
         & & & & -1\\
         & & & & & 2
      \end{bmatrix}
    \]
  \end{solucao}
\end{exercicio}

\begin{exercicio}
  Nos casos abaixo, encontre a forma de Jordan e a base de Jordan que gera essa forma:
  \begin{enumerate}[label=({\alph*})]
    \item $A = \begin{bmatrix}
      \phantom{-} 2 & \phantom{-} 2 & \phantom{-} 3\\
      \phantom{-} 1 & \phantom{-} 3 & \phantom{-} 3\\
      -1 & -2 & -2
    \end{bmatrix}$
    \item $B = \begin{bmatrix}
      5 & 4 & 2\\
      4 & 5 & 2\\
      2 & 2 & 2
    \end{bmatrix}$
    \item $C = \begin{bmatrix}
      \phantom{-} 1 & 1 & \phantom{-} 0\\
      -1 & 4 & \phantom{-} 1\\
      -4 & 13 & -3
    \end{bmatrix}$
    \item $D = \begin{bmatrix}
      \phantom{-} 4 & 0 & 1 & 0\\
      \phantom{-} 2 & 2 & 3 & 0\\
      -1 & 0 & 2 & 0\\
      \phantom{-} 4 & 0 & 1 & 2
    \end{bmatrix}$
    \item $E = \begin{bmatrix}
      5 & -1 & 0 & \phantom{-} 0\\
      9 & -1 & 0 & \phantom{-} 0\\
      0 & \phantom{-} 0 & 7 & -2\\
      0 & \phantom{-} 0 & 12 & -3
    \end{bmatrix}$
  \end{enumerate}
  \begin{solucao}
    \begin{enumerate}[label=({\alph*})]
      \item Forma de Jordan: $J(A) = \begin{bmatrix}
        1 & 0 & 0\\
        1 & 1 & 0\\
        0 & 0 & 1
      \end{bmatrix}$, uma poss{\'\i}vel escolha para a base de Jordan \'e: $\mathcal{B} = \{(1,0,0);(1,1,-1);(-3,1,0)\}$
      \item Forma de Jordan: $J(B) = \begin{bmatrix}
        0 & 0 & 0\\
        0 & 1 & 0\\
        0 & 1 & 1
      \end{bmatrix}$, uma poss{\'\i}vel escolha para a base de Jordan \'e: $\mathcal{B} = \{(1,1,3);(-3,-1,0);(1,0,-1)\}$
      \item Forma de Jordan: $J(C) = \begin{bmatrix}
        1 & 0 & 0\\
        0 & 1 & 0\\
        0 & 0 & 10
      \end{bmatrix}$, uma poss{\'\i}vel escolha para a base de Jordan \'e: $\mathcal{B} = \{(-1,0,2);(-1,1,0);(2,2,1)\}$
      \item Forma de Jordan: $J(D) = \begin{bmatrix}
        2 & 0 & 0 & 0\\
        0 & 2 & 0 & 0\\
        0 & 0 & 3 & 0\\
        0 & 0 & 1 & 3
      \end{bmatrix}$, uma poss{\'\i}vel escolha para a base de Jordan \'e: $\mathcal{B} = \{(0,1,0,0);(0,0,0,1);(1,3,0,1);(1,-1,-1,3)\}$
      \item Forma de Jordan: $J(E) = \begin{bmatrix}
        2 & 0 & 0 & 0\\
        1 & 2 & 0 & 0\\
        0 & 0 & 3 & 0\\
        0 & 0 & 0 & 1
      \end{bmatrix}$, uma poss{\'\i}vel escolha para a base de Jordan \'e: $\mathcal{B} = \{(0,-1,0,0);(1,3,0,0);(0,0,-1,2);(0,0,-1,3)\}$
    \end{enumerate}
  \end{solucao}
\end{exercicio}

\newpage
\Closesolutionfile{ans}
\hrule
\begin{center}
{\large\bf RESPOSTAS}
\end{center}
\hrule
\input{ans1}

\end{document}