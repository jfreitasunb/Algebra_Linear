%!TEX program = xelatex
% !TEX encoding = ISO-8859-1

\documentclass[12pt]{exam}

\usepackage{caption}
\usepackage{amssymb}
\usepackage{amsmath,amsfonts,amsthm,amstext}
\usepackage{hhline}
\usepackage[brazil]{babel}
% \usepackage[latin1]{inputenc}
\usepackage{graphicx}
\graphicspath{{/home/jfreitas/Dropbox/imagens-latex/}{/Volumes/Vader/Dropbox/imagens-latex/}{D:/Dropbox/imagens-latex/}}
\usepackage{enumitem}
\usepackage{multicol}
 \usepackage{answers}
\usepackage{tikz,ifthen}
\usetikzlibrary{lindenmayersystems}
\usetikzlibrary[shadings]

\Newassociation{solucao}{Solution}{ans}
\newtheorem{exercicio}{}

\setlength{\topmargin}{-1.0in}
\setlength{\oddsidemargin}{0in}
\setlength{\textheight}{10.1in}
\setlength{\textwidth}{6.5in}
\setlength{\baselineskip}{12mm}

\newcounter{exercicios}
\setcounter{exercicios}{0}
\newcommand{\questao}{
\addtocounter{exercicios}{1}
\noindent{\bf Exerc{\'\i}cio \Roman{exercicios}: }}

\newcommand{\resp}[1]{
\noindent{\bf Exerc{\'\i}cio #1: }}

\newcommand{\im}{{\rm Im\,}}
\newcommand{\sub}{\subseteq}
\newcommand{\n}{\mathbb{N}}
\newcommand{\integer}{\mathbb{Z}}
\newcommand{\rac}{\mathbb{Q}}
\newcommand{\real}{\mathbb{R}}
\newcommand{\complex}{\mathbb{C}}
\newcommand{\cp}[1]{\mathbb{#1}}
\newcommand{\ch}{\mbox{\textrm{car\,}}\nobreak}
\newcommand{\dlim}[2]{\displaystyle\lim_{#1\rightarrow #2}}
\newcommand{\minf}{+\infty}
\newcommand{\ninf}{-\infty}
\renewcommand{\sin}{{\rm sen\,}}
\renewcommand{\sinh}{{\rm senh\,}}
\renewcommand{\tan}{{\rm tg\,}}
\renewcommand{\csc}{{\rm cossec\,}}
\renewcommand{\cot}{{\rm cotg\,}}
\newcommand{\din}[4]{\displaystyle\int_{#1}^{#2}{#3}{d#4}}

\newcommand{\se}[1]{\displaystyle\sum_{n = 1}^\infty{#1}}
\newcommand{\slim}{\displaystyle\lim_{n \rightarrow \infty}}
\newcommand{\seq}[1]{\{{#1_n\}}}
\newcommand{\seg}[1]{\displaystyle\sum_{n = 1}^\infty{#1_n}}
\newcommand{\sei}[2]{\displaystyle\sum_{#1}^\infty{#2}}

\newcommand{\vesp}[1]{\vspace{ #1  cm}}

\newcommand{\compcent}[1]{\vcenter{\hbox{$#1\circ$}}}
\newcommand{\comp}{\mathbin{\mathchoice
{\compcent\scriptstyle}{\compcent\scriptstyle}
{\compcent\scriptscriptstyle}{\compcent\scriptscriptstyle}}}

\begin{document}
\pagestyle{empty}

\Opensolutionfile{ans}[ans1]

\begin{figure}[h]
        \begin{minipage}[c]{1.7cm}
        \includegraphics[width=1.7cm]{unb.pdf}
        \end{minipage}%
        \hspace{0pt}
        \begin{minipage}[c]{4in}
          {Universidade de Bras{\'\i}lia} \\
          {Departamento de Matem{\'a}tica}
\end{minipage}
\end{figure}

\vesp{-0.35} \hrule

\begin{center}
{\Large\bf \'Algebra Linear - Turma A -- 1$^{o}$/2017} \\ \vspace{9pt} {\large\bf
  $8^{\underline{a}}$ Lista de Exerc{\'\i}cios -- Forma de Jordan}\\ \vspace{9pt} Prof. Jos{\'e} Ant{\^o}nio O. Freitas
\end{center}
\hrule

\vesp{.6}

\begin{exercicio}
  Sejam $U$ e $W$ subespa\c{c}os de $\real^3$ definidos por
  \begin{align*}
    U = \{(a,b,c) \mid a = b = c\}\\
    W = \{(0,b,c) \mid b, c \in \real\}.
  \end{align*}
  Mostre que $\real^3 = U \oplus W$.
\end{exercicio}

\begin{exercicio}
  Sejam $U$, $V$ e $W$ os seguintes subespa\c{c}os de $\real^3$:
  \begin{align*}
    U = \{(a,b,c) \mid a + b + c = 0\}\\
    V = \{(a,b,c) \mid a = c\}\\
    W = \{(0,0,c) \mid c \in \real\}.
  \end{align*}
  Mostre que:
  \begin{enumerate}[label=({\alph*})]
    \item $\real^3 = U + V$
    \item $\real^3 = U + W$
    \item $\real^3 = V + W$.
  \end{enumerate}
  Em qual caso a soma \'e direta?
\end{exercicio}

\begin{exercicio}
  Seja $V$ o espa\c{c}o vetorial de todas as fun\c{c}\~oes de $\real$ em $\real$. Seja $U$ o subespa\c{c}o das fun\c{c}\~oes pares e $W$ o subespa\c{c}o das fun\c{c}\~oes {\'\i}mpares. Mostre que $V = U \oplus W$. [Lembre-se que f \'e par se, e somente se, $f(-x) = f(x)$ e que $g$ \'e {\'\i}mpar se, e somente se, $g(-x) = -g(x)$.]
\end{exercicio}

\begin{exercicio}
  Seja $W = \{(z,z) \mid z \in \complex\} \sub \complex^2$. Mostre que $W$ \'e um subespa\c{c}o de $\complex^2$ e encontre subespa\c{c}os $U$ e $V$ de $\complex^2$ tais que $W \oplus V = W \oplus U = \complex^2$ e $U \cap V = \{(0,0\}$.
\end{exercicio}

\begin{exercicio}
  Mostre que todo espa\c{c}o vetorial finitamente gerado sobre $\cp{K}$ \'e uma soma direta de subespa\c{c}os vetorias de dimens\~ao 1.
\end{exercicio}


  Seja $V$ um $\cp{K}$-espa\c{c}o vetorial. Para subespa\c{c}os $W_1$, \dots, $W_t$ de $V$ dizemos que $V$ \'e uma \textbf{soma direta} de $W_1$, \dots, $W_t$ se
  \begin{itemize}
    \item $W_1 + \cdots + W_t = \{u_1 + \cdots + u_t \mid u_i \in W_i,\ i = 1,\dots, t\}$
    \item $W_i \cap (W_1 + \cdots + W_{i - 1} + W_{i + 1} + \cdots + W_t) = \{0_V\}$, $i = 1$, \dots, $t$.
  \end{itemize}
  Neste caso escrevemos
  \[
      V = W_1 \oplus \cdots \oplus W_t.
  \]
  Se $V$ \'e um $\cp{K}$-espa\c{c}o vetorial de dimens\~ao finita tal que $V = W_1 \oplus \cdots \oplus W_t$, ent\~ao
  \[
      \dim_\cp{K}V = \sum_{i = 1}^t\dim_\cp{K}W_i.
  \]

\begin{exercicio}
  Seja $V = W_1 \oplus \cdots \oplus W_t$ e sejam $\mathcal{B}_i \sub W_i$, para cada $i = 1$, \dots, $t$. Considere $\mathcal{B} = \mathcal{B}_1 \cup \dots \cup \mathcal{B}_t$.
  \begin{enumerate}[label=({\alph*})]
    \item Mostre que se $\mathcal{B}_i$ for L.I. para cada $i = 1$, \dots, $t$, ent\~ao $\mathcal{B}$ \'e L.I..
    \item Mostre que se $\mathcal{B}_i$ uma base de $W_i$ para cada $i = 1$, \dots, $t$, ent\~ao $\mathcal{B}$ \'e uma base de $V$.
  \end{enumerate}
\end{exercicio}

\begin{exercicio}
Seja $T \in \mathcal{L}(V,V)$ um operador linear, onde $V$ é um $\cp{K}$-espaço vetorial de dimensão finita. Mostre que se $T = T_1 \oplus T_2$, ent\~ao $p_T(x) = p_{T_1}(x)p_{T_2}(x)$.
\end{exercicio}

\begin{exercicio}
  Sejam $T : V \to V$ um operador linear, $W \sub V$ um subespa\c{c}o de $V$ e $\lambda \in \cp{K}$. Mostre que $W$ \'e $(\lambda Id - T)$-invariante se, e somente se, $W$ for $T$-invariante.
\end{exercicio}

\begin{exercicio}
  Seja $T \in \mathcal{L}(V,V)$ um operador linear com polin\^omio caracter{\'\i}stico $p_T(x) = x^n$. Mostre que existe $m \ge 1$ tal que $T^m = 0$.
\end{exercicio}


\begin{exercicio}
  Mostre que se $T : V \to V$ \'e um operador linear nilpotente, ent\~ao $\ker T \ne \{0_V\}$.
\end{exercicio}

\begin{exercicio}
  Mostre que os seguintes operadores s\~ao nilpotentes e encontre seu \'indice de nilpot\^encia:
  \begin{enumerate}[label=({\alph*})]
    \item Seja $D : \mathcal{P}_3(\real) \to \mathcal{P}_3(\real)$ o operador deriva\c{c}\~ao.
    \item Seja $D : \mathcal{P}_n(\real) \to \mathcal{P}_n(\real)$ o operador deriva\c{c}\~ao.
    \item Seja $T : \real^2 \to \real^2$ o operador linear tal que
    \[
      [T] = \begin{bmatrix}
        0 & 0\\
        1 & 0
      \end{bmatrix}.
    \]
  \end{enumerate}
\end{exercicio}

\begin{exercicio}
  Encontre todas as possibilidades para o polin\^omio minimal de um operador $T : \real^5 \to \real^5$ com polin\^omio caracter{\'\i}stico:
  \begin{enumerate}[label=({\alph*})]
    \item $p_T(x) = (x - 3)^3(x - 2)^2$
    \item $p_T(x) = (x - 1)(x - 2)(x - 3)(x - 4)(x - 5)$
    \item $p_T(x) = (x - 1)^m$, $m \ge 1$
  \end{enumerate}
  \'E poss{\'\i}vel concluir que algum deles \'e necessariamente diagonaliz\'avel?
  \begin{solucao}
    \begin{enumerate}[label=({\alph*})]
      \item $m_T(x) = (x - 3)(x - 2)$ ou $m_T(x) = (x - 3)(x - 2)^2$ ou $m_T(x) = (x - 3)^2(x - 2)$ ou $m_T(x) = (x - 3)^2(x - 2)^2$ ou $m_T(x) = (x - 3)^3(x - 2)$ ou $m_T(x) = (x - 3)^3(x - 2)^2$
      \item $m_T(x) = p_T(x)$ e neste caso \'e diagonaliz\'avel.
      \item Existem $m$ possibilidades que s\~ao: $m_T(x) = (x - 1)$, $m_T(x) = (x - 1)^2$, \dots, $m_T(x) = (x - 1)^m$
      \end{enumerate}
  \end{solucao}
\end{exercicio}

\begin{exercicio}
  Encontre os polin\^omios caracter{\'\i}stico e minimal das seguintes matrizes:
  \begin{enumerate}[label=({\roman*})]
    \item $A = \begin{bmatrix}
        1 & 1 & 0 & 0\\
        -1 & -1 & 0 & 0\\
        -2 & -2 & 2 & 1\\
        1 & 1 & -1 & 0
      \end{bmatrix}$
      \item $B = \begin{bmatrix}
        2 & 5 & 0 & 0 & 0\\
        0 & 2 & 0 & 0 & 0\\
        0 & 0 & 4 & 2 & 0\\
        0 & 0 & 3 & 5 & 0\\
        0 & 0 & 0 & 0 & 7
      \end{bmatrix}$
      \item $C = \begin{bmatrix}
        3 & 1 & 0 & 0 & 0\\
        0 & 3 & 0 & 0 & 0\\
        0 & 0 & 3 & 1 & 0\\
        0 & 0 & 0 & 3 & 1\\
        0 & 0 & 0 & 0 & 3
      \end{bmatrix}$
      \item $D = \begin{bmatrix}
        \lambda & 0 & 0 & 0 & 0\\
        0 & \lambda & 0 & 0 & 0\\
        0 & 0 & \lambda & 0 & 0\\
        0 & 0 & 0 & \lambda & 0\\
        0 & 0 & 0 & 0 & \lambda
      \end{bmatrix}$
      \item $E = \begin{bmatrix}
        1 & 1 & 0\\
        0 & 2 & 0\\
        0 & 0 & 1
      \end{bmatrix}$
      \item $F = \begin{bmatrix}
        2 & 0 & 0\\
        0 & 2 & 2\\
        0 & 0 & 1
      \end{bmatrix}$
  \end{enumerate}
  \begin{solucao}
    \begin{enumerate}[label=({\roman*})]
      \item $p_A(x) = m_A(x) = x^2(x - 1)^2$.
      \item $p_B(x) = (x - 2)^3(x - 7)^2$ $m_A(x) = (x - 2)^2(x - 7)$.
      \item $p_C(x) = (x - 3)^5$ $m_A(x) = (x - 3)^3$.
      \item $p_D(x) = (x - \lambda)^5$ $m_A(x) = (x - \lambda)$.
      \item $p_E(x) = (x - 1)^2(x - 2)$ $m_A(x) = (x - 1)(x - 2)$.
      \item $p_F(x) = (x - 1)(x - 2)^2$ $m_A(x) = (x - 1)(x - 2)$.
    \end{enumerate}
  \end{solucao}
\end{exercicio}

\begin{exercicio}
  Sejam $V$ um $\cp{K}$-espa\c{c}o vetorial de dimens\~ao finita e $T : V \to V$ um operador linear. Mostre que se para algum $l > 0$, temos que $\ker T^l = \ker T^{l + 1}$, ent\~ao $\ker T^l = \ker T^{l + i}$, para todo $i \ge 0$.
\end{exercicio}

% \begin{exercicio}
%   Seja $T : \cp{K}^\cp{N} \to \cp{K}^\cp{N}$ dada por $T((x_n)) = (0,x_1,x_2,\dots,x_n,\dots)$. Mostre que $T$ n\~ao se escreve como uma soma direta de um operador nilpotente com um operador invert{\'\i}vel.
% \end{exercicio}

\begin{exercicio}
  Seja $T : \real^5 \to \real^5$ o operador linear dado por
  \[
      T(x_1,x_2,x_3,x_4,x_5) = (3x_1 -2x_5, 0 , 2x_3 - x_4 + x_5, x_5 - x_1, 2x_1 - x_5).
  \]
  Determine a decomposi\c{c}\~ao $T = T_1 \oplus T_2$ onde $T_1$ \'e nilpotente e $T_2$ \'e invert{\'\i}vel.
\end{exercicio}

\begin{exercicio}
  Determine o n\'umero de matrizes n\~ao semelhantes $A$ em $\cp{M}_5(\real)$ que satisfazem a equa\c{c}\~ao $(A + I_5)^3 = 0$.
  \begin{solucao}
    Existem 2 tipos de matrizes diferentes.
  \end{solucao}
\end{exercicio}

\begin{exercicio}
  Seja $A \in \cp{M}_4(\real)$ tal que $A^4 - 8A^2 + 16I = 0$. Quais s\~ao as poss{\'\i}veis formas de Jordan n\~ao semelhantes para $A$?
  \begin{solucao}
    \[
      [A] = \begin{bmatrix}
        \phantom{-}2 \\
        & \phantom{-}2 \\
        & & -2\\
        & & & -2
      \end{bmatrix},\qquad [A] = \begin{bmatrix}
        \phantom{-}2 & \phantom{-}0\\
        \phantom{-}1& \phantom{-}2 \\
        & & -2\\
        & & & -2
      \end{bmatrix}\]
      \[
      [A] = \begin{bmatrix}
        \phantom{-}2\\
        & \phantom{-}2 \\
        & & -2 & \phantom{-}0\\
        & & \phantom{-}1& -2
      \end{bmatrix},\qquad
      [A] = \begin{bmatrix}
        \phantom{-}2 & \phantom{-}0\\
        \phantom{-}1& \phantom{-}2 \\
        & & -2 & \phantom{-}0\\
        & & \phantom{-}1& -2
      \end{bmatrix}
    \]
  \end{solucao}
\end{exercicio}

\begin{exercicio}
  Verifique se as matrizes seguintes s\~ao semelhantes
  \[
    A = \begin{bmatrix}
      -1 & 0 & 0 & -2\\
      0 & 1 & 0 & 4\\
      -1 & 0 & 1 & 1\\
      0 & 0 & 0 & 1
    \end{bmatrix}, \quad B = \begin{bmatrix}
      1 & 0 & 0 & 0\\
      -1 & 1 & 0 & 0\\
      0 & 1 & 1 & 0\\
      0 & 0 & 0 & -1
    \end{bmatrix}.
  \]
  \begin{solucao}
    N\~ao s\~ao semelhantes.
  \end{solucao}
\end{exercicio}

\begin{exercicio}
  Ache a forma de Jordan das seguintes matrizes
  \[
      A = \begin{bmatrix}
      0 & -9 & 0 & 0\\
      1 & 6 & 0 & 0\\
      0 & 0 & 3 & 0\\
      0 & 0 & 0 & 3
    \end{bmatrix}, \quad B = \begin{bmatrix}
      5 & -9 & -4\\
      6 & -11 & -5\\
      -7 & 13 & 6
    \end{bmatrix}.
  \]
  \begin{solucao}
    \[
      J(A) = \begin{bmatrix}
        3 & 0\\
        1 & 3\\
        & & 3\\
        & & & 3
      \end{bmatrix}, \qquad J(B) = \begin{bmatrix}
        0 & 0 & 0\\
        1 & 0 & 0\\
        0 & 1 & 0
      \end{bmatrix}
    \]
  \end{solucao}
\end{exercicio}

\begin{exercicio}
  Seja $A$ uma matriz real $9 \times 9$ cujo polin\^omio caracter{\'\i}stico \'e $(x - 3)^5(x - 2)^4$ e cujo polin\^omio minimal \'e $(x - 3)^3(x - 2)^2$. D\^e as poss{\'\i}veis formas de Jordan de $A$.
  \begin{solucao}
    \[
      [J_1] = \begin{bmatrix}
        3 & 0 & 0\\
        1 & 3 & 0\\
        0 & 1 & 3\\
        & & & 3\\
        & & & & 3\\
        & & & & & 2 & 0\\
        & & & & & 1 & 2\\
        & & & & & & & 2\\
        & & & & & & & & 2
      \end{bmatrix}\qquad
      [J_2] = \begin{bmatrix}
        3 & 0 & 0\\
        1 & 3 & 0\\
        0 & 1 & 3\\
        & & & 3 & 0\\
        & & & 1 & 3\\
        & & & & & 2 & 0\\
        & & & & & 1 & 2\\
        & & & & & & & 2\\
        & & & & & & & & 2
      \end{bmatrix}
    \]
    \[
      [J_3] = \begin{bmatrix}
        3 & 0 & 0\\
        1 & 3 & 0\\
        0 & 1 & 3\\
        & & & 3\\
        & & & & 3\\
        & & & & & 2 & 0\\
        & & & & & 1 & 2\\
        & & & & & & & 2 & 0\\
        & & & & & & & 1 & 2
      \end{bmatrix}\qquad
      [J_4] = \begin{bmatrix}
        3 & 0 & 0\\
        1 & 3 & 0\\
        0 & 1 & 3\\
        & & & 3 & 0\\
        & & & 1 & 3\\
        & & & & & 2 & 0\\
        & & & & & 1 & 2\\
        & & & & & & & 2 & 0\\
        & & & & & & & 1 & 2
      \end{bmatrix}
    \]
  \end{solucao}
\end{exercicio}

\begin{exercicio}
  Seja $T$ um operador linear sobre um espa\c{c}o de dimens\~ao finita. Mostre que se $m_T(x)$ for um produto de polin\^omios de grau 1 e sem ra{\'\i}zes repetidas, ent\~ao $T$ \'e diagonaliz\'avel.
\end{exercicio}

\begin{exercicio}
  Encontre todas as poss{\'\i}veis formas de Jordan para o operador linear $T$ cujo polin\^omios caracter{\'\i}stico e minimal s\~ao como seguem:
  \begin{enumerate}[label=({\alph*})]
    \item $p_T(x) = (x - 2)^4(x - 3)^3$, $m_T(x) = (x - 2)^2(x - 3)^2$
    \item $p_T(x) = (x - 7)^5$, $m_T(x) = (x - 7)^2$
    \item $p_T(x) = (x - 2)^7$, $m_T(x) = (x - 2)^3$
    \item $p_T(x) = (x - 3)^4(x - 5)^4$, $m_T(x) = (x - 3)^2(x - 5)^2$
  \end{enumerate}
  \begin{solucao}
    \begin{enumerate}[label=({\alph*})]
      \item $[T] = \begin{bmatrix}
        2 & 0 \\
        1 & 2 \\
        & & 2 & 0\\
        & & 1 & 2\\
        & & & & 3 & 0\\
        & & & & 1 & 3\\
        & & & & & & 3
      \end{bmatrix},\quad [T] = \begin{bmatrix}
        2 & 0 \\
        1 & 2 \\
        & & 2\\
        & & & 2\\
        & & & & 3 & 0\\
        & & & & 1 & 3\\
        & & & & & & 3
      \end{bmatrix}$
      \item $[T] = \begin{bmatrix}
        7 & 0 \\
        1 & 7 \\
        & & 7 & 0\\
        & & 1 & 7\\
        & & & & 7
      \end{bmatrix},\quad [T] = \begin{bmatrix}
        7 & 0 \\
        1 & 7 \\
        & & 7\\
        & & & 7\\
        & & & & 7
      \end{bmatrix}$
      \item $[T] = \begin{bmatrix}
        2 & 0 & 0\\
        1 & 2 & 0\\
        0 & 1 & 2\\
        & & & 2 & 0 & 0\\
        & & & 1 & 2 & 0\\
        & & & 0 & 1 & 2\\
        & & & & & & 2
      \end{bmatrix},\quad [T] = \begin{bmatrix}
        2 & 0 & 0\\
        1 & 2 & 0\\
        0 & 1 & 2\\
        & & & 2 & 0\\
        & & & 1 & 2\\
        & & & & & 2 & 0\\
        & & & & & 1 & 2
      \end{bmatrix}$\\
      $[T] = \begin{bmatrix}
        2 & 0 & 0\\
        1 & 2 & 0\\
        0 & 1 & 2\\
        & & & 2 & 0\\
        & & & 1 & 2\\
        & & & & & 2\\
        & & & & & & 2
      \end{bmatrix},\quad [T] = \begin{bmatrix}
        2 & 0 & 0\\
        1 & 2 & 0\\
        0 & 1 & 2\\
        & & & 2 \\
        & & & & 2\\
        & & & & & 2 \\
        & & & & & & 2
      \end{bmatrix}$
      \item $[T] = \begin{bmatrix}
        3 & 0\\
        1 & 3\\
        & & 3 & 0\\
        & & 1 & 3\\
        & & & & 5 & 0\\
        & & & & 1 & 5\\
        & & & & & & 5 & 0\\
        & & & & & & 1 & 5
      \end{bmatrix},\quad [T] = \begin{bmatrix}
        3 & 0\\
        1 & 3\\
        & & 3 & 0\\
        & & 1 & 3\\
        & & & & 5 & 0\\
        & & & & 1 & 5\\
        & & & & & & 5\\
        & & & & & & & 5
      \end{bmatrix}$\\
      $[T] = \begin{bmatrix}
        3 & 0\\
        1 & 3\\
        & & 3\\
        & & & 3\\
        & & & & 5 & 0\\
        & & & & 1 & 5\\
        & & & & & & 5 & 0\\
        & & & & & & 1 & 5
      \end{bmatrix},\quad [T] = \begin{bmatrix}
        3 & 0\\
        1 & 3\\
        & & 3\\
        & & & 3\\
        & & & & 5 & 0\\
        & & & & 1 & 5\\
        & & & & & & 5\\
        & & & & & & & 5
      \end{bmatrix}$
    \end{enumerate}
  \end{solucao}
\end{exercicio}

\begin{exercicio}
  Se $A$ \'e uma matriz $5 \times 5$ complexa com polin\^omio caracter{\'\i}stico $p_A(x) = (x - 2)^3(x + 7)^2$ e polin\^omio minimal $m_A(x) = (x - 2)^2(x + 7)$, qual \'e a forma de Jordan de $A$?
  \begin{solucao}
    \[
      \begin{bmatrix}
        2 & 0\\
        1 & 2\\
        & & 2\\
        & & & 7\\
        & & & & 7
      \end{bmatrix}
    \]
  \end{solucao}
\end{exercicio}

\begin{exercicio}
  Quantas formas de Jordan s\~ao poss{\'\i}veis para a matriz complexa $6 \times 6$ cujo polin\^omio caracter{\'\i}stico \'e $p_A(x) = (x + 2)^4(x - 1)^2$?
  \begin{solucao}
    Existem 10 poss{\'\i}veis formas de Jordan para $A$.
  \end{solucao}
\end{exercicio}

\begin{exercicio}
  O operador deriva\c{c}\~ao sobre o espa\c{c}os dos polin\^omios reais de grau menor ou igual a 3 \'e representado em rela\c{c}\~ao \`a base can\^onica pela matriz
  \[
      [D]_\mathcal{B} = \begin{bmatrix}
        0 & 1 & 0 & 0\\
        0 & 0 & 2 & 0\\
        0 & 0 & 0 & 3\\
        0 & 0 & 0 & 0
      \end{bmatrix}.
  \]
  Qual a forma de Jordan deste operador?
  \begin{solucao}
    \[
      \begin{bmatrix}
        0 & 0 & 0 & 0\\
        1 & 0 & 0 & 0\\
        0 & 1 & 0 & 0\\
        0 & 0 & 1 & 0
      \end{bmatrix}
    \]
  \end{solucao}
\end{exercicio}

\begin{exercicio}
  Seja $A$ a matriz complexa
  \[
      A = \begin{bmatrix}
        2 & 0 & 0 & 0 & 0 & 0\\
        1 & 2 & 0 & 0 & 0 & 0\\
        -1 & 0 & 2 & 0 & 0 & 0\\
        0 & 1 & 0 & 2 & 0 & 0\\
        1 & 1 & 1 & 1 & 2 & 0\\
        0 & 0 & 0 & 0 & 1 & -1
      \end{bmatrix}.
  \]
  Determinar a forma de Jordan de $A$.
  \begin{solucao}
    \[
      \begin{bmatrix}
        2 & 0 & 0 & 0 \\
        1 & 2 & 0 & 0\\
        0 & 1 & 2 & 0\\
        0 & 0 & 1 & 2\\
         & & & & -1\\
         & & & & & 2
      \end{bmatrix}
    \]
  \end{solucao}
\end{exercicio}

\begin{exercicio}
  Nos casos abaixo, encontre a forma de Jordan e a base de Jordan que gera essa forma:
  \begin{enumerate}[label=({\alph*})]
    \item $A = \begin{bmatrix}
      2 & 2 & 3\\
      1 & 3 & 3\\
      -1 & -2 & -2
    \end{bmatrix}$
    \item $B = \begin{bmatrix}
      5 & 4 & 2\\
      4 & 5 & 2\\
      2 & 2 & 2
    \end{bmatrix}$
    \item $C = \begin{bmatrix}
      -1 & 1 & 1\\
      -2 & 2 & 1\\
      -1 & 1 & 1
    \end{bmatrix}$
    \item $D = \begin{bmatrix}
      4 & 0 & 1 & 0\\
      2 & 2 & 3 & 0\\
      -1 & 0 & 2 & 0\\
      4 & 0 & 1 & 2
    \end{bmatrix}$
    \item $E = \begin{bmatrix}
      5 & -1 & 0 & 0\\
      9 & -1 & 0 & 0\\
      0 & 0 & 7 & 2\\
      0 & 0 & 12 & 3
    \end{bmatrix}$
  \end{enumerate}
  \begin{solucao}
    \begin{enumerate}[label=({\alph*})]
      \item Forma de Jordan: $J(A) = \begin{bmatrix}
        1 & 0 & 0\\
        1 & 1 & 0\\
        0 & 0 & 1
      \end{bmatrix}$, base de Jordan: FAZER%$\mathcal{B} = \{(1,0,0);(1,1,-1);(-2,1,0)\}$
      \item Forma de Jordan: $J(B) = \begin{bmatrix}
        0 & 0 & 0\\
        0 & 1 & 0\\
        0 & 1 & 1
      \end{bmatrix}$, base de Jordan: FAZER%$\mathcal{B} = \{(1,1,0);(0,1,0);(1,1,1)\}$
      \item Forma de Jordan: $J(C) = \begin{bmatrix}
        1 & 0 & 0\\
        0 & 1 & 0\\
        0 & 0 & 10
      \end{bmatrix}$, base de Jordan: FAZER%$\mathcal{B} = \{(-1,0,2);(-1,1,0);(2,2,1)\}$
      \item Forma de Jordan: $J(D) = \begin{bmatrix}
        2 & 0 & 0 & 0\\
        0 & 2 & 0& 0\\
        0 & 0 & 3 & 0\\
        0 & 0 & 1 & 3
      \end{bmatrix}$, base de Jordan: FAZER%$\mathcal{B} = \{(0,1,0,0);(0,0,0,1);(0,4,1,-2);\linebreak(1,-1,-1,3)\}$
      \item Forma de Jordan: $J(E) = \begin{bmatrix}
        1 & -1 & -2 & 3\\
        0 & 0 & -2 & 3\\
        0 & 1 & 1 & -1\\
        0 & 0 & 1 & 2
      \end{bmatrix}$, base de Jordan: FAZER%$\mathcal{B} = \{(1,2,0,0);(1,3,0,0);(0,0,-1,2);\linebreak(0,0,-1,3)\}$
      \item Forma de Jordan: $J(F) = \begin{bmatrix}
        1 & 0 & 0 & 0\\
        1 & 1 & 0 & 0\\
        0 & 1 & 1 & 0\\
        0 & 0 & 0 & 1
      \end{bmatrix}$, base de Jordan: FAZER%$\mathcal{B} = \{(1,1,-1,0);(1,3,0,0);(0,0,-1,2);\linebreak(0,0,-1,3)\}$
    \end{enumerate}
  \end{solucao}
\end{exercicio}

\newpage
\Closesolutionfile{ans}
\hrule
\begin{center}
{\large\bf RESPOSTAS}
\end{center}
\hrule
\input{ans1}

\end{document}