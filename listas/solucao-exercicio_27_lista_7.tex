% !TEX encoding = ISO-8859-1
\documentclass[12pt]{article}

\usepackage{amssymb}
\usepackage{amsmath,amsfonts,amsthm,amstext}
\usepackage[brazil]{babel}
% \usepackage[latin1]{inputenc}
\usepackage{graphicx}
\usepackage{enumitem}
\usepackage{multicol}
\usepackage[all]{xy}

\setlength{\topmargin}{-1.0in}
\setlength{\oddsidemargin}{0in}
\setlength{\textheight}{10.1in}
\setlength{\textwidth}{6.5in}
\setlength{\baselineskip}{12mm}

\newcounter{exercicios}
\setcounter{exercicios}{0}
\newcommand{\questao}{
\addtocounter{exercicios}{15}
\noindent{\bf Exerc{\'\i}cio \arabic{exercicios}: }}

\newcommand{\equi}{\Leftrightarrow}
\newcommand{\bic}{\leftrightarrow}
\newcommand{\cond}{\rightarrow}
\newcommand{\impl}{\Rightarrow}
\newcommand{\nao}{\sim}
\newcommand{\sub}{\subseteq}
\newcommand{\e}{\ \wedge\ }
\newcommand{\ou}{\ \vee\ }
\newcommand{\vaz}{\emptyset}

\newcommand{\real}{\mathbb{R}}
\newcommand{\vesp}{\vspace{0.2cm}}
\newcommand{\z}{\mathbb{Z}}
\newcommand{\n}{\mathbb{N}}
\newcommand{\q}{\mathbb{Q}}
\newtheorem{defin}{Defini{\c c}{\~a}o}

\newcommand{\compcent}[1]{\vcenter{\hbox{$#1\circ$}}}
\newcommand{\comp}{\mathbin{\mathchoice
{\compcent\scriptstyle}{\compcent\scriptstyle}
{\compcent\scriptscriptstyle}{\compcent\scriptscriptstyle}}}

\begin{document}

\questao Seja $T : \real^5 \to \real^5$ o operador linear dado por
  \[
      T(x_1,x_2,x_3,x_4,x_5) = (3x_1 -2x_5, 0 , 2x_3 - x_4 + x_5, x_5 - x_1, 2x_1 - x_5).
  \]
  Determine a decomposi\c{c}\~ao $T = T_1 \oplus T_2$ onde $T_1$ \'e nilpotente e $T_2$ \'e invert{\'\i}vel.

\textbf{Solu{\c c}{\~a}o}: Precisamos encontrar $\real$-subespaços vetoriais $T$-invariante $W_1$ e $W_2$ tais que $T_1 = T : W_1 \to W_1$ seja nilpotente, $T_2 = T : W_2 \to W_2$ seja invertível e $\real^5 = W_1 \oplus W_2$.

Para issso, primeiro observe que
\begin{align*}
	T(e_1) &= 3e_1 - e_4 + 2e_5\\
	T(e_2) &= (0,0,0,0,0)\\
	T(e_3) &= 2e_3\\
	T(e_4) &= -e_3\\
	T(e_5) &= -2e_1 + e_3 + e_4 - e_5.
\end{align*}

Segue facilmente que
\[
 	\ker T = [e_2, e_3 + 2e_4].
\]

Defina $W_1 = \ker T$. Assim
\[
	T_1(e_2) = T_1(e_3 + 2e_4) = (0,0,0,0,0)
\]
e então $T_1$ é nilpotente de índice de nilpotência 1.

Agora, defina
\[
	W_2 = [e_1 + e_5, e_3, e_4 + e_5].
\]
Temos
\begin{align*}
	T_2(e_1 + e_5) &= e_1 + e_3 + e_5 \in W_2\\
	T_2(e_3) &= 3e_3 \in W_2\\
	T_2(e_4 + e_5) &= -2e_1 + e_4 - e_5 = -2(e_1 + e_5) + (e_4 + e_5) \in W_2.
\end{align*}

É fácil ver que $\ker T_2 = \{(0,0,0,0,0)\}$, daí $T_2 : W_2 \to W_2$ é um invertível. Portanto $T = T_1 + T_2$, onde $T_1$ é nilpotente e $T_2$ é invertível.



\end{document}