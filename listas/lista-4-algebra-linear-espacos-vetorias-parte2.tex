%!TEX program = xelatex 
% !TEX encoding = ISO-8859-1
\documentclass[12pt]{exam}

\usepackage{caption}
\usepackage{amssymb}
\usepackage{amsmath,amsfonts,amsthm,amstext}
\usepackage[brazil]{babel}
% \usepackage[latin1]{inputenc}
%\usepackage[pdftex]{graphicx}
\usepackage{graphicx}
\graphicspath{{/ArquivosUbuntu/Dropbox/imagens-latex/}{D:/Dropbox/imagens-latex/}}
\usepackage{enumitem}
\usepackage{multicol}
 \usepackage{answers}
\usepackage{tikz,ifthen}
\usetikzlibrary{lindenmayersystems}
\usetikzlibrary[shadings]

\Newassociation{solucao}{Solution}{ans}
\newtheorem{exercicio}{}

\setlength{\topmargin}{-1.0in}
\setlength{\oddsidemargin}{0in}
\setlength{\textheight}{10.1in}
\setlength{\textwidth}{6.5in}
\setlength{\baselineskip}{12mm}

\newcounter{exercicios}
\setcounter{exercicios}{0}
\newcommand{\questao}{
\addtocounter{exercicios}{1}
\noindent{\bf Exerc{\'\i}cio \Roman{exercicios}: }}

\newcommand{\resp}[1]{
\noindent{\bf Exerc{\'\i}cio #1: }}

\newcommand{\sub}{\subseteq}
\newcommand{\n}{\mathbb{N}}
\newcommand{\integer}{\mathbb{Z}}
\newcommand{\rac}{\mathbb{Q}}
\newcommand{\real}{\mathbb{R}}
\newcommand{\complex}{\mathbb{C}}
\newcommand{\cp}[1]{\mathbb{#1}}
\newcommand{\ch}{\mbox{\textrm{car\,}}\nobreak}
\newcommand{\dlim}[2]{\displaystyle\lim_{#1\rightarrow #2}}
\newcommand{\minf}{+\infty}
\newcommand{\ninf}{-\infty}
\renewcommand{\sin}{{\rm sen\,}}
\renewcommand{\sinh}{{\rm senh\,}}
\renewcommand{\tan}{{\rm tg\,}}
\renewcommand{\csc}{{\rm cossec\,}}
\renewcommand{\cot}{{\rm cotg\,}}
\newcommand{\din}[4]{\displaystyle\int_{#1}^{#2}{#3}{d#4}}

\newcommand{\se}[1]{\displaystyle\sum_{n = 1}^\infty{#1}}
\newcommand{\slim}{\displaystyle\lim_{n \rightarrow \infty}}
\newcommand{\seq}[1]{\{{#1_n\}}}
\newcommand{\seg}[1]{\displaystyle\sum_{n = 1}^\infty{#1_n}}
\newcommand{\sei}[2]{\displaystyle\sum_{#1}^\infty{#2}}

\newcommand{\vesp}[1]{\vspace{ #1  cm}}

\newcommand{\compcent}[1]{\vcenter{\hbox{$#1\circ$}}}
\newcommand{\comp}{\mathbin{\mathchoice
{\compcent\scriptstyle}{\compcent\scriptstyle}
{\compcent\scriptscriptstyle}{\compcent\scriptscriptstyle}}}

\begin{document}
\pagestyle{empty}

\Opensolutionfile{ans}[ans1]

\begin{figure}[h]
        \begin{minipage}[c]{1.7cm}
        \includegraphics[width=1.7cm]{unb.pdf}
        \end{minipage}%
        \hspace{0pt}
        \begin{minipage}[c]{4in}
          {Universidade de Bras{\'\i}lia} \\
          {Departamento de Matem{\'a}tica}
\end{minipage}
\end{figure}

\vesp{-0.35} \hrule

\begin{center}
{\Large\bf \'Algebra Linear - Turma A -- 1$^{o}$/2019} \\ \vspace{9pt} {\large\bf
  $4^{\underline{a}}$ Lista de Exerc{\'\i}cios -- Subespa\c{c}os Vetoriais}\\ \vspace{9pt} Prof. Jos{\'e} Ant{\^o}nio O. Freitas
\end{center}
\hrule

\vesp{.6}

% \begin{exercicio}
%   Sejam $V$ um espa\c{c}o vetorial sobre um corpo $\cp{K}$ e $W$ um subconjunto n\~ao vazio de $V$. Prove que:
%   \begin{enumerate}[label={\alph*})]
%     \item Se $W$ \'e um subespa\c{c}o de $V$, ent\~ao $\lambda u_1 + u_2 \in W$ para todos $u_1$, $u_2 \in W$ e todo $\lambda \in \cp{K}$.
%     \item Se $\lambda u_1 + u_2 \in W$ para todos $u_1$, $u_2 \in W$ e todo $\lambda \in \cp{K}$, ent\~ao $W$ \'e um subespa\c{c}o de $V$.
%   \end{enumerate}
% \end{exercicio}

% \begin{exercicio}
%   Se $W_1$ e $W_2$ s\~ao subespa\c{c}os de uma $\cp{K}$-espa\c{c}o vetorial $V$, mostre que:
%   \begin{enumerate}[label={\alph*})]
%     \item $W_1 \cap W_2$ \'e um subespa\c{c}o vetorial;
%     \item $W_1 + W_2 = \{u_1 + u_2 \mid u_1 \in W_1; u_2 \in W_2\} $ \'e um subespa\c{c}o vetorial.
%   \end{enumerate}
% \end{exercicio}

\begin{exercicio}
  Seja $V$ uma $\cp{K}$-espa\c{c}o vetorial finitamente gerado e seja $\mathcal{B}$ um conjunto L.I. em $V$. Mostre que existe uma base de $V$ contendo $\mathcal{B}$.
\end{exercicio}

Nos exerc{\'\i}cios \ref{subespacoinicio} \`a \ref{subespacofim}, verifique se $S$ \'e um subespa\c{c}o vetorial do espa\c{c}o vetorial $V$ sobre o corpo $\cp{K}$ em quest\~ao:
\begin{exercicio}\label{subespacoinicio}
$V = \real^n$ e $S = \{(a_1, a_2, \dots, a_n) \in \real^n \mid a_1a_2 = 0\}$; $\cp{K} = \real$.
\begin{solucao}
  N\~ao \'e subespa\c{c}o.
\end{solucao}
\end{exercicio}

\begin{exercicio}
$V = \cp{M}_2(\complex)$ e $S = \left\{\begin{pmatrix} a_{11} & a_{12}\\ a_{21} & a_{22}\end{pmatrix} \in V \mid a_{ij} = \overline{a_{ji}}, i, j = 1, 2\right\}$; $\cp{K} = \complex$, onde $\overline{a + bi} = a - bi$.
\begin{solucao}
  N\~ao \'e subespa\c{c}o.
\end{solucao}
\end{exercicio}

\begin{exercicio}
$V = \real^3$ e $S = \{(x_1, x_2, x_1x_2) \in \real^3 \mid x_1, x_2 \in \real\}$; $\cp{K} = \real$.
\begin{solucao}
  N\~ao \'e subespa\c{c}o.
\end{solucao}
\end{exercicio}

\begin{exercicio}
$V = \real^2$ e $S = \{(x, y) \in \real^2 \mid x + 3y = 0\}$; $\cp{K} = \real$.
\begin{solucao}
  \'E subespa\c{c}o.
\end{solucao}
\end{exercicio}

\begin{exercicio}
$V = \real^2$ e $S = \{(x, y) \in \real^2 \mid y = x + 1\}$; $\cp{K} = \real$.
\begin{solucao}
  N\~ao \'e subespa\c{c}o.
\end{solucao}
\end{exercicio}

\begin{exercicio}
$V = \real^2$ e $S = \{(x, y) \in \real^2 \mid x \ge 0\}$; $\cp{K} = \real$.
\begin{solucao}
  N\~ao \'e subespa\c{c}o.
\end{solucao}
\end{exercicio}

\begin{exercicio}
$V = \real^3$ e $S = \{(x, y, z) \in \real^3 \mid y = x + 2 \mbox{ e } z = 0\}$; $\cp{K} = \real$.
\begin{solucao}
  N\~ao \'e subespa\c{c}o.
\end{solucao}
\end{exercicio}

\begin{exercicio}
$V = \real^3$ e $S = \{(x, y, z) \in \real^3 \mid x + 2y -3z = 4\}$; $\cp{K} = \real$.
\begin{solucao}
  N\~ao \'e subespa\c{c}o.
\end{solucao}
\end{exercicio}

\begin{exercicio}
$V = \real^3$ e $S = \{(x, y, z) \in \real^3 \mid x = -z \mbox{ e } x = z\}$; $\cp{K} = \real$.
\begin{solucao}
  \'E subespa\c{c}o.
\end{solucao}
\end{exercicio}

\begin{exercicio}
$V = \real^3$ e $S = \left\{(x, y, z) \in \real^3 \mid \dfrac{x}{2} = \dfrac{y - 3}{5}\right\}$; $\cp{K} = \real$.
\begin{solucao}
  Não \'e subespa\c{c}o.
\end{solucao}
\end{exercicio}

\begin{exercicio}
$V = \real^4$ e $S = \{(x_1, x_2, x_3, x_4) \in \real^4 \mid x_1 + x_2 = 0, x_3 - x_4 = 0 \in \real\}$; $\cp{K} = \real$.
\begin{solucao}
  \'E subespa\c{c}o.
\end{solucao}
\end{exercicio}

\begin{exercicio}
$V = \real^4$ e $S = \{(x_1, x_2, x_3, x_4) \in \real^4 \mid 2x_1 + x_2 - x_4 = 0, x_3 = 0 \in \real\}$; $\cp{K} = \real$.
\begin{solucao}
  \'E subespa\c{c}o.
\end{solucao}
\end{exercicio}

\begin{exercicio}
$V = \cp{M}_2(\complex)$ e $S = \left\{\begin{bmatrix} a & b\\ c & d\end{bmatrix} \in V \mid b = c\right\}$; $\cp{K} = \complex$.
\begin{solucao}
  \'E subespa\c{c}o.
\end{solucao}
\end{exercicio}

\begin{exercicio}
$V = \cp{M}_2(\real)$ e $S = \left\{\begin{bmatrix} a & b\\ c & d\end{bmatrix} \in V \mid c = a + b \mbox{ e } d = 0\right\}$; $\cp{K} = \real$.
\begin{solucao}
  \'E subespa\c{c}o.
\end{solucao}
\end{exercicio}

\begin{exercicio}
$V = \cp{M}_2(\real)$ e $S = \left\{\begin{bmatrix} a & b\\ 0 & c\end{bmatrix} \in V \mid a, b, c \in \real\right\}$; $\cp{K} = \real$.
\begin{solucao}
  \'E subespa\c{c}o.
\end{solucao}
\end{exercicio}

\begin{exercicio}
$V = \cp{M}_2(\real)$ e $S = \left\{\begin{bmatrix} a & b\\ b & c\end{bmatrix} \in V \mid a, b, c \in \real\right\}$; $\cp{K} = \real$.
\begin{solucao}
  \'E subespa\c{c}o.
\end{solucao}
\end{exercicio}

\begin{exercicio}
$V = \cp{M}_2(\real)$ e $S = \left\{\begin{bmatrix} a & a+b\\ a-b & b\end{bmatrix} \in V \mid a, b \in \real \right\}$; $\cp{K} = \real$.
\begin{solucao}
  \'E subespa\c{c}o.
\end{solucao}
\end{exercicio}

\begin{exercicio}
$V = \cp{M}_n(\integer_p)$ e $S = \{A \in V \mid \det A = \overline{0}\}$; $\cp{K} = \integer_p$.
\begin{solucao}
  N\~ao \'e subespa\c{c}o.
\end{solucao}
\end{exercicio}


\begin{exercicio}\label{subespacofim}
$V = \real^n$ e $S = \{(a_1, a_2, \dots, a_n) \in \real^n \mid a_1a_2 = 0\}$; $\cp{K} = \real$.
\begin{solucao}
  N\~ao \'e subespa\c{c}o.
\end{solucao}
\end{exercicio}

\begin{exercicio}
  Sejam $W_1$ e $W_2$ subespa\c{c}os de um $\cp{K}$-espa\c{c}o vetorial $V$.
  \begin{enumerate}[label={\alph*})]
    \item D\^e um exemplo mostrando que $W_1 \cup W_2$ pode n\~ao ser subespa\c{c}o de $V$.
    \item Prove que $W_1 \cup W_2$ \'e um subespa\c{c}o de $V$ se, e somente se, $W_1 \subseteq W_2$ ou $W_2 \subseteq W_1$.
  \end{enumerate}
\end{exercicio}

\begin{exercicio}
  Sejam $W_1$ e $W_2$ subespa\c{c}os de um espa\c{c}o vetorial $V$ sobre $\cp{K}$ tais que $W_1 \cap W_2 = \{0_V\}$,
  \begin{enumerate}[label={\alph*})]
    \item Mostre que se $\mathcal{B}_1$ e $\mathcal{B}_2$ s\~ao conjuntos L.I. em $W_1$ e $W_2$, respectivamente, ent\~ao $\mathcal{B}_1 \cup \mathcal{B}_2$ \'e L.I. em $V$.
    \item Mostre que se $\mathcal{B}_1$ e $\mathcal{B}_2$ s\~ao bases de $W_1$ e $W_2$, respectivamente, ent\~ao $\mathcal{B}_1 \cup \mathcal{B}_2$ \'e base de $W_1 + W_2$.
  \end{enumerate}
\end{exercicio}

\begin{exercicio}
  Sejam $V$ um $\cp{K}$-espa\c{c}o vetorial e $S \sub V $ um subconjunto n\~ao vazio de $V$. Mostre que $S$ \'e um subespa\c{c}o vetorial de $V$ se, e somente se, $S + S \sub S$ e $\lambda S \sub S$ para cada $\lambda \in \cp{K}$.
\end{exercicio}

\begin{exercicio}
  Seja $W = \left\{\begin{pmatrix} a_{11} & a_{12}\\ a_{21} & a_{22}\end{pmatrix} \in \cp{M}_2(\complex) \mid a_{11} + a_{12} = 0\right\}$.
  \begin{enumerate}[label={\alph*})]
    \item Mostre que $W$ \'e um espa\c{c}o vetorial sobre $\real$.
    \item Determine uma base de $W$.
    \item Seja $W_1 = \{(a_{ij})_{i,j}, \in \cp{M}_2(\complex), i, j = 1, 2 \mid a_{21} = -\overline{a_{12}}\}$, onde $\overline{a + bi} = a - bi$. Prove que $W_1$ \'e um subespa\c{c}o de $\cp{M}_2(\complex)$ sobre $\real$ e ache uma base de $W_1$.
  \end{enumerate}
\end{exercicio}

\begin{exercicio}
  Seja $V$ um $\cp{K}$-espa\c{c}o vetorial e $\{v_1,\dots,v_n\} \sub V$. Mostre que $[v_1,\dots,v_n]$ \'e um $\cp{K}$-subespa\c{c}o vetorial de $V$.
\end{exercicio}

\begin{exercicio}
  Considere o subespa\c{c}o $S = [(1,1,-2,4),(1,1,-1,2),(1,4,-4,8)]$ de $\real^4$.
  \begin{enumerate}[label={\alph*})]
    \item O vetor $(2/3, 1, -1, 2)$ pertence a $S$?
    \item O vetor $(0, 0, 1, 1)$ pertence a $S$?
  \end{enumerate}
  \begin{solucao}
    \begin{enumerate}[label={\alph*})]
      \item N\~ao.
      \item N\~ao.
    \end{enumerate}
  \end{solucao}
\end{exercicio}

\begin{exercicio}
  Seja $W$ o subespa\c{c}o de $M_2(\real)$ definido por
  \[
    W = \left\{\begin{bmatrix}2a & a + 2b\\0 & a - b\end{bmatrix} \mid a, b \in \real\right\}.
  \]
  \begin{enumerate}[label={\alph*})]
      \item $\begin{bmatrix}0 & -2\\0 & 1\end{bmatrix} \in W$?
      \item $\begin{bmatrix}0 & 2\\3 & 1\end{bmatrix} \in W$?
    \end{enumerate}
  \begin{solucao}
  \begin{enumerate}[label={\alph*})]
      \item Sim.
      \item N\~ao.
    \end{enumerate}
  \end{solucao}
\end{exercicio}

\begin{exercicio}
  Seja $W$ um subespa\c{c}o vetorial de uma espa\c{c}o vetorial $V$ finitamente gerado sobre $\cp{K}$. Mostre que se $\dim_\cp{K}W = \dim_\cp{K}V$, ent\~ao $W = V$.
\end{exercicio}

\begin{exercicio}
  Mostre que 
    \[
      [1, 1 - x, (1 - x)^2, 1 - x^3] = \mathcal{P}_3(\real).
    \]
\end{exercicio}

\begin{exercicio}
  Seja $V$ o espa\c{c}o das matrizes $2 \times 2$ sobre $\real$, e seja $W$ o subespa\c{c}o gerado por
  \[
    \begin{bmatrix}
      1 & -5\\
      -4 & 2
    \end{bmatrix},
    \begin{bmatrix}
      1 & 1\\
      -1 & 5
    \end{bmatrix},
    \begin{bmatrix}
      2 & -4\\
      -5 & 7
    \end{bmatrix},
    \begin{bmatrix}
      1 & -7\\
      -5 & 1
    \end{bmatrix}.
  \]
Encontre uma base e a dimens\~ao de $W$.
\end{exercicio}

\begin{exercicio}
  Considere o subespa\c{c}o de $\real^4$ gerado pelos vetores $v_1 = (1, -1, 0, 0)$, $v_2 = (0, 0, 1, 1)$, $v_3 = (2, -2, 1, 1)$ e $v_4 = (1, 0, 0, 0)$.
  \begin{enumerate}[label={\alph*})]
    \item O vetor $(2, -3, 2, 2) \in [v_1, v_2, v_3, v_4]$? Justifique.
    \item Exiba uma base para $[v_1, v_2, v_3, v_4]$. Qual \'e a dimens\~ao?
    \item $[v_1, v_2, v_3, v_4] = \real^4$? Por qu\^e?
  \end{enumerate}
  \begin{solucao}
    \begin{enumerate}[label={\alph*})]
      \item N\~ao.
      \item $\dim_\real [v_1, v_2, v_3, v_4] = 3$.
      \item N\~ao.
    \end{enumerate}
  \end{solucao}
\end{exercicio}


\begin{exercicio}
  Considere o sistema linear
  \begin{equation}\label{sistema}
    \begin{cases}
      2x_1 + 4x_2 - 6x_3 = \alpha\\
      x_1 - x_1 + 4x_3 = \beta\\
      6x_2 - 14x_3 = \gamma
    \end{cases}.
  \end{equation}
  Seja $W = \{(x_1, x_2, x_3) \in \real^3 \mid (x_1, x_2, x_3) \mbox{\'e solu\c{c}\~ao de } \eqref{sistema}\}$.
  \begin{enumerate}[label={\alph*})]
    \item Que condi\c{c}\~oes devemos impor a $\alpha$, $\beta$ e $\gamma$ para que $W$ seja subespa\c{c}o vetorial de $\real^3$.
    \item Nas condi\c{c}\~oes determinadas em \textit{a)} encontre uma base para $W$.
    \item Que rela\c{c}\~ao existe entre a dimens\~ao de $W$ e o grau de liberdade do sistema?
  \end{enumerate}
\end{exercicio}

\begin{exercicio}
  Para quais valores de $\alpha \in \real$ vale
  \[
    [(1, 0, \alpha), (1, 2, -3) , (\alpha, 1, 0)] = \real^3?
  \]
\end{exercicio}

\begin{exercicio}
  Encontre uma base de $\real^4$ que contenha os vetores $(1,2,-2,1)$ e $(1,0,-2,2)$.
\end{exercicio}

\begin{exercicio}
  Considere o subespa\c{c}o vetorial $W$ de $\mathcal{P}_4(\real)$ gerado pelo conjunto
  \[
    \mathcal{A} = \{1+2x+x^2+3x^3+x^4, 1-2x-2x^2-2x^3-3x^4,2-x^2+x^3-2x^4,x-x^3+x^4,3x^2+6x^3+3x^4\}.
  \]
  Determine uma base de $\mathcal{B}$ de $W$ que esteja contida em $\mathcal{A}$.
\end{exercicio}

\begin{exercicio}
  Seja $S$ o subespa\c{c}o de $M_2(\real)$:
  \[
    S = \left\{\begin{bmatrix}
      a & b\\c & d
    \end{bmatrix} \mid c = a+b \mbox{ e } d = a\right\}.
  \]
  \begin{enumerate}[label={\alph*})]
    \item Qual a dimens\~ao de $S$?
    \item O conjunto
    \[
      \left\{\begin{bmatrix}
        1 & -1\\0 & 1
      \end{bmatrix}, \begin{bmatrix}
        2 & 1\\3 & 4
      \end{bmatrix}\right\}
    \]
    \'e uma base de $S$? Justifique.
  \end{enumerate}
  \begin{solucao}
    \begin{enumerate}[label={\alph*})]
      \item $\dim_\real S = 2$
      \item N\~ao.
    \end{enumerate}
  \end{solucao}
\end{exercicio}

\newpage
\Closesolutionfile{ans}
\hrule
\begin{center}
{\large\bf RESPOSTAS}
\end{center}
\hrule
\input{ans1}

\end{document}