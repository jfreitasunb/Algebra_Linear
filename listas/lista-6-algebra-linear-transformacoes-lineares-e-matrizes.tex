%!TEX program = xelatex
% !TEX encoding = ISO-8859-1
\def\ano{2019}
\def\semestre{1}

\documentclass[12pt]{exam}

\usepackage{caption}
\usepackage{amssymb}
\usepackage{amsmath,amsfonts,amsthm,amstext}
\usepackage[brazil]{babel}
% \usepackage[latin1]{inputenc}
\usepackage{graphicx}
\graphicspath{{/ArquivosUbuntu/Dropbox/imagens-latex/}{D:/Dropbox/imagens-latex/}}
\usepackage{enumitem}
\usepackage{multicol}
\usepackage{answers}
\usepackage{tikz,ifthen}
\usetikzlibrary{lindenmayersystems}
\usetikzlibrary[shadings]

\Newassociation{solucao}{Solution}{ans}
\newtheorem{exercicio}{}

\setlength{\topmargin}{-1.0in}
\setlength{\oddsidemargin}{0in}
\setlength{\textheight}{10.1in}
\setlength{\textwidth}{6.5in}
\setlength{\baselineskip}{12mm}

\extrafootheight[.25in]{.5in}
\footrule
\lfoot{Álgebra Linear \semestre$^o$/\ano}
\cfoot{}
\rfoot{Página \thepage\ de \numpages}

\newcounter{exercicios}
\setcounter{exercicios}{0}
\newcommand{\questao}{
\addtocounter{exercicios}{1}
\noindent{\bf Exerc{\'\i}cio \Roman{exercicios}: }}

\newcommand{\resp}[1]{
\noindent{\bf Exerc{\'\i}cio #1: }}

\newcommand{\im}{{\rm Im\,}}
\newcommand{\sub}{\subseteq}
\newcommand{\n}{\mathbb{N}}
\newcommand{\z}{\mathbb{Z}}
\newcommand{\rac}{\mathbb{Q}}
\newcommand{\real}{\mathbb{R}}
\newcommand{\complex}{\mathbb{C}}
\newcommand{\cp}[1]{\mathbb{#1}}
\newcommand{\ch}{\mbox{\textrm{car\,}}\nobreak}
\newcommand{\dlim}[2]{\displaystyle\lim_{#1\rightarrow #2}}
\newcommand{\minf}{+\infty}
\newcommand{\ninf}{-\infty}
\renewcommand{\sin}{{\rm sen\,}}
\renewcommand{\sinh}{{\rm senh\,}}
\renewcommand{\tan}{{\rm tg\,}}
\renewcommand{\csc}{{\rm cossec\,}}
\renewcommand{\cot}{{\rm cotg\,}}
\newcommand{\din}[4]{\displaystyle\int_{#1}^{#2}{#3}{d#4}}

\newcommand{\se}[1]{\displaystyle\sum_{n = 1}^\infty{#1}}
\newcommand{\slim}{\displaystyle\lim_{n \rightarrow \infty}}
\newcommand{\seq}[1]{\{{#1_n\}}}
\newcommand{\seg}[1]{\displaystyle\sum_{n = 1}^\infty{#1_n}}
\newcommand{\sei}[2]{\displaystyle\sum_{#1}^\infty{#2}}

\newcommand{\vesp}[1]{\vspace{ #1  cm}}

\newcommand{\compcent}[1]{\vcenter{\hbox{$#1\circ$}}}
\newcommand{\comp}{\mathbin{\mathchoice
{\compcent\scriptstyle}{\compcent\scriptstyle}
{\compcent\scriptscriptstyle}{\compcent\scriptscriptstyle}}}

\begin{document}
% \pagestyle{empty}
% \pagenumbering{arabic}

\Opensolutionfile{ans}[ans1]

\begin{figure}[h]
        \begin{minipage}[c]{1.7cm}
        \includegraphics[width=1.7cm]{unb.pdf}
        \end{minipage}%
        \hspace{0pt}
        \begin{minipage}[c]{4in}
          {Universidade de Bras{\'\i}lia} \\
          {Departamento de Matem{\'a}tica}
\end{minipage}
\end{figure}

\vesp{-0.35} \hrule

\begin{center}
{\Large\bf \'Algebra Linear - Turma A -- \semestre$^{o}$/\ano} \\ \vspace{9pt} {\large\bf
  $6^{\underline{a}}$ Lista de Exerc{\'\i}cios -- Transforma\c{c}\~oes Lineares e Matrizes}\\ \vspace{9pt} Prof. Jos{\'e} Ant{\^o}nio O. Freitas
\end{center}
\hrule

\vesp{.6}

\begin{exercicio}
  Sejam $\mathcal{B} = \{(1,0);(0,1)\}$, $\mathcal{B}_1 = \{(-1,1);(1,1)\}$, $\mathcal{B}_2 = \{(\sqrt{3},-1);(\sqrt{3},1)\}$ e $\mathcal{B}_3 = \{(2,0);(0,2)\}$ bases ordenadas de $\real^2$.
  \begin{enumerate}[label=({\alph*})]
    \item Quais s\~ao as coordenadas do vetor $v = (3,-2)$ em rela\c{c}\~ao \`a base:
    \begin{enumerate}[label=({\roman*})]
      \item $\mathcal{B}$
      \item $\mathcal{B}_1$
      \item $\mathcal{B}_2$
      \item $\mathcal{B}_3$
    \end{enumerate}
    \item Ache a matriz de mudan\c{c}a de base nos seguintes casos:
    \begin{enumerate}[label=({\roman*})]
      \item $[I]_{\mathcal{B},\mathcal{B}_1}$
      \item $[I]_{\mathcal{B}_1,\mathcal{B}}$
      \item $[I]_{\mathcal{B}_2,\mathcal{B}}$
      \item $[I]_{\mathcal{B}_3,\mathcal{B}}$
    \end{enumerate}
    \item As coordenadas de um vetor $v \in \real^2$ em rela\c{c}\~ao \`a base $\mathcal{B}_1$ s\~ao dadas por
    \[
      [v]_{\mathcal{B}_1} = \begin{bmatrix}
        4\\0
      \end{bmatrix}.
    \]
    Quais as coordenadas de $v$ em rela\c{c}\~ao \`a base:
    \begin{enumerate}[label=({\roman*})]
      \item $\mathcal{B}$
      \item $\mathcal{B}_2$
      \item $\mathcal{B}_3$
    \end{enumerate}
  \end{enumerate}
\end{exercicio}

\begin{exercicio}
  Se
  \[
    [I]_{\mathcal{B}_1,\mathcal{B}_2} = \begin{bmatrix}
      1 & \phantom{-}1 & 0\\
      0 & -1 & 1\\
      1 & \phantom{-}0 & 1
    \end{bmatrix}
  \]
  encontre:
  \begin{enumerate}[label=({\alph*})]
    \item $[v]_{\mathcal{B}_1}$ onde $[v]_{\mathcal{B}_2} = \begin{bmatrix}
      -1\\\phantom{-}2\\\phantom{-}3
    \end{bmatrix}$
    \item $[v]_{\mathcal{B}_2}$ onde $[v]_{\mathcal{B}_1} = \begin{bmatrix}
      -1\\\phantom{-}2\\\phantom{-}3
    \end{bmatrix}$
  \end{enumerate}
\end{exercicio}


\begin{exercicio}
  Encontre a matriz da transformação linear dada:
  \begin{enumerate}[label=({\alph*})]
    \item $F : \mathcal{P}_2(\real) \to \mathcal{P}_2(\real)$ dada por $F(p(t)) = t^2p''(t)$. Considere $\mathcal{P}_2(\real)$ um $\real$-espa\c{c}o vetorial e $\mathcal{B} = \{u_1 = 1, u_2 = x, u_3 = x^2\}$.
    
    \item $D : \mathcal{P}_2(\complex) \to \mathcal{P}_1(\complex)$ dada por
    \[
      D(a_0 + a_1x + a_2x^2) = a_1 + 2a_2x.
    \]
    Considere $\mathcal{P}_2(\complex)$ e $\mathcal{P}_1(\complex)$ sendo espa\c{c}os vetoriais sobre $\complex$ com bases $\mathcal{B}_1 = \{1, x, x^2\}$ e $\mathcal{B}_2 = \{1, 2 - x\}$, respectivamente.
    
    \item $D : \mathcal{P}_3(\real) \to \mathcal{P}_5(\real)$ dada por
    \[
      D(a_0 + a_1x + a_2x^2 + a_3x^3) = a_1 + 2a_2x + 3a_3x^2
    \]
    Considere $\mathcal{P}_3(\real)$ e $\mathcal{P}_5(\real)$ sendo espa\c{c}os vetoriais sobre $\real$ com bases $\mathcal{B}_1 = \{1, x, x^2, x^3\}$ e $\mathcal{B}_2 = \{1, x, x^2, x^3, x^4, x^5\}$, respectivamente..

    \item $G : \cp{M}_2(\real) \to \cp{M}_2(\real)$ dada por $G(X) = MX + X$ onde
    \[
      M = \begin{pmatrix}
        1 & 0\\
        0 & 0
      \end{pmatrix}.
    \]
    Considere $\cp{M}_2(\real)$ um $\real$-espa\c{c}o vetorial.
    \item $H : \cp{M}_2(\real) \to \cp{M}_2(\real)$ dada por $H(x) = MX - XM$ onde
    \[
      M = \begin{pmatrix}
        1 & 2\\
        0 & 1
      \end{pmatrix}.
    \]
    Considere $\cp{M}_2(\real)$ um $\real$-espa\c{c}o vetorial.
    \item $T : \mathcal{P}_3(\real) \to \real^2$ dada por
    \[
      T(p) = \left(\int_{-1}^0p(x)dx, \int_0^1p(x)dx\right).
    \]
    Considere $\mathcal{P}_3(\real)$ e $\real^2$ como $\real$-espa\c{c}os vetoriais.
    \item $T : \complex^3 \to \complex$ dada por $T(x,y,z) = x + 2y + iz$. Considere $\complex^3$ e $\complex^2$ como $\complex$-espa\c{c}os vetoriais.
    \item $F : \mathcal{P}_3(\real) \to \mathcal{P}_4(\real)$ dada por $(Fp)(x) = xp(x + 1)$. Considere $\mathcal{P}_3(\real)$ e $\mathcal{P}_4(\real)$ como $\real$-espa\c{c}os vetoriais.
    \item $G : \real^2 \to \mathcal{P}_2(\real)$ dada por $T(a,b) = ax^2 + bx + (a + b)$. Considere $\real^2$ e $\mathcal{P}_2(\real)$ como $\real$-espa\c{c}os vetoriais.

    \item Considere a fun\c{c}\~ao $T : \complex \to \cp{M}_2(\real)$ dada por
  \[
    T(x + yi) = \begin{pmatrix}
      x + 7y & 5y\\
      -10y & x - 7y
    \end{pmatrix}.
  \]
  Considere $\cp{M}_2(\real)$ e $\complex$ como espa\c{c}os vetoriais sobre $\real$.
   \end{enumerate}
   \begin{solucao}
     \begin{enumerate}[label=({\alph*})]
       \item $[F]_\mathcal{B} = \begin{bmatrix}
         0 & 0 & 0\\
         0 & 0 & 0\\
         0 & 0 & 2
       \end{bmatrix}$

       \item $[D]_{\mathcal{B}_1, \mathcal{B}_2} = \begin{bmatrix}
         0 & 1 & 2\\
         0 & 0 & -2
       \end{bmatrix}$

       \item $[D]_{\mathcal{B}_1, \mathcal{B}_2} = \begin{bmatrix}
         0 & 1 & 0 & 0\\
         0 & 0 & 2 & 0\\
         0 & 0 & 0 & 3\\
         0 & 0 & 0 & 0\\
         0 & 0 & 0 & 0\\
         0 & 0 & 0 & 0
       \end{bmatrix}$
     \end{enumerate}
   \end{solucao}
\end{exercicio}

\begin{exercicio}
  Sejam $T : V \to W$ e $S : W \to U$ transforma\c{c}\~oes lineares. Mostre que $S \comp T : V \to U$ \'e uma transforma\c{c}\~ao linear.
\end{exercicio}


\begin{exercicio}\label{nucleo_imagem_inicio}
  Seja $F : \real^4 \to \real^3$ a transforma\c{c}\~ao linear definida por
  \[
    F(x,y,s,t) = (x - y + s + t, x + 2s - t, x + y + 3s - 3t).
  \]
  \begin{solucao}
    \begin{enumerate}[label=({\alph*})]
      \item $\dim_\real\im F = 2$
      \item $\dim_\real\ker F = 2$
    \end{enumerate}
  \end{solucao}
\end{exercicio}

\begin{exercicio}
Seja $F : \mathcal{P}_3(\real) \to \mathcal{P}_4(\real)$ dada por $(Fp)(x) = xp(x + 1)$. Considere $\mathcal{P}_3(\real)$ e $\mathcal{P}_4(\real)$ como $\real$-espa\c{c}os vetoriais.
\end{exercicio}

\begin{exercicio}
  Seja $G : \real^2 \to \mathcal{P}_2(\real)$ dada por $T(a,b) = ax^2 + bx + (a + b)$. Considere $\real^2$ e $\mathcal{P}_2(\real)$ como $\real$-espa\c{c}os vetoriais.
\end{exercicio}


\begin{exercicio}
  Seja $F : \real^3 \to \real^3$ a transforma\c{c}\~ao linear definida por
  \[
    F(x,y,z) = (x + 2y - z, y + z, x + y - 2z).
  \]
  \begin{solucao}
    \begin{enumerate}[label=({\alph*})]
      \item $\dim_\real\im F = 2$
      \item $\dim_\real\ker F = 1$
    \end{enumerate}
  \end{solucao}
\end{exercicio}

\begin{exercicio}
  Seja $F : \cp{M}_3(\real) \to \cp{M}_3(\real)$ a transforma\c{c}\~ao linear definida por
  \[
    F(A) = AM - MA,
  \]
  onde $M = \begin{bmatrix}
    1 & \phantom{-}2 & 0\\0 & \phantom{-}3 & 1\\0 & -1 & 1
  \end{bmatrix}$ e $\cp{M}_3(\real)$ \'e um $\real$-espa\c{c}o vetorial.
  \begin{solucao}
    \begin{enumerate}[label=({\alph*})]
      \item $\dim_\real\im F = 2$
      \item $\dim_\real\ker F = 2$
    \end{enumerate}
  \end{solucao}
\end{exercicio}

\begin{exercicio}
  Seja $T : \real^2 \to \real^2$ uma transforma\c{c}\~ao linear definida por
  \[
    T(x,y) = (-y,x).
  \]
  \begin{enumerate}[label=({\alph*})]
    \item Qual \'e a matriz de $T$ em rela\c{c}\~ao \`a base ordenada can\^onica de $\real^2$?
    \item Qual \'e a matriz de $T$ em rela\c{c}\~ao \`a base ordenada $\mathcal{B}_1 = \{w_1 = (1,2); w_2 = (1,-1)\}$?
    \item Exiba a matriz $P$ tal que $[T]_{\mathcal{B}} = P^{-1}[T]_{\mathcal{B}_1}P$.
  \end{enumerate}
\end{exercicio}

\begin{exercicio}
  Seja $T : V \to W$ um isomorfismo, onde $V$ e $W$ s\~ao $\cp{K}$-espa\c{c}os vetoriais. Seja $G : W \to V$ definida por: $G(w) = v$ se, e somente se, $T(v) = w$, para todo $w \in V$. Mostre que:
  \begin{enumerate}[label=({\alph*})]
    \item $G$ \'e uma transforma\c{c}\~ao linear.
    \item $T\circ G = Id_W$, onde $Id_V : W \to W$ tal que $Id_W(x) = x$, para todo $x \in W$.
    \item $G\circ T = Id_V$, onde $Id_V : V \to V$ tal que $Id_V(y) = y$, para todo $y \in V$.
  \end{enumerate}
  A transforma\c{c}\~ao linear $G$ \'e chamada de \textbf{inversa} de $T$ e ser\'a denotada por $G = T^{-1}$.
\end{exercicio}

\begin{exercicio}
  Seja $T : \real^3 \to \real^3$ uma transforma\c{c}\~ao linear cuja matriz com rela\c{c}\~ao \`a base can\^onica seja
  \[
    \begin{pmatrix}
      \phantom{-}1 & \phantom{-}1 & \phantom{-}0\\
      -1 & \phantom{-}0 & \phantom{-}1\\
      \phantom{-}0 & -1 & -1
    \end{pmatrix}.
  \]
    \begin{enumerate}[label=({\alph*})]
      \item Determine $T(x,y,z)$.
      \item Qual \'e a matriz de $T$ com rela\c{c}\~ao \`a base $\mathcal{B} = \{(-1,1,0);(1,-1,1);(0,1,-1)\}$?
      \item A transforma\c{c}\~ao $T$ \'e invert{\'\i}vel? Justifique.
    \end{enumerate}
\end{exercicio}

\begin{exercicio}
  Sejam $V$ e $W$ dois espa\c{c}os vetoriais sobre $\cp{K}$ tais que $\dim_\cp{K}V = \dim_\cp{K}W = n \ge 1$ e considere $\mathcal{B}_1$ e $\mathcal{B}_2$ bases ordenadas de $V$ e $W$, respectivamente. Mostre que:
  \begin{enumerate}[label=({\alph*})]
    \item Se uma transforma\c{c}\~ao linear $T : V \to W$ \'e um isomorfismo, ent\~ao a matriz $[T]_{\mathcal{B}_1,\mathcal{B}_2}$ \'e invert{\'\i}vel.
    \item Se uma transforma\c{c}\~ao linear $T : V \to W$ tal que a matriz $[T]_{\mathcal{B}_1,\mathcal{B}_2}$ \'e invert{\'\i}vel , ent\~ao $T$ \'e um isomorfismo.
    \item Mostre que se $T$ \'e um isomorfismo, ent\~ao
    \[
      \left([T]_{\mathcal{B}_1,\mathcal{B}_2}\right)^{-1} = [T^{-1}]_{\mathcal{B}_1,\mathcal{B}_2}.
    \]
  \end{enumerate}
\end{exercicio}

\begin{exercicio}
  Mostre que cada uma das transforma\c{c}\~oes lineares de $\real^3$ em $\real^3$ a seguir \'e invert{\'\i}vel e determine a transforma\c{c}\~ao linear inversa:
  \begin{enumerate}[label=({\alph*})]
    \item $T(x,y,z) = (x - 3y - 2z, y - 4z, -z)$
    \item $T(x,y,z) = (x, x - y, 2x + y -z)$
  \end{enumerate}
\end{exercicio}

\begin{exercicio}
  Seja $\cp{K}$ um corpo e $T : \cp{K}^2 \to \cp{K}^2$ a transforma\c{c}\~ao linear dada por $T(x_1,x_2) = (x_1 + x_2, x_1)$ para todo $(x_1,x_2) \in \cp{K}^2$. Prove que $T$ \'e um isomorfismo e exiba $T^{-1}$.
\end{exercicio}

\begin{exercicio}
  Seja $T : \complex^3 \to \complex^3$ a transforma\c{c}\~ao linear definida por $T(1,0,0) = (1,0,i)$, $T(0,1,0) = (0,1,1)$ e $T(0,0,1) = (i,1,0)$. Decida se $T$ \'e invert{\'\i}vel.
\end{exercicio}

\begin{exercicio}
  Seja $T : \real^3 \to \real^2$ e $S : \real^2 \to \real^3$ transforma\c{c}\~oes lineares.
  \begin{enumerate}[label=({\alph*})]
    \item Prove que $S \circ T$ n\~ao \'e invert{\'\i}vel.
    \item Achar um exemplo em que $T\circ S$ n\~ao \'e invert{\'\i}vel e um outro em que $T\circ S$ \'e invert{\'\i}vel.
  \end{enumerate}
\end{exercicio}

\begin{exercicio}
  Seja $V$ um $\cp{K}$-espa\c{c}o vetorial de dimens\~ao 2 e seja $\mathcal{B}$ uma base ordenada de $V$. Se $T: V \to V$ \'e uma transforma\c{c}\~ao linear tal que
  \[
    [T]_\mathcal{B} = \begin{bmatrix}
      a & b\\
      c & d
    \end{bmatrix},
  \]
  mostre que $[T]_\mathcal{B}^2 - (a + d)[T]_\mathcal{B} + (ad - bc)I_2 = 0$.
\end{exercicio}

\begin{exercicio}
  Seja $T : \real^3 \to \real^3$ uma transforma\c{c}\~ao linear definida por
  \[
    T(x,y,z) = (3x + z,-2x + y,-x+2y + 4z).
  \]
  \begin{enumerate}[label=({\alph*})]
    \item Qual \'e a matriz de $T$ em rela\c{c}\~ao \`a base ordenada can\^onica de $\real^3$?
    \item Qual \'e a matriz de $T$ em rela\c{c}\~ao \`a base ordenada $\mathcal{B}_1 = \{w_1 = (1,0,1); w_2 = (-1,2,1); w_3 = (2,1,1)\}$?
    \item Exiba a matriz $P$ tal que $[T]_{\mathcal{B}} = P^{-1}[T]_{\mathcal{B}_1}P$.
    \item Mostrar que $T$ \'e invert{\'\i}vel e achar uma express\~ao para $T^{-1}$.
  \end{enumerate}
\end{exercicio}

\begin{exercicio}
  Sejam $T, S : V \to W$ duas transforma\c{c}\~oes lineares, onde $V$ e $W$ s\~ao $\cp{K}$-espa\c{c}os vetoriais de dimens\~ao finita. Sejam $\mathcal{B}$ e $\mathcal{C}$ bases de $V$ e $W$, respectivamente e $\lambda \in \cp{K}$. Mostre que:
  \begin{enumerate}[label=({\alph*})]
    \item $[T + S]_{\mathcal{B}, \mathcal{C}} = [T]_{\mathcal{B}, \mathcal{C}} + [S]_{\mathcal{B}, \mathcal{C}}$.
    \item $[\lambda T]_{\mathcal{B}, \mathcal{C}} = \lambda [T]_{\mathcal{B}, \mathcal{C}}$.
  \end{enumerate}
\end{exercicio}

\begin{exercicio}
  Seja $T : \real^3 \to \real^3$ uma transforma\c{c}\~ao linear tal que em rela\c{c}\~ao \`a base can\^onica $\mathcal{B} = \{(1,0,0); (0,1,0); (0,0,1)\}$:
  \[
    [T]_\mathcal{B} =\begin{bmatrix}
      \phantom{-}1 & 2 & 1\\
      \phantom{-} 0 & 1 & 1\\
      -1 & 3 & 4
    \end{bmatrix}.
  \]
  Ache uma base de $\im T$ e uma base de $\ker T$.
\end{exercicio}

\begin{exercicio}
  Seja $T : \cp{M}_2(\complex) \to \cp{M}_2(\complex)$ uma transforma\c{c}\~ao linear dada por
  \[
    T \begin{pmatrix}
      x & y\\
      z & w
    \end{pmatrix} = \begin{pmatrix}
      0 & x\\
      z - w & 0
    \end{pmatrix},
  \]
  onde $\cp{M}_2(\complex)$ \'e um $\complex$-espa\c{c}o vetorial
  e
  \[
      \mathcal{B}_1 = \left\{u_1 = \begin{bmatrix}
        1 & 0\\0 & 0
      \end{bmatrix}, u_2 = \begin{bmatrix}
        0 & 1\\0 & 0
      \end{bmatrix}, u_3 = \begin{bmatrix}
        0 & 0\\1 & 0
      \end{bmatrix}, u_4 = \begin{bmatrix}
        0 & 0\\0 & 1
      \end{bmatrix}\right\}
  \]
    \begin{enumerate}[label=({\alph*})]
      \item Determine a matriz de $T$ com rela\c{c}\~ao \`a base can\^onica $\mathcal{B}_1$.
      \item Determine a matriz de $T$ com rela\c{c}\~ao \`a base
      \[
        \mathcal{B}_2 = \left\{\begin{pmatrix}
          1 & 0\\
          0 & 1
        \end{pmatrix}; \begin{pmatrix}
          0 & 1\\
          1 & 0
        \end{pmatrix}; \begin{pmatrix}
          1 & 0\\
          1 & 1
        \end{pmatrix}; \begin{pmatrix}
          0 & 1\\
          0 & 1
        \end{pmatrix}\right\}
      \]
      de $\cp{M}_2(\complex)$.
      \item Exiba a matriz $P$ tal que $[T]_{\mathcal{B}_2} = P^{-1}[T]_{\mathcal{B}_1}P$.
    \end{enumerate}
\end{exercicio}

\newpage
\Closesolutionfile{ans}
\hrule
\begin{center}
{\large\bf RESPOSTAS}
\end{center}
\hrule
\input{ans1}

\end{document}