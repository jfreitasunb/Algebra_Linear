% !TEX encoding = ISO-8859-1
\documentclass[12pt]{article}

\usepackage{amssymb}
\usepackage{amsmath,amsfonts,amsthm,amstext}
\usepackage{graphicx}
\usepackage{enumitem}
\usepackage{multicol}
\usepackage[all]{xy}

\setlength{\topmargin}{-1.0in}
\setlength{\oddsidemargin}{0in}
\setlength{\textheight}{10.1in}
\setlength{\textwidth}{6.5in}
\setlength{\baselineskip}{12mm}

\newcounter{exercicios}
\setcounter{exercicios}{0}
\newcommand{\questao}{
\addtocounter{exercicios}{15}
\noindent{\bf Exerc{\'\i}cio \arabic{exercicios}: }}

\newcommand{\equi}{\Leftrightarrow}
\newcommand{\bic}{\leftrightarrow}
\newcommand{\cond}{\rightarrow}
\newcommand{\impl}{\Rightarrow}
\newcommand{\nao}{\sim}
\newcommand{\sub}{\subseteq}
\newcommand{\e}{\ \wedge\ }
\newcommand{\ou}{\ \vee\ }
\newcommand{\vaz}{\emptyset}

\newcommand{\cp}[1]{\mathbb{#1}}
\newcommand{\real}{\mathbb{R}}
\newcommand{\vesp}{\vspace{0.2cm}}
\newcommand{\z}{\mathbb{Z}}
\newcommand{\n}{\mathbb{N}}
\newcommand{\q}{\mathbb{Q}}
\newtheorem{defin}{Defini{\c c}{\~a}o}

\newcommand{\compcent}[1]{\vcenter{\hbox{$#1\circ$}}}
\newcommand{\comp}{\mathbin{\mathchoice
{\compcent\scriptstyle}{\compcent\scriptstyle}
{\compcent\scriptscriptstyle}{\compcent\scriptscriptstyle}}}

\begin{document}

Seja $\cp{K}$ um corpo. Dados $a$, $b$, $c$, $d$, $e$, $f \in \cp{K}$, mostre que as duas matrizes seguintes s\~ao linha-equivalentes, se supormos que $ad - bc \ne 0_{\cp{K}}$:
\[
    \left[\begin{array}{cc|c}
      a & b & e\\
      c & d & f
    \end{array}\right] \sim \left[\begin{array}{cc|c}
      1_{\cp{K}} & 0_{\cp{K}} & (de - bf)(ad - bc)^{-1}\\
      0_{\cp{K}} & 1_{\cp{K}} & (af - ce)(ad - bc)^{-1}
    \end{array}\right]
\]
onde $(ad - bc)^{-1}$ \'e o inverso multiplicativo de $ad - bc$ no corpo $\cp{K}$.

\textbf{Solu{\c c}{\~a}o}: Como $ad - bc \ne 0_{\cp{K}}$ devemos ter $a \ne 0_\cp{K}$ e $d \ne 0_\cp{K}$ ou $b \ne 0_\cp{K}$ e $c \ne 0_\cp{K}$. Sendo assim vamos supor que $a \ne 0_\cp{K}$ e $d \ne 0_\cp{K}$. Vamos ent\~ao aplicar opera\c{c}\~oes elementares na matriz
\[
    \left[\begin{array}{cc|c}
      a & b & e\\
      c & d & f
    \end{array}\right].
\]
Antes note que se $d - ca^{-1}b = 0_\cp{K}$, ent\~ao multiplicando essa equa\c{c}\~ao por $a$ obter{\'\i}amos $ad - bc = 0_\cp{K}$. Logo $d - ca^{-1}b \ne 0_\cp{K}$
\begin{align*}
    \left[
        \begin{array}{cc|c}
            a & b & e\\
            c & d & f
        \end{array}
    \right]
    \begin{array}{l}
        L_1 \to a^{-1}L_1\\
        \phantom{x}
    \end{array} &\sim
    \left[
        \begin{array}{cc|c}
            1_\cp{K} & a^{-1}b & a^{-1}e\\
            c & d & f
        \end{array}
    \right]
    \begin{array}{l}
        \phantom{x}\\
        L_2 \to L_2 - cL_1
    \end{array}\\ &\sim
    \left[
        \begin{array}{cc|c}
            1_\cp{K} & a^{-1}b & a^{-1}e\\
            0_\cp{K} & d - ca^{-1}b & f - ca^{-1}e
        \end{array}
    \right]
    \begin{array}{l}
        \phantom{x}\\
        L_2 \to (d - ca^{-1}b)^{-1}L_2
    \end{array}\\ &\sim
    \left[
        \begin{array}{cc|c}
            1_\cp{K} & a^{-1}b & a^{-1}e\\
            0_\cp{K} & 1_\cp{K} & (d - ca^{-1}b)^{-1}(f - ca^{-1}e)
        \end{array}
    \right]
    \begin{array}{l}
        L_1 \to L_1 - a^{-1}bL_2\\
        \phantom{x}
    \end{array}\\ &\sim
    \left[
        \begin{array}{cc|c}
            1_\cp{K} & 0_\cp{K} & a^{-1}e - a^{-1}b(d - ca^{-1}b)^{-1}(f - ca^{-1}e)\\
            0_\cp{K} & 1_\cp{K} & (d - ca^{-1}b)^{-1}(f - ca^{-1}e)
        \end{array}
    \right]
\end{align*}
Vamos agora simplificar os termos na \'ultima coluna:
\begin{align*}
    a^{-1}e - a^{-1}b(d - ca^{-1}b)^{-1}(f - ca^{-1}e) &= a^{-1}e - ba^{-1}(d - ca^{-1}b)^{-1}(f - ca^{-1}e)\\ &= a^{-1}e - b[a(d - ca^{-1}b)]^{-1}(f - ca^{-1}e)\\ &= a^{-1}e - b(ad - aca^{-1}b)^{-1}(f - ca^{-1}e) \\ &= a^{-1}e - b(ad - bc)^{-1}(f - ca^{-1}e)\\ &= a^{-1}e(ad - bc)(ad - bc)^{-1} - b(ad - bc)^{-1}(f - ca^{-1}e)\\ &= (ad - bc)^{-1}[a^{-1}e(ad - bc) - b(f - ca^{-1}e)]\\ &= (ad - bc)^{-1}[a^{-1}ead - a^{-1}ebc - bf + bca^{-1}e]\\ & = (de - bf)(ad - bc)^{-1}
\end{align*}

\begin{align*}
    (d - ca^{-1}b)^{-1}(f - ca^{-1}e) &= (a^{-1}ad - a^{-1}cd)^{-1}(a^{-1}af - a^{-1}ce) = [a^{-1}(ad - bc)]^{-1}a^{-1}(af - ce)\\ &= (ad - bc)^{-1}(a^{-1})^{-1}a^{-1}(af - ce)\\ &= (ad - bc)^{-1}aa^{-1}(af - ce)\\ &= (af - ce)(ad - bc)^{-1}
\end{align*}
Assim a \'ultima matriz obtida torna-se
\[
    \left[
        \begin{array}{cc|c}
            1_\cp{K} & 0_\cp{K} & (de - bf)(ad - bc)^{-1}\\
            0_\cp{K} & 1_\cp{K} & (af - ce)(ad - bc)^{-1}
        \end{array}
    \right]
\]
portanto
\[
    \left[\begin{array}{cc|c}
      a & b & e\\
      c & d & f
    \end{array}\right] \sim \left[\begin{array}{cc|c}
      1_{\cp{K}} & 0_{\cp{K}} & (de - bf)(ad - bc)^{-1}\\
      0_{\cp{K}} & 1_{\cp{K}} & (af - ce)(ad - bc)^{-1}
    \end{array}\right]
\]
como quer{\'\i}amos.
\end{document}