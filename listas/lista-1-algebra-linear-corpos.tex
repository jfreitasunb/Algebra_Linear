%!TEX program = xelatex 
%!TEX encoding = ISO-8859-1
\documentclass[12pt]{exam}

\usepackage{caption}
\usepackage{amssymb}
\usepackage{amsmath,amsfonts,amsthm,amstext}
\usepackage[brazil]{babel}
% \usepackage[latin1]{inputenc}
%\usepackage[pdftex]{graphicx}
\usepackage{graphicx}
\graphicspath{{/ArquivosUbuntu/Dropbox/imagens-latex/}{D:/Dropbox/imagens-latex/}}
\usepackage{enumitem}
\usepackage{multicol}
 \usepackage{answers}
\usepackage{tikz,ifthen}
\usetikzlibrary{lindenmayersystems}
\usetikzlibrary[shadings]

\Newassociation{sol}{Solution}{ans}
\newtheorem{exercicio}{}

\setlength{\topmargin}{-1.0in}
\setlength{\oddsidemargin}{0in}
\setlength{\textheight}{10.1in}
\setlength{\textwidth}{6.5in}
\setlength{\baselineskip}{12mm}

\newcounter{exercicios}
\setcounter{exercicios}{0}
\newcommand{\questao}{
\addtocounter{exercicios}{1}
\noindent{\bf Exerc{\'\i}cio \Roman{exercicios}: }}

\newcommand{\resp}[1]{
\noindent{\bf Exerc{\'\i}cio #1: }}

\newcommand{\sub}{\subseteq}
\newcommand{\n}{\mathbb{N}}
\newcommand{\z}{\mathbb{Z}}
\newcommand{\rac}{\mathbb{Q}}
\newcommand{\real}{\mathbb{R}}
\newcommand{\complex}{\mathbb{C}}
\newcommand{\cp}[1]{\mathbb{#1}}
\newcommand{\ch}{\mbox{\textrm{car\,}}\nobreak}
\newcommand{\dlim}[2]{\displaystyle\lim_{#1\rightarrow #2}}
\newcommand{\minf}{+\infty}
\newcommand{\ninf}{-\infty}
\renewcommand{\sin}{{\rm sen\,}}
\renewcommand{\sinh}{{\rm senh\,}}
\renewcommand{\tan}{{\rm tg\,}}
\renewcommand{\csc}{{\rm cossec\,}}
\renewcommand{\cot}{{\rm cotg\,}}
\newcommand{\din}[4]{\displaystyle\int_{#1}^{#2}{#3}{d#4}}

\newcommand{\se}[1]{\displaystyle\sum_{n = 1}^\infty{#1}}
\newcommand{\slim}{\displaystyle\lim_{n \rightarrow \infty}}
\newcommand{\seq}[1]{\{{#1_n\}}}
\newcommand{\seg}[1]{\displaystyle\sum_{n = 1}^\infty{#1_n}}
\newcommand{\sei}[2]{\displaystyle\sum_{#1}^\infty{#2}}

\newcommand{\vesp}[1]{\vspace{ #1  cm}}

\newcommand{\compcent}[1]{\vcenter{\hbox{$#1\circ$}}}
\newcommand{\comp}{\mathbin{\mathchoice
{\compcent\scriptstyle}{\compcent\scriptstyle}
{\compcent\scriptscriptstyle}{\compcent\scriptscriptstyle}}}

\begin{document}
\pagestyle{empty}

\Opensolutionfile{ans}[ans1]

\begin{figure}[h]
        \begin{minipage}[c]{1.7cm}
        \includegraphics[width=1.7cm]{unb.pdf}
        \end{minipage}%
        \hspace{0pt}
        \begin{minipage}[c]{4in}
          {Universidade de Bras{\'\i}lia} \\
          {Departamento de Matem{\'a}tica}
\end{minipage}
\end{figure}

\vesp{-0.35} \hrule

\begin{center}
{\Large\bf \'Algebra Linear - Turma A -- 2$^{o}$/2018} \\ \vspace{9pt} {\large\bf
  $1^{\underline{a}}$ Lista de Exerc{\'\i}cios -- Corpos}\\ \vspace{9pt} Prof. Jos{\'e} Ant{\^o}nio O. Freitas
\end{center}
\hrule

\vesp{.6}

\begin{exercicio}
Verifique se os seguintes conjuntos, com as opera\c{c}\~oes dadas, s\~ao corpos.
\begin{enumerate}[label={\alph*})]
    \item $A = \{ a + bi \mid a,\ b \in \rac\}$, onde $i$ \'e o n\'umero complexo imagin\'ario e as opera\c{c}\~oes s\~ao as mesmas opera\c{c}\~oes de $\complex$;
    \item $B = \{ a + b\sqrt{3} \mid a,\ b \in \rac\}$. Aqui dados $a + b\sqrt{3}$, $c + d\sqrt{3} \in B$ definimos
    \begin{align*}
      a + b\sqrt{3} &= c + d\sqrt{3} \mbox{ quando } a = c \mbox{ e } b = d,\\
      (a + b\sqrt{3}) \oplus (c + d\sqrt{3}) &= (a + c) + (b + d)\sqrt{3},\\
      (a + b\sqrt{3}) \otimes (c + d\sqrt{3}) &= (ac + 3bd) + (ad + bc)\sqrt{3}.
    \end{align*}
    \item $C = \{ a + b\sqrt[3]{2} \mid a,\ b \in \rac\}$. Aqui dados $a + b\sqrt[3]{2}$, $c + d\sqrt[3]{2} \in C$ definimos
    \begin{align*}
      a + b\sqrt[3]{2} &= c + d\sqrt[3]{2} \mbox{ quando } a = c \mbox{ e } b = d,\\
      (a + b\sqrt[3]{2}) \oplus (c + d\sqrt[3]{2}) &= (a + c) + (b + d)\sqrt[3]{2},\\
      (a + b\sqrt[3]{2}) \otimes (c + d\sqrt[3]{2}) &= (ac + bd\sqrt[3]{4}) + (ad + bc)\sqrt[3]{2}.
    \end{align*}
    \item $D = \{ a\sqrt{2} + b\sqrt{3} \mid a,\ b \in \rac\}$. Aqui dados $a\sqrt{2} + b\sqrt{3}$, $c\sqrt{2} + d\sqrt{3} \in D$ definimos
    \begin{align*}
      a\sqrt{2} + b\sqrt{3} &= c\sqrt{2} + d\sqrt{3} \mbox{ quando } a = c \mbox{ e } b = d,\\
      (a\sqrt{2} + b\sqrt{3}) \oplus (c\sqrt{2} + d\sqrt{3}) &= (a + c)\sqrt{2} + (b + d)\sqrt{3},\\
      (a\sqrt{2} + b\sqrt{3}) \otimes (c\sqrt{2} + d\sqrt{3}) &= (2ac + 3bd) + (ad + bc)\sqrt{6}.
    \end{align*}
    \item $E = \{ a + b\sqrt{2} \mid a,\ b \in \z\}$. Aqui dados $a + b\sqrt{3}$, $c + d\sqrt{3} \in E$ definimos
    \begin{align*}
      a + b\sqrt{3} &= c + d\sqrt{3} \mbox{ quando } a = c \mbox{ e } b = d,\\
      (a + b\sqrt{3}) \oplus (c + d\sqrt{3}) &= (a + c) + (b + d)\sqrt{3},\\
      (a + b\sqrt{3}) \otimes (c + d\sqrt{3}) &= (ac + 3bd) + (ad + bc)\sqrt{3}.
    \end{align*}
    \item $(\rac, \oplus, \odot)$ onde $a \oplus b = a + b - 1$ e $a \odot b = a + b - ab$.
  \end{enumerate}  
\end{exercicio}

\begin{exercicio}
Resolva as seguintes equa\c{c}\~oes:
\begin{enumerate}[label={\alph*})]
  \item $x \oplus \overline{2} = \overline{0}$ no conjunto $\z_4$;
  \item $x^2 \oplus \overline{2}x = \overline{2}$ no conjunto $\z_5$;
  \item $x^3 \oplus \overline{2}x = \overline{3}$ no conjunto $\z_7$.
\end{enumerate}
\end{exercicio}

\begin{exercicio}
  Resolva o sistema de equa\c{c}\~oes
  \[
    \begin{cases}
      x \oplus y = \overline{0}\\
      \overline{2}x \oplus y = \overline{2}
    \end{cases}
  \]
  no conjunto $\z_4$. Este sistema possui solu\c{c}\~ao em $\z_3$?
\end{exercicio}

\begin{exercicio}
  Resolva a equa\c{c}\~ao $x^2 \oplus \overline{2}x \oplus \overline{1} = \overline{0}$ nos conjuntos $\z_7$ e $\z_{11}$, caso ela tenha ra{\'\i}zes.
\end{exercicio}

\begin{exercicio}
  Para quais valores de $\overline{c}$ a equa\c{c}\~ao $\overline{2}\otimes x = \overline{c}$ tem solu\c{c}\~ao no conjunto $\z_5$.
\end{exercicio}

\begin{exercicio}
Seja $\cp{K}$ um corpo. Definimos a caracter{\'\i}stica \textbf{$\ch \cp{K}$} de $\cp{K}$ da seguinte maneira:
\begin{enumerate}[label=({\roman*})]
  \item se a soma $1 + 1 + \cdots + 1$ for sempre diferente de zero, ent\~ao \textbf{$\ch \cp{K} = 0$};
  \item se a soma $\underbrace{1 + 1 + \cdots + 1}_{m} = 0$, para algum $m \ge 2$, ent\~ao \textbf{$\ch \cp{K}$} \'e o menor n\'umero $m$ com esta propriedade.
\end{enumerate}
Com base nesta defini\c{c}\~ao:
\begin{enumerate}[label={\alph*})]
  \item Mostre que se $\ch \cp{K} = m \ne 0$, ent\~ao $m$ \'e um n\'umero primo.
  \item Exiba corpos com caracter{\'\i}sticas iguais a 0 e outros com caracter{\'\i}sticas distintas de 0.
\end{enumerate}
\end{exercicio}

\begin{exercicio}
  Seja $\cp{K}$ um corpo. Dizemos que um elemento $a \in \cp{K}$ \'e um \textbf{elemento quadrado} se existe um elemento $x \in \cp{K}$ tal que $x^2 = a$. Denotamos o conjunto formado por todos os elementos quadrados de um corpo $\cp{K}$ por $\cp{K}^2$. Mostre que:
  \begin{enumerate}[label={\alph*})]
    \item $0 \in \cp{K}^2$ e $1 \in \cp{K}^2$.
    \item Se $a$, $b \in \cp{K}^2$, ent\~ao $ab \in \cp{K}^2$.
    \item Se $a \in \cp{K}^2$ e $a \ne 0$, ent\~ao $a^{-1} \in \cp{K}^2$.
    \item \'E verdade que se $a$, $b \in \cp{K}^2$, ent\~ao $a + b \in \cp{K}^2$.
  \end{enumerate}
\end{exercicio}

\begin{exercicio}
  Ache todos os elementos quadrados de $\z_5$, $\z_7$ e $\z_{11}$.
\end{exercicio}

% \newpage
% \Closesolutionfile{ans}
% \hrule
% \begin{center}
% {\large\bf RESPOSTAS}
% \end{center}
% \hrule
% \begin{multicols}{2}
% \input{ans1}
% \end{multicols}

\end{document}