%!TEX program = xelatex 
%!TEX encoding = ISO-8859-1
\documentclass[12pt]{exam}

\usepackage{caption}
\usepackage{amssymb}
\usepackage{amsmath,amsfonts,amsthm,amstext}
\usepackage[brazil]{babel}
% \usepackage[latin1]{inputenc}
%\usepackage[pdftex]{graphicx}
\usepackage{graphicx}
\graphicspath{{/ArquivosUbuntu/Dropbox/imagens-latex/}{D:/Dropbox/imagens-latex/}}
\usepackage{enumitem}
\usepackage{multicol}
 \usepackage{answers}
\usepackage{tikz,ifthen}
\usetikzlibrary{lindenmayersystems}
\usetikzlibrary[shadings]

\Newassociation{solucao}{Solution}{ans}
\newtheorem{exercicio}{}

\setlength{\topmargin}{-1.0in}
\setlength{\oddsidemargin}{0in}
\setlength{\textheight}{10.1in}
\setlength{\textwidth}{6.5in}
\setlength{\baselineskip}{12mm}

\newcounter{exercicios}
\setcounter{exercicios}{0}
\newcommand{\questao}{
\addtocounter{exercicios}{1}
\noindent{\bf Exerc{\'\i}cio \Roman{exercicios}: }}

\newcommand{\resp}[1]{
\noindent{\bf Exerc{\'\i}cio #1: }}

\newcommand{\sub}{\subseteq}
\newcommand{\n}{\mathbb{N}}
\newcommand{\integer}{\mathbb{Z}}
\newcommand{\rac}{\mathbb{Q}}
\newcommand{\real}{\mathbb{R}}
\newcommand{\complex}{\mathbb{C}}
\newcommand{\cp}[1]{\mathbb{#1}}
\newcommand{\ch}{\mbox{\textrm{car\,}}\nobreak}
\newcommand{\dlim}[2]{\displaystyle\lim_{#1\rightarrow #2}}
\newcommand{\minf}{+\infty}
\newcommand{\ninf}{-\infty}
\renewcommand{\sin}{{\rm sen\,}}
\renewcommand{\sinh}{{\rm senh\,}}
\renewcommand{\tan}{{\rm tg\,}}
\renewcommand{\csc}{{\rm cossec\,}}
\renewcommand{\cot}{{\rm cotg\,}}
\newcommand{\din}[4]{\displaystyle\int_{#1}^{#2}{#3}{d#4}}

\newcommand{\se}[1]{\displaystyle\sum_{n = 1}^\infty{#1}}
\newcommand{\slim}{\displaystyle\lim_{n \rightarrow \infty}}
\newcommand{\seq}[1]{\{{#1_n\}}}
\newcommand{\seg}[1]{\displaystyle\sum_{n = 1}^\infty{#1_n}}
\newcommand{\sei}[2]{\displaystyle\sum_{#1}^\infty{#2}}

\newcommand{\vesp}[1]{\vspace{ #1  cm}}

\newcommand{\compcent}[1]{\vcenter{\hbox{$#1\circ$}}}
\newcommand{\comp}{\mathbin{\mathchoice
{\compcent\scriptstyle}{\compcent\scriptstyle}
{\compcent\scriptscriptstyle}{\compcent\scriptscriptstyle}}}

\begin{document}
\pagestyle{empty}

\Opensolutionfile{ans}[ans1]

\begin{figure}[h]
        \begin{minipage}[c]{1.7cm}
        \includegraphics[width=1.7cm]{unb.pdf}
        \end{minipage}%
        \hspace{0pt}
        \begin{minipage}[c]{4in}
          {Universidade de Bras{\'\i}lia} \\
          {Departamento de Matem{\'a}tica}
\end{minipage}
\end{figure}

\vesp{-0.35} \hrule

\begin{center}
{\Large\bf \'Algebra Linear - Turma A -- 2$^{o}$/2018} \\ \vspace{9pt} {\large\bf
  $2^{\underline{a}}$ Lista de Exerc{\'\i}cios -- Sistema Lineares}\\ \vspace{9pt} Prof. Jos{\'e} Ant{\^o}nio O. Freitas
\end{center}
\hrule

\vesp{.6}
\begin{exercicio}
  Descrever explicitamente todas as matrizes $2 \times 2$ linha-reduzidas \`a forma em escada.
\end{exercicio}

\begin{exercicio}
  Descrever explicitamente todas as matrizes $3 \times 3$ linha-reduzidas \`a forma em escada.
\end{exercicio}

\begin{exercicio}
  Seja
  \[
    A = \begin{bmatrix}
      x & y\\
      z & t
    \end{bmatrix}
  \]
  uma matriz $2 \times 2$ com elementos complexos. Suponhamos que $A$ seja linha-reduzida e tamb\'em que $x + y + z + t = 0$. Mostre que existem exatamente tr\^es destas matrizes.
\end{exercicio}

\begin{exercicio}
  Seja $\cp{K}$ um corpo. Dados $a$, $b$, $c$, $d$, $e$, $f \in \cp{K}$, mostre que as duas matrizes seguintes são linha-equivalentes, se supormos que $ad - bc \ne 0_{\cp{K}}$:
  \[
    \begin{pmatrix}
      a & b & e\\
      c & d & f
    \end{pmatrix} \sim \begin{pmatrix}
      1_{\cp{K}} & 0_{\cp{K}} & (de - bf)(ad - bc)^{-1}\\
      0_{\cp{K}} & 1_{\cp{K}} & (af - ce)(ad - bc)^{-1}
    \end{pmatrix}
  \]
  onde $(ad - bc)^{-1}$ é o inverso multiplicativo de $ad - bc$ no corpo $\cp{K}$.
\end{exercicio}

\begin{exercicio}
  Demonstrar que as duas matrizes seguintes \textbf{n\~ao} s\~ao linha-equivalentes:
  \[
    \begin{bmatrix}
      2 & 0 & 0\\
      a & -1 & 0\\
      b & c & 3
    \end{bmatrix} \qquad
    \begin{bmatrix}
      1 & 1 & 2\\
      -2 & 0 & -1\\
      1 & 3 & 5
    \end{bmatrix}.
  \]
\end{exercicio}

\begin{exercicio}
  Seja
  \[
    A = \begin{bmatrix}
      3 & -1 & 2\\
      2 & 1 & 1\\
      1 & -3 & 0
    \end{bmatrix}.
  \]
  Para que ternas $(y_1, y_2, y_3)$ o sistema $AX = Y$ admite solu\c{c}\~ao? Considere $y_i \in \real$, $i=1$, 2, 3.
\end{exercicio}

Nos exerc{\'\i}cios \ref{sistemalinearinicio} \`a \ref{sistemalinearfim}, encontre a solu\c{c}\~ao geral dos seguintes sistemas lineares. Encontre o posto e a nulidade do sistema.
\begin{exercicio}\label{sistemalinearinicio}
$\begin{cases}
  x + y + z = 4\\
  2x + 5y - 2z = 3\\
  x + 7y - 7z = 5
\end{cases}$ em $\rac$.
\begin{solucao}
  N\~ao existe solu\c{c}\~ao.
\end{solucao}
\end{exercicio}

\begin{exercicio}
$\begin{cases}
  x - 2y + 3z = 0\\
  2x + 5y + 6z = 0
\end{cases}$ em $\real$.
\begin{solucao}
  $p = 2$, Nulidade = 1, $S = \{(x, y, z) \mid z, y, z \in \real\} = \{(-3z, 0, z) \mid z \in \real\}$
\end{solucao}
\end{exercicio}


\begin{exercicio}
$\begin{cases}
  x_1 + x_2 + 2x_3 = 8\\
  -x_1 - 2x_2 + 3x_3 = 1\\
  3x_1 - 7x_2 + 4x_3 = 10
\end{cases}$ em $\real$.
\begin{solucao}
  $p = 3$, Nulidade = 0, $S = \{(3, 1, 2)\}$
\end{solucao}
\end{exercicio}

\begin{exercicio}
$\begin{cases}
  2x_1 + 2x_2 + 2x_3 = 0\\
  -2x_1 + 5x_2 + 2x_3 = 1\\
  8x_1 + x_2 + 4x_3 = -1
\end{cases}$ em $\real$.
\begin{solucao}
  $p = 2$, Nulidade = 1, $S = \{(x_1, x_2, x_3) \mid x_1, x_2, x_3 \in \real\} = \{(-1/7 - (3/7)x_3, 1/7 - (4/7)x_3, x_3) \mid x_3 \in \real\}$
\end{solucao}
\end{exercicio}

\begin{exercicio}
$\begin{cases}
  \phantom{2x_1} - 2x_2 + 3x_3 = 1\\
  3x_1 + 6x_2 - 3x_3 = -2\\
  6x_1 + 6x_2 + 3x_3 = 5
\end{cases}$ em $\real$.
\begin{solucao}
  O sistema n\~ao tem solu\c{c}\~ao.
\end{solucao}
\end{exercicio}

\begin{exercicio}
$\begin{cases}
  \overline{1}x + \overline{2}y + \overline{3}z = \overline{0}\\
  \overline{2}x + \overline{1}y + \overline{3}z = \overline{0}\\
  \overline{3}x + \overline{2}y + \overline{1}z = \overline{0}
\end{cases}$ em $\integer_5$.
\begin{solucao}
  $p = 3$, Nulidade = 0, $S = \{(\overline{0},\overline{0},\overline{0})\}$
\end{solucao}
\end{exercicio}

\begin{exercicio}
$\begin{cases}
  (2 + 3\sqrt{2})x_1 - 3x_2 + x_3 = 0\\
  (1 - \sqrt{2})x_1 + \sqrt{2}x_3 = 0
\end{cases}$ em $\rac[\sqrt{2}]$.
\begin{solucao}
  $p = 2$, Nulidade = 1, $S = \{(x, y, z) \mid x, y, z \in \rac[\sqrt{2}]\} = $ \\ $\left\{\left((2 + \sqrt{2})z, \dfrac{11 + 8\sqrt{2}}{3}z, z\right) \mid z \in \rac[\sqrt{2}]\right\}$
\end{solucao}
\end{exercicio}

\begin{exercicio}
$\begin{cases}
  ix + iy = 0\\
  2ix - y = 0\\
\end{cases}$ em $\complex$.
\begin{solucao}
  $p = 2$, Nulidade = 0, $S = \{(0, 0)\}$
\end{solucao}
\end{exercicio}

\begin{exercicio}
$\begin{cases}
  x_1 + x_2 + x_3 + x_4 = 0\\
  x_1 + x_2 + x_3 - x_4 = 4\\
  x_1 + x_2 - x_3 + x_4 = -4\\
  x_1 - x_2 + x_3 + x_4 = 2\\
\end{cases}$ em $\real$.
\begin{solucao}
  $p = 4$, Nulidade = 0, $S = \{(1, -1, 2, -2)\}$
\end{solucao}
\end{exercicio}

\begin{exercicio}
$\begin{cases}
  \overline{1}x + \overline{2}y + \overline{1}z + \overline{3}w = \overline{0}\\
  \overline{1}x + \overline{10}y + \overline{1}w = \overline{0}\\
  \overline{1}y + \overline{10}z + \overline{1}w = \overline{0}
\end{cases}$ em $\integer_{13}$.
\begin{solucao}
  $p = 3$, Nulidade = 1, $S = \{(x, y, z, w) \mid x, y, z, w \in \integer_{13}\} = \{(\overline{5}w, \overline{2}w,w, w) \mid w \in \integer_{13}\}$
\end{solucao}
\end{exercicio}

\begin{exercicio}
  $\begin{cases}
    -2x_2 + 3x_3 = 1\\
    3x_1 + 6x_2 - 3x_3 = -1\\
    6x_1 + 6x_2 + 3x_3 = 5
  \end{cases}$ em $\real$.
  \begin{solucao}
    N\~ao existe solu\c{c}\~ao.
  \end{solucao}
\end{exercicio}

\begin{exercicio}
  $\begin{cases}
    2x_1 + x_2 + (6 + 6i)x_3 + 8x_4 = 0\\
    x_1 + x_2 + (2 + 5i)x_3 + (5  i)x_4 = 0\\
    (2 + 2i)x_1 + 2x_2 + (2 + 14i)x_3 + (14 + 8i)x_4 =0\\
    (-1 - i)x_1 - x_2 - (3 + 3i)x_3 + (-6 + 2i)x_4 = 0
  \end{cases}$ em $\complex$.
  \begin{solucao}
    $p = 2$, Nulidade = 2, $S = \{(x_1, x_2, x_3, x_4) \mid x_1, x_2, x_3, x_4 \in \complex\} = $\\ $\{(-(2 + i)x_3 - (3 - i)x_4,-3ix_3 - 2x_4, x_3, x_4) \mid x_3, x_4 \in \complex\}$
  \end{solucao}
\end{exercicio}

\begin{exercicio}
  $\begin{cases}
    x_1 + 2x_2 - 3x_4 + x_5 = 2\\
    x_1 + 2x_2 + x_3 - 3x_4 + x_5 + 2x_6 = 3\\
    x_1 + 2x_2 - 3x_4 + 2x_5 + x_6 = 4\\
    3x_1 + 6x_2 + x_3 - 9x_4 + 4x_5 + 3x_6 = 9
  \end{cases}$ em $\real$.
  \begin{solucao}
    $p = 3$, Nulidade = 3, $S = \{(x_1, x_2, x_3, x_4, x_5, x_6) \mid x_1, x_2, x_3, x_4, x_5, x_6 \in \real\} = $\\ $\{(x_6 + 3x_4 - 2x_2, x_2, 1 - 2x_6, x_4, 2 - x_6, x_6) \mid x_2, x_4, x_6 \in \real\}$
  \end{solucao}
\end{exercicio}

\begin{exercicio}
  $\begin{cases}
    x_1 + 3x_2 - 2x_3 + 2x_5 = 0\\
    2x_1 + 6x_2 - 5x_3 - 2x_4 + 4x_5 - 3x_6 = -1\\
    5x_3 + 10x_4 + 15x_6 = 5\\
    2x_1 + 6x_2 + 8x_4 + 4x_5 + 18x_6 = 6
  \end{cases}$ em $\real$.
  \begin{solucao}
    $p = 3$, Nulidade = 3, $S = \{(x_1, x_2, x_3, x_4, x_5, x_6) \mid x_1, x_2, x_3, x_4, x_5, x_6 \in \real\} = $\\ $\{(-3x_2 - 4x_4 - 2x_5, x_2, -2x_4, x_4, x_5, 1/3) \mid x_2, x_4, x_5 \in \real\}$
  \end{solucao}
\end{exercicio}

\begin{exercicio}
  $\begin{cases}
    x + 2z = 1\\
    (2 - \sqrt{2})x + 2y + 6z = 2 + \sqrt{2}\\
    \sqrt{2}x - y (-1 + \sqrt{2})z = 1 + \sqrt{2}
  \end{cases}$ em $\rac[\sqrt{2}]$.
  \begin{solucao}
    N\~ao existe solu\c{c}\~ao.
  \end{solucao}
\end{exercicio}


\begin{exercicio}\label{sistemalinearfim}
  $\begin{cases}
    3x + 2y - 4z = 1\\
    x - y + z = 3\\
    x - y - 3z = -3\\
    3x + 3y - 5z =0\\
    -x + y + z = 1
  \end{cases}$ em $\rac$.
\begin{solucao}
  N\~ao existe solu\c{c}\~ao.
\end{solucao}
\end{exercicio}

\begin{exercicio}
  Seja
  \[
    A = \begin{bmatrix}
      1 & 0 & 5\\
      1 & 1 & 1\\
      0 & 1 & -4
    \end{bmatrix}.
  \]
\begin{enumerate}[label={\alph*})]
  \item Encontre a solu\c{c}\~ao geral do sistema $(A + 4I_3)X = 0$ em $\real$.
  \item Encontre a solu\c{c}\~ao geral do sistema $(A - 2I_3)X = 0$ em $\real$.
\end{enumerate}
\begin{solucao}
  \begin{enumerate}[label={\alph*})]
    \item $S = \{(-\alpha, 0, \alpha) \mid \alpha \in \real\}$.
    \item $S = \{(5\alpha, 6\alpha, \alpha) \mid \alpha \in \real\}$.
  \end{enumerate}
\end{solucao}
\end{exercicio}

Nos exerc{\'\i}cios \ref{sistemasinicio} \`a \ref{sistemasfim}, encontre o(s) valor(es) de $\lambda$ tal(is) que:
\begin{enumerate}[label={\alph*})]
  \item o sistema tem solu\c{c}\~ao \'unica;
  \item o sistema tem v\'arias solu\c{c}\~oes;
  \item o sistema n\~ao tem solu\c{c}\~ao.
\end{enumerate}

\begin{exercicio}\label{sistemasinicio}
$\begin{cases}
  (\lambda - 2)x + y = 0\\
  x + (\lambda - 2)y = 0
\end{cases}$ em $\real$.
\begin{solucao}
  \begin{enumerate}[label={\alph*})]
    \item $\lambda \ne 1$ e $\lambda \ne 3$
    \item $\lambda = 1$ e $\lambda = 3$.
    \item N\~ao existe tal $\lambda$.
  \end{enumerate}
\end{solucao}
\end{exercicio}

\begin{exercicio}
$\begin{cases}
  x + 2y - 3z = 4\\
  3x - y + 5z = 2\\
  4x + y + (\lambda^2 - 14)z = \lambda + 2
\end{cases}$ em $\real$.
\begin{solucao}
  \begin{enumerate}[label={\alph*})]
    \item $\lambda \ne \pm 4$.
    \item $\lambda = 4$.
    \item $\lambda = -4$.
  \end{enumerate}
\end{solucao}
\end{exercicio}

\begin{exercicio}
$\begin{cases}
  x + y + z = 2\\
  2x + 3y + 2z = 5\\
  2x + 3y + (\lambda^2 - 1)z = \lambda + 1
\end{cases}$ em $\rac$.
\begin{solucao}
  \begin{enumerate}[label={\alph*})]
    \item N\~ao existe tal $\lambda$.
    \item N\~ao existe tal $\lambda$.
    \item N\~ao existe tal $\lambda$.
  \end{enumerate}
\end{solucao}
\end{exercicio}

\begin{exercicio}
$\begin{cases}
  \overline{1}x_1 + \overline{1}x_2 + \overline{1}x_3= \overline{2}\\
  \overline{1}x_1 + \overline{2}x_2 + \overline{1}x_1 = \overline{4}\\
  \overline{2}x_1 + \overline{2}x_2 + \overline{4}x_3 = \overline{3}\\
  \overline{3}x_1 + \overline{3}x_2 + \overline{3}x_3 + (\lambda^2 + \overline{2})x_4 = \lambda + \overline{4}
\end{cases}$ em $\integer_5$.
\begin{solucao}
  \begin{enumerate}[label={\alph*})]
    \item O sistema sempre tem solu\c{c}\~ao \'unica, independente de $\lambda$.
    \item N\~ao existe tal $\lambda$.
    \item N\~ao existe tal $\lambda$.
  \end{enumerate}
\end{solucao}
\end{exercicio}

\begin{exercicio}
$\begin{cases}
  \overline{1}x_1 + \overline{1}x_2 + \overline{1}x_3 = \overline{3}\\
  \overline{1}x_1 + \overline{2}x_2 + \overline{1}x_1 = \overline{5}\\
  \overline{2}x_1 + \overline{2}x_2 + \overline{4}x_3 = \overline{1}\\
  \overline{3}x_1 + \overline{3}x_2 + \overline{3}x_3 + (\lambda^2 + \overline{2})x_4 = \lambda + \overline{5}
\end{cases}$ em $\integer_7$.
\begin{solucao}
  \begin{enumerate}[label={\alph*})]
    \item O sistema sempre tem solu\c{c}\~ao \'unica, independente de $\lambda$.
    \item N\~ao existe tal $\lambda$.
    \item $\lambda = \overline{4}$.
  \end{enumerate}
\end{solucao}
\end{exercicio}

\begin{exercicio}\label{sistemasfim}
$\begin{cases}
  x + y = 2\\
  y + z = 2\\
  x + z = 2\\
  x + y + \lambda z = 0
\end{cases}$ em $\real$.
\begin{solucao}
  \begin{enumerate}[label={\alph*})]
    \item $\lambda = -2$
    \item N\~ao existe tal $\lambda$.
    \item $\lambda \ne -2$.
  \end{enumerate}
\end{solucao}
\end{exercicio}

\begin{exercicio}
  Encontre condi\c{c}\~oes sobre os $b_i \in \real$'s para que cada um dos sistemas tenha solu\c{c}\~ao.
  \begin{enumerate}[label={\alph*})]
    \item $\begin{cases}
      x_1 - 2x_2 + 5x_3 = b_1\\
      4x_1 - 5x_2 + 8x_3 = b_2\\
      -3x_1 + 3x_2 - 3x_3 = b_3
    \end{cases}$

    \item $\begin{cases}
      x_1 - 2x_2 - x_3 = b_1\\
      -4x_1 + 5x_2 + 2x_3 = b_2\\
      -4x_1 + 7x_2 + 4x_3 = b_3
    \end{cases}$
  \end{enumerate}
  \begin{solucao}
    \begin{enumerate}[label={\alph*})]
      \item $b_3 - b_1 + b_2 = 0$
      \item O sistema tem solu\c{c}\~ao para todos os valores reais de $b_1$, $b_2$ e $b_3$.
    \end{enumerate}
  \end{solucao}
\end{exercicio}

\newpage
\Closesolutionfile{ans}
\hrule
\begin{center}
{\large\bf RESPOSTAS}
\end{center}
\hrule
\input{ans1}

\end{document}