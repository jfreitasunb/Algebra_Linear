% !TEX encoding = ISO-8859-1

\documentclass[12pt]{exam}

\usepackage{caption}
\usepackage{amssymb}
\usepackage{amsmath,amsfonts,amsthm,amstext}
\usepackage{hhline}
\usepackage[brazil]{babel}
% \usepackage[latin1]{inputenc}
\usepackage{graphicx}
\graphicspath{{/home/jfreitas/Dropbox/imagens-latex/}{/Volumes/Vader/Dropbox/imagens-latex/}{D:/Dropbox/imagens-latex/}}
\usepackage{enumitem}
\usepackage{multicol}
 \usepackage{answers}
\usepackage{tikz,ifthen}
\usetikzlibrary{lindenmayersystems}
\usetikzlibrary[shadings]

\Newassociation{solucao}{Solution}{ans}
\newtheorem{exercicio}{}

\setlength{\topmargin}{-1.0in}
\setlength{\oddsidemargin}{0in}
\setlength{\textheight}{10.1in}
\setlength{\textwidth}{6.5in}
\setlength{\baselineskip}{12mm}

\newcounter{exercicios}
\setcounter{exercicios}{0}
\newcommand{\questao}{
\addtocounter{exercicios}{1}
\noindent{\bf Exerc{\'\i}cio \Roman{exercicios}: }}

\newcommand{\resp}[1]{
\noindent{\bf Exerc{\'\i}cio #1: }}

\newcommand{\im}{{\rm Im\,}}
\newcommand{\sub}{\subseteq}
\newcommand{\n}{\mathbb{N}}
\newcommand{\integer}{\mathbb{Z}}
\newcommand{\rac}{\mathbb{Q}}
\newcommand{\real}{\mathbb{R}}
\newcommand{\complex}{\mathbb{C}}
\newcommand{\cp}[1]{\mathbb{#1}}
\newcommand{\ch}{\mbox{\textrm{car\,}}\nobreak}
\newcommand{\dlim}[2]{\displaystyle\lim_{#1\rightarrow #2}}
\newcommand{\minf}{+\infty}
\newcommand{\ninf}{-\infty}
\renewcommand{\sin}{{\rm sen\,}}
\renewcommand{\sinh}{{\rm senh\,}}
\renewcommand{\tan}{{\rm tg\,}}
\renewcommand{\csc}{{\rm cossec\,}}
\renewcommand{\cot}{{\rm cotg\,}}
\newcommand{\din}[4]{\displaystyle\int_{#1}^{#2}{#3}{d#4}}

\newcommand{\se}[1]{\displaystyle\sum_{n = 1}^\infty{#1}}
\newcommand{\slim}{\displaystyle\lim_{n \rightarrow \infty}}
\newcommand{\seq}[1]{\{{#1_n\}}}
\newcommand{\seg}[1]{\displaystyle\sum_{n = 1}^\infty{#1_n}}
\newcommand{\sei}[2]{\displaystyle\sum_{#1}^\infty{#2}}

\newcommand{\vesp}[1]{\vspace{ #1  cm}}

\newcommand{\compcent}[1]{\vcenter{\hbox{$#1\circ$}}}
\newcommand{\comp}{\mathbin{\mathchoice
{\compcent\scriptstyle}{\compcent\scriptstyle}
{\compcent\scriptscriptstyle}{\compcent\scriptscriptstyle}}}

\begin{document}
\pagestyle{empty}

\Opensolutionfile{ans}[ans1]

\begin{figure}[h]
        \begin{minipage}[c]{1.7cm}
        \includegraphics[width=1.7cm]{../../../imagens/unb.pdf}
        \end{minipage}%
        \hspace{0pt}
        \begin{minipage}[c]{4in}
          {Universidade de Bras{\'\i}lia} \\
          {Departamento de Matem{\'a}tica}
\end{minipage}
\end{figure}

\vesp{-0.35} \hrule

\begin{center}
{\Large\bf \'Algebra Linear - Turma A -- 2$^{o}$/2013} \\ \vspace{9pt} {\large\bf
  $10^{\underline{a}}$ Lista de Exerc{\'\i}cios -- Produto Interno}\\ \vspace{9pt} Prof. Jos{\'e} Ant{\^o}nio O. Freitas
\end{center}
\hrule

\vesp{.6}

\begin{exercicio}
  Seja $V$ um $\cp{K}$-espa\c{c}o vetorial com produto interno $\langle\ ,\ \rangle$. Mostre que
  \begin{enumerate}[label=({\alph*})]
    \item $\langle u, 0\rangle = \langle 0, u\rangle = 0_\cp{K}$ para todo $u \in V$.
    \item $\langle u, u\rangle = 0_\cp{K}$ se, e somente se, $u = 0_V$.
    \item $\langle u, v + w\rangle = \langle u, v\rangle + \langle u, w\rangle$ para todos $u$, $v$ e $w \in V$.
    \item $\langle u, \lambda w\rangle = \overline{\lambda} \langle u, w\rangle$ para todo $\lambda \in \cp{K}$ e todos $u$, $w \in V$.
    \item $\displaystyle\langle \sum_{i = 1}^n\alpha_iu_i, \sum_{j - 1}^m\beta_jw_j\rangle = \sum_{i - 1}^n\sum_{j = 1}^m\alpha_i\overline{\beta_j}\langle u_i, w_j \rangle$ para $u_i$, $w_j \in V$ e $\alpha_i$, $\beta_j \in \cp{K}$, $i = 1$, \dots, $n$ e $j = 1$, \dots, $m$.
  \end{enumerate}
\end{exercicio}

\begin{exercicio}
  Sejam $V$ e $W$ dois espa\c{c}os vetoriais sobre $\cp{K}$ e seja $\langle\ ,\ \rangle$ um produto interno em $V$. Se $T : W \to V$ \'e uma transforma\c{c}\~ao linear injetora, ent\~ao defina
    \[
        \langle u , v \rangle_T = \langle T(u) , T(v) \rangle,
    \]
    para todos $u$, $v \in W$. Mostre que $\langle\ ,\ \rangle_T$ \'e um produto interno em $W$.
\end{exercicio}

\begin{exercicio}
  Verifique se as seguintes fun\c{c}\~oes s\~ao um produto no $\cp{K}$-espa\c{c}o vetorial $V$ ou n\~ao.
  \begin{enumerate}[label=({\alph*})]
    \item $V = \real^2$, $(a,b)$, $(x, y) \in \real^2$, $\langle (a,b) , (x,y) \rangle = 2ax - ay - bx + by$;
    \item $V = \complex^2$, $(a,b)$, $(x, y) \in \complex^2$, $\langle (a,b) , (x,y) \rangle = ax$;
    \item $V = \complex^3$, $(a, b, c)$, $(x, y, z) \in \complex^3$, $\langle (a, b, c) , (x, y, z) \rangle = a\overline{x} + b\overline{y} + c\overline{z}$;
    \item $V = \real^3$, $(a, b, c)$, $(x, y, z) \in \real^3$, $\langle (a, b, c) , (x, y, z) \rangle = a^2x^2 + b^2y^2 + c^2z^2$;
    \item $V = \mathcal{C}([a,b], \real)$ o espa\c{c}o das fun\c{c}\~oes cont{\'\i}nuas de $[a,b]$ em $\real$. Defina
    \[
        \langle f , g \rangle = \int_a^bf(t)g(t)dt
    \]
    para $f$, $g \in V$.
    \item $V = \mathcal{P}_3(\complex)$, $p(x) = a_0 + a_1x + a_2x^2 + a_3x^3$, $q(x) = b_0 + b_1x + b_2x^2 + b_3x^3$,
    \[
        \langle p(x) , q(x) \rangle = a_0\overline{b_0} + a_1\overline{b_1} + a_2\overline{b_2} + a_3\overline{b_3}.
    \]
    \item $V = \cp{M}_2(\complex)$, $A = \begin{bmatrix}
      a_{11} & a_{12}\\
      a_{21} & a_{22}
    \end{bmatrix}$, $B = \begin{bmatrix}
      b_{11} & b_{12}\\
      b_{21} & b_{22}
    \end{bmatrix}$,
    \[
        \langle A, B \rangle = 2a_{11}b_{11} + a_{12}b_{12} + a_{21}b_{21} + 2a_{22}b_{22}.
    \]
    \item $V = \real^2$, $(a,b)$, $(x, y) \in \real^2$, $\langle (a,b) , (x,y) \rangle = ax - by$;
  \end{enumerate}
  \begin{solucao}
  \begin{enumerate}[label=({\alph*})]
    \item Sim.
    \item N\~ao.
    \item Sim.
    \item N\~ao.
    \item Sim.
    \item Sim.
    \item N\~ao.
    \item N\~ao.
  \end{enumerate}
  \end{solucao}
\end{exercicio}

\begin{exercicio}
  Determine se os conjuntos abaixo s\~ao ortogonais, ortonormais ou nenhum deles, considerando o produto interno can\^onico em $V$.
  \begin{enumerate}[label=({\alph*})]
    \item $\{(-4, 6); (5, 0)\}$, $V = \real^2$
    \item $\{(3/5, 4/5); (-4/5, 3/5)\}$, $V = \real^2$
    \item $\{(4, -1, 1); (-1, 0, 4); (-4, -17, -1)\}$, $V = \real^3$
    \item $\{1, x, x^2, x^3\}$, $V = \mathcal{P}_3(\complex)$
    \item $\{(\sin \theta, \cos \theta); (\cos \theta, -\sin \theta)\}$, $V = \real^2$
    \item $\{(1/\sqrt{2}, i/\sqrt{2}, 0); ((1 + 2i)/\sqrt{18}, (2 - i)/\sqrt{18}, (2 - 2i)/\sqrt{18})\}$, $V = \complex^3$
  \end{enumerate}
  \begin{solucao}
    \begin{enumerate}[label=({\alph*})]
      \item Nenhum deles.
      \item Ortonormal.
      \item Ortogonal.
      \item Ortogonal.
      \item Ortonormal.
      \item Ortonormal.
    \end{enumerate}
  \end{solucao}
\end{exercicio}

\begin{exercicio}
  Encontre uma base ortonormal para $\complex^3$ contendo o vetor $\{(1, 2i, 0)\}$, onde o produto interno de $\complex^3$ \'e dado por
  \[
      \langle (x_1, x_2, x_3) ; (y_1, y_2, y_3) \rangle = 2x_1\overline{y_1} + 4x_2\overline{y_2} + x_3\overline{y_3}.
  \]
\end{exercicio}

\begin{exercicio}
  Seja $S = [(1 + i, 3i, 2 - i); (2 - 3i, 10 + 2i, 5 - i)]$. Determine uma base ortogonal para $S$, considerando em $\complex^3$ o produto interno can\^onio.
\end{exercicio}

\begin{exercicio}
  Considere a base $\mathcal{B} = \{(1, i); (i, 1)\}$ de $\complex^2$. Determine uma base ortonormal de $\complex^2$ que contenha um dos elementos de $\mathcal{B}$, considerando em $\complex^2$ o produto interno can\^onico.
\end{exercicio}

\begin{exercicio}
  Encontre uma base ortonormal do subespa\c{c}o $W$ de $\complex^3$ gerado por $v_1 = (1, i, 0)$ e $v_2 = (1, 2, 1 - i)$.
\end{exercicio}

\begin{exercicio}
  Considere a base de $\real^3$ dada por $\{(1, 1, 1); (0, 1, 1); (0, 0, 1)\}$. Encontre uma base ortonormal para $\real^3$ em rela\c{c}\~ao  ao produto interno can\^onico.
\end{exercicio}


\newpage
\Closesolutionfile{ans}
\hrule
\begin{center}
{\large\bf RESPOSTAS}
\end{center}
\hrule
\input{ans1}

\end{document}