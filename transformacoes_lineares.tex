%!TEX program = xelatex
%!TEX root = Algebra_Linear.tex
%%Usar makeindex -s indexstyle.ist Algebra_Linear.idx arquivo no terminal para gerar o {\'\i}ndice remissivo agrupado por inicial
%%Ap\'os executar pdflatex arquivo

\chapter{Transformações Lineares}

Em todo esse capítulo $\cp{K}$ denotará um corpo.

\section{Conceitor Básicos}

\begin{definicao}
	Sejam $(V, +, \cdot)$ e $(W, \oplus, \otimes)$ espaços vetoriais sobre um corpo $\cp{K}$. Uma função $T : V \to W$ é uma \textbf{transformação linear} se
	\begin{enumerate}
		\item $T(u_1 + u_2) = T(u_1) \oplus T(u_2)$ para todos $u_1$, $u_2 \in V$;
		\item $T(\lambda \cdot u) = \lambda \otimes T(u)$ para todo $\lambda \in \cp{K}$ e todo $u \in V$.
	\end{enumerate}
\end{definicao}

\begin{observacao}
	Para simplificar a notação, vamos adotar os mesmos símbolos para indicar a soma e o produto por escalar nos espaços vetoriais que aparecerão no decorrer do texto. No entanto, o leitor deve estar ciente que estes símbolos podem ter significados diferentes, dependendo do espaço vetorial em questão.
\end{observacao}

\begin{lema}
	Sejam $V$ e $W$ espaços vetoriais sobre $\cp{K}$. Então uma função $T : V \to W$ é uma transformação linear se, e somente se,
	\[
		T(\lambda u_1 + u_2) = \lambda T(u_1) + T(u_2),
	\]
	para todos $u_1$, $u_2 \in V$ e todo $\lambda \in \cp{K}$.
\end{lema}
\begin{prova}
	Deixada a cargo do leitor.
\end{prova}

\begin{lema}
	Sejam $V$ e $W$ espaços vetoriais sobre $\cp{K}$ e $T : V \to W$ uma transformação linear. Então:
	\begin{enumerate}\label{transformacao_linear_propriedades_basicas}
		\item $T(0_V) = 0_W$, onde $0_V$ e $0_W$ denotam os vetores nulos de $V$ e $W$, respectivamente.

		\item $T(-u) = -T(u)$, para cada $u \in V$.

		\item $T(\sum_{i=1}^m\alpha_iu_i) = \sum_{i=1}^mT(u_i)$, onde $\alpha_i \in \cp{K}$ e $u_i \in V$ para $i = 1$, \dots, $m$.
	\end{enumerate}
\end{lema}
\begin{prova}
	\begin{enumerate}
		\item Note que
		\[
			0_W + T(0_V) = T(0_V + 0_V) = T(0_V) + T(0_V),
		\]
		ou seja, $T(0_V) = 0_W$.

		\item Basta observar que $-u = (-1_\cp{K})u$ e daí
		\[
			T(-u) = T((-1_\cp{K})u) = -1_\cp{K}T(u) = -T(u).
		\]

		\item Por indução em $m$. Se $m = 2$, então
		\[
			T(\alpha_1u_1 + \alpha_2u_2) = T(\alpha_1u_1) + T(\alpha_2u_2) = \alpha_1T(u_1) + \alpha_2T(u_2).
		\]
		Suponha que para $m = p$ tenhamos
		\[
			T(\sum_{i=1}^p\alpha_iu_i) = \sum_{i=1}^pT(u_i).
		\]
		Vamos mostra que é válido para $m = p + 1$. De fato,
		\begin{align*}
			T(\sum_{i=1}^{p+1}\alpha_iu_i) &= T([\sum_{i=1}^p\alpha_iu_i] + \alpha_{p + 1}u_{p + 1}) = T(\sum_{i=1}^p\alpha_iu_i) + T(\alpha_{p+1}u_{p+1}) \\ &= \sum_{i=1}^p\alpha_iT(u_i) + \alpha_{p+1}T(u_{p+1}) = \sum_{i=1}^{p+1}\alpha_iT(u_i).
		\end{align*}
	\end{enumerate}
\end{prova}

\begin{exemplo}
	\begin{enumerate}
		\item Sejam $V$ e $W$ $\cp{K}$-espaços vetoriais. A função $T : V \to W$ dada por $T(u) = 0_W$ para todo $u \in V$ é uma transformação linear.
		
		\item Seja $V$ um $\cp{K}$-espaço vetorial. A função $T : V \to V$ dada por $T(u) = u$ para todo $u \in V$ é uma transformação linear.
		
		\item Considere $\real$ como um $\real$-espaço vetorial. Dado $a \in \real$, defina $T_a : \real \to \real$ por $T_a(x) = ax$ para todo $x \in \real$. Então $T_a$ é uma transformação linear. Agora, seja $T : \real \to \real$ dada por $T(x) = e^x$. Então $T$ não é uma transformação linear pois $T(0) \ne 0$.
		
		\item Sejam $\cp{K}^3$ e $\cp{M}_2(\cp{K})$ $\cp{K}$-espaços vetoriais. Defina $T : \cp{K}^3 \to \cp{M}_2(\cp{K})$ por
		\[
			T(a, b, c) = \begin{bmatrix}
				a+b & 0_\cp{K}\\
				0_\cp{K} & c - b
			\end{bmatrix}.
		\]
		Então $T$ é uma transformação linear. De fato, dados $(a, b, c)$, $(d, e, f) \in \cp{K}^3$ e $\lambda \in \cp{K}$ temos
		\begin{align*}
			T(\lambda(a,b,c) + (d,e,f)) &= T(\lambda a + d, \lambda b + e, \lambda c + f) \\ &= \begin{bmatrix}
				(\lambda a + d) + (\lambda b + e) & 0_\cp{K}\\
				0_\cp{K} & (\lambda c + f) - (\lambda b + e)
			\end{bmatrix} \\ &= \begin{bmatrix}
				\lambda a + \lambda b & 0_\cp{K}\\
				0_\cp{K} & \lambda c - \lambda b
			\end{bmatrix} + \begin{bmatrix}
				d + e & 0_\cp{K}\\
				0_\cp{K} & f - e
			\end{bmatrix} \\ &= \lambda T(a,b,c) + T(d,e,f).
		\end{align*}

		\item Seja $\mathcal{P}(\complex)$ um $\complex$-espaço vetorial e considere $D : \mathcal{P}(\complex) \to \mathcal{P}(\complex)$ dado por
		\[
			D(a_0 + a_1x + a_2x^2 + \cdots + a_nx^n) = a_1 + 2a_2x + \cdots + na_nx^{n-1}.
		\]
		Então $D$ é uma transformação linear.

		\item Seja $\mathcal{C}([a,b], \real) = \{ f : [a,b] \to \real \mid f \mbox{ é uma função contínua}\}$. É imediato verificar que $\mathcal{C}([a,b], \real)$ é um $\real$-espaço vetorial. Defina $T : \mathcal{C}([a,b], \real) \to \real$ por
		\[
			T(f(x)) = \int_a^bf(x)dx.
		\]
		Então $T$ é uma transformação linear.

		\item Sejam $a_1$, \dots, $a_n \in \cp{K}$. Defina $T : \cp{K}^n \to \cp{K}$ por
		\[
			T(x_1, \dots, x_n) = \sum_{i=1}^na_ix_i.
		\]
		É imediato verificar que $T$ é uma transformação linear. Denote por $e_i$ o elemento de $\cp{K}^n$ contendo $1_\cp{K}$ na posição $i$ e $0_\cp{K}$ das demais. Então $\{e_1, \dots, e_n\}$ é uma base de $\cp{K}^n$ e
		\[
			T(e_i) = a_i
		\]
		para $i = 1$, \dots, $n$. Agora, se $S : \cp{K}^n \to \cp{K}$ é uma transformação linear, então pelo Lema \ref{transformacao_linear_propriedades_basicas} item (c)
		\[
			S(x_1, \dots, x_n) = S(\sum_{i=1}^nx_ie_i) = \sum_{i=1}^nS(e_i)
		\]
		onde $S(e_i) \in \cp{K}$ para $i = 1$, \dots, $n$. Logo qualquer transformação linear de $T : \cp{K}^n \to \cp{K}$ é da forma
		\[
			T(x_1, \dots, x_n) = \sum_{i=1}^na_ix_i,
		\]
		para determinados escalares $a_1$, \dots, $a_n \in \cp{K}$. Isto é, para determinarmos a transformação $T$ só precisamos conhecer seus valores na base $\{e_1, \dots, e_n\}$ de $\cp{K}^n$.
	\end{enumerate}
\end{exemplo}

\begin{teorema}\label{existencia_de_transformacao_unica_dado_valores}
	Sejam $V$ e $W$ $\cp{K}$-espaços vetoriais. Se $\{u_1, \dots, u_n\}$ é uma base de $V$ e se $\{w_1, \dots, w_n\} \subseteq W$, então existe uma única transformação linear $T : V \to W$ tal que $T(u_i) = w_i$ para cada $i = 1$, \dots, $n$.
\end{teorema}
\begin{prova}
	Dado $v \in V$, como $\{u_1, \dots, u_n\}$ é uma base de $V$, então sabemos que existem únicos $\alpha_1$, \dots, $\alpha_n \in \cp{K}$ tais que
	\[
		v = \alpha_1u_1 + \cdots + \alpha_nu_n.
	\]
	Defina então $T : V \to W$ por
	\[
		T(v) = T(\alpha_1u_1 + \cdots + \alpha_nu_n) = \alpha_1w_1 + \cdots + \alpha_nw_n.
	\]
	A unicidade dos escalares $\alpha_1$, \dots, $\alpha_n \in \cp{K}$ garantem que $T$ está bem definida, isto é, um mesmo elemento de $V$ não pode ter duas imagens distintas.

	Agora, note que $T(u_i) = w_i$ para cada $i = 1$, \dots, $n$. Assim precisamos mostrar que $T$ é linear. Sejam $v_1 = \sum_{i=1}^n\alpha_iu_i$, $v_2 = \sum_{i=1}^n\beta_iu_i$ e $\lambda \in \cp{K}$. Então
	\begin{align*}
		T(\lambda v_1 + v_2) = T(\lambda\sum_{i=1}^n\alpha_iu_i + \sum_{i=1}^n\beta_iu_i) = T(\sum_{i=1}^n(\lambda\alpha_i + \beta_i)u_i) = \sum_{i=1}^n(\lambda_i\alpha_i + \beta_i)w_i = \lambda\sum_{i=1}^n\alpha_iw_i + \sum_{i=1}^n\beta_iw_i = \lambdaT(v_1) + T(v_2).
	\end{align*}
	Logo $T$ é uma transformação linear.

	Resta agora mostrar que $T$ é única. Suponha que exista uma transformação linear $S : V \to W$ tal que $S(u_i) = w_i$ para todo $i = 1$, \dots, $n$. Para $v = \sum_{i=1}^n\alpha_iu_i$ temos
	\[
		S(v) = S(\sum_{i=1}^n\alpha_iu_i) = \sum_{i=1}^n\alpha_iS(u_i) = \sum_{i=1}^n\alpha_iw_i = \sum_{i=1}^n\alpha_iT(u_i) = T(\sum_{i=1}^n\alpha_iu_i) = T(v)
	\]
	para todo $v \in V$. Logo $T=S$, isto é, existe uma única transformação linear que satisfaz as condições do teorema.
\end{prova}