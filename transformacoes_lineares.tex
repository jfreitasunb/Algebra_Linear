%!TEX program = xelatex
%!TEX root = Algebra_Linear.tex
%%Usar makeindex -s indexstyle.ist Algebra_Linear.idx arquivo no terminal para gerar o {\'\i}ndice remissivo agrupado por inicial
%%Ap\'os executar pdflatex arquivo

\chapter{Transformações Lineares}

Em todo esse capítulo $\cp{K}$ denotará um corpo.

\section{Conceitos Básicos}

\begin{definicao}
	Sejam $(V, +, \cdot)$ e $(W, \oplus, \otimes)$ espaços vetoriais sobre um corpo $\cp{K}$. Uma função $T : V \to W$ é uma \textbf{transformação linear} se
	\begin{enumerate}
		\item $T(u_1 + u_2) = T(u_1) \oplus T(u_2)$ para todos $u_1$, $u_2 \in V$;
		\item $T(\lambda \cdot u) = \lambda \otimes T(u)$ para todo $\lambda \in \cp{K}$ e todo $u \in V$.
	\end{enumerate}
\end{definicao}

\begin{observacao}
	Para simplificar a notação, vamos adotar os mesmos símbolos para indicar a soma e o produto por escalar nos espaços vetoriais que aparecerão no decorrer do texto. No entanto, o leitor deve estar ciente que estes símbolos podem ter significados diferentes, dependendo do espaço vetorial em questão.
\end{observacao}

\begin{lema}
	Sejam $V$ e $W$ espaços vetoriais sobre $\cp{K}$. Então uma função $T : V \to W$ é uma transformação linear se, e somente se,
	\[
		T(\lambda u_1 + u_2) = \lambda T(u_1) + T(u_2),
	\]
	para todos $u_1$, $u_2 \in V$ e todo $\lambda \in \cp{K}$.
\end{lema}
\begin{prova}
	Deixada a cargo do leitor.
\end{prova}

\begin{lema}
	Sejam $V$ e $W$ espaços vetoriais sobre $\cp{K}$ e $T : V \to W$ uma transformação linear. Então:
	\begin{enumerate}\label{transformacao_linear_propriedades_basicas}
		\item $T(0_V) = 0_W$, onde $0_V$ e $0_W$ denotam os vetores nulos de $V$ e $W$, respectivamente.

		\item $T(-u) = -T(u)$, para cada $u \in V$.

		\item $T(\sum_{i=1}^m\alpha_iu_i) = \sum_{i=1}^mT(u_i)$, onde $\alpha_i \in \cp{K}$ e $u_i \in V$ para $i = 1$, \dots, $m$.
	\end{enumerate}
\end{lema}
\begin{prova}
	\begin{enumerate}
		\item Note que
		\[
			0_W + T(0_V) = T(0_V + 0_V) = T(0_V) + T(0_V),
		\]
		ou seja, $T(0_V) = 0_W$.

		\item Basta observar que $-u = (-1_\cp{K})u$ e daí
		\[
			T(-u) = T((-1_\cp{K})u) = -1_\cp{K}T(u) = -T(u).
		\]

		\item Por indução em $m$. Se $m = 2$, então
		\[
			T(\alpha_1u_1 + \alpha_2u_2) = T(\alpha_1u_1) + T(\alpha_2u_2) = \alpha_1T(u_1) + \alpha_2T(u_2).
		\]
		Suponha que para $m = p$ tenhamos
		\[
			T(\sum_{i=1}^p\alpha_iu_i) = \sum_{i=1}^pT(u_i).
		\]
		Vamos mostra que é válido para $m = p + 1$. De fato,
		\begin{align*}
			T(\sum_{i=1}^{p+1}\alpha_iu_i) &= T([\sum_{i=1}^p\alpha_iu_i] + \alpha_{p + 1}u_{p + 1}) = T(\sum_{i=1}^p\alpha_iu_i) + T(\alpha_{p+1}u_{p+1}) \\ &= \sum_{i=1}^p\alpha_iT(u_i) + \alpha_{p+1}T(u_{p+1}) = \sum_{i=1}^{p+1}\alpha_iT(u_i).
		\end{align*}
	\end{enumerate}
\end{prova}

\begin{exemplo}
	\begin{enumerate}
		\item Sejam $V$ e $W$ $\cp{K}$-espaços vetoriais. A função $T : V \to W$ dada por $T(u) = 0_W$ para todo $u \in V$ é uma transformação linear.
		
		\item Seja $V$ um $\cp{K}$-espaço vetorial. A função $T : V \to V$ dada por $T(u) = u$ para todo $u \in V$ é uma transformação linear.
		
		\item Considere $\real$ como um $\real$-espaço vetorial. Dado $a \in \real$, defina $T_a : \real \to \real$ por $T_a(x) = ax$ para todo $x \in \real$. Então $T_a$ é uma transformação linear. Agora, seja $T : \real \to \real$ dada por $T(x) = e^x$. Então $T$ não é uma transformação linear pois $T(0) \ne 0$.
		
		\item Sejam $\cp{K}^3$ e $\cp{M}_2(\cp{K})$ $\cp{K}$-espaços vetoriais. Defina $T : \cp{K}^3 \to \cp{M}_2(\cp{K})$ por
		\[
			T(a, b, c) = \begin{bmatrix}
				a+b & 0_\cp{K}\\
				0_\cp{K} & c - b
			\end{bmatrix}.
		\]
		Então $T$ é uma transformação linear. De fato, dados $(a, b, c)$, $(d, e, f) \in \cp{K}^3$ e $\lambda \in \cp{K}$ temos
		\begin{align*}
			T(\lambda(a,b,c) + (d,e,f)) &= T(\lambda a + d, \lambda b + e, \lambda c + f) \\ &= \begin{bmatrix}
				(\lambda a + d) + (\lambda b + e) & 0_\cp{K}\\
				0_\cp{K} & (\lambda c + f) - (\lambda b + e)
			\end{bmatrix} \\ &= \begin{bmatrix}
				\lambda a + \lambda b & 0_\cp{K}\\
				0_\cp{K} & \lambda c - \lambda b
			\end{bmatrix} + \begin{bmatrix}
				d + e & 0_\cp{K}\\
				0_\cp{K} & f - e
			\end{bmatrix} \\ &= \lambda T(a,b,c) + T(d,e,f).
		\end{align*}

		\item Seja $\mathcal{P}(\complex)$ um $\complex$-espaço vetorial e considere $D : \mathcal{P}(\complex) \to \mathcal{P}(\complex)$ dado por
		\[
			D(a_0 + a_1x + a_2x^2 + \cdots + a_nx^n) = a_1 + 2a_2x + \cdots + na_nx^{n-1}.
		\]
		Então $D$ é uma transformação linear.

		\item Seja $\mathcal{C}([a,b], \real) = \{ f : [a,b] \to \real \mid f \mbox{ é uma função contínua}\}$. É imediato verificar que $\mathcal{C}([a,b], \real)$ é um $\real$-espaço vetorial. Defina $T : \mathcal{C}([a,b], \real) \to \real$ por
		\[
			T(f(x)) = \int_a^bf(x)dx.
		\]
		Então $T$ é uma transformação linear.

		\item Sejam $a_1$, \dots, $a_n \in \cp{K}$. Defina $T : \cp{K}^n \to \cp{K}$ por
		\[
			T(x_1, \dots, x_n) = \sum_{i=1}^na_ix_i.
		\]
		É imediato verificar que $T$ é uma transformação linear. Denote por $e_i$ o elemento de $\cp{K}^n$ contendo $1_\cp{K}$ na posição $i$ e $0_\cp{K}$ das demais. Então $\{e_1, \dots, e_n\}$ é uma base de $\cp{K}^n$ e
		\[
			T(e_i) = a_i
		\]
		para $i = 1$, \dots, $n$. Agora, se $S : \cp{K}^n \to \cp{K}$ é uma transformação linear, então pelo Lema \ref{transformacao_linear_propriedades_basicas} item (c)
		\[
			S(x_1, \dots, x_n) = S(\sum_{i=1}^nx_ie_i) = \sum_{i=1}^nS(e_i)
		\]
		onde $S(e_i) \in \cp{K}$ para $i = 1$, \dots, $n$. Logo qualquer transformação linear de $T : \cp{K}^n \to \cp{K}$ é da forma
		\[
			T(x_1, \dots, x_n) = \sum_{i=1}^na_ix_i,
		\]
		para determinados escalares $a_1$, \dots, $a_n \in \cp{K}$. Isto é, para determinarmos a transformação $T$ só precisamos conhecer seus valores na base $\{e_1, \dots, e_n\}$ de $\cp{K}^n$.
	\end{enumerate}
\end{exemplo}

\begin{teorema}\label{existencia_de_transformacao_unica_dado_valores}
	Sejam $V$ e $W$ $\cp{K}$-espaços vetoriais. Se $\{u_1, \dots, u_n\}$ é uma base de $V$ e se $\{w_1, \dots, w_n\} \subseteq W$, então existe uma única transformação linear $T : V \to W$ tal que $T(u_i) = w_i$ para cada $i = 1$, \dots, $n$.
\end{teorema}
\begin{prova}
	Dado $v \in V$, como $\{u_1, \dots, u_n\}$ é uma base de $V$, então sabemos que existem únicos $\alpha_1$, \dots, $\alpha_n \in \cp{K}$ tais que
	\[
		v = \alpha_1u_1 + \cdots + \alpha_nu_n.
	\]
	Defina então $T : V \to W$ por
	\[
		T(v) = T(\alpha_1u_1 + \cdots + \alpha_nu_n) = \alpha_1w_1 + \cdots + \alpha_nw_n.
	\]
	A unicidade dos escalares $\alpha_1$, \dots, $\alpha_n \in \cp{K}$ garantem que $T$ está bem definida, isto é, um mesmo elemento de $V$ não pode ter duas imagens distintas.

	Agora, note que $T(u_i) = w_i$ para cada $i = 1$, \dots, $n$. Assim precisamos mostrar que $T$ é linear. Sejam $v_1 = \sum_{i=1}^n\alpha_iu_i$, $v_2 = \sum_{i=1}^n\beta_iu_i$ e $\lambda \in \cp{K}$. Então
	\begin{align*}
		T(\lambda v_1 + v_2) = T(\lambda\sum_{i=1}^n\alpha_iu_i + \sum_{i=1}^n\beta_iu_i) = T(\sum_{i=1}^n(\lambda\alpha_i + \beta_i)u_i) = \sum_{i=1}^n(\lambda_i\alpha_i + \beta_i)w_i = \lambda\sum_{i=1}^n\alpha_iw_i + \sum_{i=1}^n\beta_iw_i = \lambda T(v_1) + T(v_2).
	\end{align*}
	Logo $T$ é uma transformação linear.

	Resta agora mostrar que $T$ é única. Suponha que exista uma transformação linear $S : V \to W$ tal que $S(u_i) = w_i$ para todo $i = 1$, \dots, $n$. Para $v = \sum_{i=1}^n\alpha_iu_i$ temos
	\[
		S(v) = S(\sum_{i=1}^n\alpha_iu_i) = \sum_{i=1}^n\alpha_iS(u_i) = \sum_{i=1}^n\alpha_iw_i = \sum_{i=1}^n\alpha_iT(u_i) = T(\sum_{i=1}^n\alpha_iu_i) = T(v)
	\]
	para todo $v \in V$. Logo $T=S$, isto é, existe uma única transformação linear que satisfaz as condições do teorema.
\end{prova}

\begin{exemplo}
	Os vetores $v_1 = (1,2)$ e $v_2 = (3,4)$ são L.I em $\real^2$ e assim formam uma base de $\real^2$. Assim pelo Teorema \ref{existencia_de_transformacao_unica_dado_valores}, sabemos que existe uma única transformação linear $T : \real^2 \to \real^3$ tal que
	\begin{align*}
		T(v_1) &= T(1,2) = (3,2,1)\\
		T(v_2) &= T(3,4) = (6,5,4).
	\end{align*}
	Determine $T(1,0)$.
\end{exemplo}
\begin{solucao}
	Inicialmente escrevemos $(1,0)$ como combinação linear de $v_1$ e $v_2$:
	\[
		(1, 0) = \alpha(1,2) + \beta(3,4).
	\]
	Obtendo $\alpha = -2$ e $\beta = 1$. Assim
	\[
		T(1,0) = T(-2(1,2) + (3,4)) = -2T(1,2) + T(3,4) = (0,1,2).
	\]
\end{solucao}

\begin{definicao}
	Sejam $V$ e $W$ $\cp{K}$-espaços vetoriais e $T : V \to W$ uma transformação linear.
	\begin{enumerate}
		\item O conjunto
		\[
			\ker T = \{u \in V \mid T(u) = 0_W\}
		\]
		é chamado de \textbf{kernel} ou \textbf{núcleo} de $T$. (O núcleo de $T$ também pode ser denotado por $Nuc\ T$.)

		\item O conjunto
		\[
			\im T = \{u \in W \mid \mbox{ existe } v \in V \mbox{ tal que } T(v) = u\}
		\]
		é chamado de \textbf{imagem} de $T$.
	\end{enumerate}
\end{definicao}

\begin{proposicao}
	Sejam $V$ e $W$ $\cp{K}$-espaços vetoriais e $T : V \to W$ uma transformação linear. Então:
	\begin{enumerate}
		\item $\ker T$ é um subespaço de $V$;
		\item $\im T$ é um subespaço de $W$.
	\end{enumerate}
\end{proposicao}
\begin{prova}
	\begin{enumerate}
		\item Inicialmente $\ker T \ne \emptyset$ pois $T(0_V) = 0_W$, ou seja, $0_V \in \ker T$. Agora, sejam $u_1$, $u_2 \in \ker T$ e $\lambda \in \cp{K}$. Precisamos mostrar que $\lambda u_1 + u_2 \in cp{K}$, isto é, precisamos mostrar que $\lambda u_1 + u_2 \in \ker T$. Temos
		\[
			T(\lambda u_1 + u_2) = \lambda T(u_1) + T(u_2) = 0_W.
		\]
		Logo, $\ker T$ é um subespaço de $V$.

		\item Inicialmente $0_W \in \im T$ pois $0_W = T(0_V)$ e daí $\im T \ne \emptyset$. Sejam $w_1$, $w_2 \in \im T$ e $\lambda \in \cp{K}$. Então existem $u_1$, $u_2 \in V$ tais que $w_1 = T(u_1)$ e $w_2 = T(u_2)$. Assim
		\[
			\lambda w_1 + w_2 = \lambda T(u_1) + T(u_2) = T(\lambda u_1) + T(u_2) = T(\lambda u_1 + u_2)
		\]
		e então $\lambda w_1 + w_2 \in \im T$. Portanto, $\im T$ é um subespaço de $W$.
	\end{enumerate}
\end{prova}

\begin{exemplos}
	\begin{enumerate}
		\item Seja $T : \real^3 \to \cp{M}_2(\real)$ dada por
		\[
			T(a,b,c) = \begin{bmatrix}
				a + b & 0\\
				0 & c - b
			\end{bmatrix}.
		\]
		Determine $\ker T$ e $\im T$.
		\begin{solucao}
			Temos
			\[
				T(a,b,c) = \begin{bmatrix}
					0 & 0\\
					0 & 0
				\end{bmatrix}
			\]
			se, e só se, $a = -b$ e $c = b$. Daí
			\[
				\ker T = \{(a,b,c) \in \real^3 \mid a = -b, c = b\} = \{(-b,b,b) \mid b \in \real\}.
			\]
			Note que $\{(1,1,1)\}$ é uma base de $\ker T$, ou seja, $\dim_\real \ker T = 1$.

			Agora,
			\begin{align*}
				\im T &= \{v \in \cp{M}_2(\real) \mid \mbox{ existe } u \in \real^3 \mbox{ de modo que } T(u) = v\}\\
				\im T &= \left\{ \begin{bmatrix}
					a + b & 0\\
					0 & c - b
				\end{bmatrix} \mid a, b, c \in \real\right\}.
			\end{align*}
			Assim temos
			\begin{align*}
				\begin{bmatrix}
					a + b & 0\\
					0 & c - b
				\end{bmatrix} &= a \begin{bmatrix}
					1 & 0\\ 0 & 0
				\end{bmatrix} + b \begin{bmatrix}
					1 & 0\\ 0 & 0
				\end{bmatrix} + c \begin{bmatrix}
					0 & 0\\ 0 & 1
				\end{bmatrix} + b \begin{bmatrix}
					0 & 0\\ 0 & -1
				\end{bmatrix}
			\end{align*}
			e é fácil ver que
			\[
				\mathcal{B}' = \left\{ \begin{bmatrix}
					1 & 0\\ 0 & 0
				\end{bmatrix}; \begin{bmatrix}
					1 & 0\\ 0 & -1
				\end{bmatrix}; \begin{bmatrix}
					0 & 0\\ 0 & 1
				\end{bmatrix}\right\}
			\]
			é um conjunto gerado de $\im T$. No entanto tal conjunto não é L.I.. Mas o conjunto
			\[
				\left\{ \begin{bmatrix}
					1 & 0\\ 0 & 0
				\end{bmatrix}; \begin{bmatrix}
					0 & 0\\ 0 & 1
				\end{bmatrix}\right\}
			\]
			é uma base de $\im T$, ou seja, $\dim_\real\im T = 2$. Observe que
			\[
				\dim_\real\ker T + \dim_\real\im T = 3 = \dim_\real\real^3.
			\]
		\end{solucao}

		\item Seja $T : \real^2 \to \real$ dada por $T(x,y) = x + y$. Determine $\ker T$ e $\im T$.
		\begin{solucao}
			Temos
			\begin{align*}
				\ker T &= \{(x,y) \in \real^2 \mid T(x,y) = 0\} = \{(x,y) \in \real^2 \mid x + y = 0\}\\
				\ker T &= \{(x,-x) \in \real^2 \mid x \in \real\}.
			\end{align*}
			Assim $\{(1,-1)\}$ é uma base de $\ker T$, ou seja, $\dim_\real\ker T = 1$.

			Agora
			\[
				\im T = \{w \in \real \mid \mbox{ existe } (x,y) \in \real^2 \mbox{ tal que } T(x,y) = w\}.
			\]
			Assim dado $w \in \real$ um número real qualquer, tome o elemento $(w,0) \in \real^2$. Temos $T(w,0) = w + 0 = w$. Logo $\im T = \real$ e então $\dim_\real\im T = 1$.

			Novamente temos
			\[
				\dim_\real\ker T + \dim_\real\im T = 2 = \dim_\real\real^2.
			\]
		\end{solucao}
	\end{enumerate}
\end{exemplos}

\begin{definicao}
	Sejam $V$ e $W$ $\cp{K}$-espaços vetoriais e $T : V \to W$ uma transformação linear.
	\begin{enumerate}
		\item Dizemos que $T$ é \textbf{injetora} se dados $u_1$, $u_2 \in V$ tais que $T(u_1) = T(u_2)$, então $u_1 = u_2$. De modo equivalente, se $u_1$, $u_2 \in V$ são tais que $u_1 \ne u_2$, então $T(u_1) \ne T(u_2)$.

		\item Dizemos que $T$ é \textbf{sobrejetora} se $\im T = W$. Em outras palavras, $T$ é \textbf{sobrejetora} se para todo $w \in W$, existe $u \in V$ tal que $T(u) = w$.

		\item Se $T$ é injetora e sobrejetora, então dizemos que $T$ é um \textbf{isomorfismo}.
	\end{enumerate}
\end{definicao}

\begin{exemplos}
	\begin{enumerate}
		\item A transformação linear $T : \real^2 \to \real$ dada por $T(x,y) = x + y$ é sobrejetora, mas não é injetora.

		\item A transformação linear $T : V \to V$ dada por $T(u) = u$ é injetora e sobrejetora, ou seja, é um isomorfismo.
	\end{enumerate}
\end{exemplos}

\begin{proposicao}\label{caracteriza_transformacao_injetora}
	Sejam $V$ e $W$ $\cp{K}$-espaços vetoriais e $T : V \to W$ uma transformação linear. Então $T$ é injetora se, e somente se, $\ker T = \{0_V\}$.
\end{proposicao}
\begin{prova}
	Suponha que $T$ é injetora. Queremos mostrar que $\ker T = \{0_V\}$. Seja então $u \in \ker T$. Então $T(u) = 0_W$. Mas $T(0_V) = 0_W$ e como $T$ é injetora devemos ter $u = 0_V$. Logo $\ker T = \{0_V\}$.

	Agora suponha que $\ker T = \{0_V\}$. Queremos mostrar que $T$ é injetora. Para isso, sejam $u_1$, $u_2 \in V$ tais que $T(u_1) = T(u_2)$. Então
	\begin{align*}
		T(u_1) &= T(u_2)\\
		T(u_1) - T(u_2) & = 0_W\\
		T(u_1 - u_2) &= 0_W,
	\end{align*}
	isto é, $u_1 - u_2 \in \ker T$. Mas $\ker T = \{0_V\}$, logo $u_1 = u_2$. Portanto $T$ é injetora.
\end{prova}

\begin{exemplo}
	Seja $T : \cp{K}^4 \to \cp{M}_2(\cp{K})$ dada por
	\begin{align*}
		T(1_\cp{K},0_\cp{K},0_\cp{K},0_\cp{K}) &= \begin{bmatrix}
			1_\cp{K} & 0_\cp{K}\\
			0_\cp{K} & 0_\cp{K}
		\end{bmatrix}; T(0_\cp{K},1_\cp{K},0_\cp{K},0_\cp{K}) = \begin{bmatrix}
			0_\cp{K} & 1_\cp{K}\\
			0_\cp{K} & 0_\cp{K}
		\end{bmatrix}\\
		T(0_\cp{K},0_\cp{K},1_\cp{K},0_\cp{K}) &= \begin{bmatrix}
			0_\cp{K} & 0_\cp{K}\\
			1_\cp{K} & 0_\cp{K}
		\end{bmatrix}; T(0_\cp{K},0_\cp{K},0_\cp{K},1_\cp{K}) = \begin{bmatrix}
			0_\cp{K} & 0_\cp{K}\\
			0_\cp{K} & 1_\cp{K}
		\end{bmatrix}\\
	\end{align*}
	É fácil ver que $\ker T = \{(0_\cp{K},0_\cp{K},0_\cp{K},0_\cp{K})\}$. Além disso,
	\[
		\im T = \left[\begin{bmatrix}
			1_\cp{K} & 0_\cp{K}\\
			0_\cp{K} & 0_\cp{K}
		\end{bmatrix};\begin{bmatrix}
			0_\cp{K} & 1_\cp{K}\\
			0_\cp{K} & 0_\cp{K}
		\end{bmatrix};\begin{bmatrix}
			0_\cp{K} & 0_\cp{K}\\
			1_\cp{K} & 0_\cp{K}
		\end{bmatrix};\begin{bmatrix}
			0_\cp{K} & 0_\cp{K}\\
			0_\cp{K} & 1_\cp{K}
		\end{bmatrix}\right] = \cp{M}_2(\cp{K}).
	\]
	Daí $\dim_\cp{K}\im T = 4$ e novamente
	\[
		\dim_\cp{K}\ker T + \dim_\cp{K}\im T = 4 = \dim_cp{K}\cp{M}_2(\cp{K}).
	\]
\end{exemplo}

\begin{lema}\label{transformacao_gera_imagem}
	Sejam $V$ e $W$ $\cp{K}$-espaços vetoriais e $T : V \to W$ uma transformação linear. Se $\mathcal{B} = \{u_1, \dots, u_n\}$ é uma base de $V$, então $\{T(u_1), \dots, T(u_n)\}$ gera $\im T$.
\end{lema}
\begin{prova}
	Seja $w \in \im T$. Por definição, existe $u \in V$ tal que $T(u) = w$. Como $\mathcal{B}$ é uma base de $V$, então existem $\alpha_1$, \dots, $\alpha_n \in \cp{K}$ tais que $u = \alpha_1u_1 + \cdots + \alpha_nu_n$. Daí
	\[
		w = T(u) = \alpha_1T(u_1) + \cdots + \alpha_nT(u_n),
	\]
	ou seja, todo vetor de $\im T$ é uma combinação linear de $T(u_1)$, \dots, $T(u_n)$. Portanto $\im T = [T(u_1), \dots, T(u_n)]$ como queríamos.
\end{prova}

\begin{exemplo}
	Considere $\complex^2$ e $\real^3$ como $\real$-espaços vetoriais e seja $T : \complex^2 \to \real^3$ dada por
	\[
		T(a+bi, c+di) = (a - c, b + 2d, a + b - c + 2d)
	\]
	onde $a$, $b$ e $c \in \real$. É fácil ver que $T$ é uma transformação linear. Seja $\{(1,0);(i,0);(0,1);(0,i)\}$ uma base de $\complex^2$. Temos pelo Lema \ref{transformacao_gera_imagem} que $\{T(1,0);T(i,0);T(0,1);T(0,i)\}$ gera $\im T$. Agora
	\begin{align*}
		T(1,0) &= (1,0,1);\quad T(i,0) = (0,1,1)\\
		T(0,1) &= (-1,0,-1);\quad T(0,i) = (0,2,2)
	\end{align*}
	e temos
	\begin{align*}
		T(1,0) &= -T(0,1)\\
		T(i,0) &= 2T(0,i).
	\end{align*}
	Observe que $T(1,0)$ não é múltiplo de $T(0,i)$. Logo o conjunto $\{T(1,0); T(0,i)\}$ é uma base de $\im T$.
\end{exemplo}

\begin{teorema}[Teorema do Núcleo e da Imagem]\label{teorema_do_nucleo_e_da_imagem}
	Sejam $V$ e $W$ $\cp{K}$-espaços vetorias com $\dim_\cp{K}V$ finita. Seja $T : V \to W$ uma transformação linear. Então
	\[
		\dim_\cp{K}V = \dim_\cp{K}\ker T + \dim_\cp{K}\im T.
	\]
\end{teorema}
\begin{prova}
	Suponha primeiro que $\ker T \ne \{0_V\}$. Seja $\mathcal{B}_1 = \{u_1, \dots, u_n\}$ uma base de $\ker T$. Podemos completar o conjunto $\mathcal{B}_1$ até obter uma base $\mathcal{B} = \{u_1, \dots, u_n, v_1, \dots, v_r\}$ de $V$. Para demonstrar o teorema precisamos encontrar uma base de $W$ com $r$ elementos. Considere então os seguintes elementos de $W$:
	\[
		T(u_1), \dots, T(u_n), T(v_1), \dots, T(v_r).
	\]
	Como $u_i \in \ker T$, então $T(u_i) = 0_W$ para $i = 1$, \dots, $n$. Assim pelo Lema \ref{transformacao_gera_imagem}, $\mathcal{B}_2 = \{T(v_1), \dots, T(v_r)\}$ é um conjunto gerador de $\im T$. Precisamos mostrar que $\mathcal{B}_2$ é L.I. para que seja uma base de $\im T$. Para isso, sejam $\lambda_1$, \dots, $\lambda_r \in \cp{K}$ tais que
	\[
		\lambda_1T(v_1) + \cdots + \lambda_rT(v_r) = 0_W.
	\]
	Daí $\lambda_1v_1 + \cdots + \lambda_rv_r \in \ker T$. Então existem $\alpha_1$, \dots, $\alpha_n \in \cp{K}$ tais que
	\[
		\lambda_1v_1 + \cdots + \lambda_rv_r = \alpha_1u_1 + \cdots + \alpha_nu_n,
	\]
	ou seja, $\lambda_1 = \cdots = \lambda_r = \alpha_1 = \cdots = \alpha_n = 0_\cp{K}$ pois $\{u_1, \dots, u_n, v_1, \dots, v_r\}$ é uma base de $V$. Portanto $\{T(v_1), \dots, T(v_r)\}$ é uma base de $\im T$. Logo
	\[
		\dim_\cp{K}V = \dim_\cp{K}\ker T + \dim_\cp{K}\im T.
	\]
	Agora, se $\ker T = \{0_V\}$, considere $\mathcal{B} = \{u_1, \dots, u_n\}$ uma base de $V$. De maneira análoga ao caso anterior, mostra-se que $\{T(u_1), \dots, T(u_n)\}$ é uma base de $\im T$. Logo
	\[
		\dim_\cp{K}V = \dim_\cp{K}\ker T + \dim_\cp{K}\im T.
	\]
\end{prova}

\section{Isomorfismos}

\begin{definicao}
	Sejam $V$ e $W$ espaços vetoriais sobre o corpo $\cp{K}$ e seja $T : V \to W$ uma transformação linear. Se $T$ é um isomorfismo, então dizemos que $V$ e $W$ são \textbf{espaços vetoriais isomorfos} e denotamos por $V \cong W$.
\end{definicao}

\begin{proposicao}
	Sejam $V$ e $W$ espaços vetoriais sobre $\cp{K}$ tais que $\dim_\cp{K}V = \dim_\cp{K}W = n < \infty$ e seja $T : V \to W$ uma transformação linear. Então as seguintes afirmações são equivalentes:
	\begin{enumerate}
		\item $T$ é um isomorfismo.
		\item $T$ é injetora.
		\item $T$ é sobrejetora.
	\end{enumerate}
\end{proposicao}
\begin{prova}
	As implicações $(i) \Rightarrow (ii) \Rightarrow (iii)$ seguem da definição de isomorfismo.
	\begin{enumerate}
		\item[$(ii) \Rightarrow (i)$] Suponha que $T$ é injetora. Queremos mostrar que $T$ é um isomorfismo. Assim só precisamos mostrar que $T$ é sobrejetora. Como $T$ é injetora, então pela Proposição \ref{caracteriza_transformacao_injetora}, temos $\ker T = \{0_V\}$. Assim, pelo Teorema \ref{teorema_do_nucleo_e_da_imagem} temos
		\[
			\dim_\cp{K}\im T = \dim_\cp{K}V = \dim_\cp{K}W = n,
		\]
		logo $\im T = W$, isto é, $T$ é sobrejetora. Assim $T$ é um isomorfismo.

		\item[$(iii) \Rightarrow (i)$] Suponha que $T$ é sobrejetora. Queremos mostrar que $T$ é um isomorfismo. Para isso, basta mostrar que $T$ é injetora. Como $T$ é sobrejetora, então $\im T = W$ e daí $\dim_\cp{K}\im T = n$. Assim, pelo Teorema \ref{teorema_do_nucleo_e_da_imagem} devemos ter $\dim_\cp{K}\ker T = 0$, ou seja, $\ker T = \{0_V\}$ e então $T$ é injetora. Logo, $T$ é um isomorfismo.
	\end{enumerate}
\end{prova}

\begin{exemplos}
	\begin{enumerate}
		\item Considere a transformação linear $T : \real^2 \to \real^3$ dada por $T(x,y) = (-y, x, x+y)$. Pelo Teorema \ref{teorema_do_nucleo_e_da_imagem}, sabemos que $T$ não pode ser sobrejetora, logo $\dim_\real\im T \le 2$. Agora
		\[
			(x,y) \in \ker T \Leftrightarrow T(x,y) = (0,0,0) \Leftrightarrow (-y,x,x+y) = (0,0,0) \Leftrightarrow x = y = 0.
		\]
		Daí $\ker T = \{(0,0)\}$ e então $T$ é injetora.

		Temos
		\[
			\im T = \{(-y,x,x+y) \in \real^3 \mid x, y \in \real\}.
		\]
		Note que
		\[
			(-y,x,x+y) = y(-1,0,1) + x(0,1,1)
		\]
		e então $\{(-1,0,1);(0,1,1)\}$ gera $\im T$ e como estes vetores não são múltiplos um do outro, eles formam uma base para $\im T$. Assim $\im T = [(-1,0,1);(0,1,1)]$ e $\dim_\real\im T = 2$.

		\item Considere o espaço vetorial $\mathcal{P}(\complex)$ sobre $\complex$ e seja $\mathcal{D} : \mathcal{P}(C) \to \mathcal{P}(C)$ dada por $\mathcal{D}(p(x)) = p'(x)$ a derivada de $p(x)$. Observe que $\ker D = \{p(x) = a \mid a \in \complex\}$. Além disso, $\im \mathcal{D} = \mathcal{P}(\complex)$ pois todo elemento de $\mathcal{P}(\complex)$ tem uma primitiva.
	\end{enumerate}
\end{exemplos}