%!TEX program = xelatex
%!TEX root = Algebra_Linear.tex
%%Usar makeindex -s indexstyle.ist Algebra_Linear.idx arquivo no terminal para gerar o {\'\i}ndice remissivo agrupado por inicial
%%Ap\'os executar pdflatex arquivo

\chapter{Transforma\c{c}\~oes Lineares}

Em todo esse cap{\'\i}tulo $\cp{K}$ denotar\'a um corpo.

\section{Conceitos B\'asicos}

\begin{definicao}
	Sejam $(V, +, \cdot)$ e $(W, \oplus, \otimes)$ espa\c{c}os vetoriais sobre um corpo $\cp{K}$. Uma fun\c{c}\~ao $T : V \to W$ \'e uma \textbf{transforma\c{c}\~ao linear} se
	\begin{enumerate}
		\item $T(u_1 + u_2) = T(u_1) \oplus T(u_2)$ para todos $u_1$, $u_2 \in V$;
		\item $T(\lambda \cdot u) = \lambda \otimes T(u)$ para todo $\lambda \in \cp{K}$ e todo $u \in V$.
	\end{enumerate}
\end{definicao}

\begin{observacao}
	Para simplificar a nota\c{c}\~ao, vamos adotar os mesmos s{\'\i}mbolos para indicar a soma e o produto por escalar nos espa\c{c}os vetoriais que aparecer\~ao no decorrer do texto. No entanto, o leitor deve estar ciente que estes s{\'\i}mbolos podem ter significados diferentes, dependendo do espa\c{c}o vetorial em quest\~ao.
\end{observacao}

\begin{lema}
	Sejam $V$ e $W$ espa\c{c}os vetoriais sobre $\cp{K}$. Ent\~ao uma fun\c{c}\~ao $T : V \to W$ \'e uma transforma\c{c}\~ao linear se, e somente se,
	\[
		T(\lambda u_1 + u_2) = \lambda T(u_1) + T(u_2),
	\]
	para todos $u_1$, $u_2 \in V$ e todo $\lambda \in \cp{K}$.
\end{lema}
\begin{prova}
	Deixada a cargo do leitor.
\end{prova}

\begin{lema}
	Sejam $V$ e $W$ espa\c{c}os vetoriais sobre $\cp{K}$ e $T : V \to W$ uma transforma\c{c}\~ao linear. Ent\~ao:
	\begin{enumerate}\label{transformacao_linear_propriedades_basicas}
		\item $T(0_V) = 0_W$, onde $0_V$ e $0_W$ denotam os vetores nulos de $V$ e $W$, respectivamente.

		\item $T(-u) = -T(u)$, para cada $u \in V$.

		\item $T(\sum_{i=1}^m\alpha_iu_i) = \sum_{i=1}^mT(u_i)$, onde $\alpha_i \in \cp{K}$ e $u_i \in V$ para $i = 1$, \dots, $m$.
	\end{enumerate}
\end{lema}
\begin{prova}
	\begin{enumerate}
		\item Note que
		\[
			0_W + T(0_V) = T(0_V + 0_V) = T(0_V) + T(0_V),
		\]
		ou seja, $T(0_V) = 0_W$.

		\item Basta observar que $-u = (-1_\cp{K})u$ e da{\'\i}
		\[
			T(-u) = T((-1_\cp{K})u) = -1_\cp{K}T(u) = -T(u).
		\]

		\item Por indu\c{c}\~ao em $m$. Se $m = 2$, ent\~ao
		\[
			T(\alpha_1u_1 + \alpha_2u_2) = T(\alpha_1u_1) + T(\alpha_2u_2) = \alpha_1T(u_1) + \alpha_2T(u_2).
		\]
		Suponha que para $m = p$ tenhamos
		\[
			T(\sum_{i=1}^p\alpha_iu_i) = \sum_{i=1}^pT(u_i).
		\]
		Vamos mostra que \'e v\'alido para $m = p + 1$. De fato,
		\begin{align*}
			T(\sum_{i=1}^{p+1}\alpha_iu_i) &= T([\sum_{i=1}^p\alpha_iu_i] + \alpha_{p + 1}u_{p + 1}) = T(\sum_{i=1}^p\alpha_iu_i) + T(\alpha_{p+1}u_{p+1}) \\ &= \sum_{i=1}^p\alpha_iT(u_i) + \alpha_{p+1}T(u_{p+1}) = \sum_{i=1}^{p+1}\alpha_iT(u_i).
		\end{align*}
	\end{enumerate}
\end{prova}

\begin{exemplo}
	\begin{enumerate}
		\item Sejam $V$ e $W$ $\cp{K}$-espa\c{c}os vetoriais. A fun\c{c}\~ao $T : V \to W$ dada por $T(u) = 0_W$ para todo $u \in V$ \'e uma transforma\c{c}\~ao linear.
		
		\item Seja $V$ um $\cp{K}$-espa\c{c}o vetorial. A fun\c{c}\~ao $T : V \to V$ dada por $T(u) = u$ para todo $u \in V$ \'e uma transforma\c{c}\~ao linear.
		
		\item Considere $\real$ como um $\real$-espa\c{c}o vetorial. Dado $a \in \real$, defina $T_a : \real \to \real$ por $T_a(x) = ax$ para todo $x \in \real$. Ent\~ao $T_a$ \'e uma transforma\c{c}\~ao linear. Agora, seja $T : \real \to \real$ dada por $T(x) = e^x$. Ent\~ao $T$ n\~ao \'e uma transforma\c{c}\~ao linear pois $T(0) \ne 0$.
		
		\item Sejam $\cp{K}^3$ e $\cp{M}_2(\cp{K})$ $\cp{K}$-espa\c{c}os vetoriais. Defina $T : \cp{K}^3 \to \cp{M}_2(\cp{K})$ por
		\[
			T(a, b, c) = \begin{bmatrix}
				a+b & 0_\cp{K}\\
				0_\cp{K} & c - b
			\end{bmatrix}.
		\]
		Ent\~ao $T$ \'e uma transforma\c{c}\~ao linear. De fato, dados $(a, b, c)$, $(d, e, f) \in \cp{K}^3$ e $\lambda \in \cp{K}$ temos
		\begin{align*}
			T(\lambda(a,b,c) + (d,e,f)) &= T(\lambda a + d, \lambda b + e, \lambda c + f) \\ &= \begin{bmatrix}
				(\lambda a + d) + (\lambda b + e) & 0_\cp{K}\\
				0_\cp{K} & (\lambda c + f) - (\lambda b + e)
			\end{bmatrix} \\ &= \begin{bmatrix}
				\lambda a + \lambda b & 0_\cp{K}\\
				0_\cp{K} & \lambda c - \lambda b
			\end{bmatrix} + \begin{bmatrix}
				d + e & 0_\cp{K}\\
				0_\cp{K} & f - e
			\end{bmatrix} \\ &= \lambda T(a,b,c) + T(d,e,f).
		\end{align*}

		\item Seja $\mathcal{P}(\complex)$ um $\complex$-espa\c{c}o vetorial e considere $D : \mathcal{P}(\complex) \to \mathcal{P}(\complex)$ dado por
		\[
			D(a_0 + a_1x + a_2x^2 + \cdots + a_nx^n) = a_1 + 2a_2x + \cdots + na_nx^{n-1}.
		\]
		Ent\~ao $D$ \'e uma transforma\c{c}\~ao linear.

		\item Seja $\mathcal{C}([a,b], \real) = \{ f : [a,b] \to \real \mid f \mbox{ \'e uma fun\c{c}\~ao cont{\'\i}nua}\}$. \'E imediato verificar que $\mathcal{C}([a,b], \real)$ \'e um $\real$-espa\c{c}o vetorial. Defina $T : \mathcal{C}([a,b], \real) \to \real$ por
		\[
			T(f(x)) = \int_a^bf(x)dx.
		\]
		Ent\~ao $T$ \'e uma transforma\c{c}\~ao linear.

		\item Sejam $a_1$, \dots, $a_n \in \cp{K}$. Defina $T : \cp{K}^n \to \cp{K}$ por
		\[
			T(x_1, \dots, x_n) = \sum_{i=1}^na_ix_i.
		\]
		\'E imediato verificar que $T$ \'e uma transforma\c{c}\~ao linear. Denote por $e_i$ o elemento de $\cp{K}^n$ contendo $1_\cp{K}$ na posi\c{c}\~ao $i$ e $0_\cp{K}$ das demais. Ent\~ao $\{e_1, \dots, e_n\}$ \'e uma base de $\cp{K}^n$ e
		\[
			T(e_i) = a_i
		\]
		para $i = 1$, \dots, $n$. Agora, se $S : \cp{K}^n \to \cp{K}$ \'e uma transforma\c{c}\~ao linear, ent\~ao pelo Lema \ref{transformacao_linear_propriedades_basicas} item (c)
		\[
			S(x_1, \dots, x_n) = S(\sum_{i=1}^nx_ie_i) = \sum_{i=1}^nS(e_i)
		\]
		onde $S(e_i) \in \cp{K}$ para $i = 1$, \dots, $n$. Logo qualquer transforma\c{c}\~ao linear de $T : \cp{K}^n \to \cp{K}$ \'e da forma
		\[
			T(x_1, \dots, x_n) = \sum_{i=1}^na_ix_i,
		\]
		para determinados escalares $a_1$, \dots, $a_n \in \cp{K}$. Isto \'e, para determinarmos a transforma\c{c}\~ao $T$ s\'o precisamos conhecer seus valores na base $\{e_1, \dots, e_n\}$ de $\cp{K}^n$.
	\end{enumerate}
\end{exemplo}

\begin{teorema}\label{existencia_de_transformacao_unica_dado_valores}
	Sejam $V$ e $W$ $\cp{K}$-espa\c{c}os vetoriais. Se $\{u_1, \dots, u_n\}$ \'e uma base de $V$ e se $\{w_1, \dots, w_n\} \subseteq W$, ent\~ao existe uma \'unica transforma\c{c}\~ao linear $T : V \to W$ tal que $T(u_i) = w_i$ para cada $i = 1$, \dots, $n$.
\end{teorema}
\begin{prova}
	Dado $v \in V$, como $\{u_1, \dots, u_n\}$ \'e uma base de $V$, ent\~ao sabemos que existem \'unicos $\alpha_1$, \dots, $\alpha_n \in \cp{K}$ tais que
	\[
		v = \alpha_1u_1 + \cdots + \alpha_nu_n.
	\]
	Defina ent\~ao $T : V \to W$ por
	\[
		T(v) = T(\alpha_1u_1 + \cdots + \alpha_nu_n) = \alpha_1w_1 + \cdots + \alpha_nw_n.
	\]
	A unicidade dos escalares $\alpha_1$, \dots, $\alpha_n \in \cp{K}$ garante que $T$ est\'a bem definida, isto \'e, um mesmo elemento de $V$ n\~ao pode ter duas imagens distintas.

	Agora, note que $T(u_i) = w_i$ para cada $i = 1$, \dots, $n$. Assim precisamos mostrar que $T$ \'e linear. Sejam $v_1 = \sum_{i=1}^n\alpha_iu_i$, $v_2 = \sum_{i=1}^n\beta_iu_i$ e $\lambda \in \cp{K}$. Ent\~ao
	\begin{align*}
		T(\lambda v_1 + v_2) &= T(\lambda\sum_{i=1}^n\alpha_iu_i + \sum_{i=1}^n\beta_iu_i) = T(\sum_{i=1}^n(\lambda\alpha_i + \beta_i)u_i) \\ &= \sum_{i=1}^n(\lambda_i\alpha_i + \beta_i)w_i = \lambda\sum_{i=1}^n\alpha_iw_i + \sum_{i=1}^n\beta_iw_i \\ &= \lambda T(v_1) + T(v_2).
	\end{align*}
	Logo $T$ \'e uma transforma\c{c}\~ao linear.

	Resta mostrar que $T$ \'e \'unica. Suponha que exista uma transforma\c{c}\~ao linear $S : V \to W$ tal que $S(u_i) = w_i$ para todo $i = 1$, \dots, $n$. Para $v = \sum_{i=1}^n\alpha_iu_i$ temos
	\[
		S(v) = S(\sum_{i=1}^n\alpha_iu_i) = \sum_{i=1}^n\alpha_iS(u_i) = \sum_{i=1}^n\alpha_iw_i = \sum_{i=1}^n\alpha_iT(u_i) = T(\sum_{i=1}^n\alpha_iu_i) = T(v)
	\]
	para todo $v \in V$. Logo $T=S$, isto \'e, existe uma \'unica transforma\c{c}\~ao linear que satisfaz as condi\c{c}\~oes do teorema.
\end{prova}

\begin{exemplo}
	Os vetores $v_1 = (1,2)$ e $v_2 = (3,4)$ s\~ao L.I em $\real^2$ e assim formam uma base de $\real^2$. Assim pelo Teorema \ref{existencia_de_transformacao_unica_dado_valores}, sabemos que existe uma \'unica transforma\c{c}\~ao linear $T : \real^2 \to \real^3$ tal que
	\begin{align*}
		T(v_1) &= T(1,2) = (3,2,1)\\
		T(v_2) &= T(3,4) = (6,5,4).
	\end{align*}
	Determine $T(1,0)$.
\end{exemplo}
\begin{solucao}
	Inicialmente escrevemos $(1,0)$ como combina\c{c}\~ao linear de $v_1$ e $v_2$:
	\[
		(1, 0) = \alpha(1,2) + \beta(3,4).
	\]
	Obtendo $\alpha = -2$ e $\beta = 1$. Assim
	\[
		T(1,0) = T(-2(1,2) + (3,4)) = -2T(1,2) + T(3,4) = (0,1,2).
	\]
\end{solucao}

\begin{definicao}
	Sejam $V$ e $W$ $\cp{K}$-espa\c{c}os vetoriais e $T : V \to W$ uma transforma\c{c}\~ao linear.
	\begin{enumerate}
		\item O conjunto
		\[
			\ker T = \{u \in V \mid T(u) = 0_W\}
		\]
		\'e chamado de \textbf{kernel} ou \textbf{n\'ucleo} de $T$. (O n\'ucleo de $T$ tamb\'em pode ser denotado por $Nuc\ T$.)

		\item O conjunto
		\[
			\im T = \{u \in W \mid \mbox{ existe } v \in V \mbox{ tal que } T(v) = u\}
		\]
		\'e chamado de \textbf{imagem} de $T$.
	\end{enumerate}
\end{definicao}

\begin{proposicao}
	Sejam $V$ e $W$ $\cp{K}$-espa\c{c}os vetoriais e $T : V \to W$ uma transforma\c{c}\~ao linear. Ent\~ao:
	\begin{enumerate}
		\item $\ker T$ \'e um subespa\c{c}o de $V$;
		\item $\im T$ \'e um subespa\c{c}o de $W$.
	\end{enumerate}
\end{proposicao}
\begin{prova}
	\begin{enumerate}
		\item Inicialmente $\ker T \ne \emptyset$ pois $T(0_V) = 0_W$, ou seja, $0_V \in \ker T$. Agora, sejam $u_1$, $u_2 \in \ker T$ e $\lambda \in \cp{K}$. Precisamos mostrar que $\lambda u_1 + u_2 \in \cp{K}$, isto \'e, precisamos mostrar que $\lambda u_1 + u_2 \in \ker T$. Temos
		\[
			T(\lambda u_1 + u_2) = \lambda T(u_1) + T(u_2) = 0_W.
		\]
		Logo, $\ker T$ \'e um subespa\c{c}o de $V$.

		\item Inicialmente $0_W \in \im T$ pois $0_W = T(0_V)$ e da{\'\i} $\im T \ne \emptyset$. Sejam $w_1$, $w_2 \in \im T$ e $\lambda \in \cp{K}$. Ent\~ao existem $u_1$, $u_2 \in V$ tais que $w_1 = T(u_1)$ e $w_2 = T(u_2)$. Assim
		\[
			\lambda w_1 + w_2 = \lambda T(u_1) + T(u_2) = T(\lambda u_1) + T(u_2) = T(\lambda u_1 + u_2)
		\]
		e ent\~ao $\lambda w_1 + w_2 \in \im T$. Portanto, $\im T$ \'e um subespa\c{c}o de $W$.
	\end{enumerate}
\end{prova}

\begin{exemplos}
	\begin{enumerate}
		\item Seja $T : \real^3 \to \cp{M}_2(\real)$ dada por
		\[
			T(a,b,c) = \begin{bmatrix}
				a + b & 0\\
				0 & c - b
			\end{bmatrix}.
		\]
		Determine $\ker T$ e $\im T$.
		\begin{solucao}
			Temos
			\[
				T(a,b,c) = \begin{bmatrix}
					0 & 0\\
					0 & 0
				\end{bmatrix}
			\]
			se, e s\'o se, $a = -b$ e $c = b$. Da{\'\i}
			\[
				\ker T = \{(a,b,c) \in \real^3 \mid a = -b, c = b\} = \{(-b,b,b) \mid b \in \real\}.
			\]
			Note que $\{(1,1,1)\}$ \'e uma base de $\ker T$, ou seja, $\dim_\real \ker T = 1$.

			Agora,
			\begin{align*}
				\im T &= \{v \in \cp{M}_2(\real) \mid \mbox{ existe } u \in \real^3 \mbox{ de modo que } T(u) = v\}\\
				\im T &= \left\{ \begin{bmatrix}
					a + b & 0\\
					0 & c - b
				\end{bmatrix} \mid a, b, c \in \real\right\}.
			\end{align*}
			Assim temos
			\begin{align*}
				\begin{bmatrix}
					a + b & 0\\
					0 & c - b
				\end{bmatrix} &= (a + b) \begin{bmatrix}
					1 & 0\\ 0 & 0
				\end{bmatrix} + (c - b) \begin{bmatrix}
					0 & 0\\ 0 & 1
				\end{bmatrix}
			\end{align*}
			e \'e f\'acil ver que
			\[
				\mathcal{B}' = \left\{ \begin{bmatrix}
					1 & 0\\ 0 & 0
				\end{bmatrix}; \begin{bmatrix}
					0 & 0\\ 0 & 1
				\end{bmatrix}\right\}
			\]
			\'e um conjunto gerador de $\im T$ e \'e L.I., ou seja, \'e uma base de $\im T$, com isso $\dim_\real\im T = 2$. Observe que
			\[
				\dim_\real\ker T + \dim_\real\im T = 3 = \dim_\real\real^3.
			\]
		\end{solucao}

		\item Seja $T : \real^2 \to \real$ dada por $T(x,y) = x + y$. Determine $\ker T$ e $\im T$.
		\begin{solucao}
			Temos
			\begin{align*}
				\ker T &= \{(x,y) \in \real^2 \mid T(x,y) = 0\} = \{(x,y) \in \real^2 \mid x + y = 0\}\\
				\ker T &= \{(x,-x) \in \real^2 \mid x \in \real\}.
			\end{align*}
			Assim $\{(1,-1)\}$ \'e uma base de $\ker T$, ou seja, $\dim_\real\ker T = 1$.

			Agora
			\[
				\im T = \{w \in \real \mid \mbox{ existe } (x,y) \in \real^2 \mbox{ tal que } T(x,y) = w\}.
			\]
			Assim dado $w \in \real$ um n\'umero real qualquer, tome o elemento $(w,0) \in \real^2$. Temos $T(w,0) = w + 0 = w$. Logo $\im T = \real$ e ent\~ao $\dim_\real\im T = 1$.

			Novamente temos
			\[
				\dim_\real\ker T + \dim_\real\im T = 2 = \dim_\real\real^2.
			\]
		\end{solucao}
	\end{enumerate}
\end{exemplos}

\begin{definicao}
	Sejam $V$ e $W$ $\cp{K}$-espa\c{c}os vetoriais e $T : V \to W$ uma transforma\c{c}\~ao linear.
	\begin{enumerate}
		\item Dizemos que $T$ \'e \textbf{injetora} se dados $u_1$, $u_2 \in V$ tais que $T(u_1) = T(u_2)$, ent\~ao $u_1 = u_2$. De modo equivalente, se $u_1$, $u_2 \in V$ s\~ao tais que $u_1 \ne u_2$, ent\~ao $T(u_1) \ne T(u_2)$.

		\item Dizemos que $T$ \'e \textbf{sobrejetora} se $\im T = W$. Em outras palavras, $T$ \'e \textbf{sobrejetora} se para todo $w \in W$, existe $u \in V$ tal que $T(u) = w$.

		\item Se $T$ \'e injetora e sobrejetora, ent\~ao dizemos que $T$ \'e um \textbf{isomorfismo}.
	\end{enumerate}
\end{definicao}

\begin{exemplos}
	\begin{enumerate}
		\item A transforma\c{c}\~ao linear $T : \real^2 \to \real$ dada por $T(x,y) = x + y$ \'e sobrejetora, mas n\~ao \'e injetora.

		\item A transforma\c{c}\~ao linear $T : V \to V$ dada por $T(u) = u$ \'e injetora e sobrejetora, ou seja, \'e um isomorfismo.
	\end{enumerate}
\end{exemplos}

\begin{proposicao}\label{caracteriza_transformacao_injetora}
	Sejam $V$ e $W$ $\cp{K}$-espa\c{c}os vetoriais e $T : V \to W$ uma transforma\c{c}\~ao linear. Ent\~ao $T$ \'e injetora se, e somente se, $\ker T = \{0_V\}$.
\end{proposicao}
\begin{prova}
	Suponha que $T$ \'e injetora. Queremos mostrar que $\ker T = \{0_V\}$. Seja ent\~ao $u \in \ker T$. Ent\~ao $T(u) = 0_W$. Mas $T(0_V) = 0_W$ e como $T$ \'e injetora devemos ter $u = 0_V$. Logo $\ker T = \{0_V\}$.

	Agora suponha que $\ker T = \{0_V\}$. Queremos mostrar que $T$ \'e injetora. Para isso, sejam $u_1$, $u_2 \in V$ tais que $T(u_1) = T(u_2)$. Ent\~ao
	\begin{align*}
		T(u_1) &= T(u_2)\\
		T(u_1) - T(u_2) & = 0_W\\
		T(u_1 - u_2) &= 0_W,
	\end{align*}
	isto \'e, $u_1 - u_2 \in \ker T$. Mas $\ker T = \{0_V\}$, logo $u_1 = u_2$. Portanto $T$ \'e injetora.
\end{prova}

\begin{exemplo}
	Seja $T : \cp{K}^4 \to \cp{M}_2(\cp{K})$ dada por
	\begin{align*}
		T(1_\cp{K},0_\cp{K},0_\cp{K},0_\cp{K}) &= \begin{bmatrix}
			1_\cp{K} & 0_\cp{K}\\
			0_\cp{K} & 0_\cp{K}
		\end{bmatrix}; T(0_\cp{K},1_\cp{K},0_\cp{K},0_\cp{K}) = \begin{bmatrix}
			0_\cp{K} & 1_\cp{K}\\
			0_\cp{K} & 0_\cp{K}
		\end{bmatrix}\\
		T(0_\cp{K},0_\cp{K},1_\cp{K},0_\cp{K}) &= \begin{bmatrix}
			0_\cp{K} & 0_\cp{K}\\
			1_\cp{K} & 0_\cp{K}
		\end{bmatrix}; T(0_\cp{K},0_\cp{K},0_\cp{K},1_\cp{K}) = \begin{bmatrix}
			0_\cp{K} & 0_\cp{K}\\
			0_\cp{K} & 1_\cp{K}
		\end{bmatrix}\\
	\end{align*}
	\'E f\'acil ver que $\ker T = \{(0_\cp{K},0_\cp{K},0_\cp{K},0_\cp{K})\}$. Al\'em disso,
	\[
		\im T = \left[\begin{bmatrix}
			1_\cp{K} & 0_\cp{K}\\
			0_\cp{K} & 0_\cp{K}
		\end{bmatrix};\begin{bmatrix}
			0_\cp{K} & 1_\cp{K}\\
			0_\cp{K} & 0_\cp{K}
		\end{bmatrix};\begin{bmatrix}
			0_\cp{K} & 0_\cp{K}\\
			1_\cp{K} & 0_\cp{K}
		\end{bmatrix};\begin{bmatrix}
			0_\cp{K} & 0_\cp{K}\\
			0_\cp{K} & 1_\cp{K}
		\end{bmatrix}\right] = \cp{M}_2(\cp{K}).
	\]
	Da{\'\i} $\dim_\cp{K}\im T = 4$ e novamente
	\[
		\dim_\cp{K}\ker T + \dim_\cp{K}\im T = 4 = \dim_\cp{K}\cp{M}_2(\cp{K}).
	\]
\end{exemplo}

\begin{lema}\label{transformacao_gera_imagem}
	Sejam $V$ e $W$ $\cp{K}$-espa\c{c}os vetoriais e $T : V \to W$ uma transforma\c{c}\~ao linear. Se $\mathcal{B} = \{u_1, \dots, u_n\}$ \'e uma base de $V$, ent\~ao $\{T(u_1), \dots, T(u_n)\}$ gera $\im T$.
\end{lema}
\begin{prova}
	Seja $w \in \im T$. Por defini\c{c}\~ao, existe $u \in V$ tal que $T(u) = w$. Como $\mathcal{B}$ \'e uma base de $V$, ent\~ao existem $\alpha_1$, \dots, $\alpha_n \in \cp{K}$ tais que $u = \alpha_1u_1 + \cdots + \alpha_nu_n$. Da{\'\i}
	\[
		w = T(u) = \alpha_1T(u_1) + \cdots + \alpha_nT(u_n),
	\]
	ou seja, todo vetor de $\im T$ \'e uma combina\c{c}\~ao linear de $T(u_1)$, \dots, $T(u_n)$. Portanto $\im T = [T(u_1), \dots, T(u_n)]$ como quer{\'\i}amos.
\end{prova}

\begin{exemplo}
	Considere $\complex^2$ e $\real^3$ como $\real$-espa\c{c}os vetoriais e seja $T : \complex^2 \to \real^3$ dada por
	\[
		T(a+bi, c+di) = (a - c, b + 2d, a + b - c + 2d)
	\]
	onde $a$, $b$ e $c \in \real$. \'E f\'acil ver que $T$ \'e uma transforma\c{c}\~ao linear. Seja $\{(1,0);(i,0);(0,1);(0,i)\}$ uma base de $\complex^2$. Temos pelo Lema \ref{transformacao_gera_imagem} que $\{T(1,0);T(i,0);T(0,1);T(0,i)\}$ gera $\im T$. Agora
	\begin{align*}
		T(1,0) &= (1,0,1);\quad T(i,0) = (0,1,1)\\
		T(0,1) &= (-1,0,-1);\quad T(0,i) = (0,2,2)
	\end{align*}
	e temos
	\begin{align*}
		T(1,0) &= -T(0,1)\\
		T(i,0) &= 2T(0,i).
	\end{align*}
	Observe que $T(1,0)$ n\~ao \'e m\'ultiplo de $T(0,i)$. Logo o conjunto $\{T(1,0); T(0,i)\}$ \'e uma base de $\im T$.
\end{exemplo}

\begin{teorema}[Teorema do N\'ucleo e da Imagem]\label{teorema_do_nucleo_e_da_imagem}
	Sejam $V$ e $W$ $\cp{K}$-espa\c{c}os vetoriais com $\dim_\cp{K}V$ finita. Seja $T : V \to W$ uma transforma\c{c}\~ao linear. Ent\~ao
	\[
		\dim_\cp{K}V = \dim_\cp{K}\ker T + \dim_\cp{K}\im T.
	\]
\end{teorema}
\begin{prova}
	Suponha primeiro que $\ker T \ne \{0_V\}$. Seja $\mathcal{B}_1 = \{u_1, \dots, u_n\}$ uma base de $\ker T$. Podemos completar o conjunto $\mathcal{B}_1$ at\'e obter uma base $\mathcal{B} = \{u_1, \dots, u_n, v_1, \dots, v_r\}$ de $V$. Para demonstrar o teorema precisamos encontrar uma base de $W$ com $r$ elementos. Considere ent\~ao os seguintes elementos de $W$:
	\[
		T(u_1), \dots, T(u_n), T(v_1), \dots, T(v_r).
	\]
	Como $u_i \in \ker T$, ent\~ao $T(u_i) = 0_W$ para $i = 1$, \dots, $n$. Assim pelo Lema \ref{transformacao_gera_imagem}, $\mathcal{B}_2 = \{T(v_1), \dots, T(v_r)\}$ \'e um conjunto gerador de $\im T$. Precisamos mostrar que $\mathcal{B}_2$ \'e L.I. para que seja uma base de $\im T$. Para isso, sejam $\lambda_1$, \dots, $\lambda_r \in \cp{K}$ tais que
	\[
		\lambda_1T(v_1) + \cdots + \lambda_rT(v_r) = 0_W.
	\]
	Da{\'\i} $\lambda_1v_1 + \cdots + \lambda_rv_r \in \ker T$. Ent\~ao existem $\alpha_1$, \dots, $\alpha_n \in \cp{K}$ tais que
	\[
		\lambda_1v_1 + \cdots + \lambda_rv_r = \alpha_1u_1 + \cdots + \alpha_nu_n,
	\]
	ou seja, $\lambda_1 = \cdots = \lambda_r = \alpha_1 = \cdots = \alpha_n = 0_\cp{K}$ pois $\{u_1, \dots, u_n, v_1, \dots, v_r\}$ \'e uma base de $V$. Portanto $\{T(v_1), \dots, T(v_r)\}$ \'e uma base de $\im T$. Logo
	\[
		\dim_\cp{K}V = \dim_\cp{K}\ker T + \dim_\cp{K}\im T.
	\]
	Agora, se $\ker T = \{0_V\}$, considere $\mathcal{B} = \{u_1, \dots, u_n\}$ uma base de $V$. De maneira an\'aloga ao caso anterior, mostra-se que $\{T(u_1), \dots, T(u_n)\}$ \'e uma base de $\im T$. Logo
	\[
		\dim_\cp{K}V = \dim_\cp{K}\ker T + \dim_\cp{K}\im T.
	\]
\end{prova}

\section{Isomorfismos}

\begin{definicao}
	Sejam $V$ e $W$ espa\c{c}os vetoriais sobre o corpo $\cp{K}$ e seja $T : V \to W$ uma transforma\c{c}\~ao linear. Se $T$ \'e um isomorfismo, ent\~ao dizemos que $V$ e $W$ s\~ao \textbf{espa\c{c}os vetoriais isomorfos} e denotamos por $V \cong W$.
\end{definicao}

\begin{proposicao}\label{equivalencia_isomorfismo_dimensao_finita}
	Sejam $V$ e $W$ espa\c{c}os vetoriais sobre $\cp{K}$ tais que $\dim_\cp{K}V = \dim_\cp{K}W = n < \infty$ e seja $T : V \to W$ uma transforma\c{c}\~ao linear. Ent\~ao as seguintes afirma\c{c}\~oes s\~ao equivalentes:
	\begin{enumerate}
		\item $T$ \'e um isomorfismo.
		\item $T$ \'e injetora.
		\item $T$ \'e sobrejetora.
	\end{enumerate}
\end{proposicao}
\begin{prova}
	As implica\c{c}\~oes $(i) \Rightarrow (ii) \Rightarrow (iii)$ seguem da defini\c{c}\~ao de isomorfismo.
	\begin{enumerate}
		\item[$(ii) \Rightarrow (i)$] Suponha que $T$ \'e injetora. Queremos mostrar que $T$ \'e um isomorfismo. Assim s\'o precisamos mostrar que $T$ \'e sobrejetora. Como $T$ \'e injetora, ent\~ao pela Proposi\c{c}\~ao \ref{caracteriza_transformacao_injetora}, temos $\ker T = \{0_V\}$. Assim, pelo Teorema \ref{teorema_do_nucleo_e_da_imagem} temos
		\[
			\dim_\cp{K}\im T = \dim_\cp{K}V = \dim_\cp{K}W = n,
		\]
		logo $\im T = W$, isto \'e, $T$ \'e sobrejetora. Assim $T$ \'e um isomorfismo.

		\item[$(iii) \Rightarrow (i)$] Suponha que $T$ \'e sobrejetora. Queremos mostrar que $T$ \'e um isomorfismo. Para isso, basta mostrar que $T$ \'e injetora. Como $T$ \'e sobrejetora, ent\~ao $\im T = W$ e da{\'\i} $\dim_\cp{K}\im T = n$. Assim, pelo Teorema \ref{teorema_do_nucleo_e_da_imagem} devemos ter $\dim_\cp{K}\ker T = 0$, ou seja, $\ker T = \{0_V\}$ e ent\~ao $T$ \'e injetora. Logo, $T$ \'e um isomorfismo.
	\end{enumerate}
\end{prova}

\begin{exemplos}
	\begin{enumerate}
		\item Considere a transforma\c{c}\~ao linear $T : \real^2 \to \real^3$ dada por $T(x,y) = (-y, x, x+y)$. Pelo Teorema \ref{teorema_do_nucleo_e_da_imagem}, sabemos que $T$ n\~ao pode ser sobrejetora, logo $\dim_\real\im T \le 2$. Agora
		\[
			(x,y) \in \ker T \Leftrightarrow T(x,y) = (0,0,0) \Leftrightarrow (-y,x,x+y) = (0,0,0) \Leftrightarrow x = y = 0.
		\]
		Da{\'\i} $\ker T = \{(0,0)\}$ e ent\~ao $T$ \'e injetora.

		Temos
		\[
			\im T = \{(-y,x,x+y) \in \real^3 \mid x, y \in \real\}.
		\]
		Note que
		\[
			(-y,x,x+y) = y(-1,0,1) + x(0,1,1)
		\]
		e ent\~ao $\{(-1,0,1);(0,1,1)\}$ gera $\im T$ e como estes vetores n\~ao s\~ao m\'ultiplos um do outro, eles formam uma base para $\im T$. Assim $\im T = [(-1,0,1);(0,1,1)]$ e $\dim_\real\im T = 2$.

		\item Considere o espa\c{c}o vetorial $\mathcal{P}(\complex)$ sobre $\complex$ e seja $\mathcal{D} : \mathcal{P}(\complex) \to \mathcal{P}(\complex)$ dada por $\mathcal{D}(p(x)) = p'(x)$ a derivada de $p(x)$. Observe que $\ker D = \{p(x) = a \mid a \in \complex\}$. Al\'em disso, $\im \mathcal{D} = \mathcal{P}(\complex)$ pois todo elemento de $\mathcal{P}(\complex)$ tem uma primitiva.

		\item Seja
		\[
			\cp{K}^\n = \{(a_i)_{i \in \n} = (a_1, a_2, \dots, a_n, \dots) \mid a_i \in \cp{K}, i \ge 1\}
		\]
		onde $\cp{K} = \real$ ou $\cp{K} = \complex$. Defina
		\begin{itemize}
			\item $(a_1,a_2,\dots,a_n,\dots) + (b_1,b_2,\dots,b_n,\dots) = (a_1+b_1,a_2+b_2,\dots,a_n+b_n,\dots)$

			\item $\lambda(a_1,a_2,\dots,a_n,\dots) = (\lambda a_1,\lambda a_2,\dots,\lambda a_n,\dots)$
		\end{itemize}
		para todos $(a_1,a_2,\dots,a_n,\dots)$, $(b_1,b_2,\dots,b_n,\dots) \in \cp{K}^\n$ e todo $\lambda \in \cp{K}$. \'E f\'acil ver que $\cp{K}^\n$ \'e um $\cp{K}$-espa\c{c}o vetorial. Defina $T : \cp{K}^\n \to \cp{K}^\n$ dada por
		\[
			T((a_i)_{i\in\n}) = ((a_i)_{i+1\in\n}) = (a_2,a_3,\dots,a_n,\dots).
		\]
		\'E imediato ver que $T$ \'e uma transforma\c{c}\~ao linear. Al\'em disso, $T$ \'e sobrejetora pois dada uma sequ\^encia $(y_i)_{i\in\n}$, considere $x = (0,y_1,y_2,\dots,y_n,\dots) \in \cp{K}^\n$. Temos
		\[
			T(x) = (y_1,y_2,\dots,y_n,\dots),
		\]
		assim $\im T = \cp{K}^\n$. Al\'em disso, $T$ n\~ao \'e injetora pois dado $(x_1,0,0,\dots,0,\dots) \in \cp{K}^\n$ tempos
		\[
			T(x_1,0,0,\dots,0,\dots) = (0,0,0,\dots,0,\dots)
		\]
		e ent\~ao
		\[
			\ker T = \{(x_1,0,0,\dots,0) \mid x_1 \in \cp{K}\}.
		\]
	\end{enumerate}
\end{exemplos}

\begin{teorema}\label{teorema_espacos_isomorfos}
	Dois $\cp{K}$-espa\c{c}os vetoriais de mesma dimens\~ao finita s\~ao isomorfos.
\end{teorema}
\begin{prova}
	Se ambos os espa\c{c}os forem nulos, nada h\'a a fazer. Sejam $V$ e $W$ espa\c{c}os vetoriais sobre $\cp{K}$ tais que $\dim_\cp{K}V = \dim_\cp{K}W = n < \infty$. Seja $\mathcal{A} = \{v_1,\dots,v_n\}$ e $\mathcal{B} = \{w_1,\dots,w_n\}$ bases de $V$ e $W$, respectivamente. Pelo Teorema \ref{existencia_de_transformacao_unica_dado_valores} sabemos que existe uma \'unica transforma\c{c}\~ao linear $T : V \to W$ tal que
	\[
		T(v_i) = w_i
	\]
	para $i=1$,\dots, $n$.

	Vamos mostrar que $T$ \'e sobrejetora e da{\'\i} pela Proposi\c{c}\~ao \ref{equivalencia_isomorfismo_dimensao_finita}, $T$ ser\'a um isomorfismo. Para isto, seja $y \in W$ um elemento qualquer de $W$. Como $\mathcal{B}$ \'e uma base de $W$, existem escalares $\alpha_1$, \dots, $\alpha_n \in \cp{K}$ tais que
	\[
		y = \sum_{i=1}^n\alpha_iw_i.
	\]
	Seja $x \in V$ dado por
	\[
		x = \sum_{i=1}^n\alpha_iv_i
	\]
	pois $\mathcal{A}$ \'e uma base de $V$. Temos
	\[
		T(x) = T(\sum_{i=1}^m\alpha_iv_i) = \sum_{i=1}^n\alpha_iT(v_i) = \sum_{i=1}^n\alpha_iw_i = y,
	\]
	logo $\im T = W$, ou seja, $T$ \'e sobrejetora.

	Portanto $V$ e $W$ s\~ao isomorfos.
\end{prova}

\begin{corolario}
	Todo $\cp{K}$-espa\c{c}o vetorial de dimens\~ao $n \ge 1$ \'e isomorfo a $\cp{K}^n$.
\end{corolario}

\begin{exemplos}
	\begin{enumerate}
		\item $\mathcal{P}_n(\cp{K}) \cong \cp{K}^{n+1}$.
		\item $\cp{M}_{p\times q}(\cp{K}) \cong \cp{K}^{pq}$.
	\end{enumerate}
\end{exemplos}

\section{Transforma\c{c}\~oes Lineares e Matrizes} % (fold)
\label{sec:transformacoes_lineares_e_matrizes}

Sejam $V$ um espa\c{c}o vetorial sobre $\cp{K}$ de dimens\~ao $n \ge 1$ e $\mathcal{B} = \{v_1,\dots,v_n\}$ uma base de $V$. Sabemos que cada elemento de $V$ se escreve de modo \'unico como combina\c{c}\~ao linear dos elementos de $\mathcal{B}$, isto \'e, dado $u \in V$ existem \'unicos escalares $\alpha_1$, \dots, $\alpha_n \in \cp{K}$ tais que
\[
	u = \alpha_1v_1 + \cdots + \alpha_nv_n.
\]

Assim vamos fixar uma ordem para os elementos da base $\mathcal{B}$ e por isso vamos cham\'a-la de \textbf{base ordenada} de $V$. Pela unicidade dos elementos $\alpha_1$, \dots, $\alpha_n$ acima, podemos denotar o vetor $u$ por
\[
	[u]_\mathcal{B} = (\alpha_1,\dots,\alpha_n)_\mathcal{B}
\]
e dizemos que $\alpha_1$, \dots, $\alpha_n$ s\~ao as \textbf{coordenadas de $u$ em rela\c{c}\~ao \`a base ordenada $\mathcal{B}$}.

\begin{exemplos}
	\begin{enumerate}
		\item Considere $V = \complex^2$ como um $\complex$-espa\c{c}o vetorial e seja $\mathcal{B} = \{(1,i);(i,0)\}$ uma base de $\complex^2$ \textit{(Verifique!)}. Dado o vetor $v = (i, 2+i) \in \complex^2$, quais suas coordenadas em rela\c{c}\~ao a tal base?
		\begin{solucao}
			As coordenadas de $v$ em rela\c{c}\~ao \`a base $\mathcal{B}$ ser\~ao $[v]_\mathcal{B} = (\alpha_1,\alpha_2)_\mathcal{B}$ onde $\alpha_1$, $\alpha_2 \in \complex^2$ s\~ao tais que
			\[
				v = \alpha_1(1,i) + \alpha_2(i,0).
			\]
			Resolvendo o sistema resultante obtemos $\alpha_1 = 1 - 2i$ e $\alpha_2 = 3 + i$. Assim
			\[
				[v]_\mathcal{B} = (1 - 2i, 3 + i)_\mathcal{B}.
			\]
		\end{solucao}

		\item Agora considere $V = \complex^2$ como um $\real$-espa\c{c}o vetorial e seja $\mathcal{A} = \{(1,1);(i,0);(1,i);(0,1)\}$ uma base de $\complex^2$ \textit{(Verifique!)}. Dado o vetor $v = (i,2+i) \in \complex^2$, quais suas coordenadas em rela\c{c}\~ao a tal base?
		\begin{solucao}
			As coordenadas de $v$ em rela\c{c}\~ao \`a base $\mathcal{A}$ ser\~ao $[v]_\mathcal{A} = (\alpha_1,\alpha_2,\alpha_3,\alpha_4)_\mathcal{A}$ onde $\alpha_1$, $\alpha_2$, $\alpha_3$, $\alpha_4 \in \complex$ s\~ao tais que
			\[
				v = \alpha_1(1,1) + \alpha_2(i,0) + \alpha_3(1,i) + \alpha_4(0,1).
			\]
			Resolvendo o sistema resultante obtemos $\alpha_1 = -1$, $\alpha_2 = 1$, $\alpha_3 = 1$ e $\alpha_4 = 3$. Assim
			\[
				[v]_\mathcal{A} = (-1,1,1,3)_\mathcal{A}.
			\]
		\end{solucao}
	\end{enumerate}
\end{exemplos}

Agora, sejam $V$ e $W$ espa\c{c}os vetoriais sobre $\cp{K}$ tais que $\dim_\cp{K}V = p \ge 1$ e $\dim_\cp{K} W = q \ge 1$. Vamos fixar bases ordenadas $\mathcal{B}_1 = \{v_1,\dots,v_p\}$ e $\mathcal{B}_2 = \{w_1,\dots,w_q\}$ de $V$ e $W$, respectivamente. Seja $T : V \to W$ uma transforma\c{c}\~ao lineares. Sabemos pelo Teorema \ref{existencia_de_transformacao_unica_dado_valores} que $T$ fica completamente determinada se conhecermos seus valores na base de $V$. Assim
\begin{align*}
	T(v_1) &= b_{11}w_1 + b_{21}w_2 + \cdots + b_{q1}w_q = \sum_{i=1}^qb_{i1}w_i\\
	T(v_2) &= b_{12}w_1 + b_{22}w_2 + \cdots + b_{q2}w_q = \sum_{i=1}^qb_{i2}w_i\\
	&\vdots\\
	T(v_p) &= b_{1p}w_1 + b_{2p}w_2 + \cdots + b_{qp}w_q = \sum_{i=1}^qb_{ip}w_i
\end{align*}
para certos $b_{ij} \in \cp{K}$ onde $1 \le i \le q$ e $1 \le j \le p$.

Agora, seja $x \in V$. Ent\~ao
\[
	x = \alpha_1v_1 + \alpha_2v_2 + \cdots + \alpha_pv_p
\]
onde $\alpha_1$, $\alpha_2$, \dots, $\alpha_p \in \cp{K}$. Da{\'\i}
\begin{align*}
	T(x) &= T(\alpha_1v_1 + \alpha_2v_2 + \cdots + \alpha_pv_p) = \alpha_1T(v_1) + \alpha_2T(v_2) + \cdots + \alpha_pT(v_p)\\
	&= \alpha_1\sum_{i=1}^qb_{i1}w_i + \alpha_2\sum_{i=1}^qb_{i2}w_i + \cdots + \alpha_p\sum_{i=1}^qb_{ip}w_i\\
	&= \sum_{j=1}^p\alpha_j\left(\sum_{i=1}^qb_{ij}w_i\right) = \sum_{j=1}^p\sum_{i=1}^q(\alpha_jb_{ij}w_i)\\
	&= \sum_{i=1}^q\left(\sum_{j=1}^p\alpha_jb_{ij}\right)w_i.
\end{align*}
Note que os escalares $\sum_{j=1}^p\alpha_jb_{1j}$, $\sum_{j=1}^p\alpha_jb_{2j}$, \dots, $\sum_{j=1}^p\alpha_jb_{qj}$ s\~ao as coordenadas do vetor $T(x)$ em rela\c{c}\~ao \`a base $\mathcal{B}_2$, isto \'e,
\[
	[T(x)]_{\mathcal{B}_2} = \left(\sum_{j=1}^p\alpha_jb_{1j}, \sum_{j=1}^p\alpha_jb_{2j}, \dots, \sum_{j=1}^p\alpha_jb_{qj}\right).
\]
Por outro lado,
\[
	\underbrace{\begin{bmatrix}
		b_{11} & b_{21} & \dots & b_{q1}\\
		b_{12} & b_{22} & \dots & b_{q2}\\
		\vdots & & \ddots & \vdots\\
		b_{1p} & b_{2p} & \dots & b_{qp}
	\end{bmatrix}}_{A}\begin{bmatrix}
		\alpha_1\\
		\alpha_2\\
		\vdots\\
		\alpha_n
	\end{bmatrix} = \begin{bmatrix}
		b_{11}\alpha_1 + b_{21}\alpha_2 + \cdots + b_{q1}\alpha_p\\
		b_{12}\alpha_1 + b_{22}\alpha_2 + \cdots + b_{q2}\alpha_p\\
		\vdots \\
		b_{1p}\alpha_1 + b_{2p}\alpha_2 + \cdots + b_{qp}\alpha_p
	\end{bmatrix}
\]
e da{\'\i} podemos escrever
\[
	[T(x)]_{\mathcal{B}_2} = A[x]_{\mathcal{B}_1}
\]
onde $A \in \cp{M}_{q\times p}(\cp{K})$.

Agora seja $A \in \cp{M}_{q\times p}(\cp{K})$ onde
\[
	A = \begin{bmatrix}
		a_{11} & a_{21} & \dots & a_{1p}\\
		a_{21} & a_{22} & \dots & a_{2p}\\
		\vdots & & \ddots & \vdots\\
		a_{q1} & a_{q2} & \dots & a_{qp}
	\end{bmatrix}.
\]

Seja $V$ um espa\c{c}o vetorial sobre $\cp{K}$ de dimens\~ao $p \ge 1$ e $W$ um espa\c{c}o vetorial sobre $\cp{K}$ de dimens\~ao $q \ge 1$. Tome $\mathcal{B}_1 = \{v_1,\dots,v_p\}$ e $\mathcal{B}_2 = \{w_1,\dots,w_q\}$ bases ordenadas de $V$ e $W$, respectivamente. Defina $T : V \to W$ por
\begin{align*}
	T(v_1) &= a_{11}w_1 + a_{21}w_2 + \cdots + a_{q1}w_q\\
	&\vdots\\
	T(v_p) &= a_{1p}w_1 + a_{2p}w_2 + \cdots + a_{qp}w_q.
\end{align*}
Ent\~ao $T$ \'e uma transforma\c{c}\~ao linear  tal que
\[
	[T(x)]_{\mathcal{B}_2} = A[x]_{\mathcal{B}_1}
\]
para todo $x \in V$.

Assim provamos o seguinte teorema:
\begin{teorema}\label{teorema_toda_transformacao_matriz}
	Sejam $V$ e $W$ $\cp{K}$-espa\c{c}os vetoriais de dimens\~oes $p \ge 1$ e $q \ge 1$, respectivamente. Sejam $\mathcal{B}_1$ e $\mathcal{B}_2$ bases ordenadas de $V$ e $W$, respectivamente. Ent\~ao para cada transforma\c{c}\~ao linear $T : V \to W$, existe uma matriz $A \in \cp{M}_{q\times p}(\cp{K})$ tal que
	\[
		[T(x)]_{\mathcal{B}_2} = A[x]_{\mathcal{B}_1}
	\]
	para todo vetor $x \in B$. Al\'em disso, a cada matriz $A \in \cp{M}_{q\times p}(\cp{K})$ corresponde uma transforma\c{c}\~ao lineares $T : V \to W$ tal que
	\[
		[T(x)]_{\mathcal{B}_2} = A[x]_{\mathcal{B}_1}
	\]
	para todo $x \in V$.
\end{teorema}

\begin{definicao}
	A matriz $A \in \cp{M}_{q\times p}(\cp{K})$ do Teorema \ref{teorema_toda_transformacao_matriz} \'e chamada de \textbf{matriz da transforma\c{c}\~ao linear} $T$ com respeito \`as bases ordenadas $\mathcal{B}_1$ e $\mathcal{B}_2$ e ser\'a denotada por
	\[
		A = [T]_{\mathcal{B}_{1},\mathcal{B}_{2}}.
	\]
	No caso em que $V = W$ e $\mathcal{B}_1 = \mathcal{B}_2 = \mathcal{B}$, denotaremos $[T]_{\mathcal{B}_{1},\mathcal{B}_{2}}$ simplesmente por $[T]_\mathcal{B}$.
\end{definicao}

\begin{exemplos}
	\begin{enumerate}
		\item Seja $T : \real^2 \to \real^2$ a transforma\c{c}\~ao linear definida por
		\[
			T(x,y) = (x,0).
		\]
		Considere $\real^2$ com a base can\^onica $\mathcal{B} = \{e_1=(1,0);e_2=(0,1)\}$. Encontre $[T]_\mathcal{B}$.
		\begin{solucao}
			Temos
			\begin{align*}
				T(e_1) &= (1,0) = 1(1,0) + 0(0,1)\\
				T(e_2) &= (0,0) = 0(1,0) + 0(0,1).
			\end{align*}
			Logo a matriz de $T$ \'e
			\[
				[T]_\mathcal{B} = \begin{bmatrix}
					1 & 0\\
					0 & 0
				\end{bmatrix}.
			\]
			Al\'em disso, dado $(x,y) \in \real^2$ temos
			\[
				(x,y) = xe_1 + ye_2
			\]
			e ent\~ao
			\[
				[(x,y)]_\mathcal{B} = (x,y)_\mathcal{B}.
			\]
			Assim podemos escrever
			\[
				[T(x,y)]_\mathcal{B} = [T]_\mathcal{B}[(x,y)]_\mathcal{B} = \begin{bmatrix}
					1 & 0\\
					0 & 0
				\end{bmatrix}[(x,y)]_\mathcal{B}
			\]
			para todo $(x,y) \in \real^2$.
		\end{solucao}

		\item Seja $T : \real^3 \to \real^2$ tal que
		\[
			T(x,y,z) = (2x+y-z,3x-2y+4z).
		\]
		Considere as bases $\mathcal{B}_1 = \{(1,1,1);(1,1,0);(1,0,0)\}$ e $\mathcal{B}_2 = \{(1,3);(1,4)\}$. Encontre $[T]_{\mathcal{B}_{1},\mathcal{B}_{2}}$.
		\begin{solucao}
			Temos
			\begin{align*}
				T(1,1,1) &= (2,5) = 3(1,3) - 1(1,4)\\
				T(1,1,0) &= (3,1) = 11(1,3) - 8(1,4)\\
				T(1,0,0) &= (2,3) = 5(1,5) - 3(1,4)
			\end{align*}
			e da{\'\i}
			\[
				[T]_{\mathcal{B}_{1},\mathcal{B}_{2}} = \begin{bmatrix}
					\phantom{-}3 & \phantom{-}11 & \phantom{-}5\\
					-1 & -8 & -3
				\end{bmatrix}.
			\]
			Agora considerando as bases $\mathcal{B}_3 = \{(1,0,0);(0,1,0);(0,0,1)\}$ e $\mathcal{B}_4 = \{(1,0);(0,1)\}$ obtemos
			\begin{align*}
				T(1,0,0) &= (2,3) = 2(1,0) + 3(0,1)\\
				T(0,1,0) &= (1,-2) = 1(1,0) - 2(0,1)\\
				T(0,0,1) &= (-1,4) = -1(1,0) + 4(0,1)
			\end{align*}
			e assim a matriz de $T$ ser\'a
			\[
				[T]_{\mathcal{B}_{3},\mathcal{B}_{4}} = \begin{bmatrix}
					\phantom{-}2 & \phantom{-}1 & -1\\
					\phantom{-}3 & -2 & \phantom{-}4
				\end{bmatrix}.
			\]
		\end{solucao}
	\end{enumerate}
\end{exemplos}

\begin{teorema}\label{matriz_da_composicao_de_transformacoes}
	Sejam $F : V \to W$ e $G : W \to U$ duas transforma\c{c}\~oes lineares onde $V$, $W$ e $U$ s\~ao $\cp{K}$-espa\c{c}os vetoriais de dimens\~oes $p$, $q$ e $r$, respectivamente. Fixe bases ordenadas $\mathcal{B}_V$, $\mathcal{B}_W$ e $\mathcal{B}_U$ para $V$, $W$ e $U$, respectivamente. Ent\~ao $(G \circ F) : V \to U$ dada por $(G\circ F)(v) = G(F(v))$ \'e uma transforma\c{c}\~ao linear e
	\[
		[(G \circ F)]_{{\mathcal{B}_V},{\mathcal{B}_W}} = [G]_{{\mathcal{B}_W},{\mathcal{B}_U}}[F]_{{\mathcal{B}_V},{\mathcal{B}_W}}.
	\]
\end{teorema}
\begin{prova}
	\'E imediato verificar que $G\circ F$ \'e uma transforma\c{c}\~ao linear.

	Sejam $\mathcal{B}_V = \{v_1,\dots,v_p\}$, $\mathcal{B}_W = \{w_1,\dots,w_q\}$ e $\mathcal{B}_U = \{u_1,\dots,u_r\}$ bases ordenadas para $V$, $W$ e $U$, respectivamente. Considere as matrizes
	\begin{enumerate}
		\item $[F]_{{\mathcal{B}_V},{\mathcal{B}_W}} = (a_{ij})_{q\times p}$ onde $F(v_j) = \sum_{i=1}^qa_{ij}w_i$ para $j=1$, \dots, $p$;

		\item $[G]_{{\mathcal{B}_W},{\mathcal{B}_U}} = (b_{ij})_{r\times p}$ onde $G(w_j) = \sum_{k=1}^rb_{ij}u_i$ para $k=1$, \dots, $r$;

		\item $[G \circ F]_{{\mathcal{B}_V},{\mathcal{B}_W}} = (c_{ij})_{r\times p}$ onde $(G\circ F)(v_j) = \sum_{k=1}^rc_{ij}u_i$ para $k=1$, \dots, $r$.
	\end{enumerate}

	Vamos calcular $(G\circ F)(v_j)$
	\begin{align*}
		(G\circ F)(v_j) &= G(F(v_j)) = G(\sum_{i=1}^qa_{ij}w_i)\\
		&= \sum_{i=1}^qa_{ij}G(w_i) = \sum_{i=1}^qa_{ij}\sum_{k=1}^rb_{ki}w_i\\
		&= \sum_{k=1}^r\sum_{i=1}^qb_{ki}a_{ij}u_i.
	\end{align*}
	Assim da unicidade dos escalares obtemos
	\[
		c_{kj} = \sum_{i=1}^qb_{ki}a_{ij}
	\]
	para $j=1$, \dots, $p$ e $k = 1$, \dots, $r$. Isto \'e, para cada par $(j,k)$ o elemento $c_{kj}$ da matriz $[(G\circ F)]_{{\mathcal{B}_V},{\mathcal{B}_W}}$ \'e o elemento na posi\c{c}\~ao $(k,j)$ da matriz resultante da multiplica\c{c}\~ao de $[G]_{{\mathcal{B}_W},{\mathcal{B}_U}}$ por $[F]_{{\mathcal{B}_V},{\mathcal{B}_W}}$. Portanto
	\[
		[(G\circ F)]_{{\mathcal{B}_V},{\mathcal{B}_W}} = [G]_{{\mathcal{B}_W},{\mathcal{B}_U}}[F]_{{\mathcal{B}_V},{\mathcal{B}_W}}
	\]
	como quer{\'\i}amos.
\end{prova}

Vamos tratar, principalmente, com a representa\c{c}\~ao por matrizes de transforma\c{c}\~oes lineares de um espa\c{c}o em si mesmo. Lembre-se que esta matriz muda de acordo com a escolha da base. Assim, deve-se prestar aten\c{c}\~ao sempre \`a base escolhida. Assim como um espa\c{c}o vetorial pode ter v\'arias bases distintas, o que acontecer\'a com a matriz representante de uma transforma\c{c}\~ao linear quando mudamos a base ordenada? Vamos considerar $T : V \to V$ uma transforma\c{c}\~ao linear sobre o $\cp{K}$-espa\c{c}o vetorial de dimens\~ao finita $V$ e sejam
\[
	\mathcal{B}_1 = \{v_1,\dots,v_n\}, \quad \mathcal{B}_2 = \{w_1,\dots,w_n\}
\]
bases ordenadas de $V$. Qual a rela\c{c}\~ao entre as matrizes $[T]_{\mathcal{B}_1}$ e $[T]_{\mathcal{B}_2}$?

Para responder a esta quest\~ao, primeiro vamos determinar uma rela\c{c}\~ao entre $[x]_{\mathcal{B}_1}$ e $[x]_{\mathcal{B}_2}$, para todo $x \in V$. Como $\mathcal{B}_1$ \'e uma base de $V$, ent\~ao existem $\alpha_{ij} \in \cp{K}$, $1 \le i,\ j \le n$ tais que
\begin{align*}
	w_1 &= \alpha_{11}v_1 + \cdots + \alpha_{n1}v_n\\
	&\vdots\\
	w_n &= \alpha_{1n}v_1 + \cdots + \alpha_{nn}v_n.
\end{align*}
Agora, dado $x \in V$, existem $\beta_1$, \dots, $\beta_n \in \cp{K}$ tais que
\[
	x = \beta_1w_1 + \cdots + \beta_nw_n.
\]
Da{\'\i}
\begin{align*}
	x &= \beta_1(\alpha_{11}v_1 + \cdots + \alpha_{n1}v_n) + \cdots + (\alpha_{1n}v_1 + \cdots + \alpha_{nn}v_n)\\
	&= (\beta_1\alpha_{11} + \beta_2\alpha_{12} + \cdots + \beta_n\alpha_{1n})v_1 + \cdots + (\beta_1\alpha_{n1} + \beta_2\alpha_{n2} + \cdots + \beta_n\alpha_{nn})v_n\\
	&= \sum_{j=1}^n\beta_j\alpha_{1j}v_1 + \cdots + \sum_{j=1}^n\beta_j\alpha_{nj}v_n\\
	&= \sum_{i=1}^n\left(\sum_{j=1}^n\beta_j\alpha_{ij}\right)v_i.
\end{align*}
Assim as coordenadas de $x$ em rela\c{c}\~ao \`a base $\mathcal{B}_1$ s\~ao
\[
	[x]_{\mathcal{B}_1} = \begin{bmatrix}
		\sum_{j=1}^n\beta_j\alpha_{1j}\\
		\vdots\\
		\sum_{j=1}^n\beta_j\alpha_{nj}
	\end{bmatrix}.
\]
Seja $P$ a matriz cuja entrada $(i,j)$ \'e o escalar $\alpha_{ij}$, isto \'e,
\[
	P = \begin{bmatrix}
		\alpha_{11} & \cdots & \alpha_{1n}\\
		\vdots\\
		\alpha_{n1} & \cdots & \alpha_{nn}
	\end{bmatrix}.
\]
Ent\~ao podemos escrever
\[
	[x]_{\mathcal{B}_1} = P \begin{bmatrix}
		\beta_1\\ 
		\vdots\\
		\beta_n
	\end{bmatrix} = P[x]_{\mathcal{B}_2}.
\]
Mais ainda, como $\mathcal{B}_1$ e $\mathcal{B}_2$ s\~ao bases de $V$, ent\~ao $[x]_{\mathcal{B}_1} = [0_V]_{\mathcal{B}_1}$ se, e somente se, $[x]_{\mathcal{B}_2} = [0_V]_{\mathcal{B}_2}$. Logo $P$ possui inversa. Assim mostramos o seguinte teorema:

\begin{teorema}\label{teorema_mudanca_base}
	Sejam $V$ um $\cp{K}$-espa\c{c}o vetorial de dimens\~ao $n \ge 1$, $\mathcal{B}_1$ e $\mathcal{B}_2$ bases ordenadas de $V$. Ent\~ao existe uma \'unica matriz $P \in \cp{M}_n(\cp{K})$, necessariamente invert{\'\i}vel tal que
	\begin{enumerate}
		\item $[x]_{\mathcal{B}_1} = P[x]_{\mathcal{B}_2}$
		\item $[x]_{\mathcal{B}_2} = P^{-1}[x]_{\mathcal{B}_1}$
	\end{enumerate}
	para todo $x \in V$. As colunas de $P = \begin{bmatrix}
		P_1 & P_2 & \cdots & P_n
	\end{bmatrix}$ s\~ao dadas por
	\[
		P_j = [w_j]_{\mathcal{B}_2}
	\]
	para $j = 1$, \dots, $n$.
\end{teorema}

Al\'em disso, tamb\'em temos o seguinte teorema:

\begin{teorema}\label{teorema_matriz_mudanca_base}
	Seja $P \in \cp{M}_n(\cp{K})$ um matriz invert{\'\i}vel. Sejam $V$ um $\cp{K}$-espa\c{c}o vetorial de dimens\~ao $n \ge 1$ e $\mathcal{B}_1$ uma base ordenada de $V$. Ent\~ao existe uma \'unica base ordenada $\mathcal{B}_2$ de $V$ tal que
	\begin{enumerate}
		\item $[x]_{\mathcal{B}_1} = P[\mathcal{B}_1]_{\mathcal{B}_2}$
		\item $[x]_{\mathcal{B}_2} = P^{-1}[x]_{\mathcal{B}_1}$
	\end{enumerate}
	para todo $x \in V$.
\end{teorema}

\begin{definicao}
	A matriz $P \in \cp{M}_n(\cp{K})$ do Teorema \ref{teorema_matriz_mudanca_base} \'e chamada de \textbf{matriz de mudan\c{c}a de base} e \'e denotada por $P = [I]_{{\mathcal{B}_1},{\mathcal{B}_2}}$.
\end{definicao}

Agora, sejam $T : V \to V$ uma transforma\c{c}\~ao linear, $\mathcal{B}_1 = \{v_1,\dots,v_n\}$ e $\mathcal{B}_2 = \{w_1,\dots,w_n\}$ bases ordenadas de $V$. Sabemos que existe $P \in \cp{M}_n(\cp{K})$ invert{\'\i}vel tal que
\begin{equation}\label{equacao_mudanca_base}
	[x]_{\mathcal{B}_1} = P[x]_{\mathcal{B}_2}
\end{equation}
para todo $x \in V$. Mas, para todo $x \in V$ temos
\begin{equation}\label{equacao_coordenadas_imagem}
	[T(x)]_{\mathcal{B}_1} = [T]_{\mathcal{B}_1}[x]_{\mathcal{B}_1}.
\end{equation}
Aplicando \eqref{equacao_mudanca_base} ao vetor $T(x)$ obtemos
\begin{equation}\label{equacao_mudanca_base_imagem}
	[T(x)]_{\mathcal{B}_1} = P[T(x)]_{\mathcal{B}_2}.
\end{equation}
Combinando \eqref{equacao_mudanca_base}, \eqref{equacao_coordenadas_imagem} e \eqref{equacao_mudanca_base_imagem}:
\begin{align*}
	[T]_{\mathcal{B}_1}[x]_{\mathcal{B}_1} &= [T(x)]_{\mathcal{B}_1} = P[T(x)]_{\mathcal{B}_2}\\
	[T]_{\mathcal{B}_1}P[x]_{\mathcal{B}_2} &= P[T(x)]_{\mathcal{B}_2}\\
	P^{-1}[T]_{\mathcal{B}_1}P[x]_{\mathcal{B}_2} &= [T(x)]_{\mathcal{B}_2} = [T]_{\mathcal{B}_2}[x]_{\mathcal{B}_2}.
\end{align*}
Da{\'\i}
\[
	P^{-1}[T]_{\mathcal{B}_1} = [T]_{\mathcal{B}_2}
\]
o que responde nossa pergunta.

Por outro lado, sabemos que existe uma \'unica transforma\c{c}\~ao linear $G : V \to V$ tal que
\[
	G(v_i) = w_i
\]
para $i = 1$, \dots, $n$. Mais ainda, tal transforma\c{c}\~ao linear \'e um isomorfismo. Afirmamos que a matriz $P$ acima \'e exatamente a matriz de $G$ em rela\c{c}\~ao \`a base $\mathcal{B}_1$. De fato, as entradas de $P$ s\~ao definidas por
\[
	w_i = \sum_{i=1}^n\alpha_{ij}v_i
\]
e como $G(v_i) = w_i$ podemos escrever
\[
	G(v_i) = w_i = \sum_{i=1}^n\alpha_{ij}v_i
\]
e ent\~ao por defini\c{c}\~ao $[G]_{\mathcal{B}_1} = P$. Desse modo, temos o seguinte teorema:

\begin{teorema}
	Sejam $V$ um espa\c{c}o vetorial sobre $\cp{K}$ de dimens\~ao finita, $\mathcal{B}_1 = \{v_1,\dots,v_n\}$ e $\mathcal{B}_2 = \{w_1,\dots,w_n\}$ bases ordenadas de $V$. Suponha que $T : V \to V$ \'e uma transforma\c{c}\~ao linear. Se $P = \begin{bmatrix}P_1 & P_2 & \dots & P_n
	\end{bmatrix} \in \cp{M}_n(\cp{K})$ \'e uma matriz com colunas
	\[
		P_j = [w_j]_{\mathcal{B}_1}
	\]
	ent\~ao
	\[
		[T]_{\mathcal{B}_2} = P^{-1}[T]_{\mathcal{B}_1}P.
	\]
	Alternativamente, se $G$ \'e o isomorfismo de $V$ definido por $G(v_i) = w_i$, $i = 1$, \dots, $n$ ent\~ao
	\[
		[T]_{\mathcal{B}_2} = ([G]_{\mathcal{B}_1})^{-1}[T]_{\mathcal{B}_1}[G]_{\mathcal{B}_1}.
	\]
\end{teorema}

\begin{definicao}
	Sejam $A$, $B \in \cp{M}_n(\cp{K})$. Dizemos que $B$ \'e \textbf{semelhante} a $A$ sobre $\cp{K}$ se existe $P \in \cp{M}_n(\cp{K})$ invert{\'\i}vel tal que
	\[
		B = P^{-1}AP.
	\]
\end{definicao}

\begin{exemplos}
	\begin{enumerate}
		\item Seja $T : \real^2 \to \real^2$ dada por $T(x,y) = (x,0)$. Sabemos que com rela\c{c}\~ao \`a base $\mathcal{B}_1 = \{e_1=(1,0);e_2=(0,1)\}$ a matriz de $T$ \'e
		\[
			[T_{\mathcal{B}_1} = \begin{bmatrix}
				1 & 0\\
				0 & 0
			\end{bmatrix}.
		\]
		Agora, seja $\mathcal{B}_2$ a base de $\real^2$ formada por
		\[
			\mathcal{B}_2 = \{w_1 = (1,1);w_2=(2,1)\}.
		\]
		Ent\~ao
		\begin{align*}
			w_1 &= e_1 + e_2\\
			w_2 & 2e_1 + e_2.
		\end{align*}
		Assim $P$ \'e a matriz
		\[
			P = \begin{bmatrix}
				1 & 2\\
				1 & 1
			\end{bmatrix}
		\]
		cuja inversa \'e
		\[
			P^1 = \begin{bmatrix}
				-1 & \phantom{-}2\\
				\phantom{-}1 & -1
			\end{bmatrix}.
		\]
		Logo
		\begin{align*}
			[T]_{\mathcal{B}_2} &= P^{-1}[T]_{\mathcal{B}_1}P\\
			[T]_{\mathcal{B}_2} &= \begin{bmatrix}
				-1 & -2\\
				\phantom{-}1 & \phantom{-}2
			\end{bmatrix}.
		\end{align*}

		\item Seja $\mathcal{D} : \mathcal{P}_3(\real) \to \mathcal{P}_3(\real)$ a transforma\c{c}\~ao derivada. Considere a base $\mathcal{B}_1 = \{f_1 = 1; f_2 = x; f_3 = x^2; f_4 = x^3\}$. Tome $t \in \real$ e defina
		\[
			\mathcal{B}_2 = \{g_1 = 1; g_2 = (x + t); g_3 = (x + t)^3; f_4 = (x + t)^3\}.
		\]
		Assim
		\begin{align*}
			g_1 &= 1f_1 + 0f_2 + 0f_3 + 0f_4\\
			g_2 &= tf_1 + xf_2 + 0f_3 + 0f_4\\
			g_3 &= t^2f_1 + 2tf_2 + 1f_3 + 0f_4\\
			g_4 &= t^3f_1 + 3t^2f_2 + 3tf_3 + 1f_4\\
		\end{align*}
		e ent\~ao
		\[
			P = \begin{bmatrix}
				1 & t & t^2 & t^3\\
				0 & 1 & 2t & 3t^2\\
				0 & 0 & 1 & 3t\\
				0 & 0 & 0 & 1
			\end{bmatrix}
		\]
		\'e invert{\'\i}vel com
		\[
			P^{-1} = \begin{bmatrix}
				1 & -t & t^2 & -t^3\\
				0 & 1 & -2t & 3t^2\\
				0 & 0 & 1 & -3t\\
				0 & 0 & 0 & 1
			\end{bmatrix}.
		\]
		Agora
		\begin{align*}
			D(f_1) &= 0\\
			D(f_2) &= 1\\
			D(f_3) &= 2x\\
			D(f_4) &= 3x^2
		\end{align*}
		e ent\~ao
		\[
			[D]_{\mathcal{B}_1} = \begin{bmatrix}
				0 & 1 & 0 & 0\\
				0 & 0 & 2 & 0\\
				0 & 0 & 0 & 3\\
				0 & 0 & 0 & 0
			\end{bmatrix}.
		\]
		Logo
		\[
			[D]_{\mathcal{B}_2} = P^{-1}[D]_{\mathcal{B}_1}P = \begin{bmatrix}
				0 & 1 & 0 & 0\\
				0 & 0 & 2 & 0\\
				0 & 0 & 0 & 3\\
				0 & 0 & 0 & 0
			\end{bmatrix}.
		\]
	\end{enumerate}
\end{exemplos}
% section transformacoes_lineares_e_matrizes (end)