%!TEX program = xelatex
%!TEX root = Algebra_Linear.tex
%%Usar makeindex -s indexstyle.ist Algebra_Linear.idx arquivo no terminal para gerar o {\'\i}ndice remissivo agrupado por inicial
%%Ap\'os executar pdflatex arquivo

\chapter{Transformações Lineares}

Em todo esse capítulo $\cp{K}$ denotará um corpo.

\section{Conceitor Básicos}

\begin{definicao}
	Sejam $(V, +, \cdot)$ e $(W, \oplus, \otimes)$ espaços vetoriais sobre um corpo $\cp{K}$. Uma função $T : V \to W$ é uma \textbf{transformação linear} se
	\begin{enumerate}
		\item $T(u_1 + u_2) = T(u_1) \oplus T(u_2)$ para todos $u_1$, $u_2 \in V$;
		\item $T(\lambda \cdot u) = \lambda \otimes T(u)$ para todo $\lambda \in \cp{K}$ e todo $u \in V$.
	\end{enumerate}
\end{definicao}

\begin{observacao}
	Para simplificar a notação, vamos adotar os mesmos símbolos para indicar a soma e o produto por escalar nos espaços vetoriais que aparecerão no decorrer do texto. No entanto, o leitor deve estar ciente que estes símbolos podem ter significados diferentes, dependendo do espaço vetorial em questão.
\end{observacao}

\begin{lema}
	Sejam $V$ e $W$ espaços vetoriais sobre $\cp{K}$. Então uma função $T : V \to W$ é uma transformação linear se, e somente se,
	\[
		T(\lambda u_1 + u_2) = \lambda T(u_1) + T(u_2),
	\]
	para todos $u_1$, $u_2 \in V$ e todo $\lambda \in \cp{K}$.
\end{lema}
\begin{prova}
	Deixada a cargo do leitor.
\end{prova}

\begin{lema}
	Sejam $V$ e $W$ espaços vetoriais sobre $\cp{K}$ e $T : V \to W$ uma transformação linear. Então:
	\begin{enumerate}
		\item $T(0_V) = 0_W$, onde $0_V$ e $0_W$ denotam os vetores nulos de $V$ e $W$, respectivamente.
		\item $T(-u) = -T(u)$, para cada $u \in V$.
		\item $T(\sum_{i=1}^m\alpha_iu_i) = \sum_{i=1}^mT(u_i)$, onde $\alpha_i \in \cp{K}$ e $u_i \in V$ para $i = 1$, \dots, $m$.
	\end{enumerate}
\end{lema}
\begin{prova}
	\begin{enumerate}
		\item Note que
		\[
			0_W + T(0_V) = T(0_V + 0_V) = T(0_V) + T(0_V),
		\]
		ou seja, $T(0_V) = 0_W$.
		\item Basta observar que $-u = (-1_\cp{K})u$ e daí
		\[
			T(-u) = T((-1_\cp{K})u) = -1_\cp{K}T(u) = -T(u).
		\]
		\item Por indução em $m$. Se $m = 2$, então
		\[
			T(\alpha_1u_1 + \alpha_2u_2) = T(\alpha_1u_1) + T(\alpha_2u_2) = \alpha_1T(u_1) + \alpha_2T(u_2).
		\]
		Suponha que para $m = p$ tenhamos
		\[
			T(\sum_{i=1}^p\alpha_iu_i) = \sum_{i=1}^pT(u_i).
		\]
		Vamos mostra que é válido para $m = p + 1$. De fato,
		\begin{align*}
			T(\sum_{i=1}^{p+1}\alpha_iu_i) &= T([\sum_{i=1}^p\alpha_iu_i] + \alpha_{p + 1}u_{p + 1}) = T(\sum_{i=1}^p\alpha_iu_i) + T(\alpha_{p+1}u_{p+1}) \\ &= \sum_{i=1}^p\alpha_iT(u_i) + \alpha_{p+1}T(u_{p+1}) = \sum_{i=1}^{p+1}\alpha_iT(u_i).
		\end{align*}
	\end{enumerate}
\end{prova}