%!TEX program = xelatex
%!TEX root = Algebra_Linear.tex
%%Usar makeindex -s indexstyle.ist arquivo no terminal para gerar o {\'\i}ndice remissivo agrupado por inicial
%%Ap\'os executar pdflatex arquivo
\chapter{Sistemas Lineares E Matrizes}

%\section{Preliminares}

\section{Corpos}\label{ssub:corpos}

\begin{definicao}\index{Corpos}
Um conjunto n\~ao vazio $\cp{K}$ \'e chamado de \textbf{corpo} se em $\cp{K}$ podemos definir duas opera\c{c}\~oes, denotadas por
$\oplus$ (adi\c{c}\~ao) e $\otimes$ (multiplica\c{c}\~ao) de modo que
\begin{align*}
a \oplus b \in \cp{K}\\
a \otimes b \in \cp{K}
\end{align*}
para todos $a$, $b \in \cp{K}$ e que satisfa\c{c}am as seguintes propriedades:
\begin{itemize}

	\item[A1)] \textbf{Comutatividade da adi\c{c}\~ao}: $a \oplus b = b \oplus a$ para todos $a$, $b \in \cp{K}$;
	\item[A2)] \textbf{Associatividade da adi\c{c}\~ao}: $a \oplus (b \oplus c) = (a \oplus b) \oplus c$, para todos $a$, $b$ e $c \in \cp{K}$;
	\item[A3)] \textbf{Elemento neutro da adi\c{c}\~ao}: Existe um elemento em $\cp{K}$, denotado por $0_\cp{K}$ ou simplesmente $0$ e chamado de \textbf{elemento neutro da adi\c{c}\~ao}, que satisfaz
	\[
		a \oplus 0_\cp{K} = a = 0_\cp{K} \oplus a
	\]
	para todo $a \in \cp{K}$. (Elemento neutro da adi\c{c}\~ao)
	\item[A4)] \textbf{Elemento oposto da adi\c{c}\~ao}: Para cada $a \in \cp{K}$, existe um elemento em $\cp{K}$, denotado por $-a$ e chamado de \textbf{oposto} de $a$ ou \textbf{inverso aditivo} de $a$ tal que
	\[
		a \oplus (-a) = 0_\cp{K} = (-a) \oplus a.
	\]
	\item[M1)] \textbf{Comutatividade da multiplica\c{c}\~ao}: $a \otimes b = b \otimes a$ para todos $a$, $b \in \cp{K}$;
	\item[M2)] \textbf{Associatividade da multiplica\c{c}\~ao}: $a \otimes (b \otimes c) = (a \otimes b) \otimes c$, para todos $a$, $b$ e $c \in \cp{K}$;
	\item[M3)] \textbf{Elemento neutro da multiplica\c{c}\~ao}: Existe um elemento em $\cp{K}$, denotado por $1_\cp{K}$ ou simplesmente $1$ e chamado de \textbf{elemento neutro da multiplica\c{c}\~ao} ou \textbf{unidade}, que satisfaz
	\[
		a \otimes 1_\cp{K} = a = 1_\cp{K} \otimes a
	\]
	para todo $a \in \cp{K}$.
	\item[M4)] \textbf{Elemento inverso da multiplica\c{c}\~ao}: Para cada $a \in \cp{K}$, $a \ne 0_{\cp{K}}$, existe um elemento em $\cp{K}$, denotado por $a^{-1}$ e chamado de \textbf{inverso multiplicativo} de $a$ 
	tal que
	\[
		a \otimes a^{-1} = 1_\cp{K} = a^{-1} \otimes a.
	\]
	\item[D)] \textbf{Distributividade da soma em rela\c{c}\~ao \`a multiplica\c{c}\~ao}: $(a \oplus b)\otimes c = a\otimes c \oplus b\otimes c$, para todos $a$, $b$ e $c \in \cp{K}$.
\end{itemize}
\end{definicao}

Para simplificar a nota\c{c}\~ao vamos escrever
\[
a \otimes b = ab.
\]
Denotamos um corpo $\cp{K}$ pela terna $(\cp{K}, \otimes, \oplus)$. Quando n\~ao houver chance de confus\~ao em rela\c{c}\~ao \`as opera\c{c}\~oes de soma e multiplica\c{c}\~ao envolvidas no corpo $(\cp{K}, \otimes, \oplus)$, vamos simplesmente dizer que $\cp{K}$ \'e um corpo. Os elementos de um corpo $\cp{K}$ s\~ao chamados de \textbf{escalares}.

\begin{exemplo}
\begin{enumerate}[label={\arabic*})]
	\item S\~ao exemplos de corpos os conjuntos: $\rac$, $\real$, $\complex$ com as opera\c{c}\~oes de soma e multiplica\c{c}\~oes usuais destes conjuntos.

	\item O conjunto $\z$ com a soma e multiplica\c{c}\~oes usuais n\~ao \'e um corpo, pois por exemplo, n\~ao existe $b \in \z$ tal que $2b = 1$.

	\item Seja
	\[
	\rac{[\sqrt{2}]} = \{a + b\sqrt{2} \mid a,\ b \in \rac\}.
	\]
	Dados $a + b\sqrt{2}$, $c + d\sqrt{2} \in \rac{[\sqrt{2}]}$, defina
	\begin{align*}
	(a + b\sqrt{2}) + (c + d\sqrt{2}) = (a + c) + (b + d)\sqrt{2}\\
	(a + b\sqrt{2})(c + d\sqrt{2}) = (ac + 2bd) + (ad + bc)\sqrt{2}
	\end{align*}
	Al\'em disso, $a + b\sqrt{2} = c + d\sqrt{2}$ se, e somente se, $a = c$ e $b = d$.
	Aqui o elemento neutro da adi\c{c}\~ao \'e $0$, o elemento neutro da multiplica\c{c}\~ao \'e $1$, o oposto aditivo de $a + b\sqrt{2}$ \'e $-a - b\sqrt{2}$ e o inverso multiplicativo de $a + b\sqrt{2}$ \'e $\dfrac{a - b\sqrt{2}}{a^2 - 2b^2}$ para $a \ne 0$ ou $b \ne 0$.

	\item Considere as opera\c{c}\~oes $\oplus$ e $\otimes$ em $\rac$ definidas por
	\begin{align*}
		x \oplus y = x + y - 3\\
		x \otimes y = x + y - \dfrac{xy}{3},
	\end{align*}
	para todos $x$, $y \in \rac$. Ent\~ao $(\rac, \oplus, \otimes)$ \'e um corpo.
	\begin{solucao}
		De fato, para todos $x$, $y$ e $z \in \rac$ temos:
		\begin{enumerate}[label={\arabic*})]
			\item $x \oplus y = x + y - 3 = y + x - 3 = y \oplus x$.
			Logo $x \oplus y = y \oplus x$.

			\item Temos
			\begin{align*}
				(x \oplus y) \oplus z &= (x + y - 3) \oplus z = (x + y - 3) + z - 3 = x + y + z - 6\\
				x \oplus ( y \oplus z) &= x \oplus (y + z - 3) = x + (y + z - 3) - 3 = x + y + z - 6.
			\end{align*} Logo $(x \oplus y) \oplus z = x \oplus ( y \oplus z)$, como quer{\'\i}amos.

			\item Tome $0_{\cp{K}} = 3$. Ent\~ao para todo $x \in \rac$ temos
			\[
				x \oplus 0_{\cp{K}} = x \oplus 3 = x + 3 - x = x.
			\]
			Logo $0_{\cp{K}} = 3$ \'e o elemento neutro da opera\c{c}\~ao $\oplus$ em $\rac$.

			\item Para $x \in \rac$ tome $y = 6 - x \in \rac$. Assim
			\[
				x \oplus y = x \oplus (6 - x) = x + (6 - x ) - 3 = 3 = 0_{\cp{K}},
			\]
			logo $y = 6 - x$ \'e o oposto de $x$ na adi\c{c}\~ao $\oplus$ definida em $\rac$.

			\item $x \otimes y = x + y - \dfrac{xy}{3} = y + x - \dfrac{yx}{3} = y \otimes x$, para todos $x$, $y \in \rac$.

			\item Para $x$, $y$ e $z \in \rac$ temos
			\begin{align*}
				(x \otimes y) \otimes z &= \left(x + y - \dfrac{xy}{3}\right) \otimes z = \left(x + y - \dfrac{xy}{3}\right) + y - \dfrac{\left(x + y - \dfrac{xy}{3}\right)z}{3} \\ &= x + y - \dfrac{xy}{3} + z - \dfrac{xz}{3} - \dfrac{yz}{3} + \dfrac{xyz}{9}\\
				x \otimes (y \otimes z) &= x \otimes \left(y + z - \dfrac{yz}{3}\right) = x + \left(y + z - \dfrac{yz}{3}\right) - \dfrac{x\left(y + z - \dfrac{yz}{3}\right)}{3} \\ &= x + y + z - \dfrac{yz}{3} - \dfrac{xy}{3} - \dfrac{xz}{3} + \dfrac{xyz}{9},
			\end{align*}
			logo $(x \otimes y) \otimes z = x \otimes (y \otimes z)$, como quer{\'\i}amos.

			\item Tome $1_{\cp{K}} = 0$. Ent\~ao
			\[
				x \otimes 1_{\cp{K}} = x \otimes 0 = x + 0 - \dfrac{x0}{3} = x,
			\]
			para todo $x \in \rac$. Logo $1_{\cp{K}} = 0$ \'e a unidade da opera\c{c}\~ao $\otimes$ em $\rac$.

			\item Dado $x \in \rac$, $x \ne 3 = 0_{\cp{K}}$ tome $y = \dfrac{-3x}{3 - x} \in \rac$. Temos
			\[
				x \otimes y = x \otimes \dfrac{-3x}{3 - x} = x + \dfrac{-3x}{3 - x} - \dfrac{x\left(\dfrac{-3x}{3 - x}\right)}{3} = 0 = 1_{\cp{K}}.
			\]
			Logo $y = \dfrac{-3x}{3 - x}$ \'e o inverso multiplicativo de $x$ na opera\c{c}\~ao $\otimes$ em $\rac$.

			\item Para todos $x$, $y$ e $z \in \rac$ temos
			\begin{align*}
				(x \oplus y) \otimes z &= (x + y - 3) \otimes z = (x + y - 3) + z - \dfrac{(x + y - 3)z}{3} \\ &= x + y - 3 + z - \dfrac{xz}{3} - \dfrac{yz}{3} + z\\
				(x \otimes z) \oplus (y \otimes z) &= \left(x + z - \dfrac{xz}{3}\right) \oplus \left(y + z - \dfrac{yz}{3}\right) \\ &= x + z - \dfrac{xz}{3} + y + z - \dfrac{yz}{3} - 3,
			\end{align*}
			Logo $(x \oplus y) \otimes z = (x \otimes z) \oplus (y \otimes z)$.
		\end{enumerate}
		Portanto $(\rac, \oplus, \otimes)$ \'e um corpo.
	\end{solucao}
\end{enumerate}
\end{exemplo}

\begin{proposicao}
	Seja $(\cp{K}, +, \cdot)$ um corpo. Ent\~ao:
	\begin{enumerate}[label={\roman*})]
		\item O elemento neutro da soma \'e \'unico.
		\item O oposto aditivo de cada elemento de $\cp{K}$ \'e \'unico.
		\item Vale a lei do cancelamento, isto \'e, se $a + b = a + c$ ent\~ao $b = c$.
		\item Para todo $a \in \cp{K}$, $a\cdot 0_\cp{K} = 0_\cp{K}$.
		\item O elemento neutro da multiplica\c{c}\~ao \'e \'unico.
		\item O inverso de um elemento n\~ao nulo \'e \'unico.
		\item Se $a \in \cp{K}$ \'e tal que $a \ne 0_\cp{K}$ e $ab = ac$, ent\~ao $b = c$.
		\item Se $ab = 0_\cp{K}$, com $a$ e $b \in \cp{K}$, ent\~ao $a = 0_\cp{K}$ ou $b = 0_\cp{K}$.
	\end{enumerate}
\end{proposicao}
\begin{prova}
	Vamos provar algumas propriedades.
	\begin{enumerate}[label={\roman*})]
		\item De fato, suponha que $0_1$ e $0_2$ sejam elementos neutros da soma em $\cp{K}$. Temos
		\[
			0_1 = 0_1 + 0_2 = 0_2.
		\]
	
		\item Sejam $b$ e $c$ elementos opostos de $a$. Daí
		\begin{align*}
			a + b &= 0_\cp{K}\\
			a + c &= 0_\cp{K}.
		\end{align*}
	 	Ent\~ao
		\[
			b = b + 0_\cp{K} = b + (c + a) = (b + a) + c = 0_\cp{K} + c = c.
		\]
		
		\item Temos
		\begin{align*}
			b &= b + 0_{\cp{K}} = b + (a + (-a)) = (b + a) + (-a) \\ &= (a + b) + (-a) = (a + c) + (-a) = (c + a) + (-a) \\ &= c + (a + (-a)) = c + 0_{\cp{K}} \\ &= c
		\end{align*}
		como queríamos.
	
		\item De fato,
		\[
			a\cdot 0_\cp{K} + 0_\cp{K} = a\cdot 0_\cp{K} = a\cdot (0_\cp{K} + 0_\cp{K}) = a\cdot 0_\cp{K} + a\cdot 0_\cp{K},
		\]
		logo pela lei do cancelamento, $a\cdot 0_\cp{K} = 0_\cp{K}$ como quer{\'\i}amos.
	
	
		\item
		De fato, suponha que $1_a$ e $1_b$ sejam elementos neutros da multiplica\c{c}\~ao. Ent\~ao
		\[
			1_a = 1_a\cdot 1_b = 1_b.
		\]
	
		\item Dado $a \in \cp{K}$, $a \ne 0_{\cp{K}}$, suponha que $b$, $c \in \cp{K}$ sejam tais que
		\[
			ab = 1_{\cp{K}} \quad ac = 1_{\cp{K}}.
		\]
		Ent\~ao
		\[
			b = 1_{\cp{K}}b = (ac)b = c(ab) = c1_{\cp{K}} = c.
		\]
		
		\item De fato,
		\begin{align*}
			b &= b1_{\cp{K}} = b(aa^{-1}) = (ba)a^{-1} = (ab)a^{-1} \\ &= (ac)a^{-1} = (ca)a^{-1} = c(aa^{-1}) = c1_{\cp{K}} \\ &= c
		\end{align*}
	
		\item
		Suponha que $a \ne 0_\cp{K}$, ent\~ao existe $a^{-1}$. Da{\'\i}
		\begin{align*}
			&ab = 0_\cp{K}\\
			&a^{-1}(ab) = a^{-1}0_\cp{K}\\
			&1b = 0_\cp{K}\\
			&b = 0_\cp{K}.
		\end{align*}
	\end{enumerate}
\end{prova}


% \begin{observacao}
% Os elementos de um corpo qualquer $\cp{K}$ s\~ao chamados de \textbf{escalares}.
% \end{observacao}

\section{Corpos Finitos}\label{sec:corpor_finitos}\index{Corpos!Finitos}

Seja $p \in \z$ um n\'umero primo. Dado $a \in \z$, sempre \'e poss{\'\i}vel escrever
\[
	a = bp + r,
\]
onde $b$, $r \in \z$ e $0 \le r \le p - 1$. Assim quando efetuamos a divis\~ao inteira de qualquer n\'umero inteiro $a$ por $p$ os poss{\'\i}veis restos s\~ao: $0$, 
$1$, $2$, \dots, $p -1 $.

Assim vamos considerar o seguinte conjunto
\[
	\z_p = \{\overline{0}, \overline{1}, \overline{2}, \dots, \overline{p - 1}\}
\]
onde
\begin{align*}
	&\overline{0} = \{bp \mid b \in \z\} = \{0, \pm p, \pm 2p, \pm 3p, \dots\}\\
	&\overline{1} = \{bp + 1\mid b \in \z\} = \{\pm p + 1, \pm 2p + 1, \pm 3p + 1, \dots\}\\
	&\overline{2} = \{bp + 2\mid b \in \z\} = \{\pm p + 2, \pm 2p + 2, \pm 3p + 2, \dots\}\\
	\vdots\\
	&\overline{p - 1} = \{bp + (p - 1) \mid b \in \z\} = \{(b + 1)p - 1) \mid b \in \z\}\\ &= \{cp - 1 \mid c \in \z\} = \{-1, \pm p - 1, \pm 2p - 1, \pm 3p - 1, \dots\}.
\end{align*}

defina em $\z_p$ a soma $\oplus$ e a multiplica\c{c}\~ao $\otimes$ por: para $\overline{x}$, $\overline{y} \in \z_p$
\begin{align*}
	\overline{x} \oplus \overline{y} = \overline{x + y}\\
	\overline{x} \otimes \overline{y} = \overline{xy},
\end{align*}
onde sob a barra estamos usando a soma e multiplica\c{c}\~ao usuais dos inteiros. Para determinar o valor de $\overline{x + y}$ e de $\overline{xy}$, encontramos o resto da divis\~ao inteira de $x + y$ por $p$ e de $xy$ por $p$. Logo
\begin{align*}
	\overline{x} \oplus \overline{y} \in \z_p\\
	\overline{x} \otimes \overline{y} \in \z_p.
\end{align*}

\begin{exemplos}
\begin{enumerate}[label={\arabic*})]
	\item Para $p = 3$ os poss{\'\i}veis restos na divis\~ao inteira s\~ao: $0$, $1$ e $2$. Da{\'\i}
		\[
			\z_3 = \{\overline{0}, \overline{1}, \overline{2}\}
		\]
	e temos
	\begin{center}
		\begin{tabular}{|c|c|c|c|c|c|}
			\hline
			$\oplus$ &\rule{0pt}{2.5ex} $\overline{0}$ & $\overline{1}$ & $\overline{2}$\\\hline
			$\rule{0pt}{2.5ex}\overline{0}$ & $\overline{0}$ & $\overline{1}$ & $\overline{2}$\\\hline
			$\rule{0pt}{2.5ex}\overline{1}$ & $\overline{1}$ & $\overline{2}$ & $\overline{0}$\\\hline
			$\rule{0pt}{2.5ex}\overline{2}$ & $\overline{2}$ & $\overline{0}$ & $\overline{1}$\\\hline
			\end{tabular} \qquad \begin{tabular}{|c|c|c|c|c|c|}
			\hline
			$\otimes$ &\rule{0pt}{2.5ex} $\overline{0}$ & $\overline{1}$ & $\overline{2}$\\\hline
			$\rule{0pt}{2.5ex}\overline{0}$ & $\overline{0}$ & $\overline{0}$ & $\overline{0}$\\\hline
			$\rule{0pt}{2.5ex}\overline{1}$ & $\overline{0}$ & $\overline{1}$ & $\overline{2}$\\\hline
			$\rule{0pt}{2.5ex}\overline{2}$ & $\overline{0}$ & $\overline{2}$ & $\overline{1}$\\\hline
		\end{tabular}
	\end{center}
	Note que a soma $\oplus$ e o produto $\otimes$ em $\z_3$ s\~ao comutativos, a soma possui elemento neutro que \'e $\overline{0}$, todo elemento possui
	oposto aditivo. A multiplica\c{c}\~ao possui unidade que \'e $\overline{1}$ e todo elemento n\~ao nulo possui inverso multiplicativo. \'E simples verificar que a soma e o produto em $\z_3$ s\~ao associativos e o produto \'e distributivo em rela\c{c}\~ao \`a soma. Portanto, $(\z_3, \oplus, \otimes)$ \'e um corpo. Al\'em disso, em tal corpo temos
	\[
		(\overline{1} \oplus \overline{1}) \oplus \overline{1} = (\overline{1 + 1}) \oplus \overline{1} = \overline{2} \oplus \overline{1} = \overline{3} = \overline{0}.
	\]
	e $\overline{1} \ne \overline{0}$.

	\item Para $p = 5$ os poss{\'\i}veis restos na divis\~ao inteira s\~ao: $0$, $1$, $2$, $3$ e $4$. Da{\'\i}
		\[
			\z_5 = \{\overline{0}, \overline{1}, \overline{2}, \overline{3}, \overline{4}\}
		\]
	e temos
		\begin{center}
			\begin{tabular}{|c|c|c|c|c|c|c|c|}
				\hline
				$\oplus$ &\rule{0pt}{2.5ex}$\overline{0}$ & $\overline{1}$ & $\overline{2}$ & $\overline{3}$ & $\overline{4}$\\\hline
				$\rule{0pt}{2.5ex}\overline{0}$ & $\overline{0}$ & $\overline{1}$ & $\overline{2}$ & $\overline{3}$ & $\overline{4}$\\\hline
				$\rule{0pt}{2.5ex}\overline{1}$ & $\overline{1}$ & $\overline{2}$ & $\overline{3}$ & $\overline{4}$ & $\overline{0}$\\\hline
				$\rule{0pt}{2.5ex}\overline{2}$ & $\overline{2}$ & $\overline{3}$ & $\overline{4}$& $\overline{0}$ & $\overline{1}$ \\\hline
				$\rule{0pt}{2.5ex}\overline{3}$ & $\overline{3}$ & $\overline{4}$ & $\overline{0}$& $\overline{1}$ & $\overline{2}$ \\\hline
				$\rule{0pt}{2.5ex}\overline{4}$ & $\overline{4}$ & $\overline{0}$ & $\overline{1}$& $\overline{2}$ & $\overline{3}$ \\\hline
			\end{tabular} \qquad 
			\begin{tabular}{|c|c|c|c|c|c|}
				\hline
				$\otimes$ &\rule{0pt}{2.5ex} $\overline{0}$ & $\overline{1}$ & $\overline{2}$ & $\overline{3}$ & $\overline{4}$\\\hline
				$\rule{0pt}{2.5ex}\overline{0}$ & $\overline{0}$ & $\overline{0}$ & $\overline{0}$ & $\overline{0}$ & $\overline{0}$\\\hline
				$\rule{0pt}{2.5ex}\overline{1}$ & $\overline{0}$ & $\overline{1}$ & $\overline{2}$ & $\overline{3}$ & $\overline{4}$\\\hline
				$\rule{0pt}{2.5ex}\overline{2}$ & $\overline{0}$ & $\overline{2}$ & $\overline{4}$& $\overline{1}$ & $\overline{3}$ \\\hline
				$\rule{0pt}{2.5ex}\overline{3}$ & $\overline{0}$ & $\overline{3}$ & $\overline{1}$& $\overline{4}$ & $\overline{2}$ \\\hline
				$\rule{0pt}{2.5ex}\overline{4}$ & $\overline{0}$ & $\overline{4}$ & $\overline{3}$& $\overline{2}$ & $\overline{1}$ \\\hline
			\end{tabular}
		\end{center}
	Note que a soma $\oplus$ e o produto $\otimes$ em $\z_5$ s\~ao comutativos, a soma possui elemento neutro que \'e $\overline{0}$, todo elemento possui
	oposto aditivo. A multiplica\c{c}\~ao possui unidade que \'e $\overline{1}$ e todo elemento n\~ao nulo possui inverso multiplicativo. \'E simples verificar que a soma e o produto em $\z_5$ s\~ao associativos e o produto \'e distributivo em rela\c{c}\~ao \`a soma. Portanto, $(\z_5, \oplus, \otimes)$ \'e um corpo. Al\'em disso, em tal corpo temos
	\[
		\overline{1} \oplus \overline{1} \oplus \overline{1} \oplus \overline{1} \oplus \overline{1} = \overline{1 + 1 + 1 + 1 + 1} = \overline{5} = \overline{0}.
	\]
	e $\overline{1} \ne \overline{0}$.
\end{enumerate}
\end{exemplos}

\begin{teorema}
Para todo $p \in \z$, $p$ n\'umero primo, $(\z_p, \oplus, \otimes)$ \'e um corpo.
\end{teorema}
\begin{prova}
Dados $\overline{x}$, $\overline{y}$, $\overline{z} \in \z_p$ temos:
\begin{enumerate}
	\item[A1)] $\overline{x} \oplus \overline{y} = \overline{x + y} = \overline{y + x} = \overline{y} \oplus \overline{x}$;
	\item[A2)] $(\overline{x} \oplus \overline{y}) \oplus \overline{z} = (\overline{x + y}) \oplus \overline{z} = \overline{(x + y) + z} = \overline{x + (y + z)} = \overline{x} \oplus (\overline{y} \oplus \overline{z}) = \overline{x} \oplus (\overline{y} \oplus \overline{z})$
	\item[A3)] Temos que $\overline{0} \in \z_p$ e para todo $\overline{x} \in \z_p$:
	\[
	\overline{0} \oplus \overline{x} = \overline{x} \oplus \overline{0} = \overline{x + 0} = \overline{x}.
	\]
	Logo $\overline{0}$ \'e o elemento neutro da adi\c{c}\~ao em $\z_p$.
	\item[A4)] Dado $\overline{x} \in \z_p$, tome $\overline{p - x} \in \z_p$, pois $0 \le p - x \le p - 1$. Assim
	\[
	\overline{x} \oplus \overline{p - x} = \overline{p - x} \oplus \overline{x} = \overline{(p - x) + x} = \overline{p} = \overline{0}.
	\]
	Logo todo elemento de $\z_p$ possui oposto aditivo.
	\item[M1)] $\overline{x} \otimes \overline{y} = \overline{x\cdot y} = \overline{y \cdot x} = \overline{y} \otimes \overline{x}$.
	\item[M2)] $(\overline{x} \otimes \overline{y}) \otimes \overline{z} = (\overline{x\cdot y}) \otimes \overline{z} = \overline{(x\cdot y) \cdot z} = 
	\overline{x\cdot (y\cdot z)} = \overline{x} \otimes (\overline{y} \otimes \overline{z})$.
	\item[M3)] O elemento $\overline{1} \in \z_p$ \'e tal que
	\[
	\overline{1} \otimes \overline{x} = \overline{x} \otimes \overline{1} = \overline{x\cdot 1} = \overline{x}
	\]
	para todo $\overline{x} \in \z_p$.
	\item[M4)] Primeiramente, como $p$ \'e um n\'umero primo ent\~ao existem $y$, $z \in \z$ tais que
	\[
	xy + pz = 1
	\]
	para todo $x \in \{1, 2, \dots, p - 1\}$. Logo,
	\begin{align*}
	&\overline{xy + pz} = \overline{1}\\
	&\overline{xy} \oplus \overline{pz} = \overline{1}\\
	&\overline{x} \otimes \overline{y} + \overline{p} \otimes \overline{z} = \overline{1}\\
	&\overline{x} \otimes \overline{y} = \overline{1}
	\end{align*}
	uma vez que $\overline{p} = \overline{0}$. Como $\overline{y}$ \'e obtido pelo resto da divis\~ao inteira de $y$ por $p$, ent\~ao 
	$\overline{y} \in \z_p$. Observe que $y \ne 0$ pois $p \ge 2$ e $y \ne p$ pois sen\~ao $(x + z)p = 1$ o que \'e imposs{\'\i}vel uma vez que $p \ge 2$. Logo $\overline{y} \ne \overline{0}$ e assim todo elemento $\overline{x} \in \z_p$ possui inverso multiplicativo.
	\item[D)] $(\overline{x} \oplus \overline{y}) \otimes \overline{z}= (\overline{x + y}) \otimes \overline{z} = \overline{(x + y) \cdot z} = \overline{xz + yz} = \overline{xz} \oplus \overline{yz} = \overline{x} \otimes \overline{z} \oplus \overline{y} \otimes \overline{z}$.
\end{enumerate}
Portanto, $(\z_p, \oplus, \otimes)$ \'e um corpo, como quer{\'\i}amos demonstrar.
\end{prova}

\begin{observacao}
\begin{enumerate}
	\item Se $p$ n\~ao for um n\'umero primo, $(\z_p, \oplus, \otimes)$ pode n\~ao ser corpo. Por exemplo, em $\z_6$ temos $\overline{2} \ne \overline{0}$ e $\overline{3} \ne \overline{0}$, mas
	\[
	\overline{2}\otimes \overline{3} = \overline{2\cdot 3} = \overline{6} = \overline{0}.
	\]

	\item Para simplificar a nota\c{c}\~ao vamos denotar $\oplus$ por $+$ e $\otimes$ por $\cdot$. Assim vamos dizer simplesmente que $(\z_p, +, \cdot)$ \'e um corpo.
\end{enumerate}
\end{observacao}


\section{Sistemas Lineares}\label{ssub:sistemas_lineares}
Seja $\cp{K}$ um corpo. Consideremos o problema de determinar $n$ escalares, ou seja, $n$ elementos $x_1$, $x_2$, \dots, $x_n$ em $\cp{K}$ que satisfa\c{c}am simultaneamente as equa\c{c}\~oes
\begin{equation}\label{sistemalinear}\index{Sistema Linear}
\begin{cases}
a_{11}x_1 + a_{12}x_2 + \cdots + a_{1n}x_n = b_1\\
a_{21}x_1 + a_{22}x_2 + \cdots + a_{2n}x_n = b_2\\
\qquad \vdots\\
a_{m1}x_1 + a_{m2}x_2 + \cdots + a_{mn}x_n = b_m\\
\end{cases}
\end{equation}
onde $b_1$, \dots, $b_m$ e $a_{ij}$, $1 \le i \le m$, $1 \le j \le n$ s\~ao elementos de $\cp{K}$ previamente conhecidos. Chamamos \eqref{sistemalinear} de um \textbf{sistema de $m$ equa\c{c}\~oes lineares a $n$ inc\'ognitas} $x_1$, $x_2$, \dots, $x_n$. Toda $n$-upla $(\alpha_1, \alpha_2, \dots, \alpha_n)$ onde $\alpha_i \in \cp{K}$ para $1 \le i \le n$, que satisfazem a cada uma das equa\c{c}\~oes de \eqref{sistemalinear} \'e chamada de uma \textbf{solu\c{c}\~ao} do sistema. 

Se $b_1 = b_2 = \cdots = b_m = 0_\cp{K} \in K$, dizemos que o sistema
\begin{equation}\label{sistemalinearhomogeneo}\index{Sistema Linear}
\begin{cases}
a_{11}x_1 + a_{12}x_2 + \cdots + a_{1n}x_n = 0_\cp{K}\\
a_{21}x_1 + a_{22}x_2 + \cdots + a_{2n}x_n = 0_\cp{K}\\
\qquad \vdots\\
a_{m1}x_1 + a_{m2}x_2 + \cdots + a_{mn}x_n = 0_\cp{K}\\
\end{cases}
\end{equation}
\'e um \textbf{sistema linear homog\^eneo}, ou que cada uma de suas equa\c{c}\~oes \'e homog\^enea. Observe que tal sistema sempre possui solu\c{c}\~ao, a saber, $x_1 = x_2 = \cdots = x_n = 0_\cp{K}$.

O m\'etodo mais importante para determinar as solu\c{c}\~oes de um sistema de equa\c{c}\~oes lineares \'e o m\'etodo do \textbf{escalonamento}. Por exemplo, considere o sistema
\begin{equation}\label{exemploplo1}
\begin{cases}
2x_1 - x_2 + x_3 = 0\\
x_1 + 3x_2 + 4x_ 3 = 0
\end{cases}
\end{equation}
onde o corpo considerado \'e $\real$.

Observe que multiplicando a segunda equa\c{c}\~ao de \eqref{exemploplo1} por $-2$ e somando o resultado \`a primeira equa\c{c}\~ao obtemos
\[
-7x_2 - 7x_3 = 0
\]
o que resulta em $x_2 = -x_3$. Agora se multiplicarmos a primeira equa\c{c}\~ao de \eqref{exemploplo1} por $3$ e somarmos com a segunda, obtemos
\[
7x_1 + 7x_3 = 0
\]
e da{\'\i} $x_1 = -x_3$.

Assim para que uma terna $(x_1, x_2, x_3)$ de n\'umeros reais seja solu\c{c}\~ao de \eqref{exemploplo1} deve satisfazer
\[
x_1 = x_2 = -x_3.
\]
Por outro lado, qualquer terna da forma $(a, a, -a)$ \'e solu\c{c}\~ao de \eqref{exemploplo1}. Portanto a solu\c{c}\~ao de \eqref{exemploplo1} \'e da forma
\[
(a, a, -a)
\]
onde $a \in \real$.

No caso de um sistema linear da forma \eqref{sistemalinear}, o processo de elemina\c{c}\~ao de vari\'aveis ser\'a feito mediante o uso de 3 tipos de opera\c{c}\~oes. S\~ao elas:
\begin{itemize}
	\item[$e_1$)] Troca da posi\c{c}\~ao de duas equa\c{c}\~oes.
	\item[$e_2$)] Multiplica\c{c}\~ao de uma equa\c{c}\~ao por um escalar n\~ao nulo.
	\item[$e_3$)] Substitui\c{c}\~ao de uma equa\c{c}\~ao pela soma desta equa\c{c}\~ao com alguma outra.
\end{itemize}

Estas tr\^es opera\c{c}\~oes s\~ao chamadas de \textbf{opera\c{c}\~oes elementares}.\index{Opera\c{c}\~oes Elementares}

\begin{exemplo}
Considere o seguinte sistema sobre o corpo $\real$:
\[
\begin{cases}
x_1 + 4x_2 + 3x_3 = 1\\
2x_1 + 5x_2 + 4x_3 = 4\\
x_1 - 3x_2 - 2x_3 = 5
\end{cases}
\]
Efetuando opera\c{c}\~oes elementares podemos escrever:
\begin{align*}
&\begin{cases}
x_1 + 4x_2 + 3x_3 = 1\\
2x_1 + 5x_2 + 4x_3 = 4 & L_2 \rightarrow L_2 - 2L_1\\
x_1 - 3x_2 - 2x_3 = 5
\end{cases} \sim
\begin{cases}
x_1 + 4x_2 + 3x_3 = 1\\
\phantom{0x_1} -3x_2 - 2x_3 = 2\\
x_1 - 3x_2 - 2x_3 = 5 & L_3 \rightarrow L_2 - L_1
\end{cases}\\ & \sim
\begin{cases}
x_1 + 4x_2 + 3x_3 = 1\\
\phantom{0x_1} - 3x_2 - 2x_3 = 2 & L_2 \rightarrow (-1/3)L_2\\
\phantom{0x_1} - 7x_2 - 5x_3 = 4
\end{cases} \sim
\begin{cases}
x_1 + 4x_2 + 3x_3 = 1\\
\phantom{0x_1} x_2 + (2/3)x_3 = (-2/3)\\
\phantom{0x_1} - 7x_2 - 5x_3 = 4 & L_3 \rightarrow L_3 + 7L_2
\end{cases}\\ & \sim
\begin{cases}
x_1 + 4x_2 + 3x_3 = 1\\
\phantom{0x_1} x_2 + (2/3)x_3 = (-2/3)\\
\phantom{0x_1} \phantom{0x_2}  -(1/3)x_3 = -(2/3)
\end{cases}
\end{align*}
Assim encontramos
\[
x_1 = 3, \quad x_2 = -2, \quad x_3 = 2.
\]
\end{exemplo}

\begin{definicao}\index{Sistemas Equivalentes}
Dois sistemas de equa\c{c}\~oes lineares s\~ao chamados de \textbf{equivalentes} se, e somente, se toda solu\c{c}\~ao de qualquer um dos sistemas \'e solu\c{c}\~ao do outro.
\end{definicao}

Dado um sistema linear
\begin{equation}
\begin{cases}
a_{11}x_1 + a_{12}x_2 + \cdots + a_{1n}x_n = b_1\\
a_{21}x_1 + a_{22}x_2 + \cdots + a_{2n}x_n = b_2\\
\qquad \vdots\\
a_{m1}x_1 + a_{m2}x_2 + \cdots + a_{mn}x_n = b_m\\
\end{cases}
\end{equation}
com o objetivo de simplificar sua nota\c{c}\~ao vamos escrev\^e-lo na forma
\begin{equation}\label{formamatricial}
AX = B
\end{equation}
onde
\begin{enumerate}
	\item
	\[
	A = \begin{bmatrix}
	a_{11} & a_{12} & \cdots & a_{1n}\\
	\vdots & & & \vdots\\
	a_{m1} & a_{m2} & \cdots & a_{mn}
	\end{bmatrix}_{m\times n}; \quad a_{ij} \in \cp{K},\ 1 \le i \le m,\ 1 \le j \le n
	\]
	\'e chamada \textbf{matriz dos coeficientes do sistema};
	\item
	\[
	X = \begin{bmatrix}
	x_1\\
	x_2\\
	\vdots\\
	x_n
	\end{bmatrix}_{n \times 1}
	\]
	\item
	\[
	B = \begin{bmatrix}
	b_1\\
	b_2\\
	\vdots\\
	b_m
	\end{bmatrix}_{m \times 1}; \quad b_1, b_2, \dots, b_m \in \cp{K}.
	\]
\end{enumerate}

Uma outra matriz que podemos associar ao sistema \eqref{sistemalinear} \'e
\[
\begin{amatrix}{4}
a_{11} & a_{12} & \cdots & a_{1n} & b_1\\
a_{21} & a_{22} & \cdots & a_{2n} & b_2\\
\vdots & \vdots & \vdots & \vdots & \vdots\\
a_{m1} & a_{m2} & \cdots & a_{mn} & b_m\\
\end{amatrix}
\]
que \'e chamada de \textbf{matriz ampliada do sistema} ou \textbf{matriz aumentada do sistema}.

Na forma matricial as opera\c{c}\~oes elementares s\~ao descritas como:\index{Opera\c{c}\~oes Elementares!Sobre Matrizes}
\begin{itemize}
	\item[$e_1$)] Trocar a $i$-\'esima linha de $A$ pela $j$-\'esima linha de $A$: $L_i \leftrightarrow L_j$;
	\item[$e_2$)] Multiplica\c{c}\~ao da $i$-\'esima linha de $A$ por um escalar $\alpha \in \cp{K}$ n\~ao nulo: $L_i \rightarrow \alpha L_i$;
	\item[$e_3$)] Substitui\c{c}\~ao da $i$-\'esima linha de $A$ pela $i$-\'esima linha mais $\alpha$ vezes a $j$-\'esima linha: $L_i \rightarrow L_i + \alpha L_j$.
\end{itemize}

\begin{observacao}
Denotaremos a matriz
\[
\begin{bmatrix}
0_{\cp{K}} & 0_{\cp{K}} \cdots & 0_{\cp{K}}\\
0_{\cp{K}} & 0_{\cp{K}} \cdots & 0_{\cp{K}}\\
\vdots & & \vdots\\
0_{\cp{K}} & 0_{\cp{K}} \cdots & 0_{\cp{K}}
\end{bmatrix},
\]
onde $0_{\cp{K}}$ \'e o elemento neutro da soma no corpo $\cp{K}$, simplesmente por $0$.
\end{observacao}

Uma raz\~ao para nos restringirmos a estes tr\^es tipos simples de opera\c{c}\~oes sobre linhas \'e que, tendo efetuado uma tal opera\c{c}\~ao $e$ sobre uma matriz $A$, podemos desfazer essa opera\c{c}\~ao efetuando uma opera\c{c}\~ao de mesmo tipo sobre $e(A)$.

\begin{teorema}
A cada opera\c{c}\~ao elementar sobre linhas $e$, corresponde uma opera\c{c}\~ao elementar sobre linhas $e'$, do mesmo tipo que $e$, tal que $e'(e(A)) = A$ para qualquer matriz $A$. Em outras palavras, a opera\c{c}\~ao inversa de uma opera\c{c}\~ao elementar sobre linhas existe e \'e uma opera\c{c}\~ao elementar sobre linhas do mesmo tipo.
\end{teorema}
\begin{prova}
Vamos verificar que cada uma das opera\c{c}\~oes elementares possui uma opera\c{c}\~ao inversa. Seja $A$ uma matriz $m \times n$ sobre o corpo $\cp{K}$
\[
A = \begin{bmatrix}
a_{11} & a_{12} & \cdots & a_{1n}\\
a_{21} & a_{22} & \cdots & a_{2n}\\
\vdots & & & \vdots\\
a_{m1} & a_{m2} & \cdots & a_{mn}
\end{bmatrix}.
\]
\begin{enumerate}
	\item [e1)] Suponha que $e$ seja a opera\c{c}\~ao que troca a linha $i$ pela linha $j$ de $A$. Temos
	\[
	e(A) = 
	\begin{bmatrix}
	a_{11} & a_{12} & \cdots & a_{1n}\\
	a_{21} & a_{22} & \cdots & a_{2n}\\
	\vdots\\
	a_{j1} & a_{j2} & \cdots & a_{jn}\\
	\vdots\\
	a_{i1} & a_{i2} & \cdots & a_{in}\\
	\vdots\\
	a_{m1} & a_{m2} & \cdots & a_{mn}
	\end{bmatrix}.
	\]
	Ent\~ao, seja $e'$ a opera\c{c}\~ao que troca a linha $i$ pela linha $j$ de $e(A)$. Assim
	\[
	e'(e(A)) = A
	\]
	como quer{\'\i}amos.

	\item [e2)] Suponha que $e$ seja a opera\c{c}\~ao que multiplica a $i$-\'esima de $A$ por $\alpha \in \cp{K}$, onde $\alpha \ne 0_\cp{K}$. Temos
	\[
	e(A) = 
	\begin{bmatrix}
	a_{11} & a_{12} & \cdots & a_{1n}\\
	a_{21} & a_{22} & \cdots & a_{2n}\\
	\vdots\\
	\alpha a_{i1} & \alpha a_{i2} & \cdots & \alpha a_{in}\\
	\vdots\\
	a_{m1} & a_{m2} & \cdots & a_{mn}
	\end{bmatrix}.
	\]
	Seja $e'$ a opera\c{c}\~ao que multiplica a linha $i$ de $e(A)$ por $\alpha^{-1} \in \cp{K}$. Ent\~ao
	\[
	e'(e(A)) = A.
	\]
	\item [e3)] Suponha que $e$ seja a opera\c{c}\~ao que substitui a linha $i$ de $A$ pela linha $i$ mais $\alpha$ vezes a linha $j$. Temos
	\[
	e(A) = 
	\begin{bmatrix}
	a_{11} & a_{12} & \cdots & a_{1n}\\
	a_{21} & a_{22} & \cdots & a_{2n}\\
	\vdots\\
	a_{i1} + \alpha a_{j1} & a_{i2} + \alpha a_{j2} & \cdots & a_{in} + \alpha a_{jn}\\
	\vdots\\
	a_{m1} & a_{m2} & \cdots & a_{mn}
	\end{bmatrix}.
	\]
	Seja $e'$ a opera\c{c}\~ao que substitui a linha $i$ de $e(A)$ pela linha $i$ mais $(-\alpha)$ vezes a linha $j$. Ent\~ao
	\[
	e'(e(A)) = 
	\begin{bmatrix}
	a_{11} & a_{12} & \cdots & a_{1n}\\
	a_{21} & a_{22} & \cdots & a_{2n}\\
	\vdots\\
	a_{i1} + \alpha a_{j1} + (-\alpha)a_{j1} & a_{i2} + \alpha a_{j2} + (-\alpha)a_{j2} & \cdots & a_{in} + \alpha a_{jn} + (-\alpha)a_{jn}\\
	\vdots\\
	a_{m1} & a_{m2} & \cdots & a_{mn}
	\end{bmatrix}.
	\]
	e assim
	\[
	e'(e(A)) = A.
	\]
\end{enumerate}
Portanto cada opera\c{c}\~ao elementar sobre linhas possui uma opera\c{c}\~ao inversa.
\end{prova}

\begin{definicao}\index{Matriz!Linha Equivalente}
Se $A$ e $B$ s\~ao matrizes $m \times n$, dizemos que $B$ \'e \textbf{linha-equivalente} a $A$, se $B$ for obtida de $A$ atrav\'es de uma quantidade finita de opera\c{c}\~oes elementares sobre as linhas de $A$.
\end{definicao}

\begin{notacao}
$A \rightarrow B$ ou $A \sim B$.
\end{notacao}

\begin{exemplo}
A matriz
\[
B = \begin{bmatrix}
1 & 0\\
0 & 1\\
0 & 0
\end{bmatrix}
\]
\'e linha equivalente \`a matriz
\[
A = \begin{bmatrix}
\phantom{-}1 & \phantom{-}0\\
\phantom{-}4 & -1\\
-3 & \phantom{-}4
\end{bmatrix}
\]
pois
		\begin{align*}
			A &= \begin{gmatrix}[b]
			\phantom{-}1 & \phantom{-}0\\
			\phantom{-}4 & -1\\
			-3 & \phantom{-}4
			\rowops
			\add[-4]{0}{1}
			\end{gmatrix}\leadsto\begin{gmatrix}[b]
			\phantom{-}1 & \phantom{-}0\\
			\phantom{-}0 & -1\\
			-3 & \phantom{-}4
			\rowops
			\add[3]{0}{2}
			\end{gmatrix}\leadsto\begin{gmatrix}[b]
			1 & \phantom{-}0\\
			0 & -1\\
			0 & \phantom{-}4
			\rowops
			\mult1{\times -1}
			\end{gmatrix}\\&\leadsto\begin{gmatrix}[b]
			1 & 0\\
			0 & 1\\
			0 & 4
			\rowops
			\add[-4]{1}{2}
			\end{gmatrix}\leadsto\begin{bmatrix}
			1 & 0\\
			0 & 1\\
			0 & 0
			\end{bmatrix} = B.
		\end{align*}
\end{exemplo}

\begin{teorema}
Se $X_1$ e $X_2$ s\~ao duas solu\c{c}\~oes de
\[
AX = 0,
\]
ent\~ao $\alpha X_1 + \beta X_2$ tamb\'em \'e solu\c{c}\~ao de $AX = 0$, para quaisquer $\alpha$, $\beta \in \cp{K}$.
\end{teorema}

\begin{teorema}
Se $A$ e $B$ s\~ao matrizes $m \times n$ que s\~ao linha-equivalentes, ent\~ao os sistemas homog\^eneos de equa\c{c}\~oes lineares $AX = 0$ e $BX = 0$ t\^em exatamente as mesmas solu\c{c}\~oes.
\end{teorema}
\begin{prova}
Suponha que podemos obter a matriz $B$ \`a partir da matriz $A$ por meio de uma sequ\^encia finita de opera\c{c}\~oes elementares sobre linhas:
\[
A = A_0 \sim A_1 \sim A_2 \sim \cdots \sim A_r = B.
\]
Nesta situa\c{c}\~ao, para provar que $AX = 0$ e $BX = 0$ tem as mesmas solu\c{c}\~oes basta provar que $A_iX = 0$ e $A_{i + 1}X = 0$ tem as mesmas solu\c{c}\~oes, isto \'e, que uma opera\c{c}\~ao elementar sobre linhas n\~ao altera o conjunto das solu\c{c}\~oes.

Assim podemos supor que $B$ \'e obtida de $A$ por meio de uma \'unica opera\c{c}\~ao elementar. Qualquer que seja a opera\c{c}\~ao elementar, $e_1$ ou $e_2$ ou $e_3$, cada equa\c{c}\~ao do sistema $BX = 0$ ser\'a uma combina\c{c}\~ao das equa\c{c}\~oes do sistema $AX = 0$. Como a inversa de uma opera\c{c}\~ao elementar sobre linhas \'e ainda uma opera\c{c}\~ao elementar sobre linhas, cada equa\c{c}\~ao de $AX = 0$ tamb\'em ser\'a uma combina\c{c}\~ao das equa\c{c}\~oes em $BX = 0$. Logo toda solu\c{c}\~ao de $AX = 0$ tamb\'em \'e solu\c{c}\~ao de $BX = 0$ e toda solu\c{c}\~ao de $BX = 0$ tamb\'em \'e solu\c{c}\~ao de $AX = 0$, como quer{\'\i}amos.
\end{prova}

\begin{exemplo}
Considere o sistema homog\^eneo $AX = 0$, onde:
\begin{enumerate}[label={\arabic*})]
	\item $A = \begin{bmatrix}
	2 & -1 & \phantom{-}3 & \phantom{-}2\\
	1 & \phantom{-}4 & \phantom{-}0 & -1\\
	2 & \phantom{-}6 & -1 & \phantom{-}5
	\end{bmatrix}.$
	Para encontrar a solu\c{c}\~ao deste sistema s\'o precisamos encontrar uma matriz $B$ que seja linha equivalente \`a $A$ e que seja mais f\'acil de determinar a solu\c{c}\~ao do sistema resultante. Assim, vamos executar as opera\c{c}\~oes elementares em $A$ de modo a simplific\'a-la:
	\begin{align*}
	A = \begin{gmatrix}[b]
	2 & -1 & \phantom{-}3 & \phantom{-}2\\
	1 & \phantom{-}4 & \phantom{-}0 & -1\\
	2 & \phantom{-}6 & -1 & \phantom{-}5
	\rowops
	\swap01
	\end{gmatrix}\leadsto\begin{gmatrix}[b]
	1 & \phantom{-}4 & \phantom{-}0 & -1\\
	2 & -1 & \phantom{-}3 & \phantom{-}2\\
	2 & \phantom{-}6 & -1 & \phantom{-}5
	\rowops
	\add[-2]{0}{1}
	\add[-2]{0}{2}
	\end{gmatrix}\\\leadsto\begin{gmatrix}[b]
	1 & \phantom{-}4 & \phantom{-}0 & -1\\
	0 & -9 & \phantom{-}3 & \phantom{-}4\\
	0 & -2 & -1 & \phantom{-}7
	\rowops
	\swap12
	\end{gmatrix}\leadsto\begin{gmatrix}[b]
	1 & \phantom{-}4 & 0 & -1\\
	0 & -2 & -1 & \phantom{-}7\\
	0 & -9 & \phantom{-}3 & \phantom{-}4
	\rowops
	\mult1{\times -1/2}
	\end{gmatrix}\\\leadsto\begin{gmatrix}[b]
	1 & \phantom{-}4 & 0 & -1\\
	0 & \phantom{-}1 & 1/2 & -7/2\\
	0 & -9 & 3 & \phantom{-}4
	\rowops
	\add[9]{1}{2}
	\end{gmatrix}\leadsto\begin{bmatrix}
	1 & 4 & 0 & -1\\
	0 & 1 & 1/2 & -7/2\\
	0 & 0 & 15/2 & -55/2
	\end{bmatrix}
	\end{align*}
	assim obtemos o sistema
	\[
	\begin{cases}
	x_1 + 4x_2 - x_4 = 0\\
	x_2 + (1/2)x_3 - (7/2)x_4 = 0\\
	(15/2)/x_3 - (55/2)x_4 = 0
	\end{cases}.
	\]
	Isolando $x_3$ na \'ultima equa\c{c}\~ao temos a solu\c{c}\~ao dada por
	\[
	S = \left\{\left(\dfrac{-17}{3}x_4, \dfrac{5}{3}x_4, \dfrac{11}{3}x_4, x_4\right) \mid x_4 \in \real\right\}.
	\]

	\item $A = \begin{bmatrix}
	-1 & i\\
	-i & 1\\
	\phantom{-}1 & 2
	\end{bmatrix}.$ Temos:
	\begin{align*}
	A &= \begin{gmatrix}[b]
	-1 & i\\
	-i & 1\\
	\phantom{-}1 & 2
	\rowops
	\swap02
	\end{gmatrix}\leadsto\begin{gmatrix}[b]
	\phantom{-}1 & 2\\
	-i & 1\\
	-1 & i
	\rowops
	\add[i]{0}{1}
	\add{0}{2}
	\end{gmatrix}\\&\leadsto\begin{gmatrix}[b]
	1 & 2\\
	0 & 1 + 2i\\
	0 & 2 + i
	\rowops
	\mult1{\times \dfrac{1 - 2i}{5}}
	\end{gmatrix}\leadsto\begin{gmatrix}[b]
	1 & 2\\
	0 & 1\\
	0 & 2 + i
	\rowops
	\add[-(2 + i)]{1}{2}
	\end{gmatrix}\\&\leadsto\begin{bmatrix}
	1 & 2\\
	0 & 1\\
	0 & 0
	\end{bmatrix}.
	\end{align*}
	Assim obtemos o sistema
	\[
	\begin{cases}
	x_1 + 2x_2 = 0\\
	x_2 = 0
	\end{cases}
	\]
	cuja solu\c{c}\~ao \'e $x_1 = x_2 = 0$.
\end{enumerate}
\end{exemplo}

\begin{definicao}\label{linhareduzida}\index{Matriz!Linha-reduzida}
Uma matriz R $m \times n$ \'e chamada de \textbf{linha-reduzida} se:
\begin{enumerate}[label={\roman*})]
\item o primeiro elemento n\~ao nulo em cada linha n\~ao nula de $R$ \'e $1_\cp{K}$.
\item cada coluna de $R$ que cont\'em o primeiro elemento n\~ao nulo de alguma linha tem todos os seus outros elementos nulos.
\end{enumerate}
\end{definicao}

\begin{exemplo}
\begin{enumerate}[label={\arabic*})]
	\item Um exemplo de uma matriz linha-reduzida \'e a matriz identidade $n \times n$. Tal matriz pode ser definida por
	\[
	I = (a_{ij})_{1 \le i,j \le n}
	\]
	onde
	\[
	a_{ij} = \delta_{ij} = \begin{cases}
	1, & \mbox{ se } i = j\\
	0, & \mbox{ se } i \ne j 
	\end{cases}.
	\]
	O s{\'\i}mbolo $\delta_{ij}$ \'e chamada \textbf{s{\'\i}mbolo de Kronecher} \'e ser\'a utilizado com certa frequ\^encia.
	\item As matrizes
	\[
	A = \begin{bmatrix}
	1 & 0 & \phantom{-}0 & 0\\
	0 & 1 & -1 & 0\\
	0 & 0 & \phantom{-}1 & 0
	\end{bmatrix}; \quad B = \begin{bmatrix}
	0 & 2 & \phantom{-}1\\
	1 & 0 & -3\\
	0 & 0 & \phantom{-}0
	\end{bmatrix}
	\]
	n\~ao s\~ao linha-reduzidas.
\end{enumerate}
\end{exemplo}

\begin{teorema}
Toda matriz $m \times n$ sobre um corpo $\cp{K}$ \'e linha-equivalente a uma matriz linha-reduzida.
\end{teorema}
\begin{prova}
Seja $A$ uma matriz $m \times n$ sobre um corpo $\cp{K}$. Se todo elemento na primeira linha de $A$ \'e $0_\cp{K}$, ent\~ao a condi\c{c}\~ao (a) de \eqref{linhareduzida} est\'a satisfeita no que diz respeito a linha 1. Se a linha 1 tem um elemento n\~ao nulo, seja $r$ o menor inteiro positivo $j$ tal que $a_{1r} \ne 0$. Multiplique a linha 1 por $a_{1r}^{-1}$ e condi\c{c}\~ao (a) de \eqref{linhareduzida} est\'a satisfeita em rela\c{c}\~ao a linha 1. Agora, para cada $i \ge 2$, somemos $-a_{ir}$ vezes a linha 1 \`a linha i. Assim o primeiro elemento n\~ao nulo da linha 1 ocorre na coluna $r$, este elemento \'e $1_\cp{K}$, e todos os outros elementos da coluna $r$ s\~ao nulos.

Considere agora a matriz que resultou das opera\c{c}\~oes acima. Se todo elemento na linha 2 \'e nulo, nada h\'a a fazer. Se algum elemento na linha 2 \'e n\~ao nulo, multiplicamos a linha 2 por um escalar de modo que o primeiro elemento n\~ao nulo da linha 2 seja $1_\cp{K}$. Caso o primeiro elemento n\~ao nulo da linha 1 ocorra na coluna $r$, o primeiro elemento n\~ao nulo da linha 2 n\~ao pode ocorrer na coluna $r$. Digamos ent\~ao que ele ocorra na coluna $r'$. Somando m\'ultiplos adequados da linha 2 \`as diversas linhas, podemos fazer com que todos os elementos da coluna $r'$ seja nulos, com exce\c{c}\~ao do elemento $1_\cp{K}$ da linha 2. O importante a ser observado \'e: ao efetuarmos estas \'ultimas opera\c{c}\~oes, n\~ao alteramos os elementos da linha 1 na colunas 1, 2, \dots, $r$; al\'em disso, n\~ao alteramos nenhum elemento da coluna $r$. \'E claro que, se a linha 1 fosse identicamente nula, as opera\c{c}\~oes com a linha 2 n\~ao afetariam a linha 1.

Operando com uma linha de cada vez da maneira acima, \'e evidente que, com uma quantidade finita de passos, chegamos a uma matriz linha-reduzida.
\end{prova}

\begin{definicao}\index{Matriz!Na forma escada}
Uma matriz $R$ $m \times n$ \'e chamada uma \textbf{matriz linha-reduzida \`a forma em escada} se:
\begin{enumerate}[label={\roman*})]
	\item $R$ \'e linha-reduzida;
	\item toda linha de $R$ cujos elementos s\~ao todos nulos ocorre abaixo de todas as linhas que possuem um elemento n\~ao-nulo;
	\item se as linhas 1, 2, \dots, $r$ s\~ao as linhas n\~ao-nulas de $R$ e se o primeiro elemento n\~ao-nulo da linha $i$ ocorre na coluna $k_i$, $i = 1$, \dots, $r$, ent\~ao $k_1 < k_2 < \cdots < k_r$.
\end{enumerate}
\end{definicao}

\begin{exemplo}
\begin{enumerate}[label={\arabic*})]
	\item  A matriz identidade e a matriz nula s\~ao linha-reduzidas \`a forma escada;
	\item $\begin{bmatrix}
	1 & 0 & \phantom{-}0 & 0\\
	0 & 1 & -1 & 0\\
	0 & 0 & \phantom{-}1 & 0
	\end{bmatrix}$ N\~ao \'e linha-reduzida \`a forma escada.
	\item $\begin{bmatrix}
	0 & 2 & \phantom{-}1\\
	1 & 0 & -3\\
	0 & 0 & \phantom{-}0
	\end{bmatrix}$ N\~ao \'e linha-reduzida \`a forma escada.
	\item $\begin{bmatrix}
	0 & 1 & -3 & 0 & 2\\
	0 & 0 & \phantom{-}0 & 1 & 2\\
	0 & 0 & \phantom{-}0 & 0 & 0
	\end{bmatrix}$ \'E linha-reduzida \`a forma escada.
\end{enumerate}
\end{exemplo}

\begin{teorema}
Toda matriz $A$ $m \times n$ \'e linha-equivalente a uma matriz linha-reduzida \`a forma em escada.
\end{teorema}
\begin{prova}
Sabemos que $A$ \'e linha-equivalente a uma matriz linha-reduzida. Portanto, basta notar que, efetuando uma quantidade finita de permuta\c{c}\~oes das linhas de uma matriz linha-reduzida, podemos transform\'a-la numa matriz linha-reduzida \`a forma em escada.
\end{prova}

\begin{definicao}\index{Posto!de uma matriz}\index{Nulidade!de uma matriz}
Dada uma matriz $A$ $m \times n$, seja $B$ a matriz $m \times n$ linha-reduzida \`a forma em escada linha-equivalente a $A$. O \textbf{posto} de $A$, denotado por $p$, \'e o n\'umero de linhas n\~ao-nulas de $B$. A \textbf{nulidade} de $A$ \'e o n\'umero $n - p$.
\end{definicao}

\begin{exemplo}
Qual o posto e a nulidade da matriz $A$, onde
\[
A = \begin{bmatrix}
\phantom{-}1 & \phantom{-}2 & 1 & 0\\
-1 & \phantom{-}0 & 3 & 5\\
\phantom{-}1 & -2 & 1 & 1
\end{bmatrix}?
\]
Precisamos primeiro reduzir $A$ a sua forma escada:
\begin{align*}
A &= \begin{gmatrix}[b]
\phantom{-}1 & \phantom{-}2 & 1 & 0\\
-1 & \phantom{-}0 & 3 & 5\\
\phantom{-}1 & -2 & 1 & 1
\rowops
\add{0}{1}
\add[-1]{0}{2}
\end{gmatrix}\leadsto\begin{gmatrix}[b]
1 & \phantom{-}2 & 1 & 0\\
0 & \phantom{-}2 & 4 & 5\\
0 & -4 & 0 & 1
\rowops
\mult1{\times (1/2)}
\end{gmatrix}\\&\leadsto\begin{gmatrix}[b]
1 & \phantom{-}2 & 1 & 0\\
0 & \phantom{-}1 & 2 & 5/2\\
0 & -4 & 0 & 1
\rowops
\add[-2]{1}{0}
\add[4]{1}{2}
\end{gmatrix}\leadsto\begin{gmatrix}[b]
1 & 0 & -3 & -5\\
0 & 1 & \phantom{-}2 & \phantom{-}5/2\\
0 & 0 & \phantom{-}8 & \phantom{-}11
\rowops
\mult2{\times 1/8}
\end{gmatrix}\\&\leadsto\begin{gmatrix}[b]
1 & 0 & -3 & -5\\
0 & 1 & \phantom{-}2 & \phantom{-}5/2\\
0 & 0 & \phantom{-}1 & \phantom{-}11/8
\rowops
\add[3]{2}{0}
\add[-2]{2}{1}
\end{gmatrix}\\&\leadsto\begin{bmatrix}
1 & 0 & 0 & -7/8\\
0 & 1 & 0 & -1/4\\
0 & 0 & 1 & \phantom{-}11/8
\end{bmatrix}
\end{align*}
Logo o posto de $A$ \'e $p = 3$ e a nulidade \'e $n - p = 4 - 3 = 1$.
\end{exemplo}

Considere o sistema
\begin{equation}\label{equacaolinear}
AX = B
\end{equation}
onde $A$ \'e uma matriz $m \times n$ e $B$ \'e uma matriz $m \times 1$, ambas com entradas no corpo $\cp{K}$ e $X$ \'e uma matriz $n \times 1$. Observe que, enquanto uma sistema homog\^eneo $AX = 0$ sempre admite a solu\c{c}\~ao
\[
x_1 = x_2 = \cdots = x_n = 0_\cp{K},
\]
um sistema n\~ao homog\^eneo pode ter:
\begin{enumerate}
	\item Uma \'unica solu\c{c}\~ao $x_1 = \alpha_1$, $x_2 = \alpha_2$, \dots, $x_n = \alpha_n$, onde $\alpha_i \in \cp{K}$, para $i = 1$, 2, \dots, $n$. Neste caso dizemos que o sistema \'e \textbf{poss{\'\i}vel e determinado}.
	\item Mais de uma solu\c{c}\~ao. Neste caso dizemos que o sistema \'e \textbf{poss{\'\i}vel e indeterminado}. Caso o corpo $\cp{K}$ tenha infinitos elementos, o sistema ter\'a infinitas solu\c{c}\~oes.
	\item Nenhuma solu\c{c}\~ao. Neste caso dizemos o que sistema \'e \textbf{imposs{\'\i}vel}.
\end{enumerate}

Com o objetivo de resolver o sistema \eqref{equacaolinear} vamos come\c{c}ar formando a matriz ampliada
\[
P = [A|B] = \begin{amatrix}{4}
a_{11} & a_{12} & \dots & a_{1n} & b_1\\
a_{21} & a_{22} & \dots & a_{2n} & b_2\\
\vdots & \vdots & \vdots & \vdots & \vdots\\
a_{m1} & a_{m2} & \dots & a_{mn} & b_m\\
\end{amatrix}_{m \times (n + 1)}.
\]

Sabemos que $P$ \'e linha-equivalente a uma matriz linha-reduzida \`a forma em escada $R$. A \'ultima coluna de $R$ cont\'em elementos $z_1$, $z_2$, \dots, $z_m$ que s\~ao resultados das opera\c{c}\~oes elementares aplicadas \`a matriz $P$. Seja 
\[
Z = \begin{bmatrix}
z_1\\
z_2\\
\vdots\\
z_m
\end{bmatrix}.
\]
Ent\~ao $R$ pode ser escrita como $R = [R' \mid Z]$. Como no caso homog\^eneo, \'e poss{\'\i}vel mostrar que os sistemas
\[
AX = B \mbox{ e } R'X = Z
\]
possuem exatamente as mesmas solu\c{c}\~oes.

As possibilidades para as solu\c{c}\~oes de tal sistema s\~ao descritas no seguinte teorema:

\begin{teorema}
Considere o sistema
\[
AX = B
\]
onde $A$ \'e uma matriz $m \times n$ e $B$ \'e uma matriz $m \times 1$, ambas com entradas no corpo $\cp{K}$ e $X$ \'e uma matriz $n \times 1$. Ent\~ao:
\begin{enumerate}[label={\roman*})]
	\item O sistema tem solu\c{c}\~ao se, e somente se, o posto da matriz ampliada \'e igual ao posto da matriz dos coeficientes.

	\item Se a matriz ampliada e a matriz dos coeficientes t\^em o mesmo posto $p$ e $p = n$, ent\~ao a solu\c{c}\~ao \'e \'unica.

	\item Se a matriz ampliada e a matriz dos coeficientes t\^em o mesmo posto $p$ e $p < n$, ent\~ao podemos escolher $n - p$ vari\'aveis, e as outras $p$ vari\'aveis ser\~ao dadas em fun\c{c}\~ao destas $n - p$ vari\'aveis escolhidas.
\end{enumerate}
O n\'umero $n - p$ \'e chamado de \textbf{grau de liberdade} e as $n - p$ vari\'aveis s\~ao chamadas de \textbf{vari\'aveis livres}.
\end{teorema}
\begin{prova}
\textit{$1^a$ Parte: Se existe solu\c{c}\~ao para o sistema, ent\~ao a matriz ampliada e a matriz dos coeficientes t\^em o mesmo posto:} Para mostrar isso, vamos provar que se a matriz ampliada e a matriz dos coeficientes tiverem postos diferentes, ent\~ao o sistema n\~ao ter\'a solu\c{c}\~ao. Observe primeiro que o posto da matriz ampliada n\~ao pode ser menor que o posto da matriz dos coeficientes uma vez que a matriz ampliada \'e formada a partir da matriz dos coeficientes. Assim o \'unico caso poss{\'\i}vel \'e o posto da matriz ampliada ser maior que o posto da matriz dos coeficientes. Ent\~ao esta matriz, quando reduzida \`a forma em escada deve conter uma linha da forma
\[
\begin{bmatrix}
0_\cp{K} & 0_\cp{K} & \cdots & 0_\cp{K} & \mid & 1_\cp{K}
\end{bmatrix}.
\]
Logo o sistema associado a essa matriz tem uma equa\c{c}\~ao do tipo
\[
0_\cp{K}x_1 + 0_\cp{K}x_2 + \cdots + 0_\cp{K}x_n = 1_\cp{K}
\]
o que \'e imposs{\'\i}vel. Logo n\~ao existe solu\c{c}\~ao.

\textit{$2^a$ Parte: Se o posto \'e igual, ent\~ao existe solu\c{c}\~ao:} Nesta situa\c{c}\~ao podem ocorrer dois casos:
\begin{enumerate}
	\item Se $p = n$, ent\~ao a matriz linha-reduzida \`a forma em escada tem a forma
	\[
	\begin{amatrix}{5}
	1_\cp{K} & 0_\cp{K} & 0_\cp{K} & \cdots & 0_\cp{K} & z_1\\
	0_\cp{K} & 1_\cp{K} & 0_\cp{K} & \cdots & 0_\cp{K} & z_2\\
	\vdots & \vdots & \vdots & \vdots & \vdots & \vdots\\
	0_\cp{K} & 0_\cp{K} & 0_\cp{K} & \cdots & 1_\cp{K} & z_n\\
	0_\cp{K} & 0_\cp{K} & 0_\cp{K} & 0_\cp{K} & 0_\cp{K} & 0_\cp{K}\\
	\vdots & \vdots & \vdots & \vdots & \vdots & \vdots\\
	0_\cp{K} & 0_\cp{K} & 0_\cp{K} & 0_\cp{K} & 0_\cp{K} & 0_\cp{K}
	\end{amatrix}
	\]
	e a solu\c{c}\~ao do sistema ser\'a
	\[
	x_1 = z_1, x_2 = z_2, \cdots, x_n = z_n.
	\]
	\item Se $p \ne n$, ent\~ao devemos ter $p < n$. Caso $p > n$, como a matriz est\'a na forma escada o elemento $1_\cp{K}$ deve ocorrer em duas linhas diferentes, mas na mesma coluna. Mas neste caso, podemos anular uma destas linhas repetidas. Logo, $p < n$. Neste caso a matriz na forma escada pode ter a forma:
	\begin{enumerate}
		\item \[
		\begin{amatrix}{9}
		1_\cp{K} & 0_\cp{K} & 0_\cp{K} & \cdots & 0_\cp{K} & a_{1p+1} & a_{1p+2} & \cdots & a_{1n} & z_1\\
		0_\cp{K} & 1_\cp{K} & 0_\cp{K} & \cdots & 0_\cp{K} & a_{2p+1} & a_{2p+2} & \cdots & a_{2n} & z_2\\
		\vdots & \vdots & \vdots & \vdots & \vdots & \vdots & \vdots & \vdots & \vdots & \vdots\\
		0_\cp{K} & 0_\cp{K} & 0_\cp{K} & \cdots & 1_\cp{K} & a_{pp+1} & a_{pp+2} & \cdots & a_{pn} & z_p\\
		0_\cp{K} & 0_\cp{K} & 0_\cp{K} & 0_\cp{K} & 0_\cp{K} & 0_\cp{K} & 0_\cp{K} & \cdots & 0_\cp{K} & 0_\cp{K}\\
		\vdots & \vdots & \vdots & \vdots & \vdots & \vdots & \vdots & \vdots & \vdots & \vdots\\
		0_\cp{K} & 0_\cp{K} & 0_\cp{K} & 0_\cp{K} & 0_\cp{K} & 0_\cp{K} & 0_\cp{K} & \cdots & 0_\cp{K}  & 0_\cp{K}
		\end{amatrix}.
		\]
		Neste caso teremos
		\[
		\begin{cases}
		x_1 = z_1 + (-a_{1 p + 1})x_{p + 1} + (-a_{1 p + 2})x_{p + 2} + \cdots + (-a_{1n})x_{n}\\
		x_2 = z_2 + (-a_{2 p + 1})x_{p + 1} + (-a_{2 p + 2})x_{p + 2} + \cdots + (-a_{2n})x_{n}\\
		\qquad \vdots\\
		x_p = z_p + (-a_{p p + 1})x_{p + 1} + (-a_{p p + 2})x_{p + 2} + \cdots + (-a_{pn})x_{n}\\
		\end{cases}
		\]
		e o sistema ter\'a mais de uma solu\c{c}\~ao, sendo $x_{p + 1}$ , $x_{p + 2}$, \dots, $x_n$ as vari\'aveis livres.
		\item Uma segunda forma a ser considerada para a matriz reduzida \'e
		\[
		\begin{amatrix}{9}
		0_\cp{K} & 1_\cp{K} & 0_\cp{K} & \cdots & 0_\cp{K} & a_{1p+2} & a_{1p+3} & \cdots & a_{1n} & z_1\\
		0_\cp{K} & 0_\cp{K} & 1_\cp{K} & \cdots & 0_\cp{K} & a_{2p+2} & a_{2p+3} & \cdots & a_{2n} & z_2\\
		\vdots & \vdots & \vdots & \vdots & \vdots & \vdots & \vdots & \vdots & \vdots & \vdots\\
		0_\cp{K} & 0_\cp{K} & 0_\cp{K} & \cdots & 1_\cp{K} & a_{pp+2} & a_{pp+3} & \cdots & a_{pn} & z_p\\
		0_\cp{K} & 0_\cp{K} & 0_\cp{K} & 0_\cp{K} & 0_\cp{K} & 0_\cp{K} & 0_\cp{K} & \cdots & 0_\cp{K} & 0_\cp{K}\\
		\vdots & \vdots & \vdots & \vdots & \vdots & \vdots & \vdots & \vdots & \vdots & \vdots\\
		0_\cp{K} & 0_\cp{K} & 0_\cp{K} & 0_\cp{K} & 0_\cp{K} & 0_\cp{K} & 0_\cp{K} & \cdots & 0_\cp{K}  & 0_\cp{K}
		\end{amatrix}.
		\]
		Neste caso teremos
		\[
		\begin{cases}
		x_2 = z_1 + (-a_{1 p + 2})x_{p + 2} + (-a_{1 p + 3})x_{p + 3} + \cdots + (-a_{1n})x_{n}\\
		x_3 = z_2 + (-a_{2 p + 3})x_{p + 3} + (-a_{2 p + 3})x_{p + 3} + \cdots + (-a_{2n})x_{n}\\
		\qquad \vdots\\
		x_{p+1} = z_p + (-a_{p p + 2})x_{p + 2} + (-a_{p p + 3})x_{p + 3} + \cdots + (-a_{pn})x_{n}\\
		\end{cases}
		\]
		e o sistema ter\'a mais de uma solu\c{c}\~ao, sendo $x_1$, $x_{p + 1}$ , $x_{p + 2}$, \dots, $x_n$ as vari\'aveis livres.
	\end{enumerate}
	Prosseguindo com esse racioc{\'\i}nio, vemos que para qualquer posto $p < n$ teremos um sistema com mais de uma solu\c{c}\~ao e $n - p$ vari\'aveis livres.
\end{enumerate}
Portanto a condi\c{c}\~ao (i) do teorema est\'a provada.

Observe que os itens (ii) e (iii) foram automaticamente demonstrados nos itens (a) e (b) anteriores.

Logo o teorema est\'a provado.
\end{prova}

\begin{exemplo}
Encontre a solu\c{c}\~ao dos seguintes sistemas lineares:
\begin{enumerate}[label={\arabic*})]
	\item $\begin{cases}
	x + 3y + z = 0\\
	2x + 6y + 2z = 0\\
	-x - 3y - z = 0
	\end{cases}$ em $\real$.
	\begin{solucao}
	A matriz dos coeficentes deste sistema \'e
	\[
	\begin{bmatrix}
	\phantom{-}1 & \phantom{-}3 & 1\\
	\phantom{-}2 & \phantom{-}6 & 2\\
	-1 & -3 & -1
	\end{bmatrix}.
	\]
	Aplicando as opera\c{c}\~oes elementares para reduzir $A$ \`a forma em escada:
	\begin{align*}
	A = \begin{gmatrix}[b]
	\phantom{-}1 & \phantom{-}3 & \phantom{-}1 \\
	\phantom{-}2 & \phantom{-}6 & \phantom{-}2 \\
	-1 & -3 & -1 
	\rowops
	\add[-2]{0}{1}
	\add{0}{2}
	\end{gmatrix}\kern-6.23pt\leadsto&\begin{bmatrix}
	1 & 3 & 1 \\
	0 & 0 & 0 \\
	0 & 0 & 0 
	\end{bmatrix}.
	\end{align*}
	Assim o posto de $A$ \'e $p = 1$ e a nulidade \'e 2, ou seja, temos duas vari\'aveis livres, a saber $y$ e $z$. Logo a solu\c{c}\~ao \'e dada por
	\[
	x = -3y - z;\quad y,\ z \in \real.
	\]
	Que pode ser escrita como
	\[
	S = \{(x, y ,z) \mid x, y, z \in \real \} = \{(-3y - z, y, z) \mid y, z \in \real\}.
	\]
	\end{solucao}
	\item $\begin{cases}
	\overline{1}x + \overline{4}y + \overline{2}z = \overline{6}\\
	\overline{1}x + \overline{5}y + \overline{2}z = \overline{2}\\
	\overline{2}x + \overline{3}y + \overline{4}z = \overline{4}\\
	\overline{4}x + \overline{5}y + \overline{1}z = \overline{5}
	\end{cases}$ em $\z_7$.
	\begin{solucao}
	A matriz ampliada do sistema \'e
	\[
	A = \begin{amatrix}{3}
	\overline{1} & \overline{4} & \overline{2} & \overline{6}\\
	\overline{1} & \overline{5} & \overline{2} & \overline{2}\\
	\overline{2} & \overline{3} & \overline{4} & \overline{4}\\
	\overline{4} & \overline{5} & \overline{1} & \overline{5}
	\end{amatrix}.
	\]
	Aplicando as opera\c{c}\~oes elementares para reduzir $A$ a forma em escada:
	\begin{align*}
	A = \begin{gmatrix}[b]
	\overline{1} & \overline{4} & \overline{2} & \overline{6}\\
	\overline{1} & \overline{5} & \overline{2} & \overline{2}\\
	\overline{2} & \overline{3} & \overline{4} & \overline{4}\\
	\overline{4} & \overline{5} & \overline{1} & \overline{5}
	\rowops
	\add[\overline{6}]{0}{1}
	\add[\overline{5}]{0}{2}
	\add[\overline{3}]{0}{3}
	\end{gmatrix}\kern-6.23pt\leadsto&\begin{gmatrix}[b]
	\overline{1} & \overline{4} & \overline{2} & \overline{6}\\
	\overline{0} & \overline{1} & \overline{0} & \overline{3}\\
	\overline{0} & \overline{2} & \overline{0} & \overline{6}\\
	\overline{0} & \overline{3} & \overline{0} & \overline{2}
	\rowops
	\add[\overline{3}]{1}{0}
	\add[\overline{5}]{1}{2}
	\add[\overline{4}]{1}{3}
	\end{gmatrix}\leadsto&\begin{amatrix}{3}
	\overline{1} & \overline{0} & \overline{2} & \overline{1}\\
	\overline{0} & \overline{1} & \overline{0} & \overline{3}\\
	\overline{0} & \overline{0} & \overline{0} & \overline{0}\\
	\overline{0} & \overline{0} & \overline{0} & \overline{0}
	\end{amatrix}.
	\end{align*}
	Assim o posto de $A$ \'e $p = 2$ e a nulidade \'e 1. Logo temos uma \'unica vari\'avel livre que \'e $z$. A solu\c{c}\~ao ent\~ao \'e dada por
	\[
	x = \overline{1} + \overline{5}z,\quad y = \overline{3},\quad z \in \z_7.
	\]
	O conjunto solu\c{c}\~ao \'e
	\[
	S = \{(x, y, z) \mid x, y , z \in \z_7\} = \{(\overline{1} + \overline{5}z, \overline{3}, z) \mid z \in \z_7\}.
	\]
	Tal conjunto cont\'em exatamente 7 solu\c{c}\~oes distintas.
	\end{solucao}
	\item $\begin{cases}
	\overline{2}x_1 + \overline{1}x_2 + \overline{2}x_3 + \overline{2}x_4 = \overline{7}\\
	\overline{3}x_1 + \overline{1}x_2 + \overline{2}x_3 + \overline{1}x_4 = \overline{9}\\
	\overline{1}x_1 + \overline{4}x_3 + \overline{3}x_4 = \overline{6}\\
	\overline{5}x_1 + \overline{1}x_3 + \overline{1}x_4 = \overline{9}
	\end{cases}$ em $\z_{11}$.
	\begin{solucao}
	A matriz ampliada do sistema \'e
	\[
	A = \begin{amatrix}{4}
	\overline{2} & \overline{1} & \overline{2} & \overline{2} & \overline{7}\\
	\overline{3} & \overline{1} & \overline{2} & \overline{1} & \overline{9}\\
	\overline{1} & \overline{0} & \overline{4} & \overline{3} & \overline{6}\\
	\overline{5} & \overline{0} & \overline{1} & \overline{1} & \overline{9}
	\end{amatrix}.
	\]
	Aplicando as opera\c{c}\~oes elementares para reduzir $A$ \`a forma em escada:
	\begin{align*}
	A &= \begin{gmatrix}[b]
	\overline{2} & \overline{1} & \overline{2} & \overline{2} & \overline{7}\\
	\overline{3} & \overline{1} & \overline{2} & \overline{1} & \overline{9}\\
	\overline{1} & \overline{0} & \overline{4} & \overline{3} & \overline{6}\\
	\overline{5} & \overline{0} & \overline{1} & \overline{1} & \overline{9}
	\rowops
	\swap02
	\end{gmatrix}\leadsto\begin{gmatrix}[b]
	\overline{1} & \overline{0} & \overline{4} & \overline{3} & \overline{6}\\
	\overline{3} & \overline{1} & \overline{2} & \overline{1} & \overline{9}\\
	\overline{2} & \overline{1} & \overline{2} & \overline{2} & \overline{7}\\
	\overline{5} & \overline{0} & \overline{1} & \overline{1} & \overline{9}
	\rowops
	\add[\overline{8}]{0}{1}
	\add[\overline{9}]{0}{2}
	\add[\overline{6}]{0}{3}
	\end{gmatrix}\\&\leadsto\begin{gmatrix}[b]
	\overline{1} & \overline{0} & \overline{4} & \overline{3} & \overline{6}\\
	\overline{0} & \overline{1} & \overline{1} & \overline{3} & \overline{2}\\
	\overline{0} & \overline{1} & \overline{5} & \overline{7} & \overline{6}\\
	\overline{0} & \overline{0} & \overline{3} & \overline{8} & \overline{1}
	\rowops
	\add[\overline{10}]{1}{2}
	\end{gmatrix}\leadsto\begin{gmatrix}[b]
	\overline{1} & \overline{0} & \overline{4} & \overline{3} & \overline{6}\\
	\overline{0} & \overline{1} & \overline{1} & \overline{3} & \overline{2}\\
	\overline{0} & \overline{0} & \overline{4} & \overline{4} & \overline{4}\\
	\overline{0} & \overline{0} & \overline{3} & \overline{8} & \overline{1}
	\rowops
	\mult2{\times \overline{3}}
	\end{gmatrix}\\&\leadsto\begin{gmatrix}[b]
	\overline{1} & \overline{0} & \overline{4} & \overline{3} & \overline{6}\\
	\overline{0} & \overline{1} & \overline{1} & \overline{3} & \overline{2}\\
	\overline{0} & \overline{0} & \overline{1} & \overline{1} & \overline{1}\\
	\overline{0} & \overline{0} & \overline{3} & \overline{8} & \overline{1}
	\rowops
	\add[\overline{10}]{2}{1}
	\add[\overline{7}]{2}{0}
	\add[\overline{8}]{2}{3}
	\end{gmatrix}\leadsto\begin{gmatrix}[b]
	\overline{1} & \overline{0} & \overline{0} & \overline{10} & \overline{2}\\
	\overline{0} & \overline{1} & \overline{0} & \overline{2} & \overline{1}\\
	\overline{0} & \overline{0} & \overline{1} & \overline{1} & \overline{1}\\
	\overline{0} & \overline{0} & \overline{0} & \overline{5} & \overline{9}
	\rowops
	\mult3{\times \overline{9}}
	\end{gmatrix}\\&\leadsto\begin{gmatrix}[b]
	\overline{1} & \overline{0} & \overline{0} & \overline{10} & \overline{2}\\
	\overline{0} & \overline{1} & \overline{0} & \overline{2} & \overline{1}\\
	\overline{0} & \overline{0} & \overline{1} & \overline{1} & \overline{1}\\
	\overline{0} & \overline{0} & \overline{0} & \overline{1} & \overline{4}
	\rowops
	\add[\overline{10}]{3}{2}
	\add[\overline{9}]{3}{1}
	\add[\overline{1}]{3}{0}
	\end{gmatrix}\\&\leadsto\begin{amatrix}{4}
	\overline{1} & \overline{0} & \overline{0} & \overline{0} & \overline{6}\\
	\overline{0} & \overline{1} & \overline{0} & \overline{0} & \overline{4}\\
	\overline{0} & \overline{0} & \overline{1} & \overline{0} & \overline{8}\\
	\overline{0} & \overline{0} & \overline{0} & \overline{1} & \overline{4}
	\end{amatrix}.
	\end{align*}
	Assim o posto de $A$ \'e $p = 4$ e a nulidade \'e 0. Logo o sistema tem uma \'unica solu\c{c}\~ao dada por
	\[
	x_1 = \overline{6}, x_2 = \overline{4}, x_3 = \overline{8}, x_4 = \overline{4}.
	\]
	\end{solucao}
	\item $\begin{cases}
	x_1 - x_2 + 2x_3 = 4\\
	x_1 + x_3 = 6\\
	2x_1 - 3x_2 + 5x_3 = 4
	\end{cases}$  em $\rac$.
	\begin{solucao}
	A matriz dos coeficentes deste sistema \'e
	\[
	\begin{amatrix}{3}
	1 & -1 & 2 & 4 \\
	1 & \phantom{-}0 & 1 & 6 \\
	2 & -3 & 5 & 4 
	\end{amatrix}.
	\]
	Aplicando as opera\c{c}\~oes elementares para reduzir $A$ \`a forma em escada:
	\begin{align*}
	A &= \begin{gmatrix}[b]
	1 & -1 & 2 & 4 \\
	1 & \phantom{-}0 & 1 & 6 \\
	2 & -3 & 5 & 4 
	\rowops
	\add[-1]{0}{1}
	\add[-2]{0}{2}
	\end{gmatrix}\kern-6.23pt\leadsto\begin{gmatrix}[b]
	1 & -1 & \phantom{-}2 & 4 \\
	0 & \phantom{-}1 & -1 & 2 \\
	0 & -1 & \phantom{-}1 & -4 
	\rowops
	\add{1}{0}
	\add{1}{2}
	\end{gmatrix}\\&\leadsto\begin{amatrix}{3}
	1 & 0 & \phantom{-}1 & \phantom{-}6 \\
	0 & 1 & -1 & \phantom{-}2 \\
	0 & 0 & \phantom{-}0 & -2 
	\end{amatrix}.
	\end{align*}
	Assim o sistema n\~ao tem solu\c{c}\~ao. Note que o posto da matriz ampliada \'e $p = 3$ e a posto da matriz dos coeficientes \'e 2.
	\end{solucao}
\end{enumerate}
\end{exemplo}

\section{Matrizes e Determinantes}

Seja $\cp{K}$ um corpo. Denotamos por $\cp{M}_{p \times q}(\cp{K})$ o conjunto de todas as matrizes $p \times q$ com entradas em $\cp{K}$. A soma e o produtos de matrizes s\~ao definidos de modo usual.

Quando $p = q = n$, dizemos que uma matriz $A \in \cp{M}_{n \times n}(\cp{K})$, que denotaremos simplesmente por $\cp{M}_{n}(\cp{K})$, \'e \textbf{quadrada}.\index{Matriz!Quadrada}
Tal conjunto tem elemento neutro para a multiplica\c{c}\~ao de matrizes que \'e a \textbf{matriz identidade} $I_n$ dada por
\[
\begin{bmatrix}
1_\cp{K} & 0_\cp{K} & 0_\cp{K} & \cdots & 0_\cp{K}\\
0_\cp{K} & 1_\cp{K} & 0_\cp{K} & \cdots & 0_\cp{K}\\
\vdots\\
0_\cp{K} & 0_\cp{K} & 0_\cp{K} & \cdots & 1_\cp{K}
\end{bmatrix}.
\]

\begin{definicao}
Seja $A \in \cp{M}_{n}(\cp{K})$. Uma matriz $B \in \cp{M}_{n}(\cp{K})$ tal que $BA = I_n$ \'e chamada uma \textbf{inversa \`a esquerda} de $A$; uma matriz $C \in \cp{M}_{n}(\cp{K})$ tal que $AC = I_n$ \'e chamada uma \textbf{inversa \`a direita} de $A$. Se $AB = BA = I_n$, ent\~ao $A$ \'e chamada \textbf{invert{\'\i}vel}.
\end{definicao}

\begin{proposicao}
Se $A \in \cp{M}_{n}(\cp{K})$ possui uma inversa \`a esquerda $B$ e uma inversa \`a direita $C$, ent\~ao $B = C$.
\end{proposicao}

\begin{proposicao}
Sejam $A$, $B \in \cp{M}_{n}(\cp{K})$.
\begin{enumerate}[label={\roman*})]
	\item Se $A$ \'e invert{\'\i}vel, ent\~ao $A^{-1}$ tamb\'em o \'e e $(A^{-1})^{-1} = A$.
	\item Se $A$ e $B$ s\~ao invert{\'\i}veis, ent\~ao $AB$ tamb\'em o \'e e $(AB)^{-1} = B^{-1}A^{-1}$.
\end{enumerate}
\end{proposicao}

Dada um matriz $A$ como encontrar sua inversa? Por exemplo, para $A \in \cp{M}_2(\cp{R})$ dada por
\[
A = \begin{bmatrix}
1 & 2\\
3 & 4
\end{bmatrix}
\]
como achar
\[
B = \begin{bmatrix}
x & y\\
z & t
\end{bmatrix}
\]
tal que $AB = BA = I_2$? Queremos que
\[
\begin{bmatrix}
1 & 2\\
3 & 4
\end{bmatrix}\begin{bmatrix}
x & y\\
z & t
\end{bmatrix} = \begin{bmatrix}
1 & 0\\
0 & 1
\end{bmatrix}.
\]
Temos ent\~ao os seguintes sistemas para resolver:
\[
\begin{cases}
x + 2z = 1\\
3x + 4z = 0
\end{cases} \quad \mbox{e}\quad \begin{cases}
y + 2t = 0\\
3y + 4t = 1
\end{cases}.
\]
Assim podemos considerar a matriz ampliada contendo colunas correspondentes a cada um dos sistemas e reduz{\'\i}-la \`a forma em escada:
\begin{align*}
A &= \begin{gmatrix}[b]
1 & 2 & 1 & 0 \\
3 & 4 & 0 & 1
\rowops
\add[-3]{0}{1}
\end{gmatrix}\kern-6.23pt\leadsto&
\begin{gmatrix}[b]
1 & \phantom{-}2 & \phantom{-}1 & 0 \\
0 & -2 & -3 & 1
\rowops
\add{1}{0}
\end{gmatrix}\leadsto&\begin{gmatrix}[b]
1 & \phantom{-}0 & -2 & 1 \\
0 & -2 & -3 & 1
\rowops
\mult1{ \times (-1/2)}
\end{gmatrix}\\&\leadsto\begin{bmatrix}
1 & 0 & -2 & \phantom{-}1 \\
0 & 1 & \phantom{-}3/2 & -1/2
\end{bmatrix}.
\end{align*}

Assim a matriz
\[
B = \begin{bmatrix}
-2 & 1 \\
3/2 & -1/2
\end{bmatrix}
\]
\'e tal que $AB = BA = I_2$.

Portanto, determinar se uma matriz $A \in \cp{M}_n(\cp{K})$ possui inversa ou n\~ao \'e equivalente \`a resolver um sistema linear. Assim, temos o seguinte resultado:
\begin{teorema}
Seja $A \in \cp{M}_n(\cp{K})$. As seguintes afirma\c{c}\~oes s\~ao equivalentes:
\begin{enumerate}[label={\roman*})]

	\item $A$ \'e invert{\'\i}vel;
	\item O sistema homog\^eneo $AX = 0$ possui somente a solu\c{c}\~ao trivial;
	\item O sistema $AX = Y$, onde $Y$ \'e uma matriz $n \times 1$, possui uma \'unica solu\c{c}\~ao para qualquer $Y$.
\end{enumerate}
\end{teorema}

\begin{corolario}
Se $A \in \cp{M}_n(\cp{K})$ \'e invert{\'\i}vel e se uma sequ\^encia de opera\c{c}\~oes elementares sobre linhas reduz $A$ \`a matriz unidade, ent\~ao essa mesma sequ\^encia de opera\c{c}\~oes elementares sobre linhas quando aplicadas \`a matriz $I_n$, resulta em $A^{-1}$.
\end{corolario}

\begin{exemplo}
Seja
\[
A = \begin{bmatrix}
1 & 1 & 0\\
0 & 2 & 1\\
1 & 0 & 1
\end{bmatrix}.
\]
Determine a inversa de $A$, se existir.
\begin{solucao}
Temos
\begin{align*}
A &= \begin{gmatrix}[b]
1 & 1 & 0 & 1 & 0 & 0 \\
0 & 2 & 1 & 0 & 1 & 0\\
1 & 0 & 1 & 0 & 0 & 1
\rowops
\add[-1]{0}{2}
\end{gmatrix}\leadsto\begin{gmatrix}[b]
1 & \phantom{-}1 & 0 & \phantom{-}1 & 0 & 0 \\
0 & \phantom{-}2 & 1 & \phantom{-}0 & 1 & 0\\
0 & -1 & 1 & -1 & 0 & 1
\rowops
\swap12
\end{gmatrix}\\&\leadsto\begin{gmatrix}[b]
1 & \phantom{-}1 & 0 & \phantom{-}1 & 0 & 0 \\
0 & -1 & 1 & -1 & 0 & 1\\
0 & \phantom{-}2 & 1 & \phantom{-}0 & 1 & 0
\rowops
\mult1{-1}
\end{gmatrix}\leadsto\begin{gmatrix}[b]
1 & 1 & \phantom{-}0 & \phantom{-}1 & 0 & 0 \\
0 & 1 & -1 & \phantom{-}1 & 0 & -1\\
0 & 2 & \phantom{-}1 & \phantom{-}0 & 1 & 0
\rowops
\add[-1]{1}{0}
\add[-2]{1}{2}
\end{gmatrix}\\&\leadsto\begin{gmatrix}[b]
1 & 0 & \phantom{-}1 & \phantom{-}0 & 0 & 1 \\
0 & 1 & -1 & -1 & 0 & 1\\
0 & 0 & \phantom{-}3 & -2 & 1 & 2
\rowops
\mult2{1/3}
\end{gmatrix}\leadsto\begin{gmatrix}[b]
1 & 0 & \phantom{-}1 & \phantom{-}0 & 0 & 1 \\
0 & 1 & -1 & -1 & 0 & 1\\
0 & 0 & \phantom{-}1 & -2/3 & 1/3 & 2/3
\rowops
\add[-1]{2}{0}
\add{2}{1}
\end{gmatrix}\\&\leadsto\begin{bmatrix}
1 & 0 & 0 & \phantom{-}2/3 & -1/3 & \phantom{-}1/3 \\
0 & 1 & 0 & \phantom{-}1/3 & \phantom{-}1/3 & -1/3\\
0 & 0 & 1 & -2/3 & \phantom{-}1/3 & \phantom{-}2/3
\end{bmatrix}.
\end{align*}
Logo $A$ \'e invert{\'\i}vel e
\[
A^{-1} = \begin{bmatrix}
\phantom{-}2/3 & -1/3 & \phantom{-}1/3 \\
\phantom{-}1/3 & \phantom{-}1/3 & -1/3\\
-2/3 & \phantom{-}1/3 & \phantom{-}2/3
\end{bmatrix}.
\]
\end{solucao}
\end{exemplo}

Faremos agora a defini\c{c}\~ao de \textbf{determinante} de modo indutivo na ordem de uma dada matriz quadrada $A \in \cp{M}_n(\cp{K})$, $n \ge 1$.

Se $n = 1$, ent\~ao a matriz $A \in \cp{M}_1(\cp{K})$ \'e da forma
\[
A = (a_{11})
\]
e neste caso definimos
\[
\det A = a_{11} \in \cp{K}.
\]
Suponha que $n > 1$ e que $\det B$ esteja definido para todas as matrizes  $B \in \cp{M}_p(\cp{K})$ com $p < n$ e seja 
$A \in \cp{M}_n(\cp{K})$. Para cada $(i,j)$, defina a matriz $A_{ij}$ formada a partir de $A$ retirando-se a sua $i$-\'esima linha e a sua $j$-\'esima coluna. \'E claro que $A \in \cp{M}_{n - 1}(\cp{K})$ e portanto $\det A_{ij}$ est\'a definido. Defina ent\~ao
\[
\det A = \sum_{j = 1}^n(-1)^{i + j}a_{ij}\det A_{ij} \in \cp{K}.
\]

\begin{exemplo}
\begin{enumerate}[label={\arabic*})]
	\item Seja
	\[
	A = \begin{bmatrix}
	a & b\\
	c & d
	\end{bmatrix} \in \cp{M}_2(\cp{K}).
	\]
	Fixada a linha 1, temos
	\[
	\det A = \sum_{j = 1}^2(-1)^{1 + j}a_{1j}\det A_{1j} = (-1)^{1 + 1}a_{11}\det A_{11} + (-1)^{1 + 2}a_{12}\det A_{12} = ad - bc.
	\]
	Obter{\'\i}amos o mesmo resultado se consider\'assemos a linha 2.

	\item Seja
	\[
	A = \begin{bmatrix}
	a_{11} & a_{12} & a_{13}\\
	a_{21} & a_{22} & a_{23}\\
	a_{31} & a_{32} & a_{33}
	\end{bmatrix} \in \cp{M}_2(\cp{K}).
	\]
	Fixada a linha 2, temos
	\begin{align*}
	\det A &= \sum_{j = 1}^3(-1)^{2 + j}a_{2j}\det A_{2j} \\ &= (-1)^{2 + 1}a_{21}\det A_{21} + (-1)^{2 + 2}a_{22}\det A_{22} + (-1)^{2 + 3}a_{32}\det A_{32}\\ &= -a_{21}\det\begin{bmatrix}a_{12} & a_{13}\\a_{32} & a_{33}\end{bmatrix} + a_{22}\det\begin{bmatrix}a_{11} & a_{12}\\a_{31} & a_{33}\end{bmatrix} - a_{23}\det\begin{bmatrix}a_{11} & a_{12}\\a_{31} & a_{32}\end{bmatrix}.
	\end{align*}
	Da{\'\i}
	\[
	\det A = a_{11}a_{22}a_{33} + a_{12}a_{23}a_{31} + a_{13}a_{21}a_{32} - a_{13}a_{22}a_{31} - a_{12}a_{21}a_{33} - 
	a_{11}a_{23}a_{32}.
	\]
\end{enumerate}
\end{exemplo}

\begin{proposicao}
	Sejam $A$, $B \in \cp{M}_n(\cp{K})$ e $\lambda \in \cp{K}$. Temos:
	\begin{enumerate}[label={\roman*})]
		\item $\det(AB) = \det A \det B$,
		\item $\det(\lambda A) = \lambda^n \det(A)$,
		\item $\det(A^{-1}) = (\det A)^{-1}$.
	\end{enumerate}
\end{proposicao}


\begin{proposicao}
	Seja $A$ uma matriz $n \times n$ com entradas num corpo $\cp{K}$.
	\begin{enumerate}[label={\roman*})]
		\item Se $B$ é a matriz resultante da permutação de duas linhas de $A$, então $\det (B) = -\det (A)$.
		\item Se $B$ é a matriz resultante da multiplicação de uma linha de $A$ por um escalar não nulo $\alpha \in \cp{K}$, então $\det(B) = \alpha\det(A)$.
		\item Se $B$ é a matriz resultante da soma da linha $i$ de $A$ com um múltiplo não nulo $\alpha \in \cp{K}$ da linha $j$ de $A$, então $\det(B) = \det(A)$.
	\end{enumerate}
\end{proposicao}

\begin{observacao}
\'E poss{\'\i}vel mostrar que o determinante tamb\'em pode ser definido a partir das colunas de uma matriz $A \in \cp{M}_n(\cp{K})$.
\end{observacao}

\begin{teorema}
Uma matriz $A \in \cp{M}_n(\cp{K})$ \'e invert{\'\i}vel se, e somente se, $\det A \ne 0_\cp{K}$.
\end{teorema}